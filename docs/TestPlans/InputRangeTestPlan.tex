
%-------------------------------------------------------------------------------
%---------------------------Define Document Class and Packages------------------
%-------------------------------------------------------------------------------
\documentclass[10 pt]{book}

%  Standard packages
\usepackage{epstopdf,graphics,overcite,amsmath, color}

%  for making index, using landscape mode, for multi page tables, supertabular??
\usepackage{makeidx, lscape, longtable, supertabular}

%  Used for drawing lines in figures
\usepackage{epic}

%  for showing script lines as they appear in GMAT
\usepackage{verbatim}

%  for using multicolumn format
\usepackage{multicol}

% Use if want to click dvi to ps and ps to pdf to create pdf
%\usepackage[ dvips, bookmarks = true, bookmarksopen ]{hyperref}

% Use if want to click dvi to pdf to create pdf . Bookmarks don't open using this method.
\usepackage[ dvipdfm, bookmarks = true, bookmarksopen]{hyperref}

%\usepackage[T1]{fontenc}
%\usepackage[latin1]{inputenc}
%\usepackage{geometry}
%\usepackage{graphics,color}
%\geometry{letterpaper}
%\usepackage{array}
%\usepackage{floatflt}
%\usepackage{subfloat}
%\usepackage{lineno}
%\usepackage{amsmath}
%\usepackage{fancyvrb}
%%\usepackage[listofnumwidth=5.5em]{subfig}
%\usepackage{tocloft}
%-------------------------------------------------------------------------------
%---------------------------Define  the Page Size-------------------------------
%-------------------------------------------------------------------------------
\setlength{\textwidth}{7.0 in} \setlength{\textheight}{9.5in}
\setlength{\oddsidemargin}{-.2in} \setlength{\evensidemargin}{-.2in}
\setlength{\topmargin}{-0.35in} \setlength{\headheight}{0.0in}
\setlength{\parskip}{8pt}

%-------------------------------------------------------------------------------
%------------------------------------New Commands-------------------------------
%-------------------------------------------------------------------------------
\newcommand{\st}[1]{\begin{ttfamily}#1\end{ttfamily}}
\newcommand{\boldst}[1]{\begin{ttfamily}\textbf{#1} \end{ttfamily}}
\newcommand{\br}[0]{$\mathbf{r} $}
\newcommand{\bv}[0]{$\mathbf{v} $}
\newcommand{\ba}[0]{$\mathbf{a} $}
\newcommand{\mbr}[0]{\mathbf{r} }
\newcommand{\mbv}[0]{\mathbf{v} }
\newcommand{\mba}[0]{\mathbf{a} }

%-------------------------------------------------------------------------------
%------------------------------------New Environments----------------------------
%-------------------------------------------------------------------------------

\newenvironment{ScriptType}
  {\noindent \begin{ttfamily}   }
   { \end{ttfamily} }

\newenvironment{Script}
 { \vspace{-.15 in} \begin{ttfamily} }
 { \end{ttfamily}\vspace{-.25 in} }

%-------------------------------------------------------------------------------
%------------------------------------Miscellaneous------------------------------
%-------------------------------------------------------------------------------
\makeindex
%\let\package\textsf
%\let\env\textsf
\providecommand\finalclearpage{\clearpage}

\graphicspath{{../Common/Images/}}

%-------------------------------------------------------------------------------
%------------------------------------Begin The Doc!!----------------------------
%-------------------------------------------------------------------------------
%  Make the watermark
\usepackage{eso-pic}
\usepackage{color}
\usepackage{type1cm}
\makeatletter
  \AddToShipoutPicture{%
    \setlength{\@tempdimb}{.5\paperwidth}%
    \setlength{\@tempdimc}{.5\paperheight}%
    \setlength{\unitlength}{1pt}%
    \put(\strip@pt\@tempdimb,\strip@pt\@tempdimc){%
      \makebox(0,730){\rotatebox{0}{\textcolor[gray]{0.75}{\fontsize{1.2cm}{1.2cm}\selectfont{Draft: Work in Progress}}}}
    }
} \makeatother

\begin{document}

\thispagestyle{empty}

\begin{center}
{\renewcommand{\thefootnote}{\fnsymbol{footnote}} { \huge \bf
Input Range Testing for the \\General Mission Analysis Tool (GMAT) \\
}}
\end{center}
\begin{center}
{\renewcommand{\thefootnote}{\fnsymbol{footnote}} { \Huge \bf
 DRAFT }}
\end{center}

\begin{figure}[htbp!]
    \begin{center}
    \scalebox{1.25}{\includegraphics*[337,337]{GMATsplash.png}}
    \end{center}
\end{figure}

\begin{center}
Steve Hughes and Edwin Dove\\

NASA Goddard Space Flight Center\\
Greenbelt, MD 20771

\today
\end{center}

\clearpage \clearpage



\tableofcontents \listoftables

\chapter{Test Plan Overview}

This document contains a test plan for testing input values to the
General Mission Analysis Tool (GMAT).  The plan includes four
primary types of information, which rigorously define all tests that
should be performed to validate that GMAT will accept allowable
inputs and deny disallowed inputs. The first is a complete list of
all allowed object fields in GMAT. The second type of information,
is test input to be attempted for each field. The third type of
information is allowable input values for all objects fields in
GMAT.  The final piece of information is how GMAT should respond to
both valid and invalid information.

It is VERY important to note that the tests below must be performed
for both the Graphical User Interface and the script!!  The examples
are illustrated using a scripting perspective, because it is simpler
to write up.  However, the test must be performed for both
interfaces to GMAT.

\section{Tests to Verify Disallowed Fields are Rejected }

The tables contained in this document detail all allowed fields that
the user should have access to in the GMAT user interfaces including
the script and GUI.  The tables are organized in terms of objects.
Each object has a table that defines the allowable field names. The
tables contain a complete definition of what object fields the user
should have access to in the user interface.  The user should not
have access to or be able to set fields that are not contained in
the field lists. This includes being able to set the field from the
GUI or script, or seeing the field when the "Show Script" button is
hit on the object's associated dialogue box or a mission is saved.

These following tests should be performed to ensure that the user
does not have access to fields that are read only or some other type
internal field.

\begin{itemize}

\item Using show script and by saving missions, verify that only
fields that are defined in the tables below are user settable and
accessible.  If a field name does not appear in the tables below, it
should not appear in show script, or a saved mission.

\item Verify that the setting for fields in a saved mission and show
script meet the allowable values.  For example, currently if you hit
show script on an OpenGL plot, you'll see a line like this\\

\noindent \st{OpenGLPlot1.ViewPointVector = Vector};

The word ``Vector" is not an allowable value  for the
\st{ViewPointVector} field.

\item Try \st{GMAT Object.DoesNotExist = 1} for all objects;

\end{itemize}


\section{Input Range Tests for Fields that Accept Numeric Values }

This section documents tests to be performed to ensure that GMAT
accepts allowable values for numeric fields, and denies disallowed
values.  The tests apply to fields that accept real numbers or
integers.  The tests ensure that both allowed and disallowed values
are attempted in the test process.

The tests are described in Table \ref{Table:NumericInputTests}.
Let's illustrate one of the tests and select the case where the
input is a real number such that  $Real \geq c$.  The \st{ECC} field
on the Spacecraft object is in this category.  According to the
Table \ref{Table:NumericInputTests}, the tests to be performed on
the \st{ECC} field are to try inputing values of -1, 0, 1,
'DoesNotExist', and ' ' (Note: according to the notation in the table, $c=0$
for the \st{ECC} field). In the GMAT script syntax, the tests to be
performed on the \st{ECC} field are:\\

\noindent \st{GMAT DefaultSC.ECC = -1.0} \\
\st{GMAT DefaultSC.ECC =  0.0} \\
\st{GMAT DefaultSC.ECC =  1.0} \\
\st{GMAT DefaultSC.ECC =  DoesNotExist} \\
\st{GMAT DefaultSC.ECC =  } \\

We see in Table \ref{Table:SCOrbitFields} that the allowable options
are \st{ECC}$\geq 0$, so we know that only the second and third
lines above should pass the test. The first, fourth, and fifth lines should
throw an error message consistent with the format discussed in Sec.
\ref{Sec:ErrorHandling}.

\begin{table}[h]
    \centering \caption{Input Range Tests for Fields that Accept Real
    Numbers and Integers}
    \begin{tabular}{p{1.75 in} |p{4.0 in} }
    \hline\hline
    Allowable Input Type & Test Input \\
    \hline
    Real  & -1.01 , 1.01 , `DoesNotExist' , ' '\\
    \hline
    Real $\geq$ c & c-1 , c , c+1 , `DoesNotExist' , ' '\\
    Real $\leq$ c & \\
    Real $>$ c & \\
    Real $<$ c & \\
    Real $\neq$ c & \\
    \hline
    $ c_1 > $ Real $>  c_2$  & $c_1-1 , c_1 , (c_1 + c_2)/2 , c_2 , c2+1$, `DoesNotExist' , ' '\\
    $ c_1 \geq $ Real $\geq c_2$  & $c_1-1 , c_1 , (c_1 + c_2)/2 , c_2 , c2+1$ , `DoesNotExist' , ' '\\
    \hline
    Integer  & -1, 1, 1.01 , `DoesNotExist', ' '\\
    \hline
    Integer $\geq$ c & $c- 0.5$, $c - 1$ , c , $c + 0.5$ , $c + 1$ , `DoesNotExist' , ' '\\
    Integer $\leq$ c & \\
    Integer $>$ c & \\
    Integer $<$ c &  \\
    \hline
    $ c_1 > $ Integer $>  c_2$  & $c_1- 0.5 , c_1$ , round$( (c_1 + c_2)/2) , c_2, c2+0.5$ , `DoesNotExist' , ' '\\
    $ c_1 \geq $ Integer $\geq c_2$  & $c_1-0.5 , c_1$ , round$( (c_1 + c_2)/2) , c_2 , c2+0.5$ , `DoesNotExist' , ' '\\
    \hline
    \end{tabular} \label{Table:NumericInputTests}
\end{table}

\section{Input Range Tests for Fields that Accept Strings }

There are numerous fields in GMAT that accept strings as input
values.  For some fields, the list of strings that are acceptable is
predefined, and the user cannot add to the list of possible values.
A good example of this type of field is the \st{StateType} field on
the Spacecraft object.  The allowable inputs for the \st{StateType}
are restricted to the following:
\input{../Common/List_OrbitStateRepresentations}.  The user cannot add to this
list.

The second type of field that accepts strings as input is best
illustrated by the \st{CoordinateSystem} field on the Spacecraft
object.  For the \st{CoordinateSystem} field, the user can select
from the default list of coordinate systems such as
\st{EarthMJ2000Eq}, \st{EarthMJ2000Ec}, or \st{EarthFixed}. However,
the user can also select from any user-defined coordinate system. So
if the user has defined a coordinate system called
\st{EarthSunRotating}, then they can choose this as the coordinate
system defined by the \st{CoordinateSystem} field object on the
Spacecraft object.

For all string fields we should test that numeric values (use 1.0
for value) and the string \st{DoesNotExist} are not accepted by
GMAT. Now let's look at the tests for these two types of fields.

\subsection{Tests for Strings that are Restricted to a Predefined
List}

 For the first type, where the user can only select from a list
of predefined strings, we should attempt two of the strings.  The
first should be the default value, the second should not be a
default value. As an example, for the \st{StateType} field on the
spacecraft object, we should test the following two inputs:\\

\noindent \st{DefaultSC.StateType = Cartesian};\\
\st{DefaultSC.StateType = Keplerian};\\

We should perform similar tests for all fields of this type.

\subsection{Tests for Strings that are not Restricted to a Predefined
List}

For the second type of string input, where the user can select from
a predefined list of strings as well as a user defined list, we
should test one value from the predefined list.  Then we should
create an object of the type appropriate to the field, and test that
the object name can be used as input to the field.  Let's look again
at the \st{CoordinateSystem} field on the spacecraft object as an
example.  We should perform the following:\\

\noindent \st{Create CoordinateSystem MyCoordSys};\\
\st{DefaultSC.CoordinateSystem = EarthMJ2000Eq};\\
\st{DefaultSC.CoordinateSystem = MyCoordSys};\\

We should perform similar tests for all fields of this type.

\section{Behavior When a Disallowed Object Field is Encountered }
\label{Sec:ErrorHandling}

There are two types of error handling described in this section. The
first, is when a user tries to set a field that does not exist, or
that the user is not allowed access to.  The second type, is when
the user tries to set an allowed field, but does not use an
allowable setting for the field.

\subsection{Error Message for a Disallowed Field}

If a field that is not allowed is encountered by GMAT, GMAT should
output the following error message: \\

\noindent The field name `` " on object `` " is not permitted. As an
example, if the user inputs the
following, \\

\noindent \st{DefaultSC.DoesNotExist = 5}\\

\noindent then GMAT should respond with the following error
message:\\

\noindent  The field name ``DoesNotExist" on object DefaultSC is not
permitted.

\subsection{Error Message for an Invalid Input for an Allowed Field}

If the user inputs a disallowed value for an allowed field GMAT
should respond with the following message\\

\noindent The value of `` " for field `` " on object `` " is not an
allowed value.  The allowed values are: [insert options here].\\

For example, if the user
input the following line of script\\

\noindent \st{DefaultSC.StateType = DoesNotExist}\\

\noindent then GMAT should respond with the following error
message:\\

The value of DoesNotExist for field StateType on object DefaultSC is
not an allowed value. The allowed values are: [\st{Cartesian},
\st{Keplerian}, \st{ModifiedKeplerian}, \st{SphericalAZFPA},
\st{SphericalRADEC} , \st{Equinoctial}].

The allowed values for all other fields are found in the the tables
below.


\section{Notes}
 ABM integrator
settings.

For Differential corrector, UseCentralDifferencing is not
implemented in base code.

Normalization of quaternion, are we doing this?

Changed TargetStatus to SolverIterations on XYplots and OpenGL
plots, and added the field to Reports.

The following fields have been changed according to this document

SolarSystem.RotationDataSource = \st{DE405} or false \st{IAU2002}
(used to be 0 or 1)

Equinoctial elements have changed from the labels on the GUI.

\st{Spacecraft.DateFormat} -$>$ \st{Spacecraft.EpochFormat}  for
consistency with the rest of GMAT

ABM integrator settings changed

\st{LowerError}  -$>$  \st{MinIntegrationError} \\
\st{TargetError} -$>$  \st{NomIntegrationError}

\chapter{Objects and Resources}
\label{Ch:ObjectsNResources}

\section{Spacecraft and Hardware Fields}

\input{../Common/Table_SCFields}

\input{../Common/Table_SCAttitudeFields}

\input{../Common/Table_SCTank}

\input{../Common/Table_SCThruster}

\section{Propagator Fields}

\input{../Common/Table_ForceModel}

\input{../Common/Table_Integrator}

\section{Maneuvers}

\input{../Common/Table_ImpulsiveBurn}

\input{../Common/Table_FiniteBurn}

\section{Solver Fields}

\input{../Common/Table_fminconSolver}

\input{../Common/Table_DifferentialCorrector}

\section{Plots and Reports}

\input{../Common/Table_OpenGLPlots}

\input{../Common/Table_Reports}

\input{../Common/Table_XYPlots}

\section{Solar System, Celestial Bodies and other Space Points}

\input{../Common/Table_SolarSystem}

\input{../Common/Table_LibrationPoint}

\input{../Common/Table_BaryCenter}

\input{../Common/Table_CelestialBodies}

\input{../Common/Table_CoordinateSystems}

\input{../Common/Table_MatlabFunctions}

\chapter{Commands and Events}

\section{Propagation}
\input{../Common/Table_PropagateCommand}

\clearpage
\section{Control Flow}
\input{../Common/Table_Ifstatement}

\input{../Common/Table_WhileLoop}

\input{../Common/Table_ForLoop}

\clearpage
\section{Solver-related}
\input{../Common/Table_Target}

\input{../Common/Table_Optimize}

\input{../Common/Table_AchieveCommand}

\input{../Common/Table_VaryCommand}

\input{../Common/Table_Minimize}

\clearpage

\input{../Common/Table_NonLinearConstraint}

\clearpage
\section{Miscellaneous}

\input{../Common/Table_Maneuver}

\input{../Common/Table_BeginFiniteManeuver}

\input{../Common/Table_EndFiniteManeuver}

\input{../Common/Table_CallFunction}

\input{../Common/Table_Toggle}

\input{../Common/Table_ReportCommand}

\input{../Common/Table_ScriptEvent}

\input{../Common/Table_Pause}

\input{../Common/Table_Stop}

\input{../Common/Table_Save}

\printindex

\end{document}
