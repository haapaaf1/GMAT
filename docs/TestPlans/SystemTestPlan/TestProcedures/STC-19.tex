\begin{table}[htbp!]
\centering\footnotesize
      \begin{tabular}{|p{1.05 in} |p{4.75 in} |}
      \hline
         \rowcolor[rgb]{0.8,0.8,0.8} Name & STC-19 Orbit GUI conversion for near singular Cartesian state\\
         \hline
         Requirements & FR-1.3\\  \hline
         PreConditions & BS-1\\     \hline
        Steps &
         %====================== BEGIN: The test procedure
         \begin{compactenum}
             \item Load BS-1.
             \item Open the dialog box for DefaultSC.
             \item Set the Cartesian state to the following values
                 \begin{compactenum}
                    \item X  = 6999.998216286026
                    \item Y = 0
                    \item Z = -5.002359263770285
                    \item VX = 10.63431352889248
                    \item VY = 0
                    \item VZ =  -0.003772975815698364
                    \end{compactenum}
             \item Hit Apply.
             \item Change state type to Keplerian and ensure that the following error is thrown: ``Warning: A nearly singular conic section was encountered while converting from the Cartesian state to the Keplerian elements so conversion was aborted.  The radius of periapsis must be greater than 1 meter."
             \item Change state type to Modified Keplerian and ensure that the following error is thrown: ``Warning: A nearly singular conic section was encountered while converting from the Cartesian state to the Modified Keplerian elements so conversion was aborted.  The radius of periapsis must be greater than 1 meter."
             \item Change state type to Equinoctial and ensure that the following error is thrown:  ``Warning: A nearly singular conic section was encountered while converting from the Cartesian state to the Equinoctial elements so conversion was aborted.  The radius of periapsis must be greater than 1 meter."
             \item Change the following states.
             \begin{compactenum}
                    \item X = 1e-10
                    \item Y = 1e-10
                    \item Z = 1e-10
             \end{compactenum}
             \item Hit Apply
             \item Repeat steps 5, 6, and 7.
             \item Change state type to SphericalRADec and ensure that the following error is thrown:  ``Warning: A nearly singular conic section was encountered while converting from the SphercialRADEC state to the Cartesian State so conversion was aborted.  The Right Ascension and Declination of position are undefined."
             \item Change state type to SphericalAZFPA and ensure that the following error is thrown: ``Warning: A nearly singular conic section was encountered while converting from the SphercialAZFPA state to the Cartesian State so conversion was aborted.  The Right Ascension and Declination of position are undefined."
             \item Change the following states:
                 \begin{compactenum}
                    \item X  = 6999.998216286026
                    \item Y = 0
                    \item Z = -5.002359263770285
                    \item VX = 1e-10
                    \item VY = 1e-10
                    \item VZ = 1e-10
                 \end{compactenum}
             \item Hit Apply
             \item Repeat steps 5, 6, and 7.
             \item Change state type to SphericalRADec and ensure that the following error is thrown:  ``Warning: A nearly singular conic section was encountered while converting from the Cartesian state to the SphericalRADEC so conversion was aborted.  The Right Ascension and Declination of velocity are undefined."
             \item Change state type to SphericalAZFPA and ensure that the following error is thrown: ``Warning: A nearly singular conic section was encountered while converting from the SphercialAZFPA state to the Cartesian State so conversion was aborted.  The Right Ascension and Declination of velocity are undefined."
         \end{compactenum}
         %====================== END: The test procedure
         \\ \hline
         Expected Results & Test results are described above.\\
      \hline
      \end{tabular}
      \label{Table:STC-19}
      \caption{STC-19 Orbit GUI conversion for near singular Cartesian state}
      \index{Test Data!STC-19}
\end{table}

