


\begin{table}[htbp!]
\centering
      \begin{tabular}{|p{1.05 in} |p{4.75 in} |}
      \hline
         \rowcolor[rgb]{0.8,0.8,0.8} Name & STC-9 Orbit GUI Behavior for orbit with zero velocity\\
         \hline
         Requirements & FR-1.3\\  \hline
         Summary &
         %======================  BEGIN: The Test Summary
         This  case tests GUI behavior when attempting to convert to element representations when the
         cartesian state results in a circular, equatorial orbit.
         %======================  END: The Test Summary
         \\     \hline
         PreConditions & BS-1\\     \hline
         Data &
         %====================== BEGIN: The test procedure
         \begin{compactenum}
             \item Load BS-1.
             \item Open the dialog box for DefaultSC.
             \item Enter the following Cartesian State data:
                      \begin{compactenum}
                         \item X = 7000
                         \item Y = 7000
                         \item Z = 7000
                         \item VX = 0;
                         \item VY = 0;
                         \item VZ = 0;
                      \end{compactenum}
             \item Hit Apply.
             \item Change the state to Keplerian and verify that the following error message is returned: The orbit is a singular conic section and the Keplerian elements are undefined.
             \item Change the state to Modified Keplerian and verify that the following error message is returned: The orbit is a singular conic section and the Modified Keplerian elements are undefined.
             \item Change the state to SphericalRADEC and verify that the following error message is returned: The orbit is a singular conic section and the SphericalRADEC elements are undefined.
             \item Change the state to SphericalAZEL and verify that the following error message is returned: The orbit is a singular conic section and the SphericalAZEL elements are undefined.
             \item Change the state to Equinoctial and verify that the following error message is returned:
                   The orbit is a singular conic section and the Equinoctial elements are undefined.
         \end{compactenum}
         %====================== END: The test procedure
         \\ \hline
         Expected Results & The truth data is described above.\\
      \hline
      \end{tabular}
      \label{Table:STC-9}
      \caption{STC-9  Orbit GUI Behavior for orbit with zero velocity}
      \index{Test Data!STC-9}
\end{table} 