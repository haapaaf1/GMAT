
\begin{table}[htbp!]
\centering
      \begin{tabular}{|p{1.05 in} |p{4.75 in} |}
      \hline
         \rowcolor[rgb]{0.8,0.8,0.8} Name & STC-16 Performing Modulo on Equinoctial Angular Elements\\
         \hline
         Requirements & FR-1.3\\  \hline
         Summary &
         %======================  BEGIN: The Test Summary
         This  case tests GUI behavior when attempting to convert to element representations when the
         cartesian state results in a circular, equatorial orbit.
         %======================  END: The Test Summary
         \\     \hline
         PreConditions & BS-1\\     \hline
         Data &
         %====================== BEGIN: The test procedure
         \begin{compactenum}
             \item Load BS-1.
             \item Open the dialog box for DefaultSC.
             \item Change the State Type to Spherical
             \item Change RA to 370.0 degrees.
             \item Change DEC to 380.0 degrees.
             \item Change AZI to 390.0 degrees.
             \item Change FPA t0 400.0 degrees.
             \item Change the State Type to Cartesian.
             \item Change the State Type to Keplerian.
         \end{compactenum}
         %====================== END: The test procedure
         \\ \hline
         Expected Results & RA = 10.0 degrees, DEC = 20 degrees, AZI = 30.0 degrees, and FPA = 40.0 degrees. (All
         values match to 14 sig. figs.)\\
      \hline
      \end{tabular}
      \label{Table:STC-16}
      \caption{STC-16 Performing Modulo on Equinoctial Angular Elements}
      \index{Test Data!STC-16}
\end{table} 