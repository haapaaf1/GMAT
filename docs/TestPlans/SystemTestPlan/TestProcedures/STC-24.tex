\begin{table}[htbp!]
\centering
      \begin{tabular}{|p{1.0 in} |p{5.0 in} |}
         \hline
          \rowcolor[rgb]{0.8,0.8,0.8}  Name & STC-24 State conversion in the spacecraft orbit dialog box\\
         \hline
         Requirements & FRR-2.3\\ \hline
         Summary & This test case represents $n(n-1)$ tests where $n$ is the number of state representations
         supported as input types in GMAT.  Each test case is designated a unique number.  For example,
         STC-24.17 tests GUI conversion from SphericalRADEC to Keplerian elements.  The procedures described below
         must be performed for each test case in the table below.   \\ \hline
         PreConditions & To run this test you need to load BS-1 and have data defined in TD-1 available.\\ \hline
         Steps &
         %------- The test data is here
          \begin{compactenum}
             \item Select subtest number. ( STC-24.17, for example )
             \item Create a new spacecraft.
             \item Change the Epoch Format to the format defined in the first column of
                   the row containing the test case ID.  (SphericalRADEC, for STC-24.17)
             \item Enter the epoch in the Define Format from TD-1.
             \item Change the Epoch Format to the format defined in the first row of the column containing  the test case Id. (Keplerian  for STC-24.17)
             \item Verify that the new state matches the value for that format given in TD-1 to 14 significant figures.
          \end{compactenum}
          \vspace{.1 in}
          \begin{centering}
          \begin{tabular}{|l|c|c|c|c|c|c|c|c|}
          \hline
             & {\begin{sideways}\parbox{2.9cm}{Cartesian}\end{sideways}} &
             {\begin{sideways}\parbox{2.9cm}{Keplerian}\end{sideways}} &
             {\begin{sideways}\parbox{2.9cm}{Mod. Keplerian}\end{sideways}} &
             {\begin{sideways}\parbox{2.9cm}{SphericalRADEC}\end{sideways}} &
             {\begin{sideways}\parbox{2.9cm}{SphericalAZFPA}\end{sideways}}  &
             {\begin{sideways}\parbox{2.9cm}{Equinoctial}\end{sideways}}  \\ \hline
             Cartesian & N/A & 24.1 & 24.2 & 24.3 & 24.4 & 24.5\\ \hline
             Keplerian & 24.6 & N/A & 24.7 & 24.8 & 24.9 & 24.10 \\ \hline
             TAIGregorian & 25.11 & 24.12 & N/A & 24.13 & 24.14 & 24.15 \\ \hline
             SphericalRADEC & 24.16 & 24.17 & 24.18 & N/A & 24.19 & 24.20 \\ \hline
             SphericalAZFPA & 24.21 & 24.22 & 24.23 & 24.24 & N/A & 24.25 \\ \hline
             Equinoctial & 24.26 & 24.27 & 24.28 & 24.29 & 24.30 &  N/A\\ \hline
          \end{tabular}
          \end{centering} \vspace{0.1 in}\\
         %------- The test data is here
         \hline
         Expected Results & The expected numeric results are described above and in TD-1.\\
      \hline
\end{tabular}
   \label{Table:STC-24}
   \caption{STC-24 State Representation conversion in the spacecraft orbit dialog box}
    \index{Test Data!STC-24}
\end{table} 