\clearpage

\begin{table}[htbp!]
\centering
      \begin{tabular}{|p{1.05 in} |p{4.75 in} |}
      \hline
         \rowcolor[rgb]{0.8,0.8,0.8} Name & STC-18 Orbit GUI behavior when Orbit is near parabolic\\
         \hline
         Requirements & FR-1.1\\  \hline
         Summary &
         %======================  BEGIN: The Test Summary
         SMA is undefined for parabolic orbits.  This test check behavior as orbit approaches parabolic
         from ECC < 1 side for Keplerian state type.
         %======================  END: The Test Summary
         \\     \hline
         PreConditions & BS-1\\     \hline
         Steps &
         %====================== BEGIN: The test procedure
         \begin{compactenum}
             \item Load BS-1.
             \item Open the dialog box for DefaultSC.
             \item Change the State Type to Keplerian
             \item Change ECC to 0.99999999
             \item Hit Apply.
         \end{compactenum}
         %====================== END: The test procedure
         \\ \hline
         Expected Results & The following error message should be displayed:
         The value of "0.99999999" for field "ECC" is not an allowed value.
         The allowed values are: [0.0 $leq$ Real Number $\leq$ 0.9999999, or Real Number $\geq$ 1.0000001].
         )\\
      \hline
      \end{tabular}
      \label{Table:STC-18}
      \caption{STC-18 Orbit GUI behavior when performing Modulo on Equinoctial Angular Elements}
      \index{Test Data!STC-18}
\end{table}
\
\begin{table}[htbp!]
\centering\footnotesize
      \begin{tabular}{|p{1.05 in} |p{4.75 in} |}
      \hline
         \rowcolor[rgb]{0.8,0.8,0.8} Name & STC-19 Orbit GUI conversion for near singular Cartesian state\\
         \hline
         Requirements & FR-1.3\\  \hline
         PreConditions & BS-1\\     \hline
        Steps &
         %====================== BEGIN: The test procedure
         \begin{compactenum}
             \item Load BS-1.
             \item Open the dialog box for DefaultSC.
             \item Set the Cartesian state to the following values
                 \begin{compactenum}
                    \item X  = 6999.998216286026
                    \item Y = 0
                    \item Z = -5.002359263770285
                    \item VX = 10.63431352889248
                    \item VY = 0
                    \item VZ =  -0.003772975815698364
                    \end{compactenum}
             \item Hit Apply.
             \item Change state type to Keplerian and ensure that the following error is thrown: ``Warning: A nearly singular conic section was encountered while converting from the Cartesian state to the Keplerian elements so conversion was aborted.  The radius of periapsis must be greater than 1 meter."
             \item Change state type to Modified Keplerian and ensure that the following error is thrown: ``Warning: A nearly singular conic section was encountered while converting from the Cartesian state to the Modified Keplerian elements so conversion was aborted.  The radius of periapsis must be greater than 1 meter."
             \item Change state type to Equinoctial and ensure that the following error is thrown:  ``Warning: A nearly singular conic section was encountered while converting from the Cartesian state to the Equinoctial elements so conversion was aborted.  The radius of periapsis must be greater than 1 meter."
             \item Change the following states.
             \begin{compactenum}
                    \item X = 1e-10
                    \item Y = 1e-10
                    \item Z = 1e-10
             \end{compactenum}
             \item Hit Apply
             \item Repeat steps 5, 6, and 7.
             \item Change state type to SphericalRADec and ensure that the following error is thrown:  ``Warning: A nearly singular conic section was encountered while converting from the SphercialRADEC state to the Cartesian State so conversion was aborted.  The Right Ascension and Declination of position are undefined."
             \item Change state type to SphericalAZFPA and ensure that the following error is thrown: ``Warning: A nearly singular conic section was encountered while converting from the SphercialAZFPA state to the Cartesian State so conversion was aborted.  The Right Ascension and Declination of position are undefined."
             \item Change the following states:
                 \begin{compactenum}
                    \item X  = 6999.998216286026
                    \item Y = 0
                    \item Z = -5.002359263770285
                    \item VX = 1e-10
                    \item VY = 1e-10
                    \item VZ = 1e-10
                 \end{compactenum}
             \item Hit Apply
             \item Repeat steps 5, 6, and 7.
             \item Change state type to SphericalRADec and ensure that the following error is thrown:  ``Warning: A nearly singular conic section was encountered while converting from the Cartesian state to the SphericalRADEC so conversion was aborted.  The Right Ascension and Declination of velocity are undefined."
             \item Change state type to SphericalAZFPA and ensure that the following error is thrown: ``Warning: A nearly singular conic section was encountered while converting from the SphercialAZFPA state to the Cartesian State so conversion was aborted.  The Right Ascension and Declination of velocity are undefined."
         \end{compactenum}
         %====================== END: The test procedure
         \\ \hline
         Expected Results & Test results are described above.\\
      \hline
      \end{tabular}
      \label{Table:STC-19}
      \caption{STC-19 Orbit GUI conversion for near singular Cartesian state}
      \index{Test Data!STC-19}
\end{table}


\begin{table}[htbp!]
\centering
      \begin{tabular}{|p{1.05 in} |p{4.75 in} |}
      \hline
         \rowcolor[rgb]{0.8,0.8,0.8} Name & STC-20 Orbit GUI conversion for near singular SphericalAZEl state \\
         \hline
         Requirements & FR-1.3\\  \hline
         PreConditions & BS-1\\     \hline
        Steps &
         %====================== BEGIN: The test procedure
         \begin{compactenum}
             \item Load BS-1.
             \item Open the dialog box for DefaultSC.
             \item Change the state type to SphercialAZFPA
             \item Set the SphercialAZFPA state to the following values
                 \begin{compactenum}
                    \item RMAG  = 7000.00000
                    \item RA = 0
                    \item DEC = -0.04094487109516581
                    \item VMAG = 10.63431419820442
                    \item AZI = 0
                    \item FPA = 0.02061675296478873
                    \end{compactenum}
             \item Hit Apply.
             \item Change state type to Keplerian and ensure that the following error is thrown: ``Warning: A nearly singular conic section was encountered while converting from the SphercialAZFPA state to the Keplerian elements so conversion was aborted.  The radius of periapsis must be greater than 1 meter."
             \item Change state type to Modified Keplerian and ensure that the following error is thrown: ``Warning: A nearly singular conic section was encountered while converting from the SphercialAZFPA state to the Modified Keplerian elements so conversion was aborted.  The radius of periapsis must be greater than 1 meter."
             \item Change state type to Equinoctial and ensure that the following error is thrown:  ``Warning: A nearly singular conic section was encountered while converting from the SphercialAZFPA state to the Equinoctial elements so conversion was aborted.  The radius of periapsis must be greater than 1 meter."
                 \item Change RMAG to 0.00001 and click apply.
                 \item Repeat steps 5, 6, and 7.
                 \item Change RMAG to 70000.
                 \item Change VMAG to 1e-14.
                 \item Repeat steps 5, 6, and 7.
                 \item Change state type to SphericalRADEC and ensure that the following error is thrown: ``Warning: A nearly singular conic section was encountered while converting from the SphercialAZFPA state to the SphericalRADEC state so conversion was aborted.  The Right Ascension and Declination of velocity are undefined."
         \end{compactenum}
         %====================== END: The test procedure
         \\ \hline
         Expected Results & Test results are described above.\\
      \hline
      \end{tabular}
      \label{Table:STC-20}
      \caption{STC-20 Orbit GUI conversion for near singular SphericalAZEl state}
      \index{Test Data!STC-20}
\end{table} 

\begin{table}[htbp!]
\centering
      \begin{tabular}{|p{1.05 in} |p{4.75 in} |}
      \hline
         \rowcolor[rgb]{0.8,0.8,0.8} Name & STC-20 Orbit GUI conversion for near singular SphericalRADEC state \\
         \hline
         Requirements & FR-1.3\\  \hline
         PreConditions & BS-1\\     \hline
        Steps &
         %====================== BEGIN: The test procedure
         \begin{compactenum}
             \item Load BS-1.
             \item Open the dialog box for DefaultSC.
             \item Change the state type to SphercialRADEC.
             \item Set the SphercialRADec state to the following values:
                 \begin{compactenum}
                    \item RMAG  = 7000.00000
                    \item RA = 0
                    \item DEC = -0.04094487109516581
                    \item VMAG = 10.63431419820442
                    \item RAV = 0
                    \item DECV = -0.0203281181043372
                    \end{compactenum}
             \item Hit Apply.
             \item Change state type to Keplerian and ensure that the following error is thrown: ``Warning: A nearly singular conic section was encountered while converting from the SphercialRADEC state to the Keplerian elements so conversion was aborted.  The radius of periapsis must be greater than 1 meter."
             \item Change state type to Modified Keplerian and ensure that the following error is thrown: ``Warning: A nearly singular conic section was encountered while converting from the SphercialRADEC state to the Modified Keplerian elements so conversion was aborted.  The radius of periapsis must be greater than 1 meter."
             \item Change state type to Equinoctial and ensure that the following error is thrown:  ``Warning: A nearly singular conic section was encountered while converting from the SphercialRADEC state to the Equinoctial elements so conversion was aborted.  The radius of periapsis must be greater than 1 meter."
                 \item Change RMAG to 0.00001 and click apply.
                 \item Repeat steps 5, 6, and 7.
                 \item Change RMAG to 70000.
                 \item Change VMAG to 1e-14.
                 \item Repeat steps 5, 6, and 7.
                 \item Change state type to SphericalRADEC and ensure that the following error is thrown: ``Warning: A nearly singular conic section was encountered while converting from the SphercialRADEC state to the SphericalAZEL state so conversion was aborted.  The Azimuth and Flight Path Angle are undefined."
         \end{compactenum}
         %====================== END: The test procedure
         \\ \hline
         Expected Results & Test results are described above.\\
      \hline
      \end{tabular}
      \label{Table:STC-21}
      \caption{STC-21 Orbit GUI conversion for near singular SphericalRADEC state}
      \index{Test Data!STC-21}
\end{table} 

\begin{table}[htbp!]
\centering
      \begin{tabular}{|p{1.05 in} |p{4.75 in} |}
      \hline
         \rowcolor[rgb]{0.8,0.8,0.8} Name & STC-27 Performing Modulo on Keplerian Elements for Circular, Equatorial orbit\\
         \hline
         Requirements & FR-1.1\\  \hline
         Summary &
         %======================  BEGIN: The Test Summary
         This  case tests GUI behavior when attempting to convert to element representations when the
         cartesian state results in a circular, equatorial orbit.
         %======================  END: The Test Summary
         \\     \hline
         PreConditions & BS-1\\     \hline
         Data &
         %====================== BEGIN: The test procedure
         \begin{compactenum}
             \item Load BS-1.
             \item Open the dialog box for DefaultSC.
             \item Change the State Type to Keplerian
             \item Change ECC to 0.0.
             \item Change INC to 0.0 degrees.
             \item Change RAAN to 380.0 degrees.
             \item Change AOP to 390.0 degrees.
             \item Change TA to 430.0 degrees.
             \item Hit Apply.
         \end{compactenum}
         %====================== END: The test procedure
         \\ \hline
         Expected Results &  RAAN = 0.0 degrees, AOP = 0.0 degrees, and TA = 120.0 degrees. (All
         values match to 14 sig. figs.)\\
      \hline
      \end{tabular}
      \label{Table:STC-27}
      \caption{STC-27 Performing Modulo on Keplerian Elements for Circular, Equatorial orbit}
      \index{Test Data!STC-27}
\end{table} 

\begin{table}[htbp!]
\centering
      \begin{tabular}{|p{1.05 in} |p{4.75 in} |}
      \hline
         \rowcolor[rgb]{0.8,0.8,0.8} Name & STC-28 Performing Modulo on Keplerian Elements for Circular, Inclined Orbit\\
         \hline
         Requirements & FR-1.1\\  \hline
         PreConditions & BS-1\\     \hline
         Data &
         %====================== BEGIN: The test procedure
         \begin{compactenum}
             \item Load BS-1.
             \item Open the dialog box for DefaultSC.
             \item Change the State Type to Keplerian
             \item Change ECC to 0.0.
             \item Change INC to 45 degrees.
             \item Change RAAN to 380.0 degrees.
             \item Change AOP to 390.0 degrees.
             \item Change TA to 430.0 degrees.
             \item Hit Apply.
         \end{compactenum}
         %====================== END: The test procedure
         \\ \hline
         Expected Results &  RAAN = 20.0 degrees, AOP = 0.0 degrees, and TA = 70.0 degrees. (All
         values match to 14 sig. figs.)\\
      \hline
      \end{tabular}
      \label{Table:STC-28}
      \caption{STC-28 Performing Modulo on Keplerian Elements for Circular, Equatorial orbit}
      \index{Test Data!STC-28}
\end{table} 