

\begin{table}[htbp!]
\centering
      \begin{tabular}{|p{1.05 in} |p{4.75 in} |}
      \hline
         \rowcolor[rgb]{0.8,0.8,0.8} Name & TC-2 Differential Corrector Dialog Box Range Tests- Allowed Values\\
         \hline
         Requirements & FR-19\\  \hline
         Summary &
         %======================  BEGIN: The Test Summary
         This case verifies the Differential Corrector accepts allowed data.
         %======================  END: The Test Summary
         \\     \hline
         PreConditions & BS-1\\     \hline
         Data &
         %====================== BEGIN: The test procedure
         \begin{compactenum}
             \item Load BS-1.
             \item In the solvers folder in the mission tree, right-click on the Boundary Value Solvers folder and add a Differential Corrector.
             \item Open the dialog box for the new Differential Corrector.
             \item Set the Max Iterations to 56.
             \item In the ReportFile field type \verb".\output\DCReport.txt"
             \item Uncheck the ShowProgress box.
             \item Set the DerivativeMethod drop-down menu to CentralDifference.
             \item Set the ReportStyle drop-down menu to Verbose.
             \item Click the Apply button.
             \item Click the Show Script button.
         \end{compactenum}
         %====================== END: The test procedure
         \\ \hline
         Expected Results &
             \begin{verbatim}
 Create DifferentialCorrector DC1;
 GMAT DC1.ShowProgress = false;
 GMAT DC1.ReportStyle = 'Verbose';
 GMAT DC1.ReportFile = '.\output\DCData.txt';
 GMAT DC1.MaximumIterations = 56;
 GMAT DC1.DerivativeMethod = CentralDifference;
             \end{verbatim}
         \\   \hline
      \end{tabular}
      \label{Table:TC-2}
      \caption{TC-2 Differential Corrector Dialog Box Range Tests- Allowed Values}
      \index{Test Data!TC-2}
\end{table} 