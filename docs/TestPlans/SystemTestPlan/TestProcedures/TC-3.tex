\begin{table}[htbp!]
\centering
      \begin{tabular}{|p{1.05 in} |p{4.75 in} |}
      \hline
         \rowcolor[rgb]{0.8,0.8,0.8} Name & TC-3 Adding and Deleting All Commands\\
         \hline
         Requirements & FR-XXX\\  \hline
         Summary & This case tests adding and deleting all supported GMAT commands via the Mission
         Tree GUI.  This test does not check for adding and deleting commands inside of branch commands.\\  \hline
         PreConditions & BS-5\\     \hline
         Data &
         %====================== BEGIN: The test procedure
         \begin{compactenum}
             \item Load BS-1.
             \item Click on the mission tree tab.
             \item Right-click on the Mission Tree folder, select Append, then BeginFiniteBurn
             \item Follow the steps in the line above for the following commands:
                \begin{compactenum}
                    \item Maneuver
                    \item CallFunction
                    \item Optimize
                    \item Equation
                    \item Report
                    \item Save
                    \item Toggle
                    \item If
                    \item If/Else
                    \item BeginFiniteBurn
                    \item Propagate
                    \item For
                    \item While
                    \item PenUp
                    \item PenDown
                    \item RunEstimator
                    \item EndFiniteBurn
                    \item RunSimulator
                    \item ScriptEvent
                    \item Stop
                    \item Target
                \end{compactenum}
                \item Right-click on the Maneuver command and the top of the tree and select delete.
                \item Perform the same steps above and delete all commands in the order they were created.
         \end{compactenum}
         %====================== END: The test procedure
         \\ \hline
         Expected Results & All objects are created and deleted and no error messages or warnings should be thrown.  )\\
      \hline
      \end{tabular}
      \label{Table:TC-3}
      \caption{TC-3 TEST NAME TEXT}
      \index{Test Data! TC-3}
\end{table} 