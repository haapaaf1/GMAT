\begin{table}[htbp!]
\centering
      \begin{tabular}{|p{1.0 in} |p{5.0 in} |}
         \hline
          \rowcolor[rgb]{0.8,0.8,0.8}  Name & STC-23 Epoch conversion in the spacecraft orbit dialog box\\
         \hline
         Requirements & FRR-2.3\\ \hline
         Summary & This test case represents $n(n-1)$ tests where $n$ is the number of epoch formats
         supported as input types in GMAT.  Each test case is designated a unique number.  For example,
         STC-23.32 test GUI conversion from A1Gregorian to TAIModJulian.  The procedures described below
         must be performed for each test case in the table below.   \\ \hline
         PreConditions & To run this test you need to load BS-1 and have data defined in TD-2 available.\\ \hline
         Steps &
         %------- The test data is here
          \begin{compactenum}
             \item Select subtest number. ( STC-23.32, for example)
             \item Create a new spacecraft.
             \item Change the Epoch Format to the format defined in the first column of
                   the row containing the test case ID.  (A1Gregorian, for STC-23.32)
             \item Enter the epoch in the Define Format from TD-2.
             \item Change the Epoch Format to the format defined in the first row of the column containing  the test case Id. (TAIModJulian  for STC-23.32)
             \item Verify that the new epoch exactly matches the value for that format given in TD-2.
          \end{compactenum}
          \vspace{.1 in}
          \begin{centering}
          \begin{tabular}{|c|c|c|c|c|c|c|c|c|}
          \hline
             & {\begin{sideways}\parbox{2.5cm}{UTCGregorian}\end{sideways}} &
             {\begin{sideways}\parbox{2.5cm}{UTCModJulian}\end{sideways}} &
             {\begin{sideways}\parbox{2.5cm}{TAIGregorian}\end{sideways}} &
             {\begin{sideways}\parbox{2.5cm}{TAIModJulian}\end{sideways}} &
             {\begin{sideways}\parbox{2.5cm}{A1Gregorian}\end{sideways}}  &
             {\begin{sideways}\parbox{2.5cm}{A1ModJulian}\end{sideways}}  &
             {\begin{sideways}\parbox{2.5cm}{TTGregorian}\end{sideways}}  &
             {\begin{sideways}\parbox{2.5cm}{TTModJulian}\end{sideways}}  \\ \hline
             UTCGregorian & N/A & 23.1 & 23.2 & 23.3 & 23.4 & 23.5 & 23.6 & 23.7 \\ \hline
             UTCModJulian & 23.8 & N/A & 23.9 & 23.10 & 23.11 & 23.12 & 23.13 & 23.14\\ \hline
             TAIGregorian & 23.15 & 23.16 & N/A & 23.17 & 23.18 & 23.19 & 23.20 & 23.21\\ \hline
             TAIModJulian & 23.22 & 23.23 & 23.24 & N/A & 23.25 & 23.26 & 23.27 & 23.28\\ \hline
             A1Gregorian & 23.29 & 23.30 & 23.31 & 23.32 & N/A & 23.33 & 23.34 & 23.35\\ \hline
             A1ModJulian & 23.36 & 23.37 & 23.38 & 23.39 & 23.40 &  N/A & 23.41 & 23.42\\ \hline
             TTGregorian & 23.43 & 23.44 & 23.44 & 23.45 & 23.46 & 23.47 & N/A & 23.48\\ \hline
             TTModJulian & 23.49 & 23.50 & 23.51 & 23.52 & 23.53 & 23.54 & 23.55 & N/A\\ \hline
          \end{tabular}
          \end{centering} \vspace{0.1 in}\\
         %------- The test data is here
         \hline
         Expected Results & The expected numeric results are described above and in TD-2.\\
      \hline
\end{tabular}
   \label{Table:STC-23}
   \caption{STC-23 Epoch conversion in the spacecraft orbit dialog box}
    \index{Test Data!STC-23}
\end{table} 