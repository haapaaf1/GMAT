

\begin{table}[htbp!]
\centering
      \begin{tabular}{|p{1.05 in} |p{4.75 in} |}
      \hline
         \rowcolor[rgb]{0.8,0.8,0.8} Name & STC-6 Orbit State Conversion for Singular Conic Section\\
         \hline
         Requirements & FR-1.3\\  \hline
         Summary &
         %======================  BEGIN: The Test Summary
         This  case tests GUI behavior when attempting to convert to element representations with
         a cartesian state that results in a singular conic section.
         %======================  END: The Test Summary
         \\     \hline
         PreConditions & BS-2 and TD-4\\     \hline
         Data &
         %====================== BEGIN: The test procedure
         \begin{compactenum}
             \item Load BS-1.
             \item Open the dialog box for DefaultSC.
             \item Enter the Cartesian state data from TD-5.
             \item Hit Apply.
             \item Change the State Type to Keplerian and verify the following error message is thrown:
             "GMAT does not support parabolic orbits in conversion from Cartesian to Keplerian state".
             \item Change the State Type to Modified Keplerian and verify the following error message is thrown:
             "GMAT does not support parabolic orbits in conversion from Cartesian to Keplerian state".
             \item Change the state to SphericalRADEC and verify the numeric data with TD-5.
             \item Change the state to SphericalAZEL and verify the numeric data with TD-5.
             \item Change the State Type to Equinoctial and verify the following error message is thrown:
             "GMAT does not support parabolic orbits in conversion from Cartesian to Equinoctial state".
         \end{compactenum}
         %====================== END: The test procedure
         \\ \hline
         Expected Results & The only State Types available are Cartesian, SphericalRADEC, and SphericalAZFPA.   State Types Keplerian, Modified Keplerian, and Equinoctial are NOT available because they are undefined.\\
      \hline
\end{tabular}
      \label{Table:STC-6}
      \caption{STC-6 Orbit State Conversion for Singular Conic Section}
      \index{Test Data!STC-6}
\end{table} 