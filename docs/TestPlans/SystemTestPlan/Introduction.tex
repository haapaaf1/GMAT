\chapter{Introduction}

%\chapauthor{Darrel J. Conway\\Thinking Systems, Inc.}

\section{Overview}

The General Mission Analysis Tool (GMAT) is a spacecraft mission analysis tool tailored to support
missions involving groups of spacecraft interacting throughout a modeled time period.  The potential
complexity of this problem makes GMAT an intricate software system.  This complexity necessitates a
rigorous testing environment to ensure that the system meets its objectives.

GMAT is designed using an object-oriented architecture\cite{GDT} and coded using extensive
object-oriented structures written in C++.  The object based approach employed in GMAT's design and
implementation makes the system robust and relatively easy to use for experienced analysts.  The
extent of the object model implemented to make GMAT a complete and robust system dictates a
comprehensive testing philosophy, described in the GMAT Master Test Plan\cite{MTP}.  This document
describes one component of the overall testing strategy, the system testing.

System testing is a black box form of testing, designed to exercise the GMAT system from the user's
perspective.  The system tests are designed to exercise all of the user accessible objects in GMAT.

\section{Purpose of this Document}

This document serves as the System Test Approach for the GMAT Project.  Preparation for system
testing consists of three major stages:

\begin{itemize}
\item The Test Approach sets the scope of system testing, the overall strategy to be adopted, the
activities to be completed, the general resources required and the methods and processes to be used
to test the release. It also details the activities, dependencies and effort required to conduct the
System Test.
\item Test Planning details the activities, dependencies and effort required to conduct the System
Test.
\item Test Cases documents the tests to be applied, the data to be processed, the automated testing
coverage and the expected results.
\end{itemize}

This document covers the first two of these items, and established the framework used for the GMAT
test case development.  The test cases themselves exist as separate components, and are managed
outside of and concurrently with this System Test Plan.

\section{Overview of the GMAT Development and Testing Process}

The GMAT development process identifies several review points for the system.  GMAT development is
conducted as a cooperative effort between an analysis team, typically composed of flight dynamics
specialists, and a development team consisting of talented software developers.  New requirements
for the system are defined and written by the analysis team.  Mathematical and design specifications
are derived from these requirements and compiled into a format that can be used to code the new
functionality.  Requirements, Specifications, and Designs are reviewed by the development team prior
to implementation.  This review is typically conducted in an informal, iterative manner until the
specifications are understood by all involved parties.  The specifications and design documentation
are then used to write the software.

During the development process, new features of a component under development may be detected that
need further specification.  When that happens, the new features are discussed and collected
together.  This may result in an immediate update to the design documents, or it may result in
collection of the new feature implementation for inclusion in a final update performed when the
component is ready for integration.  In either case, the design documentation is updated to reflect
the implemented functionality prior to formal acceptance of the related components.

During development, the software undergoes internal testing in the development team at both a unit
and an integration level.  Unit testing is intended to exercise all of the executable paths through
the code, validating that the internal working of the code behaves correctly.  Integration testing
takes unit tested components and builds those components, either one at a time or collectively, into
the system.  From time to time, the development team will interact with the analysis team during
integration testing to confirm that the observed behavior of the new code conforms to the
expectations of the users.  Unit testing and integration testing are performed in the course of the
development of the software; neither will necessarily provide test results in a formal manner,
though informal communications of the component and integrated test results are strongly encouraged.

When the GMAT development team completes integration of new functionality into the system, that new
functionality is ready for system test.  GMAT system testing follows a more formal test procedure
than unit or integration testing.  New components are exercised both from the GMAT scripting
language and from the GMAT Graphical User Interface (GUI).  The test cases exercised are documented
using the procedures described later in this document.  Test cases are managed using a traceability
matrix that lists all of the elements of GMAT visible at the user level, and matches those elements
to test cases that are executed in system testing.  This master traceability matrix is used to
generate a spreadsheet of test cases each time GMAT enters a system test cycle.  All tests are
tracked using this spreadsheet; formal system test is complete when every test case has been
exercised and the results of the tests have been tabulated and accepted after review.

\section{System Test Objectives}

At a high level, System Test intends to prove that

\begin{itemize}
\item The functionality, delivered by the development team, is as specified by the Mathematical and
Design Specifications\footnote{System test does not provide a formal mechanism for mapping the
system requirements to the implemented functionality; that is the responsibility of Acceptance
testing.  The system test validates that the implemented functionality is correct.}.
\item The software is stable and of high quality.
\item The software models spacecraft missions faithfully.
\item The software interfaces correctly with other systems, specifically MATLAB.
\item The software user interfaces are stable, complete, and understandable by novice and
experienced users.
\end{itemize}

These objectives are addressed through the development of a suite of test cases exercised on builds of the GMAT system.  Each major release of GMAT is tested using this suite, and the results of the
tests are collected and reviewd by all interested parties prior to release.  This document
describes the procedures followed for system testing.

\section{Formal System Testing}

While system tests can be performed as soon as new features are available, there is not a
requirement that they must be performed at that time.  However, system tests shall be performed
prior to each major release of GMAT to the aerospace community.  Part of the GMAT release process
includes a review of the system test matrices and results to ensure that the system has maintained
its integrity for the release.  The review performed at each major release:

\begin{itemize}
\item Checks the System Test matrices to ensure full system coverage for User Elements, Parameters,
Commands, and GUI Widgets.
\item Ensures that the system tests have been run for all test cases.
\item Ensures that the data produced from GMAT is consistent with known ``truth'' data.
\item Ensures that system tests that failed have documented the cause or causes of the failure
\item Ensures that any failures that must be addressed for the release have (1) been addressed and
(2) that the resulting correction has been validated to meet the expected results.
\item Ensures that all scripting elements of GMAT have been exercised, and function correctly.
\item Ensures that all GUI elements of GMAT have been exercised, and function correctly.
\item Ensures that the system is stable.  Stability in this context means that GMAT
\begin{itemize}
\item Does not crash
\item Produces identical results on rerun
\item Produces comparable results on all supported platforms
\item Allocates and releases memory consistently, without long term memory artifacts (aka ``memory
leaks'')
\item Produces identical results when configured from the GUI, from a script file, and when saved to
file and reloaded, both into the running instance and into a new image.
\end{itemize}
\item Ensures that GMAT performs efficiently, both when executing mission sequences, and when saving
and loading missions.
\end{itemize}

System test review is performed by members of the analysis and development teams.  Detailed testing
of the system numerics and scripting is performed by the domain experts on the analysis team.  GUI
testing is performed by the development team.

While the formal test responsibilities are as described in the previous paragraph, both teams are
encouraged to exercise the features being tested by the other team to help identify any additional
issues that exist.  For example, the analysis team is encouraged to create all test cases using the
GMAT GUI, and to report any difficulties encountered when following this approach.  Similarly, the
development team is encouraged to test the GUI in such a way as to produce functional models, to
run those models, and to report any resulting anomalous behavior.  This cross checking of
functionality ensures that the system has been exercised as much as possible, given the resources
available for development of GMAT.

\section{Items Not Addressed in System Tests}

The system tests described in this document are used to validate the stability and accessibility of
GMAT components to users attempting to use the system to solve flight dynamics problems.  These
tests do not address several key system elements.  Those elements are covered by other components of
the GMAT test suite.

Specifically, the tests defined in this document do not address these items:

\begin{itemize}
\item Internal data representations and data flow in the GMAT code.  These elements are tested in
the GMAT unit and integration test processes.
\item Numerical fidelity of the models.  The detailed numerical testing of the components are part
of the GMAT acceptance tests.
\item Data range validation.  The data range tests are performed as part of the integration tests.
\item Requirements Validation.  The mapping of GMAT capabilities to the system requirements is made
and validated in the GMAT acceptance tests.
\end{itemize}

\section{Document Layout}

The remainder of this document describes the procedures followed to prepare for, conduct, and
document the GMAT system tests.  Chapter~\ref{chap:testprep} describes the procedured followed
when preparing for the system tests.  Chapters~\ref{chap:scripttests} and~\ref{chap:guitests}
document the procedures followed when running the test cases.  Chapter~\ref{chap:reporting}
describes the data collection and review procedures followed for the system.  The Appendices at the
end of the document provide additional information that may be useful during system test.
