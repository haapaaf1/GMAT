%  This is the list of tests associated with panel aesthetics
\subsubsection{Aesthetics}

Description:  This set of tests verifies platform-specific look and feel of a panel, as extended
by the GMAT GUI Philosophy document\cite{Dove}. Each criterion must be met to pass the aesthetics
tests.

\begin{itemize}
\item All of the data input fields and bounding boxes can be seen at the panel size displayed when
the panel is first opened, for all tabs on the panel.
\item The blank space surrounding the data area is not distracting, and does not dominate the
appearance of the interface.  As a guideline, for platforms that allow control of the surrounding
white space, that region should not consume more than 20\% of the total space dedicated to the
panel when it is opened.
\item The data area does not appear too crowded; the surrounding blank space is appropriately sized.
\item The window cannot be resized so that the data cannot be seen.
\end{itemize}

\subsubsection{General Panel Functionality}

Description:  This is the list of tests associated with basic panel functionality: open, close,
rename, minimize, ok, cancel, help, show script, command summary.  Additionally, the behavior of
open panels needs to be consistent with deletion actions taken on the resource and mission trees 
-- if an object in the tree is deleted, any open panel associated with that object should close.
All of these functions must pass.

\begin{itemize}
\item New objects of the type being tested can be created from the appropriate tree on the Resource
or Mission panels.
\item Double clicking in a new object opens the panel for that object.
\item Double clicking in a object that has an open panel brings the panel for that object to the
front of the displayed panels.
\item New objects can be renamed.
\item Default objects, when they exist, can be renamed.
\item Default objects, when they exist, can be deleted.
\begin{itemize}
\item The object can be renamed.
\item References to the renamed object are updated in related elements of the system.
\end{itemize}
\item Renaming works after making changes to the data on the object panel.
\begin{itemize}
\item The object can be renamed while the panel is open.
\item A change can be made on the panel, and then the object can be renamed before the change is
applied.
\item A change can be made on the panel, the change can be applied, and then the object can be
renamed.
\item For each of the above cases, references to the object's name are updated throughout the
system when the object's name is changed.
\end{itemize}
\item Changes made on the panel and applied using the OK button appear on the panel when it is
reopened.
\item Changes made on the panel and applied using the Apply button are visible in the script when
viewed using the Show Script dialog.
\item When you open the panel, make a minor change in the panel, and click button to close the
panel (on Windows, this is the small ``x'' button in the upper right hand corner; on the Mac, it is
the red button on the left side of the frame controls, and on Linux, varies based on the
configuration of the Linux window manager), you are prompted to save data before closing. Check
that:
\begin{itemize}
\item The prompt does appear.
\item Selecting ``Yes'' updates the underlying data.
\item Selecting ``No'' discards the changes.
\end{itemize}
\item Cancelling closes the opened panel without changing the underlying data.
\begin{itemize}
\item The object does not change when you open the panel and press the Cancel button without making
any changes.
\item The object does not change when you open the panel, make a minor change in the data, and press
the Cancel button.
\item The object does not change when you open the panel and click the close button in the panel's
frame to close the panel, but the panel does close without prompting.
\end{itemize}
\item The panel is minimized when the minimize button on the panel frame is pressed.
\item The panel reopens to previous size when maximize icon on the minimized panel is pressed
\item The tab key navigates the open panel in agreement with style and GUI design philosophy. 
Navigation is orderly and sensible using the tab key.
\end{itemize}

\subsubsection{Panel Data Element Completeness and Correctness}

Description: This set of tests verifies that all data elements that should appear on the panel
are present on the panel.  It also tests that all elements that should appear in ``Show Script''
dialog appear there, and that items that should not appear in show script do not appear there.

\begin{itemize}
\item Verify that only data elements that occur in the Range Test Plan appear in show script and
that the user does not see any other object fields.
\item Verify that defaults agree with the values in the Range Test Plan.
\item Press the ``Show Script'' button, and verify that all elements on the GUI panel also appear on
the show script dialog.  Verify that these elements match the description in the Range Test Plan.
\item  Verify that all data elements that appear in Show Script also appear on the GUI.  (This
step validates that all scriptable settings also appear in the GUI.)
\end{itemize}
