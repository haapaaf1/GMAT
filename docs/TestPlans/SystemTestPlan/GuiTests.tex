\chapter{\label{chap:guitests}Executing Tests for the Graphical User Interface}

The tests described in this chapter are designed to exercise all of the controls and other elements
visible from the GMAT graphical user interface (GUI).  The GMAT GUI is designed to present a
consistent, easy to use interface into the underlying engine so that users of the system can view,
configure, and interact with the elements of the system during all phases of mission modeling.
System testers work with these elements, using them both to perform the expected tasks and to
attempt to perform undesired actions.  The former set of actions exercises the engine to ensure
that the system can be configured correctly.  The latter tests are run to ensure that users cannot
configure GMAT incorrectly.

\begin{figure}[htb]
\begin{center}
\includegraphics[460,372]{Images/GuiTestTracker.png}
\caption{\label{figure:GuiTestTracker}The GUI Test Tracking Spreadsheet}
\end{center}
\end{figure}

\section{GUI Test Case Management}

The GUI test cases are managed using a test tracking spreadsheet generated at the end of test
preparation, described in Chapter~\ref{chap:testprep}.  Figure~\ref{figure:GuiTestTracker} shows an
example of this spreadsheet partway through a testing cycle.

The test procedure for GUI based tests requires extensive exercising of the components in the GUI.
Testers follow these steps when executing the system tests:

\begin{enumerate}
\item\label{item:GetGuiTestcases} Obtain the latest versions of the GUI test cases and a local copy
of the test case tracking spreadsheet\footnote{The test tracking spreadsheet is generated from the
Systen Test Matrix spreadsheet using an OpenOffice macro, as described in
Section~\ref{section:CompleteCoverage}.}.
\item Identify the tests that the tester needs to run.
\item Open GMAT\footnote{GMAT should only be opened one time for any given testing period.  All
tests run during that test period -- typically a morning or afternoon -- should be run in the same
instance of GMAT.  This helps ensure that the system is stable over long periods of time.  If the
system is shut down, either by the user or through a system crash, that event should be noted.}.
\item Run each test following the procedure in Section~\ref{section:RunningGuiTests}.
\item As each test is run, record the results of the test on the test case worksheet retrieved in
step~\ref{item:GetGuiTestcases}.
\item When anomalies are found in testing, record them on the test case worksheet and enter them in
the bug tracking database.
\item Close GMAT at the end of the test period.
\item At the end of each day or when testing is finished, whichever occurs first, gather the
completed test case worksheets and place them in the folder used to gather the test results.
\item At the end of each day or when testing is finished, whichever occurs first, save the local
gui test tracking spreadsheet with the name <spreadsheetName>\_<tester's initials> in the folder
used to gather the test results.
\item Upon completion of all assigned test cases, report that status to the system test lead.
\end{enumerate}

The procedure for running a single test case is described next.

\section{\label{section:RunningGuiTests}Running the GUI System Tests}

By their very nature, the script based tests described in Chapter~\ref{chap:scripttests} follow a
linear execution sequence once the scripts have been written and debugged.  In contrast,
interactions performed using the GMAT GUI are less structured -- users can use the controls on the
GUI in a seemingly random fashion -- so the test cases for the GUI include allowances for
interacting with the GUI elements by the tester in a more free form manner than the script based
tests allow.

\subsection{Sample GUI Test Case}

A sample GUI test case is shown here:

\begin{quote}
\VerbatimInput[numbers=left,firstnumber=1]{./SystemTests/OpenGLPanel.txt}
\end{quote}

\begin{figure}[htb]
\begin{center}
\includegraphics[300,235]{Images/OpenGLPanel.png}
\caption{\label{figure:openGLPanel}The OpenGLPlot Setup Panel}
\end{center}
\end{figure}

\noindent The test case worksheet shown here is the test case for the OpenGL plot setup panel.
The panel, shown in Figure~\ref{figure:openGLPanel}, is a fairly complex GUI panel,
containing text fields, combo boxes, check boxes, text lists, and action buttons which open color
selection dialogs.  Each element is included in the test plan worksheet, along with the standard
control processes that need to be exercised.  Each test criterion is evaluated using this worksheet,
and given a pass or fail evaluation.

\subsection{Procedure}

Each GUI test case has a worksheet like the one shown above.  A tester follows this procedure to
perform the associated system test:

\begin{enumerate}
\item Open the test case worksheet.
\item Follow the procedure outlined in the test case.
\begin{itemize}
\item Section~\ref{section:TestRules} provides detailed instructions about the process that should
be followed when testing each type of GUI element.
\item Each item identified in the worksheet is marked as either passing or failing the test.  If
the item fails, an associated bug is entered or identified in the bug tracking system and listed on
the worksheet.
\item After completing the tests on the worksheet, the tester experiments with the component for an
additional period (typically ten to fifteen minutes), checking to be sure that the component is
stable and behaves correctly when bad data is entered, and when random actions are taken using that
component.
\item Once every item on the worksheet has been evaluated and the final period of usability testing
has been performed, the number of pass and fail evaluations are counted and recorded in the
summary section of the test case worksheet. Any bugs identified on the worksheet are listed in this
section, and any additional notes that need to be recorded are also listed here\footnote{These data
are collected using an automation tool to build a status report for the system tests.}.
\end{itemize}
\item Summarize the results of the tests.
\begin{itemize}
\item Once every item on the worksheet has been evaluated, an overall pass or fail evaluation is
made and recorded in the summary section.  Any bugs identified on the worksheet are listed in this
section, and any additional notes that need to be recorded are also listed here.
\item Add the tester's name and the data the test was run to the worksheet.
\item Save the completed test case worksheet.
\end{itemize}
\item Update the local test tracking worksheet to indicate that the test was run and the results of
the run.
\item Save the test tracking worksheet.
\end{enumerate}

\subsection{Reporting Results}

At the start of the system test process, a central location was established for collection of the
test results.  The final step performed by the system testers is to copy their test case worksheets
and local test tracking worksheet to this central location.  This action is performed each day the
system tests are run so that the progress of the system test execution can be evaluated.  Upon
completion of all system testing by a specific tester, a final update is made and the system test
lead is notified that that tester has completed the assigned tests.  Chapter~\ref{chap:reporting}
describes the consolidation of the collected test results into a system test report.

\section{\label{section:TestRules}Procedural Rules}

The steps described in the preceding sections lay out the procedures followed when testing the GUI
elements of GMAT.  In this section, the criteria that must be evaluated are defined for these
tests.

\subsection{\label{Sec:GeneralTests}Test Procedures for All Elements}

%  This is the list of tests associated with panel aesthetics
\subsubsection{Aesthetics}

Description:  This set of tests verifies platform-specific look and feel of a panel, as extended
by the GMAT GUI Philosophy document\cite{Dove}. Each criterion must be met to pass the aesthetics
tests.

\begin{itemize}
\item All of the data input fields and bounding boxes can be seen at the panel size displayed when
the panel is first opened, for all tabs on the panel.
\item The blank space surrounding the data area is not distracting, and does not dominate the
appearance of the interface.  As a guideline, for platforms that allow control of the surrounding
white space, that region should not consume more than 20\% of the total space dedicated to the
panel when it is opened.
\item The data area does not appear too crowded; the surrounding blank space is appropriately sized.
\item The window cannot be resized so that the data cannot be seen.
\end{itemize}

\subsubsection{General Panel Functionality}

Description:  This is the list of tests associated with basic panel functionality: open, close,
rename, minimize, ok, cancel, help, show script, command summary.  Additionally, the behavior of
open panels needs to be consistent with deletion actions taken on the resource and mission trees 
-- if an object in the tree is deleted, any open panel associated with that object should close.
All of these functions must pass.

\begin{itemize}
\item New objects of the type being tested can be created from the appropriate tree on the Resource
or Mission panels.
\item Double clicking in a new object opens the panel for that object.
\item Double clicking in a object that has an open panel brings the panel for that object to the
front of the displayed panels.
\item New objects can be renamed.
\item Default objects, when they exist, can be renamed.
\item Default objects, when they exist, can be deleted.
\begin{itemize}
\item The object can be renamed.
\item References to the renamed object are updated in related elements of the system.
\end{itemize}
\item Renaming works after making changes to the data on the object panel.
\begin{itemize}
\item The object can be renamed while the panel is open.
\item A change can be made on the panel, and then the object can be renamed before the change is
applied.
\item A change can be made on the panel, the change can be applied, and then the object can be
renamed.
\item For each of the above cases, references to the object's name are updated throughout the
system when the object's name is changed.
\end{itemize}
\item Changes made on the panel and applied using the OK button appear on the panel when it is
reopened.
\item Changes made on the panel and applied using the Apply button are visible in the script when
viewed using the Show Script dialog.
\item When you open the panel, make a minor change in the panel, and click button to close the
panel (on Windows, this is the small ``x'' button in the upper right hand corner; on the Mac, it is
the red button on the left side of the frame controls, and on Linux, varies based on the
configuration of the Linux window manager), you are prompted to save data before closing. Check
that:
\begin{itemize}
\item The prompt does appear.
\item Selecting ``Yes'' updates the underlying data.
\item Selecting ``No'' discards the changes.
\end{itemize}
\item Cancelling closes the opened panel without changing the underlying data.
\begin{itemize}
\item The object does not change when you open the panel and press the Cancel button without making
any changes.
\item The object does not change when you open the panel, make a minor change in the data, and press
the Cancel button.
\item The object does not change when you open the panel and click the close button in the panel's
frame to close the panel, but the panel does close without prompting.
\end{itemize}
\item The panel is minimized when the minimize button on the panel frame is pressed.
\item The panel reopens to previous size when maximize icon on the minimized panel is pressed
\item The tab key navigates the open panel in agreement with style and GUI design philosophy. 
Navigation is orderly and sensible using the tab key.
\end{itemize}

\subsubsection{Panel Data Element Completeness and Correctness}

Description: This set of tests verifies that all data elements that should appear on the panel
are present on the panel.  It also tests that all elements that should appear in ``Show Script''
dialog appear there, and that items that should not appear in show script do not appear there.

\begin{itemize}
\item Verify that only data elements that occur in the Range Test Plan appear in show script and
that the user does not see any other object fields.
\item Verify that defaults agree with the values in the Range Test Plan.
\item Press the ``Show Script'' button, and verify that all elements on the GUI panel also appear on
the show script dialog.  Verify that these elements match the description in the Range Test Plan.
\item  Verify that all data elements that appear in Show Script also appear on the GUI.  (This
step validates that all scriptable settings also appear in the GUI.)
\end{itemize}


\subsection{Procedures for Specific Control Types}
The following table provides additional guidelines that should be followed when testing each
specific type of control.
 \begin{longtable}{p{1.25 in} |p{4.5 in} }
 \caption
 [Tests for Data Objects on All Panels]
 {Tests for Data Objects on All Panels \label{Table:DataElementTests}}\\
 \hline\hline
% \multicolumn{2}{@{*}c@{*}}%
%      {This part appears at the top of the table}\\
 Element Type & Tests\\
 \hline
 \endfirsthead
 \caption[]{(Tests for Data Objects on All Panels...continued)}\\
 \hline\hline
 Element Type & Tests\\
 \hline\hline
 \endhead
 \hline
 % This goes at the&bottom.\\
 \hline
 \endfoot
 \hline
 \hline
 \endlastfoot
%-----------------------------------------------------------------------
%-----------------------Begin Table Here--------------------------------
%-----------------------------------------------------------------------
% ---- Column 1--------%
Check Boxes &
% ---- Column 2--------%
\begin{itemize} \vspace{-.25 in}
\item Set all check boxes to off (unchecked), hit show script, and verify that the functionality is
indeed turned off for each radio button and check box.
%
\item Set all check buttons to on (checked), hit apply, and show script
and verify that the functionality is indeed turned on for each radio
button and check box.
\end{itemize} \\
\hline
% ---- Column 1--------%
Radio Buttons &
% ---- Column 2--------%
\begin{itemize} \vspace{-.25 in}
   \item For each radio button on panel, select the button, and ensure that it activates
   and all others are deactivated.  Hit Apply, and then check show
   script to ensure that the configuration was properly saved.
\end{itemize} \\
\hline
%-------------- This is the beginning of a new row!!!----------------------
% ---- Column 1--------%
Combo Boxes &
% ---- Column 2--------%
\begin{itemize} \vspace{-.25 in}
\item  For each combo box on the panel, ensure that all options that
appear in Range Test Plan appear in the pull down menu.
%
\item  For each Combo box on the panel, select each allowable option, hit apply and show script
and check to see that the option was correctly saved.
%
\item Check to ensure that the combo box is not editable.
\end{itemize} \\
\hline
%-------------- This is the beginning of a new row!!!----------------------
% ---- Column 1--------%
Text Fields &
% ---- Column 2--------%
\begin{itemize} \vspace{-.2 in}
\item For each text field enter ``DNE" and ensure that if GMAT should reject this string that the string is
rejected. ( Currently, this is not an acceptable value for any GMAT
field unless the user has created an appropriate object type and
named it DNE, and is using it correctly in the GUI. )
%
\item Perform all range tests as described in Range Test Plan.
%
\item For all numeric fields, enter an allowed numeric value, hit
apply and show script and check that the value was saved.
%
\item If user-defined objects can appear in the combo box, create
one object for all allowable object types for the particular combo
box, and ensure that it appears in the combo box.  Also, hit apply
and ensure that each case appears in show script.
%
\end{itemize} \\
\hline
%-------------- This is the beginning of a new row!!!----------------------
% ---- Column 1--------%
Action Buttons &
% ---- Column 2--------%
\begin{itemize} \vspace{-.2 in}
\item For each button ensure that clicking on the button brings up the appropriate panel.
\item For the panel opened up, perform all tests defined in Section \ref{Sec:GeneralTests} and Table \ref{Table:DataElementTests}
\end{itemize} \\
\hline
%-------------- This is the beginning of a new row!!!----------------------
% ---- Column 1--------%
Selection Lists &
% ---- Column 2--------%
\begin{itemize} \vspace{-.2 in}
\item First Item
\item Second Item
\end{itemize} \\
\hline

%-------------- This is the beginning of a new row!!!----------------------
% ---- Column 1--------%
Tabbed Panels &  % ---- Column 2--------%
\begin{itemize} \vspace{-.2 in}
\item First Item
\item Second Item
\end{itemize} \\
\end{longtable}


%% ---- Column 2--------%
%\begin{itemize} \vspace{-.2 in}
%\item First Item
%\item Second Item
%\end{itemize} \\
%\hline


\subsection{Usability Testing}

The tests described in the preceding paragraphs are meant to exercise all of the elements of the
graphical user interface.  One important aspect of the interface not covered by those tests is the
usability of the system: the GUI may perform error free as designed, and still be difficult to use
in practice.  Usability testing is performed to capture information about this aspect of the GUI.

