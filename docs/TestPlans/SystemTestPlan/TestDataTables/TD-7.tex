

\begin{table}[htbp!]
\centering
      \begin{tabular}{|p{1.0 in} |p{5.0 in} |}
         \hline
          \rowcolor[rgb]{0.8,0.8,0.8}  Name & TD-4 Equivalent State Representations for a Singular Conic Section\\
         \hline
         Description & This table contains equivalent states in all GMAT state representations
         that have a central body at the origin.    \\ \hline
         Source &  STK.  The data below is to ensure that the GUI state conversions agree with conversions in the script. (The test data assumes $\mu = 398600.4415$)\\
         \hline
         Data &
         %------- The test data is here
          \begin{compactenum}
              \item Cartesian State
              \begin{compactenum}
                   \item X  = 7000
                   \item Y  = 7000
                   \item Z  = 7000
                   \item VX = 0;
                   \item VY = 0;
                   \item VZ = 0;
              \end{compactenum}
              \item Keplerian State: Undefined (e = 1);
              \item Modified Keplerian: Undefined (e = 1);
              \item Spherical RADec
             \begin{compactenum}
                 \item RMAG = 12124.355652982142
                 \item RA   = 45
                 \item DEC  = 35.2643896827546470
                 \item VMAG = 0;
                 \item RAV  = 0;
                 \item DECV = 0;
              \end{compactenum}
              \item Spherical RADec
               \begin{compactenum}
                  \item RMAG = 12124.355652982142
                  \item RA = 45
                  \item DEC = 35.2643896827546470
                  \item VMAG = 6.999999999999998
                  \item FPA = 225
                  \item VAZI = 35.264389682754661
              \end{compactenum}
              \item Equinoctial: Undefined (e = 1);
          \end{compactenum}\\

         %------- The test data is here
         \hline
\end{tabular}
   \label{Table:TD-4}
   \caption{TD-4 Equivalent State Representations for a Singular Conic Section}
    \index{Test Data!TD-4}
\end{table} 