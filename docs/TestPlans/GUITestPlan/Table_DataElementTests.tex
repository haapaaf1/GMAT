 \begin{longtable}{p{1.25 in} |p{4.5 in} }
 \caption
 [Tests for Data Objects on All Panels]
 {Tests for Data Objects on All Panels \label{Table:DataElementTests}}\\
 \hline\hline
% \multicolumn{2}{@{*}c@{*}}%
%      {This part appears at the top of the table}\\
 Element Type & Tests\\
 \hline
 \endfirsthead
 \caption[]{(Tests for Data Objects on All Panels...continued)}\\
 \hline\hline
 Element Type & Tests\\
 \hline\hline
 \endhead
 \hline
 % This goes at the&bottom.\\
 \hline
 \endfoot
 \hline
 \hline
 \endlastfoot
%-----------------------------------------------------------------------
%-----------------------Begin Table Here--------------------------------
%-----------------------------------------------------------------------
% ---- Column 1--------%
Check Boxes &
% ---- Column 2--------%
\begin{itemize} \vspace{-.25 in}
\item Set all check boxes to off (unchecked), hit show script, and verify that the functionality is
indeed turned off for each radio button and check box.
%
\item Set all check buttons to on (checked), hit apply, and show script
and verify that the functionality is indeed turned on for each radio
button and check box.
\end{itemize} \\
\hline
% ---- Column 1--------%
Radio Buttons &
% ---- Column 2--------%
\begin{itemize} \vspace{-.25 in}
   \item For each radio button on panel, select the button, and ensure that it activates
   and all others are deactivated.  Hit Apply, and then check show
   script to ensure that the configuration was properly saved.
\end{itemize} \\
\hline
%-------------- This is the beginning of a new row!!!----------------------
% ---- Column 1--------%
Combo Boxes &
% ---- Column 2--------%
\begin{itemize} \vspace{-.25 in}
\item  For each combo box on the panel, ensure that all options that
appear in Range Test Plan appear in the pull down menu.
%
\item  For each Combo box on the panel, select each allowable option, hit apply and show script
and check to see that the option was correctly saved.
%
\item Check to ensure that the combo box is not editable.
\end{itemize} \\
\hline
%-------------- This is the beginning of a new row!!!----------------------
% ---- Column 1--------%
Text Fields &
% ---- Column 2--------%
\begin{itemize} \vspace{-.2 in}
\item For each text field enter ``DNE" and ensure that if GMAT should reject this string that the string is
rejected. ( Currently, this is not an acceptable value for any GMAT
field unless the user has created an appropriate object type and
named it DNE, and is using it correctly in the GUI. )
%
\item Perform all range tests as described in Range Test Plan.
%
\item For all numeric fields, enter an allowed numeric value, hit
apply and show script and check that the value was saved.
%
\item If user-defined objects can appear in the combo box, create
one object for all allowable object types for the particular combo
box, and ensure that it appears in the combo box.  Also, hit apply
and ensure that each case appears in show script.
%
\end{itemize} \\
\hline
%-------------- This is the beginning of a new row!!!----------------------
% ---- Column 1--------%
Action Buttons &
% ---- Column 2--------%
\begin{itemize} \vspace{-.2 in}
\item For each button ensure that clicking on the button brings up the appropriate panel.
\item For the panel opened up, perform all tests defined in Section \ref{Sec:GeneralTests} and Table \ref{Table:DataElementTests}
\end{itemize} \\
\hline
%-------------- This is the beginning of a new row!!!----------------------
% ---- Column 1--------%
Selection Lists &
% ---- Column 2--------%
\begin{itemize} \vspace{-.2 in}
\item First Item
\item Second Item
\end{itemize} \\
\hline

%-------------- This is the beginning of a new row!!!----------------------
% ---- Column 1--------%
Tabbed Panels &  % ---- Column 2--------%
\begin{itemize} \vspace{-.2 in}
\item First Item
\item Second Item
\end{itemize} \\
\end{longtable}


%% ---- Column 2--------%
%\begin{itemize} \vspace{-.2 in}
%\item First Item
%\item Second Item
%\end{itemize} \\
%\hline
