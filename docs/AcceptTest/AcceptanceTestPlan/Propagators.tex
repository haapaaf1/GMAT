\chapter{Propagation}
\label{Ch:Propagators}
\providecommand{\tabref}[1]{Table~\ref{#1}}
\providecommand{\pref}[1]{Page~\pageref{#1}}

In order to validate the accuracy of GMAT's propagation, the
fundamental unit-level components need to be combined and propagated
on a system level. From a software development point of view, if the
program under development is tested in a wide range of core
applications, it is more likely that problems would be found before
each new version is released to the public. This Acceptance Test
Plan tests GMAT by comparing many possible scenarios users of GMAT
would encounter to reference software packages. Although it is
impossible to create all the possible scenarios each user would
encounter in GMAT, this is a start to eliminate possible
frustrations a user could experience if a component did not work
correctly.

Propagation\index{Propagation} is one of the most important aspects
of GMAT. Everything from outputting parameters to performing a
thruster burns at the correct stopping condition depend on whether
or not GMAT is able to propagate the satellite/object for a defined
time period with acceptable accuracy.

Parts of the Initial Orbit State Conditions section are referenced
from Emergent Space Technologies' Orbit Determination Toolbox
(ODTBX)\index{ODTBX} Spiral 1 DEMO document~\cite{ODTBX1:2005}, due to a similar
objective of testing the numerical implementation of the
programs base functions.\\

\section{Initial Orbit State Conditions}
\label{Sec: InitOrbitState}
\subsection{Earth Based Test Cases}
\label{SubSec: EarthInitState}

The initial orbit states for the Sun-Synchronous
(SunSync)\index{SunSync}, Geostationary (GEO)\index{GEO},
Molniya\index{Molniya}, International Space Station (ISS)\index{ISS}
and the Global Positioning Satellite (GPS)\index{GPS} orbits were
obtained from Emergent Space Technologies' ODTBX\index{ODTBX} Spiral
1 Demo ~\cite{ODTBX1:2005}. Emergent used STK-High Precision Orbit
Propagator (STK-HPOP)\index{STK!HPOP} models and two-Line Element
(TLE) sets with an initial UTC orbit epoch of June 1st 2004,
12:00:00:00. The initial orbit states that were used for the test
case orbits can be seen in \tabref{Tab:Initial}, on
\pref{Tab:Initial}. The perigee and apogee altitudes of the test
case orbits can be seen in
\tabref{Tab:Apogee/Perigee}, on \pref{Tab:Apogee/Perigee}.\\

\begin{table}[htb]
\centering \caption{ Satellite Initial Conditions}
    \begin{tabular}{lccccccc}
    \hline\hline
        Category & Orbit Type & X(km) & Y(km) & Z(km) & Vx(km/s) & Vy(km/s) & Vz(km/s) \\
        \hline
         %---New Row---%
         LEO & ISS & -4453.783586 & -5038.203756 & -426.384456 & 3.831888 & -2.887221 & -6.018232 \\
         %---New Row---%
         LEO & Sun-Sync & -2290.301063 & -6379.471940 & 0 & -0.883923 & 0.317338 & 7.610832 \\
         %---New Row---%
         MEO & GPS & 5525.33668 & -15871.18494 & -20998.992446 & 2.750341 & 2.434198 & -1.068884 \\
         %---New Row---%
         HEO & Molniya & -1529.894287 & -2672.877357 & -6150.115340 & 8.717518 & -4.989709 & 0 \\
         %---New Row---%
         GEO & GEO & 36607.358256 & -20921.723703 & 0 & 1.525636 & 2.669451 & 0 \\
      \hline\hline
      \label{Tab:Initial}
\end{tabular}
\end{table}
\begin{table}[htb]
\centering \caption{Apogee and Perigee Altitudes for Test
Satellites}
    \begin{tabular}{lcc}
    \hline\hline
        Orbit Type & Perigee Altitude(km) & Apogee Altitude(km) \\
        \hline
         %---New Row---%
         ISS & 358.168 & 380.387 \\
         %---New Row---%
         Sun-Sync & 400 & 400 \\
         %---New Row---%
         GPS & 19757.6 & 20603.8 \\
         %---New Row---%
         Molniya & 500 & 39850.5 \\
         %---New Row---%
         GEO & 35786 & 35786 \\
      \hline\hline
      \label{Tab:Apogee/Perigee}
\end{tabular}
\end{table}

The propagation duration, report output step size, and integrator
step sizes were varied for the different test cases. For the ISS,
SunSync\index{SunSync}, GPS\index{GPS}, Molynia\index{Molynia}, and
GEO\index{GEO} cases, the propagation length and report output step
size were chosen based on a study performed by The Aerospace
Corporation to validate STK-HPOP's\index{STK!HPOP}~\cite{Chao:2000}.
The integrator time steps were chosen to allow for the most accurate
comparison of the test case results. These time steps were based on
Vallado's analysis of state vector propagation~\cite{Vallado:2005}.\\

The chosen parameters can be seen in \tabref{Tab:Frequency}, on \pref{Tab:Frequency}.\\

\begin{table}[htb]
\centering \caption{Integrator,Propagator, and Output Frequency}
    \begin{tabular}{lccc}
    \hline\hline
        Orbit Type & Integrator Step Size(s) & Propagator Length(days) & Output Frequency(mins) \\
        \hline
         %---New Row---%
         ISS & 5 & 1 & 1\\
         %---New Row---%
         Sun-Sync & 5 & 1 & 1 \\
         %---New Row---%
         GPS & 60 & 2 & 2 \\
         %---New Row---%
         Molniya & 5 & 3 & 5 \\
         %---New Row---%
         GEO & 60 & 7 & 10 \\
      \hline\hline
      \label{Tab:Frequency}
\end{tabular}
\end{table}

Several test cases were created for each satellite orbit to verify
GMAT's ability to perform accurately, while applying various forces.
The forces used for Earth-based test cases were two-body,
JGM2\index{JGM2}, EGM96\index{EGM96}, and JGM3\index{JGM3} gravity
models, third-body perturbation effects from other planets, the
Jacchia-Roberts (JR)\index{JR} and the Mass Spectrometer and
Incoherent Scatter Radar Exosphere (MSISE\index{MSISE} 1990 \& 2000)
Atmospheric Drag Model, and Solar Radiation Pressure
(SRP)\index{SRP}. Each of the force models were run independently
within GMAT to verify their individual accuracy, as well as a test
case that includes an atmospheric drag model, the SRP\index{SRP}
model, a non-spherical gravity model, and third-body perturbations.
These last test cases were performed to validate the capability of
the GMAT to accurately propagate satellite orbits while multiple
force models were applied.\\

The Degree and Order for Earth-based non-spherical gravity cases was
set at a constant 20 by 20.\\

Refer to Appendix ~\ref{Sec:initCondsProp} for an alternate listing
of all Propagator initial orbit state conditions.

\clearpage
\subsection{Non-Earth Based Test Cases}
\label{SubSec: NonEarthInitState}

GMAT is designed for accuracy in non-Earth mission scenarios. Test
cases for Mars, Mercury, the Moon, Neptune, Pluto, Saturn, Uranus,
Venus, L2 orbits, and deep space orbits were created to test various
forces,individually and jointly, affecting a spacecraft in an orbit.
Many satellite parameters, such as Cd,Cr, satellite area, and
satellite mass, were kept the same as the Earth test cases for
simplicity and consistency. The initial Keplerian satellite state
only varied in Semi-Major Axis and gravity field degree \& order for
the non-Earth test cases.\\

Refer to Table~\ref{Tab:NonEarthInitial} for the initial Keplerian
orbital elements and the Degree \& Order used for the non-spherical
gravity force cases. The integrator step size, propagation length,
and output frequency for all the non-Earth cases are 5 seconds, 3
days, and 5 minutes, respectively. Table~\ref{Tab:NonEarthInitial}
excludes the deep space and L2 orbit test cases. \\

\begin{table}[htb!]
\centering \caption{Non-Earth Keplerian Orbital Elements}
    \begin{tabular}{lccccccc}
    \hline\hline
        Orbit Type & SMA (km) & Ecc. & Inc. (deg) & AOP (deg) & RAAN (deg) & TA (deg) & Deg. x Ord. used\\
        \hline
         Mars & 4603 & 0.2 & 45 & 45 & 90 & 45 & 20x20\\
         Mercury & 3640 & 0.2 & 45 & 45 & 90 & 45 & 4x0\\
         Moon & 2500 & 0.2 & 45 & 45 & 90 & 45 & 20x20\\
         Neptune & 34999 & 0.2 & 45 & 45 & 90 & 45 & 4x0\\
         Pluto & 1795 & 0.2 & 45 & 45 & 90 & 45 & 4x0\\
         Saturn & 80000 & 0.2 & 45 & 45 & 90 & 45 & 4x0\\
         Uranus & 45000 & 0.2 & 45 & 45 & 90 & 45 & 4x0\\
         Venus & 8125 & 0.2 & 45 & 45 & 90 & 45 & 20x20\\
      \hline\hline
      \multicolumn{7}{l}{SMA: Semi-Major Axis $|$ Ecc.: Eccentricity $|$ Inc.: Inclination $|$ AOP: Argument of Perigee}\\
      \multicolumn{7}{l}{RAAN: Right Ascension of Ascending Node $|$ TA: True Anomaly}\\
      \label{Tab:NonEarthInitial}
\end{tabular}
\end{table}

The DeepSpace\index{DeepSpace}, Earth Moon L2 (EML2)\index{EML2},
and Earth Sun L2 (ESL2)\index{ESL2} cases involve propagating about
libration points and propagating deep space orbits. Table
~\ref{Table: InitStateDeepSpace}, ~\ref{Table: InitStateEML2}, and
~\ref{Table: InitStateESL2}, in Appendix~\ref{Ch:initConds}, provide
the initial states for the DeepSpace\index{DeepSpace},
EML2\index{EML2}, and ESL2\index{ESL2} test cases, respectively.\\

Refer to Appendix ~\ref{Sec:initCondsProp} for a listing of all
Propagator initial orbit state conditions.

\section{Other Initial State Conditions}
\label{Sec: OtherInitStateCond}

In order to reduce the complications of the comparisons, certain
initial orbit parameters were kept constant throughout all of the
cases. These parameters are Cd, Cr, Spacecraft Area, each programs
integrator, and software settings affecting the results of various
force models.

The GMAT integrator used for all the test cases was Runga Kutta
8(9), except for the STK-HPOP\index{STK!HPOP} test cases, to avoid
any additional differences that could occur from changing
integrators. The Integrators Section compares the differences
between the various integrators GMAT can use.

The parameters in Table~\ref{Table: InitialCond} shows the
differences between GMAT and the reference programs. The ideal
situation would be for all the programs to match perfectly, but that
is not realistic due to the different approaches each program takes
to solve each problem.

\begin{table}[htbp!]
\centering \caption{Universal Test Case
Parameters}\index{STK!HPOP}\index{STK!Astrogator}\index{FreeFlyer}
      \begin{tabular}{lcccc}
      \hline\hline
         Parameter & GMAT & STK-HPOP & FF & STK-Astro\\
         \hline
         Cd & 2.2 & 2.2 & 2.2 & 2.2\\
         Cr & 1.2 & 1.2 & 1.2 & 1.2\\
         Area($m^2$) & 20 & 20 & 20 & 20\\
         Satellite Mass(kg) & 1000 & 1000 & 1000 & 1000\\
         Integrator (excluding integrator test cases) & RK8(9) & RK7(8) & RK8(9) & RK8(9)\\
         Integrator error tolerance & 1e-013 & 1e-013 & 1e-013 & 1e-013\\
         Integrator error control. Relative to & Step & Step? & ? & Step\\
         Drag: Altitude Calculation & Approximate & Approximate & Exact & ?\\
         Drag: Sun Position & True & True & True & True\\
         SRP: Sun Position  & True & True & True? & True\\
         Solar Flux ($W/m^2$) at 1 AU & 1359.38857 & 1359.38857 & 1358 & 1359.38857\\
         Solid Tides & N/A & Disabled & N/A & N/A\\
         Ocean Tides & N/A & Disabled & N/A & N/A\\
         Daily F10.7: JR and MSISE models & 150 & 150 & N/A & 150\\
         Average F10.7 & 150 & 150 & 150 & 150\\
         Geomagnetic Index(Kp): JR and MSISE only  & 3 & 3 & N/A & 3\\
         Drag: Geomagnetic Flux Update  & Constant & Constant & N/A & Constant\\
         Boundary Mitigation & N/A & Disabled & N/A & N/A\\
         Relativistic Accelerations & N/A & Disabled & N/A & N/A\\
         Shadow Modeling & Dual Cone & Dual Cone & Dual Cone & ?\\
         IERS EOP format used: & Long term C04 & Bulletin A & Bulletin A & Bulletin A\\
         Polar Motion calculation: & Enabled & Enabled & Enabled & Enabled?\\
         Nutation update interval: Earth (sec) & 60 & 60 & 60 & 60\\
         Planetary Ephemeris update interval (sec) & 0 & 0? & 0 & 0?\\
      \hline\hline
      \label{Table: InitialCond}
\end{tabular}
\end{table}

\clearpage
Exceptions to Table~\ref{Table: InitialCond} are as followed:\\
\begin{itemize}
    \item STK has trouble propagating the EML2\index{EML2} test case at an error
    tolerance of 1e-013 with a relative to step error control. The GMAT
    and STK cases were changed to both use relative to state error control
\end{itemize}

\subsection{Earth Orientation Parameters(EOP) data}\index{EOP}
\label{Sec: EOP}

\emph{"International Earth Rotation Service (IERS)\index{IERS}
Bulletins A and B provide current information on the Earth's
orientation in the IERS Reference System. This includes Universal
Time, coordinates of the terrestrial pole, and celestial pole
offsets.  Bulletin A gives an advanced solution updated weekly by
e-mail subscription or daily by anonymous ftp; the standard solution
is given monthly in Bulletin B and updated every week in the
(IERS)\index{IERS} C04 solution."}~\cite{IERSab:1999}

\emph{"Bulletin A is issued by the IERS\index{IERS} Rapid
Service/Prediction Centre at the U.S. Naval Observatory(USNO),
Washington, DC and Bulletin B"}, as well as the C04 data, \emph{"is
issued by the IERS\index{IERS} Earth Orientation Centre at the Paris
Observatory."}~\cite{IERSab:2004}

\emph{"Bulletin A is intended for users who need accurate
information before the Bulletin B �finals� series is available,
i.e., those who reduce data in the very recent past (require rapid
service) or those who operate in real-time (require predictions).
Bulletin B is intended for standard use.  For scientific and
long-term analyses of the Earth's orientation, the long-term
continuous series"}~\cite{IERSab:2004}, C04 (1962- present), can be
used.

\emph{"EOP\index{EOP} (IERS)\index{IERS} C04 is regularly recomputed
to take advantage on one hand of the improvement of the various
individual contributions and in the other hand of the refinement of
the analysis procedures. To date, it is twice-weekly
updated."}~\cite{IERSc04}

\emph{"The EOP\index{EOP} (IERS)\index{IERS} C04 is given at one-day
intervals, it is free from the diurnal/subdiurnal terms due to the
oceanic effects and can be interpolated linearly. The oscillations
in UT and duration of the days due to zonal tides for periods under
35 days are present in full size in the series."}~\cite{IERSc04}

GMAT retrieves long term earth orientation IERS\index{IERS}
EOP\index{EOP} CO4 data, which includes UTC-UT1 data, from a file.
This file includes smoothed values at 1-day intervals and data from
1962-present.

STK and FF retrieves its EOP\index{EOP} data from the USNO series 7
/ IERS\index{IERS} Bulletin A.

The differences between the EOP\index{EOP} data sets are displayed
in Table~\ref{Table: EOPAccuracy}. The Terrestrial Pole column
refers to the accuracy of the pole position [x,y] and the UT1 column
refers to the accuracy of the rotation angle about the pole UT1.

\begin{table}[htbp!]
\centering \caption{EOP Format Accuracy}\index{EOP}
      \begin{tabular}{lcc}
      \hline\hline
         EOP Format & Pole Position (mas)& UT1 (ms)\\
         \hline
         Bulletin A obs. 1-d (1)& 0.10 & 0.02\\
         Bulletin A pred. 1-d (2)& 0.50 & 0.14\\
         Bulletin A pred. 4-d (2)& 1.60 & 0.52\\
         Bulletin A pred. 10-d (2)& 3.9 & 1.60\\
         Bulletin A pred. 40-d (2)& 11.2 & 7.70\\
         Bulletin B obs. smooth 5-d (3)& 0.15 & 0.02\\
         Bulletin B obs. raw 5-d (3)& 0.15 & 0.02\\
         Bulletin B pred. 5-d (3)& 1.6 & 0.60\\
         Bulletin B pred. 10-d (3)& 3.0 & 1.60\\
         Bulletin B pred. 30-d (3)& 10.0 & 4.0\\
         Long-Term C04 '62-'67 & 30.0 & 2.0\\
         Long-Term C04 '68-'71 & 20.0 & 1.5\\
         Long-Term C04 '72-'79 & 15.0 & 1.0\\
         Long-Term C04 '80-'83 & 2.0 & 0.4\\
         Long-Term C04 '84-'95 & 0.7 & 0.04\\
         Long-Term C04 '96-present & 0.2 & 0.02\\
      \hline\hline
      \multicolumn{3}{l}{NOTES: (1) Based on data after 1997; applies only to latest epochs in each update.}\\
      \multicolumn{3}{l}{(2) Based on data since 1995.}\\
      \multicolumn{3}{l}{(3) Based on data since 1996.}\\
      \multicolumn{3}{l}{The Terrestrial Pole and UT1 data is free from the diurnal/subdiurnal}\\
      \multicolumn{3}{l}{terms due to the oceanic effects and can be interpolated linearly.}\\
      \multicolumn{3}{l}{These terms can be added after interpolation.}\\
      \label{Table: EOPAccuracy}
\end{tabular}
\end{table}

As shown in Table~\ref{Table: EOPAccuracy}, there are differences
between STK\index{STK} and GMAT, but the Terrestrial Pole data
agrees to within the thousandth place of a milli-arcseconds and the
UT1 data agrees to within the hundredth place of second.

\clearpage
\subsection{Other Planetary Parameters}
\label{Sec: PlanetInfo}
\subsubsection{Gravitational Constant}\index{mu}
All programs used for comparisons utilize the DE405\index{DE405}
Planetary Gravitational Constants listed in Table~\ref{Table:
muValues}, except when non-spherical gravity was used. When
non-spherical gravity files are used they typically call the
gravitational constants located in the file, unless the program
creates an exception.
\begin{table}[htbp!]
\centering \caption{Planetary Gravitational Constants(mu
values)}\index{mu}
      \begin{tabular}{lc}
      \hline\hline
         Planet & mu ($km^3/s^2$)\\
         \hline
         Sun & 132712440017.99\\
         Mercury & 22032.080486418\\
         Venus & 324858.59882646\\
         Earth & 398600.44150000\\
         Moon & 4902.8005821478\\
         Mars & 42828.314258067\\
         Jupiter & 126712767.85780\\
         Saturn & 37940626.061137\\
         Uranus & 5794549.0070719\\
         Neptune & 6836534.0638793\\
         Pluto & 981.60088770700\\
      \hline\hline
      \label{Table: muValues}
\end{tabular}
\end{table}

\subsubsection{Flattening Coefficient}
Whenever possible the flattening coefficients listed in
Table~\ref{Table: flatCoeff} were used. Without the use of these
values the planetary bodies could have wildly different shapes,
which would result in large differences in parameters such as
longitude, latitude, and altitude.

\begin{table}[htbp!]
\centering \caption{Flattening Coefficient}
      \begin{tabular}{lc}
      \hline\hline
         Planet & Flattening Coefficient\\
         \hline
         Sun & 0.00000000\\
         Mercury & 0.00000000\\
         Venus & 0.00000000\\
         Earth & 0.00335270\\
         Moon & 0.00000000\\
         Mars & 0.00647630\\
         Jupiter & 0.06487439\\
         Saturn & 0.09796243\\
         Uranus & 0.02292734\\
         Neptune & 0.01856029\\
         Pluto & 0.00000000\\
      \hline\hline
      \label{Table: flatCoeff}
\end{tabular}
\end{table}

\clearpage
\subsubsection{Equatorial Radius}
The several celestial body equatorial radii used are listed in
Table~\ref{Table: equRadii}. Similar differences as the flattening
coefficient occur when GMAT and the reference programs don't use the
same values.

\begin{table}[htbp!]
\centering \caption{Equatorial Radius}
      \begin{tabular}{lc}
      \hline\hline
         Planet & Radius (km)\\
         \hline
         Sun & 695990.0\\
         Mercury & 2439.7\\
         Venus & 6051.90\\
         Earth & 6378.1363\\
         Moon & 1738.2\\
         Mars & 3397.0\\
         Jupiter & 71492.0\\
         Saturn & 60268.0\\
         Uranus & 25559.0\\
         Neptune & 25269.0\\
         Pluto & 1162.0\\
      \hline\hline
      \label{Table: equRadii}
\end{tabular}
\end{table}

\subsubsection{Leap Seconds}
The amount of leap seconds, or $\Delta$\emph{AT}, has been used
since 1972 in order to keep |UTC-UT1|\index{UTC-UT1}$\leq$ 0.9sec.
GMAT and the reference software packages use all the leap seconds up
until 2004. In 2004 the amount of leap seconds in use were 32
seconds.

\clearpage
\section{Naming Convention} \label{nameConvProp}
This section describes the naming convention for propagator scripts
and output reports. The naming convention consists of an ordered
series of option strings, separated by underscores (\_ ). Currently,
options are allowed for the following fields, and will be present in
the file name in order:
\begin{enumerate}
  \item \emph{tool} - The tool used to generate the test case
  \item \emph{traj} - The trajectory to use.  This includes initial conditions, physical parameters, and time step
  \item \emph{pmg} - The point-mass gravity model to use
  \item \emph{nsg} - The non-spherical gravity model to use
  \item \emph{drag} - The atmospheric drag model to use
  \item \emph{other} - Any other forces to include, such as SRP, secondary body gravity, etc
\end{enumerate}
The \emph{tool} field should always be the first field.  Future
additional fields should be added to the end of the list of fields.
If multiple \emph{other} options are required, they should be added
to the end of the file as required.  For example, the file name will
be \emph{tool\_traj\_pmg\_nsg\_drag\_other1\_other2.report} (file
extensions are described later.)  Each field has a finite list of
options, as follows (future options should be added to this list):
\begin{enumerate}
  \item \emph{tool}

  \begin{tabular}{ll}
    STK  & - Satellite Toolkit HPOP or Astrogator\\
    FF   & - FreeFlyer\\
    GMAT & - General Mission Analysis Tool\\
  \end{tabular}

  \item \emph{traj}

  \begin{tabular}{ll}
    ISS & - LEO orbit\\
    SunSync & - LEO orbit\\
    GPS & - MEO orbit\\
    GEO & - GEO orbit\\
    Molniya & - HEO orbit\\
    Mars1 & - eccentric low orbit\\
    Mercury1 & - eccentric low orbit\\
    Moon & - eccentric low orbit\\
    Neptune1 & - eccentric low orbit\\
    Pluto1 & - eccentric low orbit\\
    Saturn1 & - eccentric low orbit\\
    Uranus1 & - eccentric low orbit\\
    Venus1 & - eccentric low orbit\\
    DeepSpace & deep space orbit\\
    EML2 & - Earth Moon L2 orbit\\
    ESL2 & - Earth Sun L2 orbit\\
  \end{tabular}

NOTE:  Some test cases contain \emph{traj} variations. In these
cases, \emph{traj} precedes the modification. For example, if ISS
trajectory is needed with no output, then \emph{traj} can be
ISSnoOut. The lack of a report file is shortened to noOut.

  \item \emph{pmg}

  \begin{tabular}{ll}
    Earth & - Earth point mass gravity \\
    Sun & - Sun point mass gravity\\
    Luna & - Lunar point mass gravity\\
    AllPlanets & - Sun, Mercury, Venus, Earth, Moon, Mars, Mercury, Jupiter, and
    Pluto point mass gravity included.\\
  \end{tabular}

NOTE:  When dealing with a combination of \emph{pmg}'s the first
point mass is the primary body and the following are third body
point masses. For example, LunaSunEarth would be a Lunar primary
body with the Earth and Sun as third body point masses. The
\emph{pmg}'s after the primary body are arranged based on the order
from the sun, in order to reduce repeat filenames.

  \item \emph{nsg}

  \begin{tabular}{ll}
    0       & - no non-spherical gravity included \\
    JGM2    & - Earth JGM2 20x20 gravity \\
    JGM3    & - Earth JGM3 20x20 gravity \\
    EGM96   & - Earth EGM96 20x20 gravity \\
    MARS50C & - Mars Mars50c 20x20 gravity\\
    LP165P  & - Moon LP165P 20x20 gravity\\
   \end{tabular}
  \item \emph{drag}

  \begin{tabular}{ll}
    0 & - drag not included\\
    HP & - Harris Priester \\
    JRXX & - Jacchia-Roberts\\
    MSISEXX & - NRL MSISE\\
  \end{tabular}

NOTE:  XX in the \emph{drag} field refers to the year.  For example,
JR77 would be the Jacchia-Roberts 1977 model, and MSISE00 would be
NRL MSISE 2000. Refer to Table~\ref{Table: InitialCond} for the drag
settings used.

  \item \emph{other}

  \begin{tabular}{ll}
    0 & - no other forces included\\
    SRP & - Solar Radiation Pressure\\
  \end{tabular}

NOTE:  Any of the above options may be included as an \emph{other}
field.  Refer to Table~\ref{Table: InitialCond} for the SRP settings
used.

\end{enumerate}


\subsection{Comparison Script Information}
The script used to perform the position and velocity comparisons
needed for the Propagator section is Comparison\_Tool1\_Tool2\_PV.m.
This script takes the normalized position and velocity vector
difference between two programs.

Refer to Appendix~\ref{Ch:CompScripts} for more details about this
script and others used in the Acceptance Test Plan document.

\section{Test Case Results}
The following results are for the Propagator section. The current
GMAT Build is compared to STK\index{STK} and
FreeFlyer\index{FreeFlyer} for this section, with the maximum
normalized position and velocity difference displayed in table
format.

To determine if a propagator test case comparison value was
acceptable, an acceptance matrix\index{Acceptance Matrix}, presented
in Table~\ref{Table: AcceptMatrix}, was created. The values in
Table~\ref{Table: AcceptMatrix} were obtained from the lower
position difference bounds of David Vallado�s �An Analysis of State
Vector Propagation Using Differing Flight Dynamics Programs�,
presented at the 2005 American Astronomical Society (AAS)/AIAA
Astrodynamics Specialist Conference. These lower bounds are
difficult to meet in some orbits due to the wide range of orbits
that are possible but they give a order of magnitude number to
strive for. If a case has a combination of either Drag,
Non-Spherical Gravity, Solar Radiation Pressure(SRP)\index{SRP}, or
Point Mass gravity, the largest acceptable position difference is
used.

The next step beyond this acceptance matrix is to compare GMAT's
comparison data to differences seen in FF\index{FreeFlyer} and
STK\index{STK} comparisons, and peer reviews.

\begin{table}[htbp!]
\centering \caption{Acceptance Matrix}\index{Acceptance Matrix}
      \begin{tabular}{lc}
      \hline\hline
         Difference in & Acceptable Position Difference (m)\\
         \hline
         Non-Spherical Gravity & $<$ 0.001\\
         Point Mass Gravity & $<$ 0.001\\
         Solar Radiation Pressure & $<$ 0.6\\
         Drag & $<$ 20\\
      \hline\hline
      \label{Table: AcceptMatrix}
\end{tabular}
\end{table}

\begin{table}[htbp!]
\centering
\caption{ WinGMAT/STK GEO Test Case Comparison}
      \begin{tabular}{lcc}
      \hline\hline
          Test Case & Position Difference(m) & Velocity Difference(m/s) \\
         \hline
         %---New Row---%
         AllPlanets-0-0-0 & 0.0002573600464 & 1.793782766e-008 \\
         %---New Row---%
         EarthLuna-0-0-0 & 0.000319469177 & 2.245968491e-008 \\
         %---New Row---%
         EarthSunLuna-EGM96-JR-SRP & 0.000157111986 & 1.10145998e-008 \\
         %---New Row---%
         EarthSunLuna-EGM96-MSISE90-SRP & 0.0001571119996 & 1.101459883e-008 \\
         %---New Row---%
         EarthSunLuna-JGM2-JR-SRP & 0.0001132816464 & 7.86017199e-009 \\
         %---New Row---%
         EarthSunLuna-JGM2-MSISE90-SRP & 0.0001132816525 & 7.860172342e-009 \\
         %---New Row---%
         EarthSunLuna-JGM3-JR-SRP & 0.0002438654858 & 1.730861641e-008 \\
         %---New Row---%
         EarthSunLuna-JGM3-MSISE90-SRP & 0.0002438654861 & 1.730861648e-008 \\
         %---New Row---%
         EarthSun-0-0-0 & 2.279168371e-005 & 1.61733327e-009 \\
         %---New Row---%
         Earth-0-0-0 & 2.111395905e-005 & 1.406994785e-009 \\
         %---New Row---%
         Earth-0-0-SRP & 4.571760608e-005 & 2.615384024e-009 \\
         %---New Row---%
         Earth-0-JR-0 & 2.111395905e-005 & 1.406994785e-009 \\
         %---New Row---%
         Earth-0-MSISE90-0 & 2.111395905e-005 & 1.406994785e-009 \\
         %---New Row---%
         Earth-EGM96-0-0 & 0.0001110202624 & 8.026471479e-009 \\
         %---New Row---%
         Earth-EGM96full-0-0 & 0.0001110202624 & 8.026471479e-009 \\
         %---New Row---%
         Earth-JGM2-0-0 & 2.954895496e-005 & 2.054073235e-009 \\
         %---New Row---%
         Earth-JGM2full-0-0 & 2.954895496e-005 & 2.054073235e-009 \\
         %---New Row---%
         Earth-JGM3-0-0 & 1.603674008e-005 & 1.085055615e-009 \\
         %---New Row---%
         Earth-JGM3full-0-0 & 1.603674009e-005 & 1.085055615e-009 \\
      \hline\hline
      \label{Table: GEO WinGMAT-STK Table} 
\end{tabular}
\end{table}
\index{GEO}\index{STK}
\begin{table}[htbp!]
\centering
\caption{ WinGMAT/STK GPS Test Case Comparison}
      \begin{tabular}{lcc}
      \hline\hline
          Test Case & Position Difference(m) & Velocity Difference(m/s) \\
         \hline
         %---New Row---%
         AllPlanets-0-0-0 & 1.857890303e-005 & 2.692489413e-009 \\
         %---New Row---%
         EarthLuna-0-0-0 & 2.280562281e-005 & 3.313474128e-009 \\
         %---New Row---%
         EarthSunLuna-EGM96-JR-SRP & 0.04423319518 & 4.494744959e-006 \\
         %---New Row---%
         EarthSunLuna-EGM96-MSISE90-SRP & 0.04425083865 & 4.497443366e-006 \\
         %---New Row---%
         EarthSunLuna-JGM2-JR-SRP & 0.04423757254 & 4.4953771e-006 \\
         %---New Row---%
         EarthSunLuna-JGM2-MSISE90-SRP & 0.0442455997 & 4.496600974e-006 \\
         %---New Row---%
         EarthSunLuna-JGM3-JR-SRP & 0.04423934565 & 4.495691689e-006 \\
         %---New Row---%
         EarthSunLuna-JGM3-MSISE90-SRP & 0.04423592869 & 4.495186505e-006 \\
         %---New Row---%
         EarthSun-0-0-0 & 1.81527305e-006 & 2.214883804e-010 \\
         %---New Row---%
         Earth-0-0-0 & 2.704817671e-006 & 3.709235187e-010 \\
         %---New Row---%
         Earth-0-0-SRP & 0.04675382861 & 4.841397638e-006 \\
         %---New Row---%
         Earth-0-JR-0 & 2.704817671e-006 & 3.709235187e-010 \\
         %---New Row---%
         Earth-0-MSISE90-0 & 1.54909103e-005 & 2.225873381e-009 \\
         %---New Row---%
         Earth-EGM96-0-0 & 3.230673797e-005 & 4.258922497e-009 \\
         %---New Row---%
         Earth-EGM96full-0-0 & 3.230673797e-005 & 4.258922497e-009 \\
         %---New Row---%
         Earth-JGM2-0-0 & 3.207859422e-005 & 4.296841145e-009 \\
         %---New Row---%
         Earth-JGM2full-0-0 & 3.207859422e-005 & 4.296841145e-009 \\
         %---New Row---%
         Earth-JGM3-0-0 & 3.197574489e-005 & 4.249781094e-009 \\
         %---New Row---%
         Earth-JGM3full-0-0 & 3.197574489e-005 & 4.249781094e-009 \\
      \hline\hline
      \label{Table: GPS WinGMAT-STK Table} 
\end{tabular}
\end{table}
\index{GPS}\index{STK}
\begin{table}[htbp!]
\centering
\caption{ WinGMAT/STK ISS Test Case Comparison}
      \begin{tabular}{lcc}
      \hline\hline
          Test Case & Position Difference(m) & Velocity Difference(m/s) \\
         \hline
         %---New Row---%
         AllPlanets-0-0-0 & 1.236533809e-005 & 1.385227256e-008 \\
         %---New Row---%
         EarthLuna-0-0-0 & 2.656381117e-005 & 3.027412372e-008 \\
         %---New Row---%
         EarthSunLuna-EGM96-JR-SRP & 278.7566272 & 0.3182101779 \\
         %---New Row---%
         EarthSunLuna-EGM96-MSISE90-SRP & 56.2421725 & 0.06448778396 \\
         %---New Row---%
         EarthSunLuna-JGM2-JR-SRP & 265.0511827 & 0.3028951491 \\
         %---New Row---%
         EarthSunLuna-JGM2-MSISE90-SRP & 56.24215684 & 0.06448774979 \\
         %---New Row---%
         EarthSunLuna-JGM3-JR-SRP & 278.7568014 & 0.3182104054 \\
         %---New Row---%
         EarthSunLuna-JGM3-MSISE90-SRP & 56.24206048 & 0.06448766318 \\
         %---New Row---%
         EarthSun-0-0-0 & 1.433764674e-005 & 1.600823115e-008 \\
         %---New Row---%
         Earth-0-0-0 & 2.047203147e-005 & 2.325189569e-008 \\
         %---New Row---%
         Earth-0-0-SRP & 0.130272568 & 0.0001262632038 \\
         %---New Row---%
         Earth-0-JR-0 & 251.765217 & 0.2867219205 \\
         %---New Row---%
         Earth-0-MSISE90-0 & 54.12970949 & 0.06180127344 \\
         %---New Row---%
         Earth-EGM96-0-0 & 0.0003663016096 & 4.164100958e-007 \\
         %---New Row---%
         Earth-EGM96full-0-0 & 0.0003621628298 & 4.134604421e-007 \\
         %---New Row---%
         Earth-JGM2-0-0 & 0.0002104591217 & 2.411864887e-007 \\
         %---New Row---%
         Earth-JGM2full-0-0 & 0.0002185235364 & 2.48797737e-007 \\
         %---New Row---%
         Earth-JGM3-0-0 & 0.0002166929687 & 2.47503413e-007 \\
         %---New Row---%
         Earth-JGM3full-0-0 & 0.0002197092927 & 2.507716249e-007 \\
      \hline\hline
      \label{Table: ISS WinGMAT-STK Table} 
\end{tabular}
\end{table}
\index{ISS}\index{STK}
\begin{table}[htbp!]
\centering
\caption{ WinGMAT/STK Molniya Test Case Comparison}
      \begin{tabular}{lcc}
      \hline\hline
          Test Case & Position Difference(m) & Velocity Difference(m/s) \\
         \hline
         %---New Row---%
         AllPlanets-0-0-0 & 0.0001645654561 & 1.252146304e-007 \\
         %---New Row---%
         EarthLuna-0-0-0 & 0.0001587779228 & 1.205681387e-007 \\
         %---New Row---%
         EarthSunLuna-EGM96-JR-SRP & 13.29848959 & 0.009564197805 \\
         %---New Row---%
         EarthSunLuna-EGM96-MSISE90-SRP & 7.084035367 & 0.004827064426 \\
         %---New Row---%
         EarthSunLuna-JGM2-JR-SRP & 13.29849439 & 0.009564215374 \\
         %---New Row---%
         EarthSunLuna-JGM2-MSISE90-SRP & 7.083986451 & 0.004827011437 \\
         %---New Row---%
         EarthSunLuna-JGM3-JR-SRP & 13.2985248 & 0.009564255446 \\
         %---New Row---%
         EarthSunLuna-JGM3-MSISE90-SRP & 7.08473615 & 0.004827602744 \\
         %---New Row---%
         EarthSun-0-0-0 & 0.000348717827 & 2.914219951e-007 \\
         %---New Row---%
         Earth-0-0-0 & 0.0002791429809 & 2.328580346e-007 \\
         %---New Row---%
         Earth-0-0-SRP & 0.6111776321 & 0.0005124550223 \\
         %---New Row---%
         Earth-0-JR-0 & 15.70095053 & 0.01312840386 \\
         %---New Row---%
         Earth-0-MSISE90-0 & 7.009963698 & 0.005861408 \\
         %---New Row---%
         Earth-EGM96-0-0 & 0.001747777921 & 1.46609689e-006 \\
         %---New Row---%
         Earth-EGM96full-0-0 & 0.001874691883 & 1.572010942e-006 \\
         %---New Row---%
         Earth-JGM2-0-0 & 0.00156516582 & 1.313726266e-006 \\
         %---New Row---%
         Earth-JGM2full-0-0 & 0.00153824279 & 1.291402628e-006 \\
         %---New Row---%
         Earth-JGM3-0-0 & 0.001286385724 & 1.080852205e-006 \\
         %---New Row---%
         Earth-JGM3full-0-0 & 0.001517774713 & 1.273995917e-006 \\
      \hline\hline
      \label{Table: Molniya WinGMAT-STK Table} 
\end{tabular}
\end{table}
\index{Molniya}\index{STK}
\begin{table}[htbp!]
\centering
\caption{ WinGMAT/STK SunSync Test Case Comparison}
      \begin{tabular}{lcc}
      \hline\hline
          Test Case & Position Difference(m) & Velocity Difference(m/s) \\
         \hline
         %---New Row---%
         AllPlanets-0-0-0 & 4.699222822e-005 & 5.325243128e-008 \\
         %---New Row---%
         EarthLuna-0-0-0 & 1.738032476e-005 & 1.943482273e-008 \\
         %---New Row---%
         EarthSunLuna-EGM96-JR-SRP & 185.4905295 & 0.2108184432 \\
         %---New Row---%
         EarthSunLuna-EGM96-MSISE90-SRP & 45.11411956 & 0.05130834281 \\
         %---New Row---%
         EarthSunLuna-JGM2-JR-SRP & 185.4896262 & 0.2108173466 \\
         %---New Row---%
         EarthSunLuna-JGM2-MSISE90-SRP & 45.1146799 & 0.05130896317 \\
         %---New Row---%
         EarthSunLuna-JGM3-JR-SRP & 185.4905136 & 0.2108184389 \\
         %---New Row---%
         EarthSunLuna-JGM3-MSISE90-SRP & 45.11393481 & 0.0513081406 \\
         %---New Row---%
         EarthSun-0-0-0 & 4.687777606e-006 & 5.019368405e-009 \\
         %---New Row---%
         Earth-0-0-0 & 4.483249915e-005 & 5.068336761e-008 \\
         %---New Row---%
         Earth-0-0-SRP & 0.173198102 & 0.0001547951558 \\
         %---New Row---%
         Earth-0-JR-0 & 172.0259163 & 0.1950745493 \\
         %---New Row---%
         Earth-0-MSISE90-0 & 42.66516434 & 0.04839009285 \\
         %---New Row---%
         Earth-EGM96-0-0 & 7.308356983e-005 & 8.466320674e-008 \\
         %---New Row---%
         Earth-EGM96full-0-0 & 7.478889446e-005 & 8.088061938e-008 \\
         %---New Row---%
         Earth-JGM2-0-0 & 8.928348594e-005 & 9.655246367e-008 \\
         %---New Row---%
         Earth-JGM2full-0-0 & 6.955767447e-005 & 7.95181581e-008 \\
         %---New Row---%
         Earth-JGM3-0-0 & 6.997911597e-005 & 7.571096195e-008 \\
         %---New Row---%
         Earth-JGM3full-0-0 & 7.300904469e-005 & 7.891681204e-008 \\
      \hline\hline
      \label{Table: SunSync WinGMAT-STK Table} 
\end{tabular}
\end{table}
\index{Sunsync}\index{STK}
\clearpage
\begin{table}[htbp!]
\centering
\caption{ WinGMAT/STK Mars1 Test Case Comparison}
      \begin{tabular}{lcc}
      \hline\hline
          Test Case & Position Difference(m) & Velocity Difference(m/s) \\
         \hline
         %---New Row---%
         AllPlanets-0-0-0 & 0.1994303898 & 0.0001681334573 \\
         %---New Row---%
         Mars-0-0-0 & 0.1873874394 & 0.0001577557948 \\
         %---New Row---%
         Mars-0-0-SRP & 0.6688079458 & 0.0005576090938 \\
         %---New Row---%
         Mars-MARS50C-0-0 & 0.4488316637 & 0.0003800813667 \\
         %---New Row---%
         Mars-MARS50C-0-SRP & 1.105089608 & 0.0009291075949 \\
      \hline\hline
      \label{Table: Mars1 WinGMAT-STK Table} 
\end{tabular}
\end{table}
\index{STK}
\begin{table}[htbp!]
\centering
\caption{ WinGMAT/STK Mercury1 Test Case Comparison}
      \begin{tabular}{lcc}
      \hline\hline
          Test Case & Position Difference(m) & Velocity Difference(m/s) \\
         \hline
         %---New Row---%
         AllPlanets-0-0-0 & 0.1124163711 & 9.591759819e-005 \\
         %---New Row---%
         Mercury-0-0-0 & 0.02585481636 & 2.218352736e-005 \\
         %---New Row---%
         Mercury-0-0-SRP & 0.07778801121 & 6.658571119e-005 \\
      \hline\hline
      \label{Table: Mercury1 WinGMAT-STK Table} 
\end{tabular}
\end{table}
\index{STK}
\begin{table}[htbp!]
\centering
\caption{ WinGMAT/STK Moon Test Case Comparison}
      \begin{tabular}{lcc}
      \hline\hline
          Test Case & Position Difference(m) & Velocity Difference(m/s) \\
         \hline
         %---New Row---%
         AllPlanets-0-0-0 & 0.0002227517348 & 1.414654829e-007 \\
         %---New Row---%
         Luna-0-0-0 & 0.0001960891057 & 1.399292279e-007 \\
         %---New Row---%
         Luna-0-0-SRP & 5.486875354e-005 & 3.881689429e-008 \\
         %---New Row---%
         Luna-LP165P-0-0 & 0.0002701673647 & 1.922174669e-007 \\
         %---New Row---%
         Luna-LP165P-0-SRP & 0.0002393214314 & 1.693094815e-007 \\
      \hline\hline
      \label{Table: Moon WinGMAT-STK Table} 
\end{tabular}
\end{table}
\index{STK}
\begin{table}[htbp!]
\centering
\caption{ WinGMAT/STK Neptune1 Test Case Comparison}
      \begin{tabular}{lcc}
      \hline\hline
          Test Case & Position Difference(m) & Velocity Difference(m/s) \\
         \hline
         %---New Row---%
         AllPlanets-0-0-0 & 0.9974264518 & 0.0005074875215 \\
         %---New Row---%
         Neptune-0-0-0 & 1.171036908 & 0.0005920828399 \\
         %---New Row---%
         Neptune-0-0-SRP & 0.9524248353 & 0.0004889922186 \\
      \hline\hline
      \label{Table: Neptune1 WinGMAT-STK Table} 
\end{tabular}
\end{table}
\index{STK}
\begin{table}[htbp!]
\centering
\caption{ WinGMAT/STK Pluto1 Test Case Comparison}
      \begin{tabular}{lcc}
      \hline\hline
          Test Case & Position Difference(m) & Velocity Difference(m/s) \\
         \hline
         %---New Row---%
         AllPlanets-0-0-0 & 0.8883195792 & 0.0004683900657 \\
         %---New Row---%
         Pluto-0-0-0 & 0.2632397208 & 0.0001382385307 \\
         %---New Row---%
         Pluto-0-0-SRP & 0.7534556636 & 0.0003932303114 \\
      \hline\hline
      \label{Table: Pluto1 WinGMAT-STK Table} 
\end{tabular}
\end{table}
\index{STK}
\begin{table}[htbp!]
\centering
\caption{ WinGMAT/STK Saturn1 Test Case Comparison}
      \begin{tabular}{lcc}
      \hline\hline
          Test Case & Position Difference(m) & Velocity Difference(m/s) \\
         \hline
         %---New Row---%
         AllPlanets-0-0-0 & 0.1414864331 & 4.873107384e-005 \\
         %---New Row---%
         Saturn-0-0-0 & 0.4497282767 & 0.0001561546333 \\
         %---New Row---%
         Saturn-0-0-SRP & 0.4994435851 & 0.0001686750308 \\
      \hline\hline
      \label{Table: Saturn1 WinGMAT-STK Table} 
\end{tabular}
\end{table}
\index{STK}
\begin{table}[htbp!]
\centering
\caption{ WinGMAT/STK Uranus1 Test Case Comparison}
      \begin{tabular}{lcc}
      \hline\hline
          Test Case & Position Difference(m) & Velocity Difference(m/s) \\
         \hline
         %---New Row---%
         AllPlanets-0-0-0 & 0.2465709269 & 7.912323063e-005 \\
         %---New Row---%
         Uranus-0-0-0 & 1.320413817 & 0.0004243963023 \\
         %---New Row---%
         Uranus-0-0-SRP & 1.156382201 & 0.0002904419051 \\
      \hline\hline
      \label{Table: Uranus1 WinGMAT-STK Table} 
\end{tabular}
\end{table}
\index{STK}
\begin{table}[htbp!]
\centering
\caption{ WinGMAT/STK Venus1 Test Case Comparison}
      \begin{tabular}{lcc}
      \hline\hline
          Test Case & Position Difference(m) & Velocity Difference(m/s) \\
         \hline
         %---New Row---%
         AllPlanets-0-0-0 & 0.02561929021 & 2.548815625e-005 \\
         %---New Row---%
         Venus-0-0-0 & 0.01793453836 & 1.768517227e-005 \\
         %---New Row---%
         Venus-0-0-SRP & 0.006332820969 & 6.249825126e-006 \\
         %---New Row---%
         Venus-MGNP180U-0-0 & 0.007510965788 & 7.525754724e-006 \\
         %---New Row---%
         Venus-MGNP180U-0-SRP & 0.006805940089 & 6.750530974e-006 \\
      \hline\hline
      \label{Table: Venus1 WinGMAT-STK Table} 
\end{tabular}
\end{table}
\index{STK}
\clearpage
\begin{table}[htbp!]
\centering
\caption{ WinGMAT/STK DeepSpace Test Case Comparison}
      \begin{tabular}{lcc}
      \hline\hline
          Test Case & Position Difference(m) & Velocity Difference(m/s) \\
         \hline
         %---New Row---%
         AllPlanets-0-0-0 & 0.02376142886 & 4.703584393e-009 \\
      \hline\hline
      \label{Table: DeepSpace WinGMAT-STK Table} 
\end{tabular}
\end{table}
\index{DeepSpace}\index{STK}
\begin{table}[htbp!]
\centering
\caption{ WinGMAT/STK EML2 Test Case Comparison}
      \begin{tabular}{lcc}
      \hline\hline
          Test Case & Position Difference(m) & Velocity Difference(m/s) \\
         \hline
         %---New Row---%
         AllPlanets-0-0-0 & 522765.3537 & 2.863727241 \\
         %---New Row---%
         AllPlanets-0-0-SRP & 304992.5809 & 1.654543949 \\
         %---New Row---%
         EarthSunLuna-0-0-0 & 191050.7911 & 1.043593334 \\
         %---New Row---%
         EarthSunLuna-JGM2-0-0 & 83171.00019 & 0.4533443656 \\
      \hline\hline
      \label{Table: EML2 WinGMAT-STK Table} 
\end{tabular}
\end{table}
\index{EML2}\index{STK}
\begin{table}[htbp!]
\centering
\caption{ WinGMAT/STK ESL2 Test Case Comparison}
      \begin{tabular}{lcc}
      \hline\hline
          Test Case & Position Difference(m) & Velocity Difference(m/s) \\
         \hline
         %---New Row---%
         AllPlanets-0-0-0 & 29036.71985 & 0.01713684894 \\
         %---New Row---%
         AllPlanets-0-0-SRP & 162456.0169 & 0.1246828624 \\
      \hline\hline
      \label{Table: ESL2 WinGMAT-STK Table} 
\end{tabular}
\end{table}
\index{ESL2}\index{STK}
\clearpage
\begin{table}[htbp!]
\centering
\caption{ FF/WinGMAT GEO Test Case Comparison}
      \begin{tabular}{lcc}
      \hline\hline
          Test Case & Position Difference(m) & Velocity Difference(m/s) \\
         \hline
         %---New Row---%
         AllPlanets-0-0-0 & 0.02466296086 & 2.348952999e-006 \\
         %---New Row---%
         EarthLuna-0-0-0 & 0.02466037174 & 2.264158243e-006 \\
         %---New Row---%
         EarthSun-0-0-0 & 4.79785384e-005 & 8.493947765e-007 \\
         %---New Row---%
         Earth-0-0-0 & 5.214381207e-005 & 6.89627309e-007 \\
         %---New Row---%
         Earth-0-0-SRP & 2.899321865 & 0.0001245219489 \\
         %---New Row---%
         Earth-JGM2-0-0 & 0.02515954067 & 2.313999898e-006 \\
      \hline\hline
      \label{Table: GEO FF-WinGMAT Table} 
\end{tabular}
\end{table}
\index{GEO}\index{FreeFlyer}
\begin{table}[htbp!]
\centering
\caption{ FF/WinGMAT GPS Test Case Comparison}
      \begin{tabular}{lcc}
      \hline\hline
          Test Case & Position Difference(m) & Velocity Difference(m/s) \\
         \hline
         %---New Row---%
         AllPlanets-0-0-0 & 0.001885974435 & 1.052020705e-006 \\
         %---New Row---%
         EarthLuna-0-0-0 & 0.001874007099 & 9.43361654e-007 \\
         %---New Row---%
         EarthSun-0-0-0 & 3.432176702e-006 & 8.317293859e-007 \\
         %---New Row---%
         Earth-0-0-0 & 5.489921041e-006 & 7.9829921e-007 \\
         %---New Row---%
         Earth-0-0-SRP & 0.6535523199 & 6.830245231e-005 \\
         %---New Row---%
         Earth-JGM2-0-0 & 0.01104317452 & 2.140216499e-006 \\
      \hline\hline
      \label{Table: GPS FF-WinGMAT Table} 
\end{tabular}
\end{table}
\index{GPS}\index{FreeFlyer}
\begin{table}[htbp!]
\centering
\caption{ FF/WinGMAT ISS Test Case Comparison}
      \begin{tabular}{lcc}
      \hline\hline
          Test Case & Position Difference(m) & Velocity Difference(m/s) \\
         \hline
         %---New Row---%
         AllPlanets-0-0-0 & 5.610904349e-006 & 8.224014796e-007 \\
         %---New Row---%
         EarthLuna-0-0-0 & 1.133489824e-005 & 8.407000067e-007 \\
         %---New Row---%
         EarthSun-0-0-0 & 9.638402495e-006 & 8.134703987e-007 \\
         %---New Row---%
         Earth-0-0-0 & 2.358180759e-005 & 8.186468434e-007 \\
         %---New Row---%
         Earth-0-0-SRP & 0.03352070462 & 2.995220929e-005 \\
         %---New Row---%
         Earth-JGM2-0-0 & 0.2076314901 & 0.0002404721693 \\
      \hline\hline
      \label{Table: ISS FF-WinGMAT Table} 
\end{tabular}
\end{table}
\index{ISS}\index{FreeFlyer}
\begin{table}[htbp!]
\centering
\caption{ FF/WinGMAT Molniya Test Case Comparison}
      \begin{tabular}{lcc}
      \hline\hline
          Test Case & Position Difference(m) & Velocity Difference(m/s) \\
         \hline
         %---New Row---%
         AllPlanets-0-0-0 & 0.01444526657 & 1.203396044e-005 \\
         %---New Row---%
         EarthLuna-0-0-0 & 0.01424235005 & 1.19546892e-005 \\
         %---New Row---%
         EarthSun-0-0-0 & 0.0003311394147 & 8.232647373e-007 \\
         %---New Row---%
         Earth-0-0-0 & 5.789754939e-005 & 8.078947532e-007 \\
         %---New Row---%
         Earth-0-0-SRP & 2.774957384 & 0.002291583263 \\
         %---New Row---%
         Earth-JGM2-0-0 & 3.064693947 & 0.002562937467 \\
      \hline\hline
      \label{Table: Molniya FF-WinGMAT Table} 
\end{tabular}
\end{table}
\index{Molniya}\index{FreeFlyer}
\begin{table}[htbp!]
\centering
\caption{ FF/WinGMAT SunSync Test Case Comparison}
      \begin{tabular}{lcc}
      \hline\hline
          Test Case & Position Difference(m) & Velocity Difference(m/s) \\
         \hline
         %---New Row---%
         AllPlanets-0-0-0 & 1.624452888e-005 & 8.363014362e-007 \\
         %---New Row---%
         EarthLuna-0-0-0 & 3.641972152e-005 & 8.139327674e-007 \\
         %---New Row---%
         EarthSun-0-0-0 & 1.62185169e-005 & 8.470799538e-007 \\
         %---New Row---%
         Earth-0-0-0 & 8.93851495e-006 & 8.215195e-007 \\
         %---New Row---%
         Earth-0-0-SRP & 0.03205041923 & 2.81000432e-005 \\
         %---New Row---%
         Earth-JGM2-0-0 & 0.09458368297 & 0.0001055216409 \\
      \hline\hline
      \label{Table: SunSync FF-WinGMAT Table} 
\end{tabular}
\end{table}
\index{SunSync}\index{FreeFlyer}
\clearpage
\begin{table}[htbp!]
\centering
\caption{ FF/WinGMAT EML2 Test Case Comparison}
      \begin{tabular}{lcc}
      \hline\hline
          Test Case & Position Difference(m) & Velocity Difference(m/s) \\
         \hline
         %---New Row---%
         AllPlanets-0-0-0 & 732436.6987 & 4.007738969 \\
         %---New Row---%
         AllPlanets-0-0-SRP & 512825.9651 & 2.780610367 \\
         %---New Row---%
         EarthSunLuna-0-0-0 & 400326.9215 & 2.185454594 \\
         %---New Row---%
         EarthSunLuna-JGM2-0-0 & 292445.684 & 1.595241714 \\
      \hline\hline
      \label{Table: EML2 FF-WinGMAT Table} 
\end{tabular}
\end{table}
\index{EML2}\index{FreeFlyer}
\begin{table}[htbp!]
\centering
\caption{ FF/WinGMAT ESL2 Test Case Comparison}
      \begin{tabular}{lcc}
      \hline\hline
          Test Case & Position Difference(m) & Velocity Difference(m/s) \\
         \hline
         %---New Row---%
         AllPlanets-0-0-0 & 5690806.264 & 3.131093893 \\
         %---New Row---%
         AllPlanets-0-0-SRP & 6718791.552 & 5.235648755 \\
      \hline\hline
      \label{Table: ESL2 FF-WinGMAT Table} 
\end{tabular}
\end{table}
\index{ESL2}\index{FreeFlyer}
\clearpage

\section{Win/Mac GMAT Comparison}
\begin{table}[htbp!]
\centering
\caption{ WinGMAT/MacGMAT GEO Test Case Comparison}
      \begin{tabular}{lcc}
      \hline\hline
          Test Case & Position Difference(m) & Velocity Difference(m/s) \\
         \hline
         %---New Row---%
         AllPlanets-0-0-0 & 7.889320184e-006 & 4.981092583e-010 \\
         %---New Row---%
         EarthLuna-0-0-0 & 3.862310091e-005 & 2.829034403e-009 \\
         %---New Row---%
         EarthSunLuna-EGM96-JR-SRP & 0.0001255674449 & 9.1445459e-009 \\
         %---New Row---%
         EarthSunLuna-EGM96-MSISE90-SRP & 0.0001255674449 & 9.144545899e-009 \\
         %---New Row---%
         EarthSunLuna-JGM2-JR-SRP & 7.353496438e-005 & 5.462181017e-009 \\
         %---New Row---%
         EarthSunLuna-JGM2-MSISE90-SRP & 7.353496437e-005 & 5.462181016e-009 \\
         %---New Row---%
         EarthSunLuna-JGM3-JR-SRP & 0.0001352482846 & 9.775905712e-009 \\
         %---New Row---%
         EarthSunLuna-JGM3-MSISE90-SRP & 0.0001352482846 & 9.775905715e-009 \\
         %---New Row---%
         EarthSun-0-0-0 & 9.928130768e-005 & 7.154327853e-009 \\
         %---New Row---%
         Earth-0-0-0 & 4.990522323e-005 & 3.634067315e-009 \\
         %---New Row---%
         Earth-0-0-SRP & 4.520549659e-005 & 3.269987835e-009 \\
         %---New Row---%
         Earth-0-JR-0 & 4.990522323e-005 & 3.634067315e-009 \\
         %---New Row---%
         Earth-0-MSISE90-0 & 4.990522323e-005 & 3.634067315e-009 \\
         %---New Row---%
         Earth-EGM96-0-0 & 6.81671185e-005 & 4.950072943e-009 \\
         %---New Row---%
         Earth-EGM96full-0-0 & 6.81671185e-005 & 4.950072943e-009 \\
         %---New Row---%
         Earth-JGM2-0-0 & 1.538671514e-005 & 1.112479746e-009 \\
         %---New Row---%
         Earth-JGM2full-0-0 & 1.538671514e-005 & 1.112479746e-009 \\
         %---New Row---%
         Earth-JGM3-0-0 & 5.489787175e-006 & 4.047244602e-010 \\
         %---New Row---%
         Earth-JGM3full-0-0 & 5.489787175e-006 & 4.047244602e-010 \\
      \hline\hline
      \label{Table: GEO WinGMAT-MacGMAT Table} 
\end{tabular}
\end{table}
\index{GEO}
\begin{table}[htbp!]
\centering
\caption{ WinGMAT/MacGMAT GPS Test Case Comparison}
      \begin{tabular}{lcc}
      \hline\hline
          Test Case & Position Difference(m) & Velocity Difference(m/s) \\
         \hline
         %---New Row---%
         AllPlanets-0-0-0 & 1.524285564e-005 & 2.13023419e-009 \\
         %---New Row---%
         EarthLuna-0-0-0 & 7.689731018e-006 & 1.062224728e-009 \\
         %---New Row---%
         EarthSunLuna-EGM96-JR-SRP & 1.627453194e-005 & 2.270918492e-009 \\
         %---New Row---%
         EarthSunLuna-EGM96-MSISE90-SRP & 9.265605105e-006 & 1.328441506e-009 \\
         %---New Row---%
         EarthSunLuna-JGM2-JR-SRP & 2.80275826e-006 & 3.762387107e-010 \\
         %---New Row---%
         EarthSunLuna-JGM2-MSISE90-SRP & 7.908282521e-006 & 1.128542622e-009 \\
         %---New Row---%
         EarthSunLuna-JGM3-JR-SRP & 6.035842385e-006 & 7.962769859e-010 \\
         %---New Row---%
         EarthSunLuna-JGM3-MSISE90-SRP & 6.490395836e-006 & 9.21768e-010 \\
         %---New Row---%
         EarthSun-0-0-0 & 3.222076727e-006 & 4.833482138e-010 \\
         %---New Row---%
         Earth-0-0-0 & 5.234750509e-006 & 7.706152625e-010 \\
         %---New Row---%
         Earth-0-0-SRP & 1.171446476e-006 & 1.822703558e-010 \\
         %---New Row---%
         Earth-0-JR-0 & 5.234750509e-006 & 7.706152625e-010 \\
         %---New Row---%
         Earth-0-MSISE90-0 & 2.782120359e-006 & 3.443304045e-010 \\
         %---New Row---%
         Earth-EGM96-0-0 & 1.050744267e-005 & 1.447584802e-009 \\
         %---New Row---%
         Earth-EGM96full-0-0 & 1.050744267e-005 & 1.447584802e-009 \\
         %---New Row---%
         Earth-JGM2-0-0 & 2.283784887e-006 & 3.435638023e-010 \\
         %---New Row---%
         Earth-JGM2full-0-0 & 2.283784887e-006 & 3.435638023e-010 \\
         %---New Row---%
         Earth-JGM3-0-0 & 7.101270799e-006 & 1.060395602e-009 \\
         %---New Row---%
         Earth-JGM3full-0-0 & 7.101270799e-006 & 1.060395602e-009 \\
      \hline\hline
      \label{Table: GPS WinGMAT-MacGMAT Table} 
\end{tabular}
\end{table}
\index{GPS}
\begin{table}[htbp!]
\centering
\caption{ WinGMAT/MacGMAT ISS Test Case Comparison}
      \begin{tabular}{lcc}
      \hline\hline
          Test Case & Position Difference(m) & Velocity Difference(m/s) \\
         \hline
         %---New Row---%
         AllPlanets-0-0-0 & 1.767688707e-005 & 1.987826967e-008 \\
         %---New Row---%
         EarthLuna-0-0-0 & 4.058801998e-005 & 4.617387375e-008 \\
         %---New Row---%
         EarthSunLuna-EGM96-JR-SRP & 2.350033054e-005 & 2.662367385e-008 \\
         %---New Row---%
         EarthSunLuna-EGM96-MSISE90-SRP & 0.0006669672891 & 7.368745763e-007 \\
         %---New Row---%
         EarthSunLuna-JGM2-JR-SRP & 1.840611847e-005 & 2.086797767e-008 \\
         %---New Row---%
         EarthSunLuna-JGM2-MSISE90-SRP & 0.0006391410111 & 7.093227198e-007 \\
         %---New Row---%
         EarthSunLuna-JGM3-JR-SRP & 4.906955953e-005 & 5.585723624e-008 \\
         %---New Row---%
         EarthSunLuna-JGM3-MSISE90-SRP & 0.0001694547514 & 1.723059642e-007 \\
         %---New Row---%
         EarthSun-0-0-0 & 3.963497881e-006 & 4.502622513e-009 \\
         %---New Row---%
         Earth-0-0-0 & 3.476512236e-005 & 3.980331596e-008 \\
         %---New Row---%
         Earth-0-0-SRP & 1.709387551e-005 & 1.961412784e-008 \\
         %---New Row---%
         Earth-0-JR-0 & 1.516771704e-006 & 1.633830233e-009 \\
         %---New Row---%
         Earth-0-MSISE90-0 & 0.0006871347746 & 7.902102731e-007 \\
         %---New Row---%
         Earth-EGM96-0-0 & 1.608009486e-005 & 1.792593523e-008 \\
         %---New Row---%
         Earth-EGM96full-0-0 & 1.796350308e-005 & 2.023666e-008 \\
         %---New Row---%
         Earth-JGM2-0-0 & 8.522888644e-006 & 9.778070887e-009 \\
         %---New Row---%
         Earth-JGM2full-0-0 & 7.900862621e-006 & 8.784310979e-009 \\
         %---New Row---%
         Earth-JGM3-0-0 & 6.904868565e-006 & 7.605149901e-009 \\
         %---New Row---%
         Earth-JGM3full-0-0 & 8.703769381e-006 & 9.962336035e-009 \\
      \hline\hline
      \label{Table: ISS WinGMAT-MacGMAT Table} 
\end{tabular}
\end{table}
\index{ISS}
\begin{table}[htbp!]
\centering
\caption{ WinGMAT/MacGMAT Molniya Test Case Comparison}
      \begin{tabular}{lcc}
      \hline\hline
          Test Case & Position Difference(m) & Velocity Difference(m/s) \\
         \hline
         %---New Row---%
         AllPlanets-0-0-0 & 0.0001161355335 & 9.734627028e-008 \\
         %---New Row---%
         EarthLuna-0-0-0 & 0.0001765185068 & 1.474675389e-007 \\
         %---New Row---%
         EarthSunLuna-EGM96-JR-SRP & 9.149240024e-005 & 7.550332195e-008 \\
         %---New Row---%
         EarthSunLuna-EGM96-MSISE90-SRP & 0.000337018456 & 2.517462331e-007 \\
         %---New Row---%
         EarthSunLuna-JGM2-JR-SRP & 0.0001687015187 & 1.180293741e-007 \\
         %---New Row---%
         EarthSunLuna-JGM2-MSISE90-SRP & 0.0003458685231 & 2.567145268e-007 \\
         %---New Row---%
         EarthSunLuna-JGM3-JR-SRP & 2.89102084e-005 & 1.439959413e-008 \\
         %---New Row---%
         EarthSunLuna-JGM3-MSISE90-SRP & 0.0004306462955 & 3.433138109e-007 \\
         %---New Row---%
         EarthSun-0-0-0 & 0.0004166915691 & 3.484312793e-007 \\
         %---New Row---%
         Earth-0-0-0 & 0.0004592790738 & 3.839183118e-007 \\
         %---New Row---%
         Earth-0-0-SRP & 3.343036684e-005 & 2.780900576e-008 \\
         %---New Row---%
         Earth-0-JR-0 & 0.0002482930089 & 2.078309993e-007 \\
         %---New Row---%
         Earth-0-MSISE90-0 & 9.929293779e-005 & 8.265330135e-008 \\
         %---New Row---%
         Earth-EGM96-0-0 & 4.303627343e-005 & 2.617173925e-008 \\
         %---New Row---%
         Earth-EGM96full-0-0 & 0.0001105407449 & 9.218740603e-008 \\
         %---New Row---%
         Earth-JGM2-0-0 & 0.0002507869521 & 1.785682667e-007 \\
         %---New Row---%
         Earth-JGM2full-0-0 & 0.0002346500837 & 1.623140036e-007 \\
         %---New Row---%
         Earth-JGM3-0-0 & 5.099576082e-005 & 4.255038887e-008 \\
         %---New Row---%
         Earth-JGM3full-0-0 & 0.0001446553101 & 8.368203852e-008 \\
      \hline\hline
      \label{Table: Molniya WinGMAT-MacGMAT Table} 
\end{tabular}
\end{table}
\index{Molniya}
\begin{table}[htbp!]
\centering
\caption{ WinGMAT/MacGMAT SunSync Test Case Comparison}
      \begin{tabular}{lcc}
      \hline\hline
          Test Case & Position Difference(m) & Velocity Difference(m/s) \\
         \hline
         %---New Row---%
         AllPlanets-0-0-0 & 6.500418074e-006 & 6.999302677e-009 \\
         %---New Row---%
         EarthLuna-0-0-0 & 4.322773976e-005 & 4.846536744e-008 \\
         %---New Row---%
         EarthSunLuna-EGM96-JR-SRP & 1.07874072e-005 & 1.222942998e-008 \\
         %---New Row---%
         EarthSunLuna-EGM96-MSISE90-SRP & 0.0002907736973 & 3.198710152e-007 \\
         %---New Row---%
         EarthSunLuna-JGM2-JR-SRP & 5.336201438e-006 & 5.644390406e-009 \\
         %---New Row---%
         EarthSunLuna-JGM2-MSISE90-SRP & 8.135754448e-005 & 8.017598457e-008 \\
         %---New Row---%
         EarthSunLuna-JGM3-JR-SRP & 4.666661247e-005 & 5.293796337e-008 \\
         %---New Row---%
         EarthSunLuna-JGM3-MSISE90-SRP & 0.0003578443346 & 3.980896788e-007 \\
         %---New Row---%
         EarthSun-0-0-0 & 1.500943262e-005 & 1.689722434e-008 \\
         %---New Row---%
         Earth-0-0-0 & 2.653939383e-005 & 2.998036763e-008 \\
         %---New Row---%
         Earth-0-0-SRP & 3.969621364e-005 & 4.465304759e-008 \\
         %---New Row---%
         Earth-0-JR-0 & 3.885692582e-005 & 4.338304594e-008 \\
         %---New Row---%
         Earth-0-MSISE90-0 & 0.0001846260209 & 2.024641677e-007 \\
         %---New Row---%
         Earth-EGM96-0-0 & 2.749076581e-006 & 2.902225971e-009 \\
         %---New Row---%
         Earth-EGM96full-0-0 & 3.242311524e-006 & 3.49137102e-009 \\
         %---New Row---%
         Earth-JGM2-0-0 & 1.047153324e-005 & 1.171331382e-008 \\
         %---New Row---%
         Earth-JGM2full-0-0 & 1.439187158e-005 & 1.599554939e-008 \\
         %---New Row---%
         Earth-JGM3-0-0 & 1.416756213e-005 & 1.591877652e-008 \\
         %---New Row---%
         Earth-JGM3full-0-0 & 1.85849559e-005 & 2.086516383e-008 \\
      \hline\hline
      \label{Table: SunSync WinGMAT-MacGMAT Table} 
\end{tabular}
\end{table}
\index{Sunsync}
\clearpage
\begin{table}[htbp!]
\centering
\caption{ WinGMAT/MacGMAT Mars1 Test Case Comparison}
      \begin{tabular}{lcc}
      \hline\hline
          Test Case & Position Difference(m) & Velocity Difference(m/s) \\
         \hline
         %---New Row---%
         AllPlanets-0-0-0 & 0.03972223921 & 3.343331462e-005 \\
         %---New Row---%
         Mars-0-0-0 & 0.005311861757 & 4.268418571e-006 \\
         %---New Row---%
         Mars-0-0-SRP & 0.05222305978 & 4.393980991e-005 \\
         %---New Row---%
         Mars-MARS50C-0-0 & 0.01297658486 & 1.093110803e-005 \\
         %---New Row---%
         Mars-MARS50C-0-SRP & 0.1117967472 & 9.414441698e-005 \\
      \hline\hline
      \label{Table: Mars1 WinGMAT-MacGMAT Table} 
\end{tabular}
\end{table}

\begin{table}[htbp!]
\centering
\caption{ WinGMAT/MacGMAT Mercury1 Test Case Comparison}
      \begin{tabular}{lcc}
      \hline\hline
          Test Case & Position Difference(m) & Velocity Difference(m/s) \\
         \hline
         %---New Row---%
         AllPlanets-0-0-0 & 0.004516330784 & 3.600005592e-006 \\
         %---New Row---%
         Mercury-0-0-0 & 0.03778257814 & 3.190578299e-005 \\
         %---New Row---%
         Mercury-0-0-SRP & 0.02607980485 & 2.236210801e-005 \\
      \hline\hline
      \label{Table: Mercury1 WinGMAT-MacGMAT Table} 
\end{tabular}
\end{table}

\begin{table}[htbp!]
\centering
\caption{ WinGMAT/MacGMAT Moon Test Case Comparison}
      \begin{tabular}{lcc}
      \hline\hline
          Test Case & Position Difference(m) & Velocity Difference(m/s) \\
         \hline
         %---New Row---%
         AllPlanets-0-0-0 & 0.0003918908681 & 2.810115186e-007 \\
         %---New Row---%
         Luna-0-0-0 & 8.768096011e-005 & 6.221838761e-008 \\
         %---New Row---%
         Luna-0-0-SRP & 0.000593804859 & 4.250992344e-007 \\
         %---New Row---%
         Luna-LP165P-0-0 & 0.0004664933125 & 3.338481962e-007 \\
         %---New Row---%
         Luna-LP165P-0-SRP & 0.0001597775244 & 1.143221729e-007 \\
      \hline\hline
      \label{Table: Moon WinGMAT-MacGMAT Table} 
\end{tabular}
\end{table}

\begin{table}[htbp!]
\centering
\caption{ WinGMAT/MacGMAT Neptune1 Test Case Comparison}
      \begin{tabular}{lcc}
      \hline\hline
          Test Case & Position Difference(m) & Velocity Difference(m/s) \\
         \hline
         %---New Row---%
         AllPlanets-0-0-0 & 1.455191523e-008 & 6.20685014e-010 \\
         %---New Row---%
         Neptune-0-0-0 & 1.543463691e-008 & 1.126411949e-010 \\
         %---New Row---%
         Neptune-0-0-SRP & 1.458030924e-008 & 1.741877302e-010 \\
      \hline\hline
      \label{Table: Neptune1 WinGMAT-MacGMAT Table} 
\end{tabular}
\end{table}

\begin{table}[htbp!]
\centering
\caption{ WinGMAT/MacGMAT Pluto1 Test Case Comparison}
      \begin{tabular}{lcc}
      \hline\hline
          Test Case & Position Difference(m) & Velocity Difference(m/s) \\
         \hline
         %---New Row---%
         AllPlanets-0-0-0 & 1.286219742e-009 & 3.384705673e-011 \\
         %---New Row---%
         Pluto-0-0-0 & 1.286219742e-009 & 9.024959195e-012 \\
         %---New Row---%
         Pluto-0-0-SRP & 1.023181539e-009 & 1.649647961e-010 \\
      \hline\hline
      \label{Table: Pluto1 WinGMAT-MacGMAT Table} 
\end{tabular}
\end{table}

\begin{table}[htbp!]
\centering
\caption{ WinGMAT/MacGMAT Saturn1 Test Case Comparison}
      \begin{tabular}{lcc}
      \hline\hline
          Test Case & Position Difference(m) & Velocity Difference(m/s) \\
         \hline
         %---New Row---%
         AllPlanets-0-0-0 & 1.626953583e-008 & 1.628929876e-010 \\
         %---New Row---%
         Saturn-0-0-0 & 2.057951587e-008 & 1.669113889e-010 \\
         %---New Row---%
         Saturn-0-0-SRP & 2.057951587e-008 & 1.124589615e-010 \\
      \hline\hline
      \label{Table: Saturn1 WinGMAT-MacGMAT Table} 
\end{tabular}
\end{table}

\begin{table}[htbp!]
\centering
\caption{ WinGMAT/MacGMAT Uranus1 Test Case Comparison}
      \begin{tabular}{lcc}
      \hline\hline
          Test Case & Position Difference(m) & Velocity Difference(m/s) \\
         \hline
         %---New Row---%
         AllPlanets-0-0-0 & 1.458030924e-008 & 2.642610672e-011 \\
         %---New Row---%
         Uranus-0-0-0 & 1.455191523e-008 & 8.413351831e-011 \\
         %---New Row---%
         Uranus-0-0-SRP & 2.057951587e-008 & 6.76066571e-011 \\
      \hline\hline
      \label{Table: Uranus1 WinGMAT-MacGMAT Table} 
\end{tabular}
\end{table}

\begin{table}[htbp!]
\centering
\caption{ WinGMAT/MacGMAT Venus1 Test Case Comparison}
      \begin{tabular}{lcc}
      \hline\hline
          Test Case & Position Difference(m) & Velocity Difference(m/s) \\
         \hline
         %---New Row---%
         AllPlanets-0-0-0 & 0.001945532067 & 1.947340584e-006 \\
         %---New Row---%
         Venus-0-0-0 & 0.01882594263 & 1.859722326e-005 \\
         %---New Row---%
         Venus-0-0-SRP & 0.02322953755 & 2.299817313e-005 \\
         %---New Row---%
         Venus-MGNP180U-0-0 & 0.03289233273 & 3.254048475e-005 \\
         %---New Row---%
         Venus-MGNP180U-0-SRP & 0.03086632412 & 3.063642165e-005 \\
      \hline\hline
      \label{Table: Venus1 WinGMAT-MacGMAT Table} 
\end{tabular}
\end{table}

\clearpage
\begin{table}[htbp!]
\centering
\caption{ WinGMAT/MacGMAT DeepSpace Test Case Comparison}
      \begin{tabular}{lcc}
      \hline\hline
          Test Case & Position Difference(m) & Velocity Difference(m/s) \\
         \hline
         %---New Row---%
         AllPlanets-0-0-0 & 0.004474133849 & 9.33402511e-010 \\
      \hline\hline
      \label{Table: DeepSpace WinGMAT-MacGMAT Table} 
\end{tabular}
\end{table}
\index{DeepSpace}
\begin{table}[htbp!]
\centering
\caption{ WinGMAT/MacGMAT EML2 Test Case Comparison}
      \begin{tabular}{lcc}
      \hline\hline
          Test Case & Position Difference(m) & Velocity Difference(m/s) \\
         \hline
         %---New Row---%
         AllPlanets-0-0-0 & 55804.23343 & 0.3054834604 \\
         %---New Row---%
         AllPlanets-0-0-SRP & 124016.2439 & 0.6662855702 \\
         %---New Row---%
         EarthSunLuna-0-0-0 & 33336.13696 & 0.1885640618 \\
         %---New Row---%
         EarthSunLuna-JGM2-0-0 & 214046.0535 & 1.167672328 \\
      \hline\hline
      \label{Table: EML2 WinGMAT-MacGMAT Table} 
\end{tabular}
\end{table}
\index{EML2}
\begin{table}[htbp!]
\centering
\caption{ WinGMAT/MacGMAT ESL2 Test Case Comparison}
      \begin{tabular}{lcc}
      \hline\hline
          Test Case & Position Difference(m) & Velocity Difference(m/s) \\
         \hline
         %---New Row---%
         AllPlanets-0-0-0 & 380849.9527 & 0.2067925122 \\
         %---New Row---%
         AllPlanets-0-0-SRP & 234119.7653 & 0.1824594525 \\
      \hline\hline
      \label{Table: ESL2 WinGMAT-MacGMAT Table} 
\end{tabular}
\end{table}
\index{ESL2}
\clearpage

\section{Win/Linux GMAT Comparison}
\begin{table}[htbp!]
\centering
\caption{ WinGMAT/LinuxGMAT GEO Test Case Comparison}
      \begin{tabular}{lcc}
      \hline\hline
          Test Case & Position Difference(m) & Velocity Difference(m/s) \\
         \hline
         %---New Row---%
         AllPlanets-0-0-0 & 1.818989404e-008 & 1.256073967e-012 \\
         %---New Row---%
         EarthLuna-0-0-0 & 1.818989404e-008 & 1.256073967e-012 \\
         %---New Row---%
         EarthSunLuna-EGM96-JR-SRP & 1.818989404e-008 & 1.256073967e-012 \\
         %---New Row---%
         EarthSunLuna-EGM96-MSISE90-SRP & 1.818989404e-008 & 1.256073967e-012 \\
         %---New Row---%
         EarthSunLuna-JGM2-JR-SRP & 1.818989404e-008 & 1.256073967e-012 \\
         %---New Row---%
         EarthSunLuna-JGM2-MSISE90-SRP & 1.818989404e-008 & 1.256073967e-012 \\
         %---New Row---%
         EarthSunLuna-JGM3-JR-SRP & 1.818989404e-008 & 1.256073967e-012 \\
         %---New Row---%
         EarthSunLuna-JGM3-MSISE90-SRP & 1.818989404e-008 & 1.256073967e-012 \\
         %---New Row---%
         EarthSun-0-0-0 & 1.818989404e-008 & 1.256073967e-012 \\
         %---New Row---%
         Earth-0-0-0 & 1.818989404e-008 & 1.256073967e-012 \\
         %---New Row---%
         Earth-0-0-SRP & 1.626953583e-008 & 1.256073967e-012 \\
         %---New Row---%
         Earth-0-JR-0 & 1.818989404e-008 & 1.256073967e-012 \\
         %---New Row---%
         Earth-0-MSISE90-0 & 1.818989404e-008 & 1.256073967e-012 \\
         %---New Row---%
         Earth-EGM96-0-0 & 1.818989404e-008 & 1.256073967e-012 \\
         %---New Row---%
         Earth-EGM96full-0-0 & 1.818989404e-008 & 1.256073967e-012 \\
         %---New Row---%
         Earth-JGM2-0-0 & 1.818989404e-008 & 8.950904183e-013 \\
         %---New Row---%
         Earth-JGM2full-0-0 & 1.818989404e-008 & 8.950904183e-013 \\
         %---New Row---%
         Earth-JGM3-0-0 & 1.466516141e-008 & 1.256073967e-012 \\
         %---New Row---%
         Earth-JGM3full-0-0 & 1.466516141e-008 & 1.256073967e-012 \\
      \hline\hline
      \label{Table: GEO WinGMAT-LinuxGMAT Table} 
\end{tabular}
\end{table}
\index{GEO}
\begin{table}[htbp!]
\centering
\caption{ WinGMAT/LinuxGMAT GPS Test Case Comparison}
      \begin{tabular}{lcc}
      \hline\hline
          Test Case & Position Difference(m) & Velocity Difference(m/s) \\
         \hline
         %---New Row---%
         AllPlanets-0-0-0 & 1.543463691e-008 & 1.256073967e-012 \\
         %---New Row---%
         EarthLuna-0-0-0 & 1.164721514e-008 & 1.256073967e-012 \\
         %---New Row---%
         EarthSunLuna-EGM96-JR-SRP & 1.543463691e-008 & 1.256073967e-012 \\
         %---New Row---%
         EarthSunLuna-EGM96-MSISE90-SRP & 1.286219742e-008 & 1.25611709e-012 \\
         %---New Row---%
         EarthSunLuna-JGM2-JR-SRP & 1.311691913e-008 & 1.256073967e-012 \\
         %---New Row---%
         EarthSunLuna-JGM2-MSISE90-SRP & 1.311691913e-008 & 1.256073967e-012 \\
         %---New Row---%
         EarthSunLuna-JGM3-JR-SRP & 1.314841238e-008 & 1.538370149e-012 \\
         %---New Row---%
         EarthSunLuna-JGM3-MSISE90-SRP & 1.311691913e-008 & 1.256073967e-012 \\
         %---New Row---%
         EarthSun-0-0-0 & 1.42067614e-008 & 1.256073967e-012 \\
         %---New Row---%
         Earth-0-0-0 & 1.311691913e-008 & 1.256073967e-012 \\
         %---New Row---%
         Earth-0-0-SRP & 1.311691913e-008 & 1.256073967e-012 \\
         %---New Row---%
         Earth-0-JR-0 & 1.311691913e-008 & 1.256073967e-012 \\
         %---New Row---%
         Earth-0-MSISE90-0 & 1.42067614e-008 & 1.256073967e-012 \\
         %---New Row---%
         Earth-EGM96-0-0 & 1.311691913e-008 & 1.256073967e-012 \\
         %---New Row---%
         Earth-EGM96full-0-0 & 1.311691913e-008 & 1.256073967e-012 \\
         %---New Row---%
         Earth-JGM2-0-0 & 1.311691913e-008 & 1.256073967e-012 \\
         %---New Row---%
         Earth-JGM2full-0-0 & 1.311691913e-008 & 1.256073967e-012 \\
         %---New Row---%
         Earth-JGM3-0-0 & 1.311691913e-008 & 1.256073967e-012 \\
         %---New Row---%
         Earth-JGM3full-0-0 & 1.311691913e-008 & 1.256073967e-012 \\
      \hline\hline
      \label{Table: GPS WinGMAT-LinuxGMAT Table} 
\end{tabular}
\end{table}
\index{GPS}
\begin{table}[htbp!]
\centering
\caption{ WinGMAT/LinuxGMAT ISS Test Case Comparison}
      \begin{tabular}{lcc}
      \hline\hline
          Test Case & Position Difference(m) & Velocity Difference(m/s) \\
         \hline
         %---New Row---%
         AllPlanets-0-0-0 & 2.087012916e-009 & 2.512147934e-012 \\
         %---New Row---%
         EarthLuna-0-0-0 & 2.033691978e-009 & 2.512147934e-012 \\
         %---New Row---%
         EarthSunLuna-EGM96-JR-SRP & 2.033691978e-009 & 2.512147934e-012 \\
         %---New Row---%
         EarthSunLuna-EGM96-MSISE90-SRP & 0.0008470424638 & 9.587548246e-007 \\
         %---New Row---%
         EarthSunLuna-JGM2-JR-SRP & 2.572439484e-009 & 2.512147934e-012 \\
         %---New Row---%
         EarthSunLuna-JGM2-MSISE90-SRP & 0.0007669211768 & 8.667589808e-007 \\
         %---New Row---%
         EarthSunLuna-JGM3-JR-SRP & 2.572439484e-009 & 1.986027323e-012 \\
         %---New Row---%
         EarthSunLuna-JGM3-MSISE90-SRP & 0.0008251276656 & 9.357517454e-007 \\
         %---New Row---%
         EarthSun-0-0-0 & 2.572439484e-009 & 2.512147934e-012 \\
         %---New Row---%
         Earth-0-0-0 & 2.572439484e-009 & 1.986027323e-012 \\
         %---New Row---%
         Earth-0-0-SRP & 2.033691978e-009 & 2.512147934e-012 \\
         %---New Row---%
         Earth-0-JR-0 & 2.033691978e-009 & 2.512147934e-012 \\
         %---New Row---%
         Earth-0-MSISE90-0 & 0.0005086050402 & 5.7095113e-007 \\
         %---New Row---%
         Earth-EGM96-0-0 & 2.033691978e-009 & 2.512147934e-012 \\
         %---New Row---%
         Earth-EGM96full-0-0 & 2.033691978e-009 & 2.664535259e-012 \\
         %---New Row---%
         Earth-JGM2-0-0 & 2.572439484e-009 & 2.512147934e-012 \\
         %---New Row---%
         Earth-JGM2full-0-0 & 2.572439484e-009 & 2.512147934e-012 \\
         %---New Row---%
         Earth-JGM3-0-0 & 2.033691978e-009 & 1.986027323e-012 \\
         %---New Row---%
         Earth-JGM3full-0-0 & 2.033691978e-009 & 2.512147934e-012 \\
      \hline\hline
      \label{Table: ISS WinGMAT-LinuxGMAT Table} 
\end{tabular}
\end{table}
\index{ISS}
\begin{table}[htbp!]
\centering
\caption{ WinGMAT/LinuxGMAT Molniya Test Case Comparison}
      \begin{tabular}{lcc}
      \hline\hline
          Test Case & Position Difference(m) & Velocity Difference(m/s) \\
         \hline
         %---New Row---%
         AllPlanets-0-0-0 & 1.543463691e-008 & 1.776356839e-012 \\
         %---New Row---%
         EarthLuna-0-0-0 & 1.818989404e-008 & 1.776356839e-012 \\
         %---New Row---%
         EarthSunLuna-EGM96-JR-SRP & 1.818989404e-008 & 1.986027323e-012 \\
         %---New Row---%
         EarthSunLuna-EGM96-MSISE90-SRP & 0.0002835450278 & 1.639475709e-007 \\
         %---New Row---%
         EarthSunLuna-JGM2-JR-SRP & 1.818989404e-008 & 1.776356839e-012 \\
         %---New Row---%
         EarthSunLuna-JGM2-MSISE90-SRP & 0.0003637177938 & 2.536251812e-007 \\
         %---New Row---%
         EarthSunLuna-JGM3-JR-SRP & 1.818989404e-008 & 1.986027323e-012 \\
         %---New Row---%
         EarthSunLuna-JGM3-MSISE90-SRP & 0.0001377244321 & 6.842684735e-008 \\
         %---New Row---%
         EarthSun-0-0-0 & 1.455191523e-008 & 2.512147934e-012 \\
         %---New Row---%
         Earth-0-0-0 & 1.818989404e-008 & 1.986027323e-012 \\
         %---New Row---%
         Earth-0-0-SRP & 1.455191523e-008 & 1.776356839e-012 \\
         %---New Row---%
         Earth-0-JR-0 & 1.818989404e-008 & 2.512147934e-012 \\
         %---New Row---%
         Earth-0-MSISE90-0 & 0.0005204508808 & 4.351107742e-007 \\
         %---New Row---%
         Earth-EGM96-0-0 & 1.543463691e-008 & 1.986027323e-012 \\
         %---New Row---%
         Earth-EGM96full-0-0 & 1.818989404e-008 & 2.512147934e-012 \\
         %---New Row---%
         Earth-JGM2-0-0 & 1.626953583e-008 & 1.779822905e-012 \\
         %---New Row---%
         Earth-JGM2full-0-0 & 1.818989404e-008 & 1.776356839e-012 \\
         %---New Row---%
         Earth-JGM3-0-0 & 1.626953583e-008 & 2.512147934e-012 \\
         %---New Row---%
         Earth-JGM3full-0-0 & 1.626953583e-008 & 1.986027323e-012 \\
      \hline\hline
      \label{Table: Molniya WinGMAT-LinuxGMAT Table} 
\end{tabular}
\end{table}
\index{Molniya}
\begin{table}[htbp!]
\centering
\caption{ WinGMAT/LinuxGMAT SunSync Test Case Comparison}
      \begin{tabular}{lcc}
      \hline\hline
          Test Case & Position Difference(m) & Velocity Difference(m/s) \\
         \hline
         %---New Row---%
         AllPlanets-0-0-0 & 2.033691978e-009 & 1.986027323e-012 \\
         %---New Row---%
         EarthLuna-0-0-0 & 2.033691978e-009 & 2.512147934e-012 \\
         %---New Row---%
         EarthSunLuna-EGM96-JR-SRP & 2.033691978e-009 & 2.664535259e-012 \\
         %---New Row---%
         EarthSunLuna-EGM96-MSISE90-SRP & 0.0004937360699 & 5.662979365e-007 \\
         %---New Row---%
         EarthSunLuna-JGM2-JR-SRP & 2.033691978e-009 & 2.512147934e-012 \\
         %---New Row---%
         EarthSunLuna-JGM2-MSISE90-SRP & 0.0006551562744 & 7.5461681e-007 \\
         %---New Row---%
         EarthSunLuna-JGM3-JR-SRP & 2.033691978e-009 & 1.986027323e-012 \\
         %---New Row---%
         EarthSunLuna-JGM3-MSISE90-SRP & 0.0005964945835 & 6.857578925e-007 \\
         %---New Row---%
         EarthSun-0-0-0 & 2.033691978e-009 & 1.986027323e-012 \\
         %---New Row---%
         Earth-0-0-0 & 2.572439484e-009 & 1.986027323e-012 \\
         %---New Row---%
         Earth-0-0-SRP & 2.572439484e-009 & 2.512147934e-012 \\
         %---New Row---%
         Earth-0-JR-0 & 2.033691978e-009 & 1.986027323e-012 \\
         %---New Row---%
         Earth-0-MSISE90-0 & 0.0006288991645 & 7.221314608e-007 \\
         %---New Row---%
         Earth-EGM96-0-0 & 2.033691978e-009 & 1.986027323e-012 \\
         %---New Row---%
         Earth-EGM96full-0-0 & 2.036867143e-009 & 2.512147934e-012 \\
         %---New Row---%
         Earth-JGM2-0-0 & 2.033691978e-009 & 1.986027323e-012 \\
         %---New Row---%
         Earth-JGM2full-0-0 & 2.572439484e-009 & 1.986027323e-012 \\
         %---New Row---%
         Earth-JGM3-0-0 & 2.033691978e-009 & 2.512147934e-012 \\
         %---New Row---%
         Earth-JGM3full-0-0 & 2.572439484e-009 & 2.512147934e-012 \\
      \hline\hline
      \label{Table: SunSync WinGMAT-LinuxGMAT Table} 
\end{tabular}
\end{table}
\index{Sunsync}
\clearpage
\begin{table}[htbp!]
\centering
\caption{ WinGMAT/LinuxGMAT Mars1 Test Case Comparison}
      \begin{tabular}{lcc}
      \hline\hline
          Test Case & Position Difference(m) & Velocity Difference(m/s) \\
         \hline
         %---New Row---%
         AllPlanets-0-0-0 & 2.033691978e-009 & 1.256073967e-012 \\
         %---New Row---%
         Mars-0-0-0 & 1.818989404e-009 & 1.256073967e-012 \\
         %---New Row---%
         Mars-0-0-SRP & 1.818989404e-009 & 1.256073967e-012 \\
         %---New Row---%
         Mars-MARS50C-0-0 & 1.818989404e-009 & 1.256073967e-012 \\
         %---New Row---%
         Mars-MARS50C-0-SRP & 1.822538655e-009 & 1.256073967e-012 \\
      \hline\hline
      \label{Table: Mars1 WinGMAT-LinuxGMAT Table} 
\end{tabular}
\end{table}

\begin{table}[htbp!]
\centering
\caption{ WinGMAT/LinuxGMAT Mercury1 Test Case Comparison}
      \begin{tabular}{lcc}
      \hline\hline
          Test Case & Position Difference(m) & Velocity Difference(m/s) \\
         \hline
         %---New Row---%
         AllPlanets-0-0-0 & 1.818989404e-009 & 1.256073967e-012 \\
         %---New Row---%
         Mercury-0-0-0 & 1.286219742e-009 & 1.256073967e-012 \\
         %---New Row---%
         Mercury-0-0-SRP & 1.822538655e-009 & 1.26097096e-012 \\
      \hline\hline
      \label{Table: Mercury1 WinGMAT-LinuxGMAT Table} 
\end{tabular}
\end{table}

\begin{table}[htbp!]
\centering
\caption{ WinGMAT/LinuxGMAT Moon Test Case Comparison}
      \begin{tabular}{lcc}
      \hline\hline
          Test Case & Position Difference(m) & Velocity Difference(m/s) \\
         \hline
         %---New Row---%
         AllPlanets-0-0-0 & 1.286219742e-009 & 8.950904183e-013 \\
         %---New Row---%
         Luna-0-0-0 & 1.575291033e-009 & 8.950904183e-013 \\
         %---New Row---%
         Luna-0-0-SRP & 1.286219742e-009 & 8.881784197e-013 \\
         %---New Row---%
         Luna-LP165P-0-0 & 0.0001310647334 & 9.401096987e-008 \\
         %---New Row---%
         Luna-LP165P-0-SRP & 0.0001872108595 & 1.337602461e-007 \\
      \hline\hline
      \label{Table: Moon WinGMAT-LinuxGMAT Table} 
\end{tabular}
\end{table}

\begin{table}[htbp!]
\centering
\caption{ WinGMAT/LinuxGMAT Neptune1 Test Case Comparison}
      \begin{tabular}{lcc}
      \hline\hline
          Test Case & Position Difference(m) & Velocity Difference(m/s) \\
         \hline
         %---New Row---%
         AllPlanets-0-0-0 & 1.455191523e-008 & 1.50728876e-011 \\
         %---New Row---%
         Neptune-0-0-0 & 1.543463691e-008 & 1.080515682e-011 \\
         %---New Row---%
         Neptune-0-0-SRP & 1.458030924e-008 & 1.080515682e-011 \\
      \hline\hline
      \label{Table: Neptune1 WinGMAT-LinuxGMAT Table} 
\end{tabular}
\end{table}

\begin{table}[htbp!]
\centering
\caption{ WinGMAT/LinuxGMAT Pluto1 Test Case Comparison}
      \begin{tabular}{lcc}
      \hline\hline
          Test Case & Position Difference(m) & Velocity Difference(m/s) \\
         \hline
         %---New Row---%
         AllPlanets-0-0-0 & 1.286219742e-009 & 1.570092459e-013 \\
         %---New Row---%
         Pluto-0-0-0 & 1.286219742e-009 & 1.570092459e-013 \\
         %---New Row---%
         Pluto-0-0-SRP & 1.023181539e-009 & 1.570092459e-013 \\
      \hline\hline
      \label{Table: Pluto1 WinGMAT-LinuxGMAT Table} 
\end{tabular}
\end{table}

\begin{table}[htbp!]
\centering
\caption{ WinGMAT/LinuxGMAT Saturn1 Test Case Comparison}
      \begin{tabular}{lcc}
      \hline\hline
          Test Case & Position Difference(m) & Velocity Difference(m/s) \\
         \hline
         %---New Row---%
         AllPlanets-0-0-0 & 1.626953583e-008 & 1.50728876e-011 \\
         %---New Row---%
         Saturn-0-0-0 & 2.057951587e-008 & 1.280949134e-011 \\
         %---New Row---%
         Saturn-0-0-SRP & 2.057951587e-008 & 1.280949134e-011 \\
      \hline\hline
      \label{Table: Saturn1 WinGMAT-LinuxGMAT Table} 
\end{tabular}
\end{table}

\begin{table}[htbp!]
\centering
\caption{ WinGMAT/LinuxGMAT Uranus1 Test Case Comparison}
      \begin{tabular}{lcc}
      \hline\hline
          Test Case & Position Difference(m) & Velocity Difference(m/s) \\
         \hline
         %---New Row---%
         AllPlanets-0-0-0 & 1.458030924e-008 & 1.065814104e-011 \\
         %---New Row---%
         Uranus-0-0-0 & 1.455191523e-008 & 1.065814104e-011 \\
         %---New Row---%
         Uranus-0-0-SRP & 2.057951587e-008 & 1.065814104e-011 \\
      \hline\hline
      \label{Table: Uranus1 WinGMAT-LinuxGMAT Table} 
\end{tabular}
\end{table}

\begin{table}[htbp!]
\centering
\caption{ WinGMAT/LinuxGMAT Venus1 Test Case Comparison}
      \begin{tabular}{lcc}
      \hline\hline
          Test Case & Position Difference(m) & Velocity Difference(m/s) \\
         \hline
         %---New Row---%
         AllPlanets-0-0-0 & 2.572439484e-009 & 1.986027323e-012 \\
         %---New Row---%
         Venus-0-0-0 & 2.572439484e-009 & 1.986027323e-012 \\
         %---New Row---%
         Venus-0-0-SRP & 2.033691978e-009 & 2.514600007e-012 \\
         %---New Row---%
         Venus-MGNP180U-0-0 & 2.572439484e-009 & 2.512147934e-012 \\
         %---New Row---%
         Venus-MGNP180U-0-SRP & 2.572439484e-009 & 2.512147934e-012 \\
      \hline\hline
      \label{Table: Venus1 WinGMAT-LinuxGMAT Table} 
\end{tabular}
\end{table}

\clearpage
\begin{table}[htbp!]
\centering
\caption{ WinGMAT/LinuxGMAT DeepSpace Test Case Comparison}
      \begin{tabular}{lcc}
      \hline\hline
          Test Case & Position Difference(m) & Velocity Difference(m/s) \\
         \hline
         %---New Row---%
         AllPlanets-0-0-0 & 0.009386010342 & 1.886732646e-009 \\
      \hline\hline
      \label{Table: DeepSpace WinGMAT-LinuxGMAT Table} 
\end{tabular}
\end{table}
\index{DeepSpace}
\begin{table}[htbp!]
\centering
\caption{ WinGMAT/LinuxGMAT EML2 Test Case Comparison}
      \begin{tabular}{lcc}
      \hline\hline
          Test Case & Position Difference(m) & Velocity Difference(m/s) \\
         \hline
         %---New Row---%
         AllPlanets-0-0-0 & 135364.0795 & 0.7390519353 \\
         %---New Row---%
         AllPlanets-0-0-SRP & 269258.1081 & 1.459047458 \\
         %---New Row---%
         EarthSunLuna-0-0-0 & 13791.8204 & 0.07140281919 \\
         %---New Row---%
         EarthSunLuna-JGM2-0-0 & 104302.9029 & 0.5713126463 \\
      \hline\hline
      \label{Table: EML2 WinGMAT-LinuxGMAT Table} 
\end{tabular}
\end{table}
\index{EML2}
\begin{table}[htbp!]
\centering
\caption{ WinGMAT/LinuxGMAT ESL2 Test Case Comparison}
      \begin{tabular}{lcc}
      \hline\hline
          Test Case & Position Difference(m) & Velocity Difference(m/s) \\
         \hline
         %---New Row---%
         AllPlanets-0-0-0 & 161033.6102 & 0.0878775284 \\
         %---New Row---%
         AllPlanets-0-0-SRP & 70022.39058 & 0.05565825527 \\
      \hline\hline
      \label{Table: ESL2 WinGMAT-LinuxGMAT Table} 
\end{tabular}
\end{table}
\index{ESL2}
\clearpage

\section{FF/STK Comparison}
\begin{table}[htbp!]
\centering
\caption{ FF/STK GEO Test Case Comparison}
      \begin{tabular}{lcc}
      \hline\hline
          Test Case & Position Difference(m) & Velocity Difference(m/s) \\
         \hline
         %---New Row---%
         AllPlanets-0-0-0 & 0.02441863704 & 2.331183059e-006 \\
         %---New Row---%
         EarthLuna-0-0-0 & 0.02435631273 & 2.242953735e-006 \\
         %---New Row---%
         EarthSunLuna-JGM2-HP-SRP & 2.852626845 & 0.0001210061727 \\
         %---New Row---%
         EarthSun-0-0-0 & 4.231055926e-005 & 8.493736892e-007 \\
         %---New Row---%
         Earth-0-0-0 & 6.836800608e-005 & 6.889863428e-007 \\
         %---New Row---%
         Earth-0-0-SRP & 2.899330477 & 0.0001245233005 \\
         %---New Row---%
         Earth-0-HP-0 & 6.836800608e-005 & 6.889863428e-007 \\
         %---New Row---%
         Earth-JGM2-0-0 & 0.02513336324 & 2.312005083e-006 \\
      \hline\hline
      \label{Table: GEO FF-STK Table} 
\end{tabular}
\end{table}
\index{GEO}\index{STK}\index{FreeFlyer}
\begin{table}[htbp!]
\centering
\caption{ FF/STK GPS Test Case Comparison}
      \begin{tabular}{lcc}
      \hline\hline
          Test Case & Position Difference(m) & Velocity Difference(m/s) \\
         \hline
         %---New Row---%
         AllPlanets-0-0-0 & 0.001903432726 & 1.053971199e-006 \\
         %---New Row---%
         EarthLuna-0-0-0 & 0.001896072444 & 9.455916149e-007 \\
         %---New Row---%
         EarthSunLuna-JGM2-HP-SRP & 0.6785588953 & 6.968633698e-005 \\
         %---New Row---%
         EarthSun-0-0-0 & 3.872950958e-006 & 8.316397374e-007 \\
         %---New Row---%
         Earth-0-0-0 & 8.164858288e-006 & 7.983006024e-007 \\
         %---New Row---%
         Earth-0-0-SRP & 0.6692784196 & 6.82998214e-005 \\
         %---New Row---%
         Earth-0-HP-0 & 8.164858288e-006 & 7.983006024e-007 \\
         %---New Row---%
         Earth-JGM2-0-0 & 0.01104063243 & 2.139211093e-006 \\
      \hline\hline
      \label{Table: GPS FF-STK Table} 
\end{tabular}
\end{table}
\index{GPS}\index{STK}\index{FreeFlyer}
\begin{table}[htbp!]
\centering
\caption{ FF/STK ISS Test Case Comparison}
      \begin{tabular}{lcc}
      \hline\hline
          Test Case & Position Difference(m) & Velocity Difference(m/s) \\
         \hline
         %---New Row---%
         AllPlanets-0-0-0 & 1.17832665e-005 & 8.274218823e-007 \\
         %---New Row---%
         EarthLuna-0-0-0 & 2.023778135e-005 & 8.422726373e-007 \\
         %---New Row---%
         EarthSunLuna-JGM2-HP-SRP & 3.468138188 & 0.003970584682 \\
         %---New Row---%
         EarthSun-0-0-0 & 2.067991746e-005 & 8.1799381e-007 \\
         %---New Row---%
         Earth-0-0-0 & 4.404163183e-005 & 8.186528591e-007 \\
         %---New Row---%
         Earth-0-0-SRP & 0.1637932094 & 0.0001561216566 \\
         %---New Row---%
         Earth-0-HP-0 & 3.213782043 & 0.003648358681 \\
         %---New Row---%
         Earth-JGM2-0-0 & 0.2076446785 & 0.0002404801574 \\
      \hline\hline
      \label{Table: ISS FF-STK Table} 
\end{tabular}
\end{table}
\index{ISS}\index{STK}\index{FreeFlyer}
\begin{table}[htbp!]
\centering
\caption{ FF/STK Molniya Test Case Comparison}
      \begin{tabular}{lcc}
      \hline\hline
          Test Case & Position Difference(m) & Velocity Difference(m/s) \\
         \hline
         %---New Row---%
         AllPlanets-0-0-0 & 0.01457313187 & 1.215436806e-005 \\
         %---New Row---%
         EarthLuna-0-0-0 & 0.01409431455 & 1.184376446e-005 \\
         %---New Row---%
         EarthSunLuna-JGM2-HP-SRP & 7.784661066 & 0.004770671087 \\
         %---New Row---%
         EarthSun-0-0-0 & 0.0006798546862 & 1.003660696e-006 \\
         %---New Row---%
         Earth-0-0-0 & 0.0003365057467 & 8.083469211e-007 \\
         %---New Row---%
         Earth-0-0-SRP & 2.16384821 & 0.001779529358 \\
         %---New Row---%
         Earth-0-HP-0 & 15.27354394 & 0.01277094127 \\
         %---New Row---%
         Earth-JGM2-0-0 & 3.063128789 & 0.002561623783 \\
      \hline\hline
      \label{Table: Molniya FF-STK Table} 
\end{tabular}
\end{table}
\index{Molniya}\index{STK}\index{FreeFlyer}
\begin{table}[htbp!]
\centering
\caption{ FF/STK SunSync Test Case Comparison}
      \begin{tabular}{lcc}
      \hline\hline
          Test Case & Position Difference(m) & Velocity Difference(m/s) \\
         \hline
         %---New Row---%
         AllPlanets-0-0-0 & 3.408772257e-005 & 8.692082437e-007 \\
         %---New Row---%
         EarthLuna-0-0-0 & 1.971678896e-005 & 8.200621923e-007 \\
         %---New Row---%
         EarthSunLuna-JGM2-HP-SRP & 0.7800584002 & 0.0008377097699 \\
         %---New Row---%
         EarthSun-0-0-0 & 1.533280696e-005 & 8.472050762e-007 \\
         %---New Row---%
         Earth-0-0-0 & 3.656123033e-005 & 8.215198955e-007 \\
         %---New Row---%
         Earth-0-0-SRP & 0.2052485046 & 0.0001828464597 \\
         %---New Row---%
         Earth-0-HP-0 & 0.6524853805 & 0.0007361489035 \\
         %---New Row---%
         Earth-JGM2-0-0 & 0.09463135069 & 0.0001055896904 \\
      \hline\hline
      \label{Table: SunSync FF-STK Table} 
\end{tabular}
\end{table}
\index{SunSync}\index{STK}\index{FreeFlyer}
\clearpage
\begin{table}[htbp!]
\centering
\caption{ FF/STK EML2 Test Case Comparison}
      \begin{tabular}{lcc}
      \hline\hline
          Test Case & Position Difference(m) & Velocity Difference(m/s) \\
         \hline
         %---New Row---%
         AllPlanets-0-0-0 & 209675.0986 & 1.14402647 \\
         %---New Row---%
         AllPlanets-0-0-SRP & 207842.3945 & 1.126072479 \\
         %---New Row---%
         EarthSunLuna-0-0-0 & 209277.3047 & 1.141861418 \\
         %---New Row---%
         EarthSunLuna-JGM2-0-0 & 209283.0825 & 1.141942183 \\
      \hline\hline
      \label{Table: EML2 FF-STK Table} 
\end{tabular}
\end{table}
\index{EML2}\index{STK}\index{FreeFlyer}
\begin{table}[htbp!]
\centering
\caption{ FF/STK ESL2 Test Case Comparison}
      \begin{tabular}{lcc}
      \hline\hline
          Test Case & Position Difference(m) & Velocity Difference(m/s) \\
         \hline
         %---New Row---%
         AllPlanets-0-0-0 & 5661795.485 & 3.114044517 \\
         %---New Row---%
         AllPlanets-0-0-SRP & 6881239.331 & 5.360316234 \\
      \hline\hline
      \label{Table: ESL2 FF-STK Table} 
\end{tabular}
\end{table}
\index{ESL2}\index{STK}\index{FreeFlyer}
