\begin{Script}\begin{verbatim}
%  Build 3, Test Case 1

% -------------------------------------------------------------------------
% --------------------------- Create Objects ------------------------------
% -------------------------------------------------------------------------

% Create Sat1 and define its orbit
Create Spacecraft Sat1;
GMAT Sat1.Epoch.TAIGregorian  = 09 Oct 2004 16:30:00.120;
GMAT Sat1.StateType = Keplerian;
GMAT Sat1.SMA      = 8000;
GMAT Sat1.ECC      = .01;
GMAT Sat1.INC      = 28.5;
GMAT Sat1.AOP      = 0;
GMAT Sat1.RAAN     = 90;
GMAT Sat1.TA       = 180;
GMAT Sat1.Cd       = 2.0;
GMAT Sat1.Cr       = 1.4;
GMAT Sat1.DragArea = 1;
GMAT Sat1.SRPArea  = 1;
GMAT Sat1.DryMass  = 100;

% Create Sat2 and define its orbit
Create Spacecraft Sat2;
GMAT Sat2.Epoch.TAIGregorian  = 09 Oct 2004 16:30:00.120;
GMAT Sat2.StateType = Keplerian;
GMAT Sat2.SMA     = 8000;
GMAT Sat2.ECC     = .01;
GMAT Sat2.INC     = 28.5;
GMAT Sat2.AOP     = 0;
GMAT Sat2.RAAN    = 90;
GMAT Sat2.TA      = 181;
GMAT Sat2.Cd      = 2.0;
GMAT Sat2.Cr      = 1.4;
GMAT Sat2.DragArea = 1;
GMAT Sat2.SRPArea  = 1;
GMAT Sat2.DryMass  = 100;

% Create Sat3 and define its orbit
Create Spacecraft Sat3;
GMAT Sat3.Epoch.TAIGregorian  = 09 Oct 2004 16:30:00.120;
GMAT Sat3.StateType = Keplerian;
GMAT Sat3.SMA   = 8000;
GMAT Sat3.ECC     = .01;
GMAT Sat3.INC     = 28.5;
GMAT Sat3.AOP   = 0;
GMAT Sat3.RAAN  = 90;
GMAT Sat3.TA    = 182;
GMAT Sat3.Cd    = 2.0;
GMAT Sat3.Cr    = 1.4;
GMAT Sat3.DragArea = 1;
GMAT Sat3.SRPArea  = 1;
GMAT Sat3.DryMass  = 100;

%  Define Force Model with JR drag
Create ForceModel LowEarth_JR;
GMAT LowEarth_JR.PrimaryBodies      = {Earth};
GMAT LowEarth_JR.PointMasses        = {Sun, Luna, Jupiter};
GMAT LowEarth_JR.Drag               = JacchiaRoberts;
GMAT LowEarth_JR.Drag.F107       = 100;
GMAT LowEarth_JR.Drag.F107A      = 120;
GMAT LowEarth_JR.Drag.MagneticIndex = 20;
GMAT LowEarth_JR.Gravity.Earth.Model  = JGM2;
GMAT LowEarth_JR.Gravity.Earth.Degree = 20;
GMAT LowEarth_JR.Gravity.Earth.Order  = 20;

%  Define Force Model with no drag
Create ForceModel HighEarth;
GMAT HighEarth.PrimaryBodies        = {Earth};
GMAT HighEarth.PointMasses          = {Sun, Luna, Jupiter};
GMAT HighEarth.Drag                 = None;
GMAT HighEarth.Gravity.Earth.Model  = JGM2;
GMAT HighEarth.Gravity.Earth.Degree = 20;
GMAT HighEarth.Gravity.Earth.Order  = 20;
GMAT HighEarth.SRP                  = On;

% Create Propagator
Create Propagator RKV_LowEarth_JR;
GMAT RKV_LowEarth_JR.Type     = RungeKutta89;
GMAT RKV_LowEarth_JR.MinStep  = .01;
GMAT RKV_LowEarth_JR.MaxStep  = 900;
GMAT RKV_LowEarth_JR.FM       = LowEarth_JR;

% Create Propagator
Create Propagator RKF_HighEarth;
GMAT RKF_HighEarth.Type    = RungeKuttaFehlberg56;
GMAT RKF_HighEarth.MinStep = .01;
GMAT RKF_HighEarth.MaxStep = 900;
GMAT RKF_HighEarth.FM      = HighEarth;

%  Create XYPlot
Create XYPlot Sat1XYZvsTime;
GMAT Sat1XYZvsTime.TargetStatus = Off;
GMAT Sat1XYZvsTime.IndVar   = Sat1.ElapsedDays;
GMAT Sat1XYZvsTime.Add = Sat1.X;
GMAT Sat1XYZvsTime.Add = Sat1.Y;
GMAT Sat1XYZvsTime.Add = Sat1.Z;

%  Create OpenGL Plot
Create OpenGLPlot SatOpenGL;
GMAT SatOpenGL.TargetStatus = Off;
GMAT SatOpenGL.Add = Sat1;
GMAT SatOpenGL.Add = Sat2;
GMAT SatOpenGL.Add = Sat3;


% -------------------------------------------------------------------------
% -------------------- Begin Mission Sequence -----------------------------
% -------------------------------------------------------------------------

Toggle Sat1XYZvsTime On;
Toggle SatOpenGL On;

%  First Propagation Sequence
Propagate Synchronized RKV_LowEarth_JR(Sat1,{Sat1.Periapsis}) RKF_HighEarth(Sat2,Sat3,{Sat2.TA = 155});

%  Second Propagation Sequence
Propagate RKF_HighEarth(Sat1,Sat2,Sat3,{Sat2.MA = 0});

\end{verbatim}\end{Script}
