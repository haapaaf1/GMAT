\chapter{ControlFlow}
\label{Ch:ControlFlow}

The Control Flow tests were designed to verify that the control flow
commands (If, While, and For) function as expected. There are
scripts that test the control flow commands by themselves, nested,
and using different user defined parameters, such as arrays,
numbers, variables, and spacecraft parameters. Each test script was
designed to store flags that contained details of the command
execution for each test case and reported to a text file.

Due to the layout of the report, a Matlab script can easily create a
pass and fail table based on the values of the flag variables. In
the report output the main columns to pay attention to are the ranOK
and ansFlag columns. The ansFlag variable tells us if there was an
incorrect control flow execution (ansFlag=-99), correct control flow
execution (ansFlag=1), or the control flow didn't get executed
(ansFlag=1). The ranOK variable tells us if each test case inside
the test script ran correctly (ranOK = 1) or not (ranOK = 0). The
only scripts that do not generate the ranOK column are the
IfLoopTest\#\#\_\_\#\#, IfLoopTest\#\#, and
IfIfLoopTest\#\#\_\_\#\#.m, because the ansFlag column is
sufficient.

\section{Test Case Results}
The results in Table~\ref{Table: LoopResults} display the Pass and
Fail outcome of each Control Flow test script.

\begin{table}[htbp!]
\centering
\caption{ Loop Test Case Results}
      \begin{tabular}{lcc}
      \hline\hline
          TestName & Pass/Fail & Failed/TotalTests \\
         \hline
         %---New Row---%
         For & Pass & 0/15 \\
         %---New Row---%
         ForFor & Pass & 0/9 \\
         %---New Row---%
         ForIf41-14 & Pass & 0/20 \\
         %---New Row---%
         ForIf42-24 & Pass & 0/21 \\
         %---New Row---%
         ForIf43-34 & Pass & 0/20 \\
         %---New Row---%
         ForIf51-15 & Pass & 0/16 \\
         %---New Row---%
         ForIf52-25 & Pass & 0/20 \\
         %---New Row---%
         ForIf53-35 & Pass & 0/20 \\
         %---New Row---%
         ForWhile42 & Pass & 0/8 \\
         %---New Row---%
         If11 & Pass & 0/9 \\
         %---New Row---%
         If12-21 & Pass & 0/18 \\
         %---New Row---%
         If22 & Pass & 0/9 \\
         %---New Row---%
         If32-23 & Pass & 0/18 \\
         %---New Row---%
         If33 & Pass & 0/9 \\
         %---New Row---%
         If42-24 & Pass & 0/16 \\
         %---New Row---%
         If44 & Pass & 0/8 \\
         %---New Row---%
         If52-25 & Pass & 0/16 \\
         %---New Row---%
         If55 & Pass & 0/8 \\
         %---New Row---%
         IfFor & Pass & 0/9 \\
         %---New Row---%
         IfIf41-14 & Pass & 0/16 \\
         %---New Row---%
         IfIf42-24 & Pass & 0/16 \\
         %---New Row---%
         IfIf43-34 & Pass & 0/20 \\
         %---New Row---%
         IfIf51-15 & Pass & 0/16 \\
         %---New Row---%
         IfIf52-25 & Pass & 0/16 \\
         %---New Row---%
         IfIf53-35 & Pass & 0/16 \\
         %---New Row---%
         IfWhile & Pass & 0/8 \\
         %---New Row---%
         While42-24 & Pass & 0/16 \\
         %---New Row---%
         While43-34 & Pass & 0/16 \\
         %---New Row---%
         While52-25 & Pass & 0/16 \\
         %---New Row---%
         While53-35 & Pass & 0/16 \\
         %---New Row---%
         WhileFor & Pass & 0/9 \\
         %---New Row---%
         WhileIf41-14 & Pass & 0/16 \\
         %---New Row---%
         WhileIf42-24 & Pass & 0/16 \\
         %---New Row---%
         WhileIf43-34 & Pass & 0/16 \\
         %---New Row---%
         WhileIf51-15 & Pass & 0/17 \\
         %---New Row---%
         WhileIf52-25 & Pass & 0/16 \\
         %---New Row---%
         WhileIf53-35 & Pass & 0/16 \\
         %---New Row---%
         WhileTarget & Pass & 0/1 \\
         %---New Row---%
         WhileWhile42-24 & Pass & 0/16 \\
      \hline\hline
      \label{Table: LoopResults} 
\end{tabular}
\end{table}

