\begin{table}[htbp!]
\centering
      \begin{tabular}{|p{1.0 in} |p{5.0 in} |}
         \hline
          \rowcolor[rgb]{0.8,0.8,0.8}  Name & STC-26 Attitude conversion in the spacecraft attitude dialog box\\
         \hline
         Requirements & FRR-3.3\\ \hline
         Summary & This test case represents $n(n-1)$ tests where $n$ is the number of attitude representations
         supported as input types in GMAT.  Each test case is designated a unique number.  For example,
         STC-26.32 tests conversion from a 231 to a 232 Euler angle sequence.  The procedures described below
         must be performed for each test case in the table below.   \\ \hline
         PreConditions & To run this test you need to load BS-1 and have data defined in TD-8 available.\\ \hline
         Steps &
         %------- The test data is here
          \begin{compactenum}
             \item Select subtest number. ( STC-26.32, for example)
             \item Create a new spacecraft.
             \item Open the dialog box for the new spacecraft.
             \item Click on the Attitude tab.
             \item Change the AttitudeStateType to the format defined in the first column of
                   the row containing the test case ID.  (Euler Angles, for STC-26.32)
             \item If the AttitudeStateType is EulerAngles, change the EulerAngleSequence to the 
                   sequence defined in the first column of
                   the row containing the test case ID.  (231, for STC-26.32)
             \item Enter the attitude state for the test ID using the data from TD-8.
             \item Hit Apply.
             \item Change the AttitudeStateType to the format defined in the first row of the column containing  the test case Id. (Euler Angles, for STC-26.32)
             \item If the AttitudeStateType is EulerAngles, Change the EulerAngleSequence to the format defined in the first row of the column containing  the test case Id. (232, for STC-26.32)
             \item Verify that the new epoch exactly matches the value for that format given in TD-8.
             \item Compare the new Euler Angles to those in TD-8 for the new attitude representation.  The values should agree to at least 13 significant figures.
          \end{compactenum}
          \vspace{.1 in}
          \begin{centering} \footnotesize
          \begin{tabular}{|l|c|c|c|c|c|c|c|c|c|c|c|c|c|c|c|c|c|}
          \hline
             & {\begin{sideways}\parbox{1.0cm}{$\mathbf{q}$}\end{sideways}} &
             {\begin{sideways}\parbox{1.0cm}{123}\end{sideways}} &
             {\begin{sideways}\parbox{1.0cm}{231}\end{sideways}} &
             {\begin{sideways}\parbox{1.0cm}{312}\end{sideways}} &
             {\begin{sideways}\parbox{1.0cm}{132}\end{sideways}}  &
             {\begin{sideways}\parbox{1.0cm}{321}\end{sideways}}  &
             {\begin{sideways}\parbox{1.0cm}{213}\end{sideways}}  &
             {\begin{sideways}\parbox{1.0cm}{121}\end{sideways}} &
             {\begin{sideways}\parbox{1.0cm}{232}\end{sideways}} &
             {\begin{sideways}\parbox{1.0cm}{313}\end{sideways}} &
             {\begin{sideways}\parbox{1.0cm}{131}\end{sideways}} &
             {\begin{sideways}\parbox{1.0cm}{323}\end{sideways}} &
             {\begin{sideways}\parbox{1.0cm}{212}\end{sideways}}  \\ \hline
             $\mathbf{q}$ & X & 1 & 2 & 3 & 4 & 5 & 6 & 7 & 8 & 9 & 10 & 11 & 12 \\ \hline
             123 & 13 & X & 14 & 15 & 16 & 17 & 18 & 19 & 20 & 21 & 22 & 23 & 24 \\ \hline
             231 & 25 & 26 & X & 27 & 28 & 29 & 30 & 31 & 32 & 33 & 34 &35 & 36 \\ \hline
             312 & 37 & 38 & 39 & X & 40 & 41 & 42 & 43 & 44 & 45 & 46 & 47 & 48 \\ \hline
             132 & 49 & 50 & 51 & 52 & X & 53 & 54 & 55 & 56 & 57 & 58 & 59 & 60 \\ \hline
             321 & 61 & 62 & 63 & 64 & 65 & X & 66 & 67 & 68 & 69 & 70 & 71 & 72 \\ \hline
             213 & 73 & 74 & 75 & 76 & 77 & 78 & X & 79 & 80 & 81 & 82 & 83 & 84  \\ \hline
             121 & 85 & 86 & 87 & 88 & 89 & 90 & 91 & X & 92 & 93 & 94 & 95 & 96  \\ \hline
             232 & 97 & 98 & 99 & 100 & 101 & 102 & 103 & 104 & X & 105 & 106 & 107 & 108  \\ \hline
             313 & 109 & 110 & 111 & 112 & 113 & 114 & 115 & 116  & X  & 117 & 118 & 119 & 120  \\ \hline
             131 & 121 & 122 & 123 & 124 & 125 & 126 & 127 & 128 & 129 & X & 130 & 131 & 132  \\ \hline
             323 & 133 & 134 & 135 & 136 & 137 & 138 & 139 & 140 & 141 & 142 & 143 & X & 144  \\ \hline
             212 & 145 & 146 & 146 & 148 & & 149 & 150 & 151 & 152 & 153 & 154 & 155 & X  \\ \hline
          \end{tabular}
          \end{centering} \vspace{0.1 in}\\
         %------- The test data is here
         \hline
         Expected Results & The expected numeric results are described above and in TD-8.\\
      \hline
\end{tabular}
   \label{Table:STC-26}
   \caption{STC-26 Attitude conversion in the spacecraft attitude dialog box}
    \index{Test Data!STC-26}
\end{table} 