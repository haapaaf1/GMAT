\chapter{Acceptance Test Plan Overview} \label{Ch:AcceptPlanIntro}

\section{GMAT Introduction}
The information presented in this Acceptance Test Plan document
shows the current status of the General Mission Analysis Tool
(GMAT). GMAT is a software system developed by NASA Goddard Space
Flight Center (GSFC) in collaboration with the private sector. The
GMAT development team continuously performs acceptance tests in
order to verify that the software continues to operate properly
after updates are made. The GMAT Development team consists of
NASA/GSFC Code 583 software developers, NASA/GSFC Code 595 analysts,
and contractors of varying professions.

GMAT was developed to provide a development approach that maintains
involvement from the private sector and academia, encourages
collaborative funding from multiple government agencies and the
private sector, and promotes the transfer of technology from
government funded research to the private sector.

GMAT contains many capabilities, such as integrated formation flying
modeling and MATLAB\index{MATLAB} compatibility. The propagation
capabilities in GMAT allow for fully coupled dynamics modeling of
multiple spacecraft, in any flight regime. Other capabilities in
GMAT include: user definable coordinate systems, 3-D graphics in any
coordinate system GMAT can calculate, 2-D plots, branch commands,
solvers, optimizers, GMAT functions, planetary ephemeris sources
including DE405\index{DE405}, DE200\index{DE200}, SLP\index{SLP} and
analytic models, script events, impulsive and finite maneuver
models, and many more.

GMAT runs on Windows, Mac, and Linux platforms. Both the Graphical
User Interface (GUI) and the GMAT engine were built and tested on
all of the mentioned platforms. GMAT was designed for intuitive use
from both the GUI and with an importable script language similar to
that of MATLAB\index{MATLAB}.

\section{Testing Methodology}
\subsubsection{Purpose}
GMAT needs to undergo a series of rigorous tests to validate the
numerical implementations of its models and establish a set of
acceptable performance times. The 595 analysts created the
acceptance test plan to achieve this goal by comparing GMAT with
flight-operational reference software packages and documenting the
results. Results can be reproduced with the initial conditions and
software setups presented in this document.

\subsubsection{Reference software}
For this comparative study to have merit, GMAT was tested against
reliable, trustworthy, and flight operational programs, such as
STK-HPOP\index{STK!HPOP},STK-Astrogator\index{STK!Astrogator},
Free-Flyer\index{FreeFlyer}, Swingby\index{Swingby}, and previous
GMAT Builds that were comparable to the aforementioned programs. To
achieve accurate comparison results, each program was compared with
equivalent, or close to equivalent, test case setups.

\subsubsection{Testing Categories}
The Acceptance Test Plan divides into the following testing
categories: Propagation\index{Propagation}, Calculation Parameters
[Central body(Cb)\index{Cb} and Coordinate System(CS)\index{CS}
dependent], Integrators\index{Integrators}, Libration
Points\index{Libration Points}, Stopping Conditions\index{Stopping
Conditions}, Delta V\index{Delta V}, and
Performance\index{Performance}.

\subsubsection{Scripts Used}
MATLAB\index{MATLAB} scripts were created to make comparisons
between GMAT Builds and the reference software. The majority of the
comparisons involved taking the difference of the data and
extracting the maximum absolute difference observed over the
propagation duration. Scripts were also created to compare
performance times for individual GMAT test cases to the reference
software. The scripts created are as followed:
Comparison\_Tool1\_Tool2\_PV.m, Comparison\_Tool1\_Tool2\_CS.m,
Comparison\_Tool1\_Tool2\_Cb.m, Comparison\_Integ.m, TimeComparo.m,
BuildRun\_Script\_GMAT.m,
Comparison\_Tool1\_Tool2\_Libr.m, Comparison\_StopCond, and \\
STK\_Repropagate.m.

The user of the semi-automated scripts provides input when
requested, in order to perform the script's core functions. For
example, a user that wants to see the position and velocity
differences between STK\index{STK} and GMAT would select a few
choices from a menu. Next, the script would generate the comparisons
based on the report data available. The semi-automation scripts
adhere to the naming conventions outlined in their relevant testing
category chapter.

Most of the scripts generate output in at least one of the following
formats: ASCII, LaTex\index{LaTex}, MATLAB\index{MATLAB} .mat, or
Excel .xls files. The report files are in an ASCII space delimited
format and contain the different test case parameters outputted
after propagation. The LaTex\index{LaTex} files contain the
comparison data between two programs and provide an easy way to
include that data into a PDF document. The .mat and .xls file are
two other methods used to save the comparison data that proved
useful from the software development team.

The details of each script and how to use them are outlined in the
relevant Testing Category section and/or the Comparison Scripts
Guide section, located in Appendix ~\ref{Ch:CompScripts}.

\subsection{Propagation}\index{Propagation}
The propagation test cases account for various orbits about Earth,
as well as other celestial bodies. The main propagation parameters
to monitor for differences are the position and velocity vectors.
The following script was generated to perform the
comparisons for this category:\\
Comparison\_Tool1\_Tool2\_CS.m

See the Propagators section (Chapter~\ref{Ch:Propagators}) for more
detail and comparison results.\\

\subsection{Calculation Parameters}
The calculation parameter test cases verify the internal
calculations used to output the various parameters presented in the
list below. This section consists of two subsections: Coordinate
System(CS)\index{CS} and Central Body(Cb)\index{Cb} dependent
parameters. The following scripts
were generated to perform the comparisons for this testing category: \\
Comparison\_Tool1\_Tool2\_CS.m \& Comparison\_Tool1\_Tool2\_Cb.m\\

\begin{itemize}
    \item Coordinate Systems
        \begin{itemize}
            \item Earth Fixed
            \item Earth Mean J2000 Equator (MJ2000Eq)
            \item Earth Mean J2000 Ecliptic (MJ2000Ec)
            \item Earth Mean of Date Equator (MODEq)
            \item Earth Mean of Date Ecliptic (MODEc)
            \item Earth True of Date Equator (TODEq)
            \item Earth True of Date Ecliptic (TODEq)
            \item Earth Geocentric Solar Ecliptic (GSE)
            \item Earth Geocentric Solar Magnetic (GSM)
            \item Mars Fixed
            \item Mars MJ2000Eq
            \item Mars MJ2000Ec
            \item Mercury Fixed
            \item Mercury MJ2000Eq
            \item Mercury MJ2000Ec
            \item Moon Fixed
            \item Moon MJ2000Eq
            \item Moon MJ2000Ec
            \item Neptune Fixed
            \item Neptune MJ2000Eq
            \item Neptune MJ2000Ec
            \item Pluto Fixed
            \item Pluto MJ2000Eq
            \item Pluto MJ2000Ec
            \item Saturn Fixed
            \item Saturn MJ2000Eq
            \item Saturn MJ2000Ec
            \item Uranus Fixed
            \item Uranus MJ2000Eq
            \item Uranus MJ2000Ec
            \item Venus Fixed
            \item Venus MJ2000Eq
            \item Venus MJ2000Ec
        \end{itemize}
    \item Coordinate System Parameters
        \begin{itemize}
            \item Position (X,Y,Z)
            \item Velocity(X,Y,Z)
            \item Magnitude of Velocity
            \item Right Ascension of Velocity
            \item Specific Angular Momentum
            \item Argument of Periapsis
            \item Declination
            \item Declination of Velocity
            \item Inclination
            \item Right Ascension
            \item Right Ascension of Ascending Node
        \end{itemize}
    \item Central Body Parameters
        \begin{itemize}
            \item Altitude
            \item Beta Angle
            \item C3 Energy
            \item Eccentricity
            \item Latitude
            \item Longitude
            \item Specific Angular Momentum
            \item Mean Anomaly
            \item Mean Motion
            \item Period
            \item Apoapsis Radius
            \item Perigee Radius
            \item Position Magnitude
            \item Semi-major Axis
            \item True Anomaly
            \item Semilatus Rectum
            \item Apoapsis Velocity
            \item Periapsis Velocity
            \item Greenwich Hour Angle
            \item Local Sidereal Time
        \end{itemize}
\end{itemize}

See the Calculation Parameters Section
(Chapter~\ref{Ch:CalcParameters}) for more detail and comparison
results.

\subsection{Integrators}\index{Integrators}
The integrator test cases isolate the differences that would occur
when changing the integrators for the same orbit. The following
script was generated to perform the test case comparisons for this category:\\
Comparison\_Integ.m

\begin{itemize}
    \item RungaKutta(RKV) 8(9)
    \item DormandElMikkawyPrince(RKN) 6(8)
    \item RungeKuttaFehlberg(RKF) 5(6)
    \item PrinceDormand(PD) 4(5)
    \item PrinceDormand(PD) 7(8)
    \item BulirschStoer(BS)
    \item AdamsBashforthMoulton(ABM)
\end{itemize}

See the Integrators\index{Integrators} Section
(Chapter~\ref{Ch:Integrators}) for more detail and comparison
results.

\subsection{Stopping Conditions}\index{Stopping Conditions}
The stopping condition test cases determine how effective GMAT is at
stopping satellite propagation on certain conditions. The following
script was created to
perform the test case comparisons for this category: \\
Comparison\_StopCond.m

The stopping conditions tested are as followed:
\begin{itemize}
    \item Epoch (A1 Modified Julian Date)
    \item Apoapsis
    \item Elapsed Days
    \item Mean Anomaly
    \item Periapsis
    \item Elapsed Seconds
    \item True Anomaly
    \item XY Plane Intersection
    \item XZ Plane Intersection
    \item YZ Plane Intersection
\end{itemize}

See the Stopping Conditions\index{Stopping Conditions} Section
(Chapter~\ref{Ch:StopCond}) for more detail and comparison results.

\subsection{Libration Point}\index{Libration Points}

The libration point test cases create data about the location of
several libration points. Current and future satellite missions use
libration points as part of their mission architecture. It is
important to have accurate data for these libration points. The
following script was created to perform the test case
comparisons for this category: \\
Comparison\_Tool1\_Tool2\_Libr.m

See the Libration Point\index{Libration Point} Section
(Chapter~\ref{Ch:LibPoint}) for more detail and comparison results.

\subsection{Delta V}\index{Delta V}

The delta v test cases determine the effectiveness of the delta v
capabilities built into GMAT. When thruster burns are added to the
mission sequence it is important that they are added correctly. The
following script was created to perform the test case comparisons
for this category: \\
Comparison\_DeltaV.m

See the Delta V\index{Delta V} Section (Chapter~\ref{Ch:DeltaV}) for
more detail and comparison results.

\subsection{Performance}\index{Performance}

The performance test cases generate performance time data for later
comparison between GMAT and the reference software packages.
Numerical calculation accuracy is important, but the amount of
computing time it takes for the software to run is equally as
important. We extracted several test cases from previous sections
and ran them on the reference software packages, in order to check
to make sure GMAT can perform just as good or better.

See the Performance Section\index{Performance}
(Chapter~\ref{Ch:Performance}) for more detail and comparison
results.

\subsection{Control Flow}\index{Control Flow}

The control flow tests generate report data that easily allows a
Matlab script to produce a table of Pass and Fail cases.The
following script was created to generate the Pass/Fail table
for this category: \\
LoopTestSummary.m

See the Control Flow Section\index{Control Flow}
(Chapter~\ref{Ch:ControlFlow}) for more detail and results.
