\chapter{Comparison Scripts Guide}
\label{Ch:CompScripts}

Using specific naming conventions, outlined in this acceptance test
plan, and a folder architecture, highlighted below for the test
cases, several semi-automated scripts were generated to compare all
of the GMAT test case results with other tools. Most of these
scripts have the ability to also compare results of older versions
of GMAT.

\section{Folder Architecture}\label{Sec:FolderArch}
The folder architecture for the files needed for the comparison
scripts is presented below:

\begin{itemize}
    \itemindent -0.1in
    \item GMAT\_RegSetup/
        \itemindent 0.1in
        \item output/AcceptTest/CompareResults/
            \itemindent 0.3in
            \item $[$Tool1$]$\_$[$Tool2$]$
        \itemindent 0.1in
        \item input/AcceptTest/
        \item output/AcceptTest/$[$Tool$]$\_reports/
        \item output/AcceptTest/Good\_reports/
            \itemindent 0.3in
            \item FF
            \item STK
    \itemindent -0.1in
    \item GMATDocuments/
        \itemindent 0.1in
        \item AcceptTest
\end{itemize}

\section{Install Instructions}
\begin{enumerate}
  \item Copy the GMAT\_RegSetup and GMATDocuments folders to the same location on your hard drive.
  \item Check to make sure the folders listed in Section~\ref{Sec:FolderArch}
  \item Make sure the GMAT executable folder has a Matlab folder with the latest GMAT commands
and keywords
  \item Open Matlab
  \item Set the path in Matlab to include the GMAT matlab folder
  \item Open GMAT and start the Matlab server
  \item Make sure Excel is closed
  \item Set current directory to the main\\
  GMAT\_results/GMAT\_scripts folder and then
  type BuildCompare\_GMATteam in the command window or
  \item Open one of the following files and run (F5 in Windows):\\
\emph{Click ok to change the current Matlab directory if prompted}
    \begin{enumerate}
    \item BuildRun\_Script\_GMAT.m
    \item Comparison\_Tool1\_Tool2\_PV.m
    \item Comparison\_Tool1\_Tool2\_CS.m
    \item Comparison\_Tool1\_Tool2\_Cb.m
    \item Comparison\_Tool1\_Tool2\_Libr.m
    \item Comparison\_Integ.m
    \item Comparison\_DeltaV.m
    \item Comparison\_StopCond.m
    \item TimeComparo.m
    \item LoopTestSummary.m
    \end{enumerate}
\end{enumerate}

\section{Warnings/Script Hints}
\begin{itemize}
  \item The following scripts were not designed for the user to hit one
  button, run multiple calculations, and output data without user
  interaction. User interaction is necessary in all of these scripts.
  \item The $[$ $]$ notation indicates multiple words can be used.
  For example,[Tool] means replace the bracketed expression with
  words such as FF,STK, and GMAT.
  \item As of November 2005, the Excel data created by the comparison
  scripts will be saved in one file name $[$Tool1$]$\_$[$Tool2$]$\_Results\_$[$DD-MMM-YYYY$]$.xls
  in their respective $[$Tool1$]$\_$[$Tool2$]$ folder in the CompareResults folder.
  \item Be careful adding MATLAB .m files that contain the text GMAT
  in their file name. These scripts use the text GMAT as an indicator
  to know if the file is a GMAT compatible script.
  \item The adherence to the naming conventions are very strict in these
  scripts. Make sure when adding reports, scripts, or folders to use
  case sensitive filenames that agree with the naming conventions
  at all times.
  \item Several output formats are overwritten when running each script. The
  Excel documents are saved with the current date as part of the
  filename. If the filename exists it will be replaced. The same
  overwriting process occurs with the Matlab .mat files.
  \item In order to compare old GMAT Builds to one another, a new folder
  must be created with the date of the GMAT Build. For example, the
  $[$Month$]$ $[$Day$]$ build performed well so after running the scripts that
  generate all the comparison data for the $[$Month$]$ $[$Day$]$ Build, a new
  folder must be created. The folder can be named YYMMDDGMAT\_reports
  with the appropriate GMAT Build Year Month and Date replacing
  YYMMDD. Now simply copy the contents of GMAT\_reports into
  YYMMDDGMAT\_reports and the data can be used to compare future build
  of GMAT to one another.
  \item All the .mat files, except for the TimeComparo files, are formatted
  in a similar way. They contain the following variables mat\_Tool11,
  mat\_Tool21, mat\_header, maxDiffs, and diffMat\_Tool1\_Tool2 or
  normMat\_Tool1\_Tool2.\\

  The mat\_header variable contains what parameters each column
  represents for the other variables The mat\_Tool11 and mat\_Tool21
  variables contain the report file data minus any headers. Tool1 is
  the alphabetical first tool selected and Tool2 is the other tool
  selected
  \item Common mistakes made with these scripts:
    \begin{itemize}
        \item Adding report files and not following the naming convention
        (i.e. FF\_ISS\_Earth\_0\_0\_0 was misnamed as FF\_ISS\_Earth\_0\_0).
        \item Outputting other tools and not following the proper ordering of
        parameters for comparison
        \item Report files were outputted in the
        wrong time interval increment
        \item  Report files were outputted without enough numerical precision.
        GMAT outputs data at a fixed width of 12 numerical characters (default).
        Other programs should be the same or better.
    \end{itemize}
\end{itemize}

\section{BuildRun\_Script\_GMAT.m script}
Win compatible / Mac \& Linux ? \\
\subsection{Purpose}
This script was designed to send multiple GMAT scripts from Matlab
to GMAT to be built and ran. In early versions of GMAT, there was no
capability to run multiple scripts, but since current versions of
GMAT contain this capability this script is not as vital in the
Acceptance test plan.

Its secondary purpose is to record the individual time elapsed to
run each test case and output the results to a .mat file for later
use.
\subsection{Inputs}
\begin{itemize}
    \item All .m files located in the GMAT\_RegSetup/input/AcceptTest/ folder that follow the
    GMAT comparison naming convention described in
    Sections~\ref{nameConvProp},~\ref{nameConvCb},~\ref{nameConvCS},
    and~\ref{nameConvInteg}
\end{itemize}
\subsection{Outputs}
\begin{itemize}
    \item When choosing to build multiple scripts and the all command is
    used a mat file is created, which keeps a log of the time it takes
    to run each case. The script also displays the case name and the
    time it takes for GMAT to run the file in Matlab's command window.
    \item Any other output data is dependent on the GMAT script.
\end{itemize}
\subsection{Script Algorithm}

\section{Comparison\_Tool1\_Tool2\_PV.m}
Win/Mac/Linux compatible
\subsection{Purpose}
This script is used to perform the position and velocity comparisons
needed for the Propagator Section (~\ref{Ch:Propagators}) of the
Acceptance Test Plan. This script takes the normalized position and
velocity vector differences between the two selected programs.

\subsection{Inputs}
\begin{itemize}
    \item Folder to search for files$:$ $[Root Folder]/[$Tool$]$\_reports. \\
    (Refer to Section~\ref{nameConvProp} for the naming convention
    of these report files.)
    \item The report files must be formatted the same way. The first
    column is time (Mod. Julian Date), second-fourth columns are the position vector
    components $(x,y,z)$, and fifth-seventh columns are the velocity
    vector components $(Vx,Vy,Vz)$. The data must be separated by
    spaces.
    \item Currently the location of the report's first row of numerical data is coded into the script.
    FreeFlyer is the fourth row, STK is the seventh row, GMAT is the
    first row, OD toolbox is the first row, and any other tools added
    will automatically search the first row.
\end{itemize}
\subsection{Outputs}
\begin{itemize}
    \item Comparison data is displayed in MATLAB's command window
    for all test cases.
    \item Excel documents with comparison data and pass/fail
    information.
    \item MATLAB .mat files with comparison data.
    \item Latex documents with comparison data.
\end{itemize}
\subsection{Script Algorithm}
\begin{itemize}
    \item Display welcome message
    \item Display menu for Tool1 options based on [Tool]\_reports folder
    \item Wait for user to choose tool from menu
        \subitem Implement error system for incorrect choice
    \item Display menu for Tool2 options based on [Tool]\_reports folder
    \item Wait for user to choose tool from menu
        \subitem Implement error system for incorrect choice
    \item Alphabetize Tool1 and Tool2 for naming purposes and rename
    if necessary
    \item Store the row location of the first instance of numerical
    data for both Tools' report files
    \item Initialize variables and folder locations
    \item Display a menu of Tool1 $*$.report files
        \subitem Generate error report if no $*$.report files are located
    in [Tool1]\_report folder
    \item Wait for user to choose report comparison option
        \subitem Implement error system for incorrect choice
        \subitem Open Excel Connection if compare all choice selected
    \item Display filename, position difference, and velocity difference header
    \item Begin Loop. Loop once for single comparison and several
    times for comparing all files.
    \begin{itemize}
        \item Check the Tool2 folder for the same report
            \subitem Display error message if no report found
        \item Continue if match found or exit loop if no match found
        \item Read both output files and save the data to different matrices
        \item Check to see if the row sizes are the same in
        both matrices
            \subitem Display error if row sizes do not match
        \item Take difference of both Tools report data
        \item Normalize the results based on position and velocity
        \item Determine the maximum normalized position and velocity
        difference
        \item Store propagation duration of the test cases in a
        variable
        \item Add acceptable differences values for Excel output
        \item Save comparison data to .mat file
        \begin{itemize}
            \item If compare all reports chosen, format data for
            output to Latex
            \item Use BasicLatexTable script to save data to LaTex
            file
            \item Save comparison results, acceptance errors, and
            duration to Excel
        \end{itemize}
        \item Close Excel connection if open.
    \end{itemize}
    \item End Loop. Allow user to rerun script
\end{itemize}

\section{Comparison\_Tool1\_Tool2\_CS.m}
Win/Mac/Linux compatible
\subsection{Purpose}
This script is used to perform the coordinate system dependent
comparisons needed for the Calculation Parameters Section
(~\ref{Ch:CalcParameters}) of the Acceptance Test Plan. The
comparison involves taking the maximum absolute value of the
differences of the variables listed in the Inputs section of this
help guide.
\subsection{Inputs}
\begin{itemize}
    \item Folder to search for files$:$ $[Root Folder]/[$Tool$]$\_reports. \\
    (Refer to Section~\ref{nameConvCS} for the naming convention
    of these report files.)
    \item The report files must be formatted the same way. Time, $[X,Y,and Z]$
    Position(km), $[X,Y,and Z]$ Velocity(km/sec), Mag. of
    Velocity(km/sec), Right Ascension of Velocity(deg), $[X,Y,and Z]$
    RxV-Specific Angular Momentum($km^2/sec$), Arg. of Perigee(deg),
    Declination(deg), Declination of Velocity(deg), Inclination(deg),
    Right Ascension(deg), Right Ascension of Ascending Node(deg)
    \item Currently the location of the reports first row of numerical
    data is coded into the script. FreeFlyer is the fourth row, STK is
    the seventh row, GMAT is the second row, OD toolbox is the first
    row, and any other tools added will automatically search the first
    row.
\end{itemize}
\subsection{Outputs}
\begin{itemize}
    \item Excel documents with comparison data
    \item MATLAB .mat files with comparison data
    \item Latex documents with comparison data
\end{itemize}
\subsection{Script Algorithm}
\begin{itemize}
    \item Display welcome message
    \item Display menu for Tool1 options based on [Tool]\_reports folder
    \item Wait for user to choose tool from menu
        \subitem Implement error system for incorrect choice
    \item Display menu for Tool2 options based on [Tool]\_reports folder
    \item Wait for user to choose tool from menu
        \subitem Implement error system for incorrect choice
    \item Alphabetize Tool1 and Tool2 for naming purposes and rename
    if necessary
    \item Store the row location of the first instance of numerical
    data for both Tools' report files
    \item Initialize variables and folder locations
    \item Display a menu of Tool1 $*$.report files
        \subitem Generate error report if no $*$.report files are located
    in [Tool1]\_report folder
    \item Wait for user to choose report comparison option
        \subitem Implement error system for incorrect choice
        \subitem Open Excel Connection if compare all choice selected
    \item Display filename, position difference, and velocity difference header
    \item Begin Loop. Loop once for single comparison and several
    times for comparing all files.
    \begin{itemize}
        \item Check the Tool2 folder for the same report
            \subitem Display error message if no report found
        \item Continue if match found or exit loop if no match found
        \item Read both output files and save the data to different matrices
        \item Check to see if the row sizes are the same in
        both matrices
            \subitem Display error if row sizes do not match
        \item Take difference of both Tools report data
        \item Determine the maximum difference for each coordinate
        system dependent parameter
        \item Store propagation duration of the test cases in a
        variable
        \item Save comparison data to .mat file
        \begin{itemize}
            \item If compare all reports chosen, format data for
            output to Latex
            \item Use BasicLatexTable script to save data to LaTex
            file
            \item Save comparison results, acceptance errors, and
            duration to Excel
        \end{itemize}
        \item Close Excel connection if open.
    \end{itemize}
    \item End Loop. Allow user to rerun script
\end{itemize}

\section{Comparison\_Tool1\_Tool2\_Cb.m}
Win/Mac/Linux compatible
\subsection{Purpose}
This script is used to perform the central body dependent
comparisons needed for the Calculation Parameters Section
(~\ref{Ch:CalcParameters}) of the Acceptance Test Plan. The
comparison involves taking the maximum absolute value of the
differences of the variables listed in the Inputs section of this
help guide.
\subsection{Inputs}
\begin{itemize}
    \item Folder to search for files$:$ $[Root Folder]/[Tool]$\_reports. \\
    (Refer to Section~\ref{nameConvCb} for the naming convention
    of these report files.)
    \item The report files must be formatted the same way. Time, Altitude
    (km), Beta Angle (deg), C3\_Energy ($km^2/sec^2$), Eccentricity,
    Latitude (deg), Longitude (deg), (RxV)\_Mag ($km^2/sec$), Mean Anomaly
    (deg), Mean Motion ($rad/sec$), Period (sec), Apoapsis Radius (km),
    Perigee Radius (km), R\_Mag (km), Semi-major Axis (km), True Anomaly
    (deg), Semilatus Rectum(km), Apoapsis Velocity ($km/sec$), Periapsis
    Velocity ($km/sec$), Greenwich Hour Angle(deg), Local Sidereal Time
    \item Due to the inability of FF and STK tools to output all the
    parameters listed in the previous bullet, exceptions for FF and STK
    are built into the code.\\

    STK: Semilatus Rectum, Apoapsis Velocity, Perigee Velocity,
    Greenwich Hour Angle, and Local Sidereal Time can not be outputted
    in the same report file. Out of the aforementioned parameters
    Greenwich Hour Angle is the only parameter that can be outputted in
    a separate file. All the other parameters are calculated in MATLAB
    based on the results STK could generate.\\

    FF: (RxV)\_Mag and R\_Mag could not be outputted easily. Instead the
    $[$XYZ$]$ components were outputted and the script computes the
    magnitude of the vectors. Apoapsis Velocity, Periapsis Velocity, and
    Local Sidereal Time are all created in this script based on
    available parameters.\\

    Modifications to the code will need to be made to add a new tool
    that cannot output all of the central body parameters that GMAT
    does.\\
    \item Currently the location of the reports first row of
    numerical data is coded into the script. FreeFlyer is the fourth
    row, STK is the seventh row, GMAT is the second row, OD toolbox is
    the first row, and any other tools added will automatically search
    the first row.
\end{itemize}
\subsection{Outputs}
\begin{itemize}
    \item Excel documents with comparison data
    \item MATLAB .mat files with comparison data
    \item Latex documents with comparison data
\end{itemize}
\subsection{Script Algorithm}
\begin{itemize}
    \item Display welcome message
    \item Display menu for Tool1 options based on [Tool]\_reports folder
    \item Wait for user to choose tool from menu
        \subitem Implement error system for incorrect choice
    \item Display menu for Tool2 options based on [Tool]\_reports folder
    \item Wait for user to choose tool from menu
        \subitem Implement error system for incorrect choice
    \item Alphabetize Tool1 and Tool2 for naming purposes and rename
    if necessary
    \item Store the row location of the first instance of numerical
    data for both Tools' report files
    \item Initialize variables and folder locations
    \item Display a menu of Tool1 $*$.report files
        \subitem Generate error report if no $*$.report files are located
    in [Tool1]\_report folder
    \item Wait for user to choose report comparison option
        \subitem Implement error system for incorrect choice
        \subitem Open Excel Connection if compare all choice selected
    \item Display filename, position difference, and velocity difference header
    \item Begin Loop. Loop once for single comparison and several
    times for comparing all files.
    \begin{itemize}
        \item Check the Tool2 folder for the same report
            \subitem Display error message if no report found
        \item Continue if match found or exit loop if no match found
        \item Read both output files and save the data to different matrices
        \item Check to see if the row sizes are the same in
        both matrices
            \subitem Display error if row sizes do not match
        \item Take difference of both Tools report data
        \item Code in exceptions for STK, FF, and any other tools
        that don't output all the desired GMAT central body
        dependent parameters
        \item Determine the maximum difference for each central body
        dependent parameter
        \item Store propagation duration of the test cases in a
        variable
        \item Save comparison data to .mat file
        \begin{itemize}
            \item If compare all reports chosen, format data for
            output to Latex
            \item Use BasicLatexTable script to save data to LaTex
            file
            \item Save comparison results, acceptance errors, and
            duration to Excel
        \end{itemize}
        \item Close Excel connection if open.
    \end{itemize}
    \item End Loop. Allow user to rerun script
\end{itemize}

\section{Comparison\_Tool1\_Tool2\_Libr.m}
Win/Mac/Linux compatible
\subsection{Purpose}
This script is used to perform the position and velocity comparisons
needed for the Libration Points Section (~\ref{Ch:LibPoint}) of the
Acceptance Test Plan. This script takes the normalized position and
velocity vector differences between the two selected programs.

\subsection{Inputs}
\begin{itemize}
    \item Folder to search for files$:$ $[$Root Folder$]/[$Tool$]$\_reports. \\
    (Refer to Section~\ref{nameConvLibPoint} for the naming convention
    of these report files.)
    \item The report files must be formatted the same way. The first
    column is time (Mod. Julian Date), second-fourth columns are the position vector
    components $(x,y,z)$, and fifth-seventh columns are the velocity
    vector components $(Vx,Vy,Vz)$. The data must be separated by
    spaces.
    \item Currently the location of the report's first row of numerical data is coded into the script.
    FreeFlyer is the fourth row, STK is the seventh row, GMAT is the
    first row, OD toolbox is the first row, and any other tools added
    will automatically search the first row.
\end{itemize}
\subsection{Outputs}
\begin{itemize}
    \item Comparison data is displayed in MATLAB's command window
    for all test cases.
    \item Excel documents with comparison data and pass/fail
    information.
    \item MATLAB .mat files with comparison data.
    \item Latex documents with comparison data.
\end{itemize}
\subsection{Script Algorithm}
[INSERT script Algorithm]

\section{Comparison\_Integ.m}
Win/Mac/Linux compatible
\subsection{Purpose}
This script is used to perform the integrator comparisons needed for
the Integrator Section (~\ref{Ch:Integrators}) of the Acceptance
Test Plan. The comparison involves taking the difference of the
position and velocity vector and then normalizing these two vectors
to get the position and velocity difference. This script behaves
similar to the Comparison\_Tool1\_Tool2\_PV.m script but the
components being varied are the integrators for two body test cases.
\subsection{Inputs}
\begin{itemize}
    \item Folder to search for files: $[Root Folder]/[Tool]$\_reports.
    \item Naming convention: Integrator\_$[Tool]$\_$[Trajectory]$\_$[IntegratorType]$\_2Body.report
    \item The report files must be formatted the same way. The first column is
    time, second-fourth columns are the position vector (x,y,z), and
    fifth-seventh columns are the velocity vector (x,y,z). The data must
    be separated by spaces.
    \item Currently the location of the reports'
    first row of numerical data is coded into the script. FreeFlyer is
    the fourth row, STK is the seventh row, GMAT is the second row, OD
    toolbox is the first row, and any other tools added will
    automatically search the first row.
\end{itemize}
\subsection{Outputs}
\begin{itemize}
    \item Excel documents with comparison data into the following
    folder:\\
    $[Root Folder]/$CompareResults$/[Tool1]$\_$[Tool2]$
    \item MATLAB .mat files with comparison data into the following
    folder:\\
    $[Root Folder]/$CompareResults$/[Tool1]$\_$[Tool2]$
    \item Latex documents with comparison data into the following
    folder:\\
    $[Root Folder]/$Latex\_Docs
\end{itemize}
\subsection{Script Algorithm}
\begin{itemize}
    \item Display welcome message
    \item Display menu for Tool1 options based on [Tool]\_reports
    folder (Can only be Exact or GMAT folders)
    \item Wait for user to choose tool from menu
        \subitem Implement error system for incorrect choice
    \item Display menu for Tool2 options based on [Tool]\_reports
    folder (Can only be Exact or GMAT folders)
    \item Wait for user to choose tool from menu
        \subitem Implement error system for incorrect choice
    \item Alphabetize Tool1 and Tool2 for naming purposes and rename
    if necessary
    \item Store the row location of the first instance of numerical
    data for both Tools' report files
    \item Initialize variables and folder locations
    \item Display a menu of Tool1 $*$.report files
        \subitem Generate error report if no $*$.report files are located
    in [Tool1]\_report folder
    \item Wait for user to choose report comparison option
        \subitem Implement error system for incorrect choice
        \subitem Open Excel Connection if compare all choice selected
    \item Display filename, position difference, and velocity difference header
    \item Begin Loop. Loop once for single comparison and several
    times for comparing all files.
    \begin{itemize}
        \item Check the Tool2 folder for the same report
            \subitem Display error message if no report found
        \item Continue if match found or exit loop if no match found
        \item Read both output files and save the data to different matrices
        \item Check to see if the row sizes are the same in
        both matrices
            \subitem Display error if row sizes do not match
        \item Take difference of both Tools report data
        \item Normalize the results based on position and velocity
        \item Determine the maximum normalized position and velocity
        difference
        \item Store propagation duration of the test cases in a
        variable
        \item Save comparison data to .mat file
        \begin{itemize}
            \item If compare all reports chosen, format data for
            output to Latex
            \item Use BasicLatexTable script to save data to LaTex
            file
            \item Save comparison results, acceptance errors, and
            duration to Excel
        \end{itemize}
        \item Close Excel connection if open.
    \end{itemize}
    \item End Loop. Allow user to rerun script
\end{itemize}

\section{Comparison\_DeltaV.m}
Win/Mac/Linux compatible
\subsection{Purpose}
\subsection{Inputs}
\subsection{Outputs}
\subsection{Script Algorithm}

\section{Comparison\_StopCond.m}
Win/Mac/Linux compatible
\subsection{Purpose}
\subsection{Inputs}
\subsection{Outputs}
\subsection{Script Algorithm}

\section{STK\_Repropagate.m}
Win compatible
\subsection{Purpose}
The STK\_Repropagate script was designed to reduce the time it took
to generate STK report files, after modifications to the STK
scenario were made, and obtain more accurate STK run times. Through
STK's connect module Matlab connect with STK and propagates
satellites, generates reports, and outputs run times.
\subsection{Inputs}
\begin{itemize}
    \item STK scenario folders that follow the GMAT Acceptance Test Plan
    naming convention, in the following folder:
    $[Root Folder]/$TruthFiles/STK
\end{itemize}
\subsection{Outputs}
\begin{itemize}
    \item STK report file saved into $[Root Folder]/$STK\_reports.
    \item Matlab .mat file with the time taken to propagate each satellite.
\end{itemize}
\subsection{Script Algorithm}

\section{TimeComparo.m}
Win/Mac/Linux compatible
\subsection{Purpose}
When running the BuildRun\_Script\_GMAT.m script there is an all
option after selecting the build \& run multiple cases choice. By
using this all option, the GMAT performance times for all the test
cases are saved to a .mat file. This script uses those saved
performance times and, based on a pre-selected amount of test cases,
creates a new excel file that contains GMAT, FF, and STK performance
times for those pre-selected cases.
\subsection{Inputs}
\begin{itemize}
    \item Template file containing pre-selected cases:
    $[Root Folder]/$NonGMATrunTimes.xls
    \item Folder to search for files:
    $[Root Folder]/$CompareResults
    \item Naming convention:\\
    $[Date]/$\_Time2RunAll.mat
\end{itemize}
\subsection{Outputs}
\begin{itemize}
    \item Excel document with comparison data
\end{itemize}
\subsection{Script Algorithm}
