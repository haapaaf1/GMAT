This section describes the naming convention for propagator scripts
and output reports. The naming convention consists of an ordered
series of option strings, separated by underscores (\_ ). Currently,
options are allowed for the following fields, and will be present in
the file name in order:
\begin{enumerate}
  \item \emph{tool} - The tool used to generate the test case
  \item \emph{traj} - The trajectory to use.  This includes initial conditions, physical parameters, and time step
  \item \emph{pmg} - The point-mass gravity model to use
  \item \emph{nsg} - The non-spherical gravity model to use
  \item \emph{drag} - The atmospheric drag model to use
  \item \emph{other} - Any other forces to include, such as SRP, secondary body gravity, etc
\end{enumerate}
The \emph{tool} field should always be the first field.  Future
additional fields should be added to the end of the list of fields.
If multiple \emph{other} options are required, they should be added
to the end of the file as required.  For example, the file name will
be \emph{tool\_traj\_pmg\_nsg\_drag\_other1\_other2.report} (file
extensions are described later.)  Each field has a finite list of
options, as follows (future options should be added to this list):
\begin{enumerate}
  \item \emph{tool}

  \begin{tabular}{ll}
    STK  & - Satellite Toolkit HPOP or Astrogator\\
    FF   & - FreeFlyer\\
    GMAT & - General Mission Analysis Tool\\
  \end{tabular}

  \item \emph{traj}

  \begin{tabular}{ll}
    ISS & - LEO orbit\\
    SunSync & - LEO orbit\\
    GPS & - MEO orbit\\
    GEO & - GEO orbit\\
    Molniya & - HEO orbit\\
    Mars1 & - eccentric low orbit\\
    Mercury1 & - eccentric low orbit\\
    Moon & - eccentric low orbit\\
    Neptune1 & - eccentric low orbit\\
    Pluto1 & - eccentric low orbit\\
    Saturn1 & - eccentric low orbit\\
    Uranus1 & - eccentric low orbit\\
    Venus1 & - eccentric low orbit\\
    DeepSpace & deep space orbit\\
    EML2 & - Earth Moon L2 orbit\\
    ESL2 & - Earth Sun L2 orbit\\
  \end{tabular}

NOTE:  Some test cases contain \emph{traj} variations. In these
cases, \emph{traj} precedes the modification. For example, if ISS
trajectory is needed with no output, then \emph{traj} can be
ISSnoOut. The lack of a report file is shortened to noOut.

  \item \emph{pmg}

  \begin{tabular}{ll}
    Earth & - Earth point mass gravity \\
    Sun & - Sun point mass gravity\\
    Luna & - Lunar point mass gravity\\
    AllPlanets & - Sun, Mercury, Venus, Earth, Moon, Mars, Mercury, Jupiter, and
    Pluto point mass gravity included.\\
  \end{tabular}

NOTE:  When dealing with a combination of \emph{pmg}'s the first
point mass is the primary body and the following are third body
point masses. For example, LunaSunEarth would be a Lunar primary
body with the Earth and Sun as third body point masses. The
\emph{pmg}'s after the primary body are arranged based on the order
from the sun, in order to reduce repeat filenames.

  \item \emph{nsg}

  \begin{tabular}{ll}
    0       & - no non-spherical gravity included \\
    JGM2    & - Earth JGM2 20x20 gravity \\
    JGM3    & - Earth JGM3 20x20 gravity \\
    EGM96   & - Earth EGM96 20x20 gravity \\
    MARS50C & - Mars Mars50c 20x20 gravity\\
    LP165P  & - Moon LP165P 20x20 gravity\\
   \end{tabular}
  \item \emph{drag}

  \begin{tabular}{ll}
    0 & - drag not included\\
    HP & - Harris Priester \\
    JRXX & - Jacchia-Roberts\\
    MSISEXX & - NRL MSISE\\
  \end{tabular}

NOTE:  XX in the \emph{drag} field refers to the year.  For example,
JR77 would be the Jacchia-Roberts 1977 model, and MSISE00 would be
NRL MSISE 2000. Refer to Table~\ref{Table: InitialCond} for the drag
settings used.

  \item \emph{other}

  \begin{tabular}{ll}
    0 & - no other forces included\\
    SRP & - Solar Radiation Pressure\\
  \end{tabular}

NOTE:  Any of the above options may be included as an \emph{other}
field.  Refer to Table~\ref{Table: InitialCond} for the SRP settings
used.

\end{enumerate}
