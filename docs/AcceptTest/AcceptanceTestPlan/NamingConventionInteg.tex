This section describes the naming convention for integrator scripts
and output reports. The naming convention consists of an ordered
series of option strings, separated by underscores (\_ ). Currently,
options are allowed for the following fields, and will be present in
the file name in order:
\begin{enumerate}
  \item \emph{tool} - The tool used to generate the test case
  \item \emph{traj} - The trajectory to use.  This includes initial conditions, physical parameters, and time step
  \item \emph{integ} - The integrator to use
\end{enumerate}

The word Integrator precedes the \emph{tool} field and 2Body follows
the \emph{integ} field. The final integrator file format is as followed:\\
Integrator\_\emph{tool}\_\emph{traj}\_\emph{integ}\_2Body.report\\

The \emph{tool} field should always be the first option field.
Future additional fields should be added to the end of the list of
fields. Each field has a finite list of options, as follows (future
options should be added to this list):
\begin{enumerate}
  \item \emph{tool}
  \begin{tabular}{ll}
    STK  & - Satellite Toolkit HPOP or Astrogator\\
    FF   & - FreeFlyer\\
    GMAT & - General Mission Analysis Tool\\
  \end{tabular}

  \item \emph{traj}
  \begin{tabular}{ll}
    ISS & - leo orbit\\
    GEO & - geo orbit\\
  \end{tabular}

NOTE:  Some test cases contain \emph{traj} variations. In this case
\emph{traj} precedes the modification. For example, if an ISS
trajectory is needed with a different Cd, \emph{traj} could be
ISSdiffCd1.

  \item \emph{integ}
  \begin{tabular}{ll}
    RKV89 & - RungaKutta 8(9)\\
    RKN68 & - DormandElMikkawyPrince 6(8)\\
    RKF56 & - RungeKuttaFehlberg 5(6)\\
    PD45  & - PrinceDormand 4(5)\\
    PD78  & - PrinceDormand 7(8)\\
    BS & - BulirschStoer\\
    ABM & - AdamsBashforthMoulton\\
  \end{tabular}
\end{enumerate}
