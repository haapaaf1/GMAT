This section describes the naming convention for loop test scripts
and output reports. The naming convention consists of an ordered
series of option strings. Currently, options are allowed for the
following fields, and will be present in the file name in order, if
applicable to the test:

\begin{enumerate}
  \item \emph{LoopType1} - Type of loop test
  \item \emph{LoopType2} - Type of loop test, if applicable
  \item \emph{LHS/RHS} - Parameter type in LHS/RHS of loop
  condition, if applicable
\end{enumerate}

Loop\_GMAT\_ precedes the \emph{LoopType1} field. The
\emph{LoopType1} field should always be the first option field. No
additional fields should be added to the list of fields. If
\emph{LoopType2} is present, then \emph{LoopType2} is nested inside
\emph{LoopType1} in the test. Examples of the final Loop Test file
format are as followed:\\

Loop\_GMAT\_\emph{LoopType1}\#\#\_\#\#.report or\\
Loop\_GMAT\_\emph{LoopType1}\emph{LoopType2}\#\#.report or\\
Loop\_GMAT\_\emph{LoopType1}\emph{LoopType2}.report\\

Each field has a finite list of options, as follows (future options
should be added to this list):

\begin{enumerate}
  \item \emph{LoopType1}

  \begin{tabular}{ll}
    For  & - For Loop\\
    While   & - While Loop\\
    If & - If or If-Else statement\\
  \end{tabular}

  \item \emph{LoopType2}

  \begin{tabular}{ll}
    For  & - For Loop\\
    While   & - While Loop\\
    If & - If or If-Else statement\\
  \end{tabular}

  \item \emph{LHS/RHS}

  \begin{tabular}{ll}
    \#\# & - \# can either be a 1, 2, 3, 4, or 5$*$\\
    \#\#\_\#\# & - \# can either be a 1, 2, 3, 4, or 5$*$\\
  \end{tabular}

  NOTE:  $*$ 1=number $|$ 2=variable $|$ 3=array $|$ 4=S/C non-time parameter $|$ 5=S/C time
  parameter. No value means there is no LHS/RHS of the loop condition. For loops are a good example of this.
\end{enumerate}
