\begin{table}[htbp!]
\centering
      \begin{tabular}{|p{1.0 in} |p{5.0 in} |}
         \hline
          \rowcolor[rgb]{0.8,0.8,0.8}  Name & TD-6 Equivalent State Representations for a Circular, Inclined (retrograde) Orbit\\
         \hline
         Description & This table contains equivalent states in all GMAT state representations
         that have a central body at the origin.    \\ \hline
         Source &  Hand calculations based on Math spec for all except Equinoctial which is from STK.   (The test data assumes $\mu = 398600.4415$)\\
         \hline
         Data &
         %------- The test data is here
          \begin{compactenum}
              \item Cartesian State
              \begin{compactenum}
                  \item X = -5975.5752861126311
                  \item Y = 480.14719831222595
                  \item Z = -3416.4248371584213
                  \item VX = 3.8002690670377621
                  \item VY = 0.9160734111800478
                  \item VZ = -6.5182010133917370
              \end{compactenum}
              \item Keplerian State
              \begin{compactenum}
                  \item SMA = 6900
                  \item ECC = 0.0
                  \item INC = 98
                  \item RAAN = 0.0
                  \item AOP = 0.0
                  \item TA = 210.0
              \end{compactenum}
              \item Modified Keplerian
               \begin{compactenum}
                  \item RadPer = 6900
                  \item RadApo = 6900
                  \item INC = 98
                  \item RAAN = 0.0
                  \item AOP = 0.0
                  \item TA = 210.0
              \end{compactenum}
              \item Spherical RADec
               \begin{compactenum}
                  \item RMAG = 6900
                  \item RA = 175.4060606593105
                  \item DEC = -29.67858910292156
                  \item VMAG = 7.6005381340755180
                  \item RAV = 13.55286811093926
                  \item DECV= -59.04786932043024
              \end{compactenum}
              \item Spherical RADec
               \begin{compactenum}
                  \item RMAG = 6900
                  \item RA = 175.4060606593105
                  \item DEC = -29.67858910292156
                  \item VMAG = 7.6005381340755180
                  \item AZI = 189.2177489242794
                  \item FPA = 90
              \end{compactenum}
              \item Equinoctial
               \begin{compactenum}
                  \item SMA = 6900
                  \item h = 0.0
                  \item k = 0.0
                  \item p = 0.0
                  \item q = 1.1503684072210094
                  \item MLONG = 210.0
              \end{compactenum}
          \end{compactenum}\\
         %------- The test data is here
         \hline
\end{tabular}
   \label{Table:TD-6}
   \caption{TD-6 Equivalent State Representations for a Circular, Inclined (retrograde) Orbit}
    \index{Test Data!TD-6}
\end{table} 