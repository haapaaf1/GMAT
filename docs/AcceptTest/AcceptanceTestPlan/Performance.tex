\chapter{Performance}
\label{Ch:Performance}

One of the many goals of GMAT is to perform at satisfactory speed
when running a script. By running similar scripts in GMAT and
reference software, time performance comparisons could be made. We
generated several test cases centered around the Earth and a minimal
set of test cases about a few non-Earth bodies. Many of the test
cases were extracted from the previous Chapters and used in these
tests.\\

The performance tests cases were created from the propagator test
cases with some slight variations. We were interested in test cases
focused on only point mass bodies, non-spherical gravity present,
solar radiation pressure turned on, or drag effects. Isolation of
the mentioned perturbation forces allows us to see how long each
process take for GMAT to complete. The deepspace, non-earth, and
libration point test cases only focused on point mass bodies.\\

Propagation duration times for the GEO, GPS, LEO, Molniya, and
SunSync cases were increased to obtain a more accurate measurement
of the time between GMAT and the reference software packages. This
could have been done for the other test cases but the amount time
needed to modify the automation scripts was not worth the extra
data. The reason extra code would have been needed was to account
for the differences in propagation modules for STK. STK-HPOP is used
for the GEO, GPS, LEO, Molniya, and Sunsyc test cases and
STK-Astrogator is used for the others.\\

\section{Test Machine Specifications}
\label{Sec: MachineSpecs}

All tests presented in this Acceptance Test Plan document were
performed on a desktop machine with the following
specifications:\\

Computer Type - PC\\
OS            - Microsoft Windows XP\\
RAM           - 2.0 GB\\
Processor     - Intel(R) Pentium(R) 4 CPU 3.40GHz\\
Architecture  - 32-bit\\
Manufacturer  - Dell\\
STK Version   - STK 6.2.0\index{STK}\\
FF Version    - FF 5.6.5.35\index{FreeFlyer}\\

\clearpage
\section{Initial Orbit Conditions}

All of the initial orbit parameters for the test cases in this
section were obtained from the Propagation\index{Propagation}
section (Chapter~\ref{Ch:Propagators}) test cases. One of the
differences is the propagation duration is extended to 30 days,
excluding the DeepSpace, ESL2, and EML2 cases. Another difference is
the use of NSG, PMG, Drag, and SRP. Refer to the Performance naming
conventions section (Section ~\ref{nameConvPerf}) for the types of
settings used.

Refer to Appendix ~\ref{Sec:initCondsProp} Tables~\ref{Table:
InitStateISS}-~\ref{Table: InitStateVenus} for a listing of all
Propagator initial orbit states used for the Performance test cases.

\clearpage
\section{Naming Convention}
\label{nameConvPerf}
This section describes the naming convention for Performance scripts
and output reports generated for use in GMAT's Acceptance Test Plan.
The naming convention consists of an ordered series of option
strings, separated by underscores (\_ ). Currently, options are
allowed for the following fields, and will be present in the file
name in order:
\begin{enumerate}
  \item \emph{tool} - Tool used to generate the trajectory.
  \item \emph{traj} - Type of trajectory.
  \item \emph{forces} - Forces applied.
  \item \emph{output} - Flag noting if there is output.
\end{enumerate}

The word Performance precedes the \emph{tool} field. The final
stopping condition file format is as followed:\\
Performance\_\emph{tool}\_\emph{traj}\_\emph{forces}\_\emph{output}.report or\\
Performance\_\emph{tool}\_\emph{traj}PM\_\emph{forces}\_\emph{output}.report

The \emph{tool} field should always be the first option field.
Future additional fields should be added to the end of the list of
fields. Each field has a finite list of options, as follows (future
options should be added to this list):
\begin{enumerate}
  \item \emph{tool}
  \begin{tabular}{ll}
    STK  & - Satellite Toolkit HPOP or Astrogator\\
    FF   & - FreeFlyer\\
    GMAT & - General Mission Analysis Tool\\
    OD   & - Orbital Determination Toolbox\\
  \end{tabular}

  \item \emph{traj}
  \begin{tabular}{ll}
    Deepspace & - deep space orbit\\
    EML2      & - Earth Moon L2 orbit\\
    GEO       & - GEO orbit\\
    GPS       & - MEO orbit\\
    ISS       & - LEO orbit\\
    Mars1     & - eccentric low orbit\\
    Molniya   & - HEO orbit\\
    Moon      & - eccentric low orbit\\
    SunSync   & - LEO orbit\\
  \end{tabular}

NOTE:  Some test cases contain \emph{traj} variations. In this case
\emph{traj} precedes the modification. For example, an ISS
trajectory is used and the point mass code is used in GMAT instead
of JGM2 0x0. In this case the \emph{traj} is ISSPM.

  \item \emph{forces}
  \begin{tabular}{ll}
    AllPlanets  & - Sun, Mercury, Venus, Earth, Moon, Mars, Mercury, Jupiter,\\
    &   and Pluto point mass gravity included.\\
    1PM         & - Central body is the only point mass included.\\
    Drag        & - JR is used for Earth cases.\\
    NSG         & - JGM2 20x20 is used.\\
    SRP         & - SRP is turned on
  \end{tabular}
\end{enumerate}


\section{Comparison Script Information}
TimeComparo.m is the script used to perform the comparison data
needed for the Performance chapter of the Acceptance Test Plan. This
script was designed to allow the user to select from a list of dates
and then generates output that includes all available performance
results from GMAT, STK, and FF. The dates represent the times when
all of the performance test cases were run from the
BuildRun\_Script\_GMAT.m script.\\

STK\_Repropagate.m is the script that generates the performance
times for STK. The data generated is then exported into an excel
document for use with the TimeComparo.m script.\\

Refer to Appendix~\ref{Ch:CompScripts} for more details of this
script and others used in the Acceptance Test Plan document.

\section{Test Results}
We strive to make GMAT perform just as good or better than the
reference software packages. For the test cases with performance
numbers that are not nearly equivalent or better, work is being done
to improve GMAT's performance without affecting its numerical
accuracy.

The main performance output parameters of interest are the time
taken to run the script, the amount of integration step taken during
that run, and the time per integration step. The results in this
section are outputted in seconds. Tables~\ref{Table:
Performance1-1}-~\ref{Table: Performance2-3} contain these
performance parameters by tool.

Tables~\ref{Table: Performance3-1}-~\ref{Table: Performance4-3}
contain a percentage difference between GMAT and the other reference
software used. The data indicates GMAT's performance data was
divided into the reference softwares same performance data to obtain
a percentage. A value of 100\% indicates both tools were the same,
greater than 100\% indicates GMAT performed worse, and less than
100\% indicates GMAT performed better.

\begin{table}[htbp!]
\centering
\caption{ Performance Test Case Comparisons}
      \begin{tabular}{lccc}
      \hline\hline
          Test Case & GMAT TimeToRun(sec) & STK TimeToRun(sec) & FF TimeToRun(sec) \\
         \hline
         %---New Row---%
         DeepSpacePM-AllPlanets-noOutput.m & NaN & NaN & NaN \\
         %---New Row---%
         DeepSpacePM-AllPlanets-output.m & NaN & NaN & NaN \\
         %---New Row---%
         DeepSpace-AllPlanets-noOutput.m & NaN & NaN & NaN \\
         %---New Row---%
         DeepSpace-AllPlanets-output.m & NaN & NaN & NaN \\
         %---New Row---%
         EML2PM-AllPlanets-noOutput.m & NaN & NaN & NaN \\
         %---New Row---%
         EML2PM-AllPlanets-output.m & NaN & NaN & NaN \\
         %---New Row---%
         EML2-AllPlanets-noOutput.m & NaN & NaN & NaN \\
         %---New Row---%
         EML2-AllPlanets-output.m & NaN & NaN & NaN \\
         %---New Row---%
         GEOPM-1PM-noOutput.m & 2.365369677 & 0.9100139264 & NaN \\
         %---New Row---%
         GEOPM-1PM-output.m & 3.624228889 & 2.025697121 & NaN \\
         %---New Row---%
         GEOPM-AllPlanets-noOutput.m & 9.262963421 & 4.056012761 & NaN \\
         %---New Row---%
         GEOPM-AllPlanets-output.m & 10.63963856 & 5.203490891 & NaN \\
         %---New Row---%
         GEOPM-SRP-noOutput.m & 3.221224485 & 1.527696956 & NaN \\
         %---New Row---%
         GEOPM-SRP-output.m & 4.613150554 & 2.711619773 & NaN \\
         %---New Row---%
         GEO-1PM-noOutput.m & 5.080978504 & 0.9100139264 & NaN \\
         %---New Row---%
         GEO-1PM-output.m & 9.065021783 & 2.025697121 & NaN \\
         %---New Row---%
         GEO-AllPlanets-noOutput.m & 12.82652415 & 4.056012761 & NaN \\
         %---New Row---%
         GEO-AllPlanets-output.m & 14.62352863 & 5.203490891 & NaN \\
         %---New Row---%
         GEO-Drag-noOutput.m & 9.083949943 & 1.076942908 & NaN \\
         %---New Row---%
         GEO-Drag-output.m & 9.863043934 & 2.161930469 & NaN \\
         %---New Row---%
         GEO-NSG-noOutput.m & 9.079661186 & 3.27783425 & NaN \\
         %---New Row---%
         GEO-NSG-output.m & 9.063319947 & 4.456671393 & NaN \\
         %---New Row---%
         GEO-SRP-noOutput.m & 9.086620394 & 1.527696956 & NaN \\
         %---New Row---%
         GEO-SRP-output.m & 9.037045099 & 2.711619773 & NaN \\
         %---New Row---%
         GPSPM-1PM-noOutput.m & 2.324964709 & 0.9494172025 & NaN \\
         %---New Row---%
         GPSPM-1PM-output.m & 3.695722836 & 2.657454174 & NaN \\
         %---New Row---%
         GPSPM-AllPlanets-noOutput.m & 9.27329574 & 4.106867269 & NaN \\
         %---New Row---%
         GPSPM-AllPlanets-output.m & 10.67132599 & 5.382557951 & NaN \\
         %---New Row---%
         GPSPM-SRP-noOutput.m & 4.068783658 & 1.55785984 & NaN \\
         %---New Row---%
         GPSPM-SRP-output.m & 5.080100348 & 2.793106498 & NaN \\
         %---New Row---%
         GPS-1PM-noOutput.m & 5.085208763 & 0.9494172025 & NaN \\
         %---New Row---%
         GPS-1PM-output.m & 9.064168882 & 2.657454174 & NaN \\
         %---New Row---%
         GPS-AllPlanets-noOutput.m & 12.82362876 & 4.106867269 & NaN \\
         %---New Row---%
         GPS-AllPlanets-output.m & 14.21737953 & 5.382557951 & NaN \\
         %---New Row---%
         GPS-Drag-noOutput.m & 9.078539423 & 1.11248804 & NaN \\
         %---New Row---%
         GPS-Drag-output.m & 9.829883966 & 2.366961555 & NaN \\
         %---New Row---%
         GPS-NSG-noOutput.m & 6.989305121 & 3.277017722 & NaN \\
         %---New Row---%
         GPS-NSG-output.m & 9.049662625 & 4.556902511 & NaN \\
         %---New Row---%
         GPS-SRP-noOutput.m & 9.063283294 & 1.55785984 & NaN \\
         %---New Row---%
         GPS-SRP-output.m & 10.14604839 & 2.793106498 & NaN \\
         %---New Row---%
         ISSPM-1PM-noOutput.m & 3.371743671 & 3.532383306 & NaN \\
         %---New Row---%
         ISSPM-1PM-output.m & 9.505570289 & 8.654091176 & NaN \\
         %---New Row---%
         ISSPM-AllPlanets-noOutput.m & 34.23165034 & 17.8158481 & NaN \\
         %---New Row---%
         ISSPM-AllPlanets-output.m & 40.2429021 & 23.11573611 & NaN \\
         %---New Row---%
         ISSPM-SRP-noOutput.m & 14.61508594 & 6.507057014 & NaN \\
      \hline\hline
      \label{Table: Performance1-1} 
\end{tabular}
\end{table}

\begin{table}[htbp!]
\centering
\caption{ Performance Test Case Comparisons}
      \begin{tabular}{lccc}
      \hline\hline
          Test Case & GMAT TimeToRun(sec) & STK TimeToRun(sec) & FF TimeToRun(sec) \\
         \hline
         %---New Row---%
         ISSPM-SRP-output.m & 25.07031637 & 11.69117678 & NaN \\
         %---New Row---%
         ISS-1PM-noOutput.m & 9.069769649 & 3.532383306 & NaN \\
         %---New Row---%
         ISS-1PM-output.m & 14.29912081 & 8.654091176 & NaN \\
         %---New Row---%
         ISS-AllPlanets-noOutput.m & 39.95640016 & 17.8158481 & NaN \\
         %---New Row---%
         ISS-AllPlanets-output.m & 46.03288368 & 23.11573611 & NaN \\
         %---New Row---%
         ISS-Drag-noOutput.m & 25.20274035 & 6.702129963 & NaN \\
         %---New Row---%
         ISS-Drag-output.m & 31.62018206 & 11.94099205 & NaN \\
         %---New Row---%
         ISS-NSG-noOutput.m & 18.25562254 & 14.51133002 & NaN \\
         %---New Row---%
         ISS-NSG-output.m & 26.18732275 & 19.74322161 & NaN \\
         %---New Row---%
         ISS-SRP-noOutput.m & 27.42110369 & 6.507057014 & NaN \\
         %---New Row---%
         ISS-SRP-output.m & 37.81563076 & 11.69117678 & NaN \\
         %---New Row---%
         Mars1PM-1PM-noOutput.m & NaN & NaN & NaN \\
         %---New Row---%
         Mars1PM-1PM-output.m & NaN & NaN & NaN \\
         %---New Row---%
         Mars1PM-AllPlanets-noOutput.m & NaN & NaN & NaN \\
         %---New Row---%
         Mars1PM-AllPlanets-output.m & NaN & NaN & NaN \\
         %---New Row---%
         Mars1-1PM-noOutput.m & NaN & NaN & NaN \\
         %---New Row---%
         Mars1-1PM-output.m & NaN & NaN & NaN \\
         %---New Row---%
         Mars1-AllPlanets-noOutput.m & NaN & NaN & NaN \\
         %---New Row---%
         Mars1-AllPlanets-output.m & NaN & NaN & NaN \\
         %---New Row---%
         MolniyaPM-1PM-noOutput.m & 2.520149293 & 1.138387756 & NaN \\
         %---New Row---%
         MolniyaPM-1PM-output.m & 4.727116994 & 2.810948095 & NaN \\
         %---New Row---%
         MolniyaPM-AllPlanets-noOutput.m & 13.49447664 & 5.357828495 & NaN \\
         %---New Row---%
         MolniyaPM-AllPlanets-output.m & 15.48603786 & 7.016379791 & NaN \\
         %---New Row---%
         MolniyaPM-SRP-noOutput.m & 4.93236051 & 2.014259857 & NaN \\
         %---New Row---%
         MolniyaPM-SRP-output.m & 9.064659894 & 3.669778142 & NaN \\
         %---New Row---%
         Molniya-1PM-noOutput.m & 5.931093645 & 1.138387756 & NaN \\
         %---New Row---%
         Molniya-1PM-output.m & 9.067935171 & 2.810948095 & NaN \\
         %---New Row---%
         Molniya-AllPlanets-noOutput.m & 17.02801742 & 5.357828495 & NaN \\
         %---New Row---%
         Molniya-AllPlanets-output.m & 18.94455575 & 7.016379791 & NaN \\
         %---New Row---%
         Molniya-Drag-noOutput.m & 11.14562516 & 1.48759834 & NaN \\
         %---New Row---%
         Molniya-Drag-output.m & 12.81850503 & 3.148300748 & NaN \\
         %---New Row---%
         Molniya-NSG-noOutput.m & 9.06083695 & 4.249729672 & NaN \\
         %---New Row---%
         Molniya-NSG-output.m & 9.881694573 & 5.892727931 & NaN \\
         %---New Row---%
         Molniya-SRP-noOutput.m & 9.723377636 & 2.014259857 & NaN \\
         %---New Row---%
         Molniya-SRP-output.m & 11.63239417 & 3.669778142 & NaN \\
         %---New Row---%
         MoonPM-1PM-noOutput.m & NaN & NaN & NaN \\
         %---New Row---%
         MoonPM-1PM-output.m & NaN & NaN & NaN \\
         %---New Row---%
         Moon-1PM-noOutput.m & NaN & NaN & NaN \\
         %---New Row---%
         Moon-1PM-output.m & NaN & NaN & NaN \\
         %---New Row---%
         SunSyncPM-1PM-noOutput.m & 3.357036607 & 3.408378327 & NaN \\
         %---New Row---%
         SunSyncPM-1PM-output.m & 9.442151447 & 8.631484337 & NaN \\
         %---New Row---%
         SunSyncPM-AllPlanets-noOutput.m & 33.3628759 & 17.53388808 & NaN \\
         %---New Row---%
         SunSyncPM-AllPlanets-output.m & 39.61642969 & 22.77039115 & NaN \\
         %---New Row---%
         SunSyncPM-SRP-noOutput.m & 14.48378093 & 6.372961703 & NaN \\
         %---New Row---%
         SunSyncPM-SRP-output.m & 25.21344992 & 11.62540558 & NaN \\
      \hline\hline
      \label{Table: Performance1-2} 
\end{tabular}
\end{table}

\begin{table}[htbp!]
\centering
\caption{ Performance Test Case Comparisons}
      \begin{tabular}{lccc}
      \hline\hline
          Test Case & GMAT TimeToRun(sec) & STK TimeToRun(sec) & FF TimeToRun(sec) \\
         \hline
         %---New Row---%
         SunSync-1PM-noOutput.m & 9.077130362 & 3.408378327 & NaN \\
         %---New Row---%
         SunSync-1PM-output.m & 14.25260635 & 8.631484337 & NaN \\
         %---New Row---%
         SunSync-AllPlanets-noOutput.m & 39.61202214 & 17.53388808 & NaN \\
         %---New Row---%
         SunSync-AllPlanets-output.m & 45.78612287 & 22.77039115 & NaN \\
         %---New Row---%
         SunSync-Drag-noOutput.m & 24.86261533 & 6.570531171 & NaN \\
         %---New Row---%
         SunSync-Drag-output.m & 31.29207997 & 11.86876567 & NaN \\
         %---New Row---%
         SunSync-NSG-noOutput.m & 18.61135222 & 14.13227078 & NaN \\
         %---New Row---%
         SunSync-NSG-output.m & 27.37641393 & 19.3581896 & NaN \\
         %---New Row---%
         SunSync-SRP-noOutput.m & 26.72110975 & 6.372961703 & NaN \\
         %---New Row---%
         SunSync-SRP-output.m & 37.03988272 & 11.62540558 & NaN \\
      \hline\hline
      \label{Table: Performance1-3} 
\end{tabular}
\end{table}

\begin{table}[htbp!]
\centering
\caption{ Performance Test Case Comparisons}
      \begin{tabular}{lccc}
      \hline\hline
          Test Case & GMAT 10000xTime/Step & STK 10000xTime/Step & FF 10000xTime/Step \\
         \hline
         %---New Row---%
         DeepSpacePM-AllPlanets-noOutput.m & NaN & NaN & NaN \\
         %---New Row---%
         DeepSpacePM-AllPlanets-output.m & NaN & NaN & NaN \\
         %---New Row---%
         DeepSpace-AllPlanets-noOutput.m & NaN & NaN & NaN \\
         %---New Row---%
         DeepSpace-AllPlanets-output.m & NaN & NaN & NaN \\
         %---New Row---%
         EML2PM-AllPlanets-noOutput.m & NaN & NaN & NaN \\
         %---New Row---%
         EML2PM-AllPlanets-output.m & NaN & NaN & NaN \\
         %---New Row---%
         EML2-AllPlanets-noOutput.m & NaN & NaN & NaN \\
         %---New Row---%
         EML2-AllPlanets-output.m & NaN & NaN & NaN \\
         %---New Row---%
         GEOPM-1PM-noOutput.m & 2.736746126 & 1.052891272 & NaN \\
         %---New Row---%
         GEOPM-1PM-output.m & 4.193253372 & 2.343743053 & NaN \\
         %---New Row---%
         GEOPM-AllPlanets-noOutput.m & 10.71730119 & 4.69282976 & NaN \\
         %---New Row---%
         GEOPM-AllPlanets-output.m & 12.31012213 & 6.020468461 & NaN \\
         %---New Row---%
         GEOPM-SRP-noOutput.m & 3.726974992 & 1.767554039 & NaN \\
         %---New Row---%
         GEOPM-SRP-output.m & 5.337441344 & 3.13735945 & NaN \\
         %---New Row---%
         GEO-1PM-noOutput.m & 5.878720935 & 1.052891272 & NaN \\
         %---New Row---%
         GEO-1PM-output.m & 10.4882816 & 2.343743053 & NaN \\
         %---New Row---%
         GEO-AllPlanets-noOutput.m & 14.84036116 & 4.69282976 & NaN \\
         %---New Row---%
         GEO-AllPlanets-output.m & 16.91950553 & 6.020468461 & NaN \\
         %---New Row---%
         GEO-Drag-noOutput.m & 10.51018158 & 1.24602905 & NaN \\
         %---New Row---%
         GEO-Drag-output.m & 11.41159775 & 2.501365809 & NaN \\
         %---New Row---%
         GEO-NSG-noOutput.m & 10.50521947 & 3.79247281 & NaN \\
         %---New Row---%
         GEO-NSG-output.m & 10.48631256 & 5.156394068 & NaN \\
         %---New Row---%
         GEO-SRP-noOutput.m & 10.51327131 & 1.767554039 & NaN \\
         %---New Row---%
         GEO-SRP-output.m & 10.45591241 & 3.13735945 & NaN \\
         %---New Row---%
         GPSPM-1PM-noOutput.m & 2.68999735 & 1.098481086 & NaN \\
         %---New Row---%
         GPSPM-1PM-output.m & 4.275972274 & 3.074689545 & NaN \\
         %---New Row---%
         GPSPM-AllPlanets-noOutput.m & 10.72925574 & 4.751668714 & NaN \\
         %---New Row---%
         GPSPM-AllPlanets-output.m & 12.34678467 & 6.227650065 & NaN \\
         %---New Row---%
         GPSPM-SRP-noOutput.m & 3.450168454 & 1.801410545 & NaN \\
         %---New Row---%
         GPSPM-SRP-output.m & 4.307725217 & 3.229771621 & NaN \\
         %---New Row---%
         GPS-1PM-noOutput.m & 5.883615368 & 1.098481086 & NaN \\
         %---New Row---%
         GPS-1PM-output.m & 10.48729478 & 3.074689545 & NaN \\
         %---New Row---%
         GPS-AllPlanets-noOutput.m & 14.83701117 & 4.751668714 & NaN \\
         %---New Row---%
         GPS-AllPlanets-output.m & 16.44958872 & 6.227650065 & NaN \\
         %---New Row---%
         GPS-Drag-noOutput.m & 10.50392158 & 1.287154969 & NaN \\
         %---New Row---%
         GPS-Drag-output.m & 11.37323148 & 2.738587938 & NaN \\
         %---New Row---%
         GPS-NSG-noOutput.m & 8.085730126 & 3.791528082 & NaN \\
         %---New Row---%
         GPS-NSG-output.m & 10.46929966 & 5.27236204 & NaN \\
         %---New Row---%
         GPS-SRP-noOutput.m & 7.738459097 & 1.801410545 & NaN \\
         %---New Row---%
         GPS-SRP-output.m & 8.662951156 & 3.229771621 & NaN \\
         %---New Row---%
         ISSPM-1PM-noOutput.m & 0.8663044812 & 0.9223173727 & NaN \\
         %---New Row---%
         ISSPM-1PM-output.m & 2.442272883 & 2.25961283 & NaN \\
         %---New Row---%
         ISSPM-AllPlanets-noOutput.m & 8.795162081 & 4.651536018 & NaN \\
         %---New Row---%
         ISSPM-AllPlanets-output.m & 10.33963724 & 6.035282657 & NaN \\
         %---New Row---%
         ISSPM-SRP-noOutput.m & 2.173053102 & 1.699636154 & NaN \\
      \hline\hline
      \label{Table: Performance2-1} 
\end{tabular}
\end{table}

\begin{table}[htbp!]
\centering
\caption{ Performance Test Case Comparisons}
      \begin{tabular}{lccc}
      \hline\hline
          Test Case & GMAT 10000xTime/Step & STK 10000xTime/Step & FF 10000xTime/Step \\
         \hline
         %---New Row---%
         ISSPM-SRP-output.m & 3.727595511 & 3.05372255 & NaN \\
         %---New Row---%
         ISS-1PM-noOutput.m & 2.330302317 & 0.9223173727 & NaN \\
         %---New Row---%
         ISS-1PM-output.m & 3.673883202 & 2.25961283 & NaN \\
         %---New Row---%
         ISS-AllPlanets-noOutput.m & 10.26602609 & 4.651536018 & NaN \\
         %---New Row---%
         ISS-AllPlanets-output.m & 11.82726129 & 6.035282657 & NaN \\
         %---New Row---%
         ISS-Drag-noOutput.m & 6.416339609 & 1.741039086 & NaN \\
         %---New Row---%
         ISS-Drag-output.m & 8.05014946 & 3.101959229 & NaN \\
         %---New Row---%
         ISS-NSG-noOutput.m & 3.455671717 & 3.778008335 & NaN \\
         %---New Row---%
         ISS-NSG-output.m & 4.957091456 & 5.140125387 & NaN \\
         %---New Row---%
         ISS-SRP-noOutput.m & 4.07785136 & 1.699636154 & NaN \\
         %---New Row---%
         ISS-SRP-output.m & 5.623643858 & 3.05372255 & NaN \\
         %---New Row---%
         Mars1PM-1PM-noOutput.m & NaN & NaN & NaN \\
         %---New Row---%
         Mars1PM-1PM-output.m & NaN & NaN & NaN \\
         %---New Row---%
         Mars1PM-AllPlanets-noOutput.m & NaN & NaN & NaN \\
         %---New Row---%
         Mars1PM-AllPlanets-output.m & NaN & NaN & NaN \\
         %---New Row---%
         Mars1-1PM-noOutput.m & NaN & NaN & NaN \\
         %---New Row---%
         Mars1-1PM-output.m & NaN & NaN & NaN \\
         %---New Row---%
         Mars1-AllPlanets-noOutput.m & NaN & NaN & NaN \\
         %---New Row---%
         Mars1-AllPlanets-output.m & NaN & NaN & NaN \\
         %---New Row---%
         MolniyaPM-1PM-noOutput.m & 1.809412186 & 0.9989362548 & NaN \\
         %---New Row---%
         MolniyaPM-1PM-output.m & 3.393966825 & 2.46660942 & NaN \\
         %---New Row---%
         MolniyaPM-AllPlanets-noOutput.m & 9.698488317 & 4.708523152 & NaN \\
         %---New Row---%
         MolniyaPM-AllPlanets-output.m & 11.12982454 & 6.166077679 & NaN \\
         %---New Row---%
         MolniyaPM-SRP-noOutput.m & 2.835504748 & 1.767514792 & NaN \\
         %---New Row---%
         MolniyaPM-SRP-output.m & 5.211072086 & 3.22023354 & NaN \\
         %---New Row---%
         Molniya-1PM-noOutput.m & 4.258395782 & 0.9989362548 & NaN \\
         %---New Row---%
         Molniya-1PM-output.m & 6.510579531 & 2.46660942 & NaN \\
         %---New Row---%
         Molniya-AllPlanets-noOutput.m & 12.23804615 & 4.708523152 & NaN \\
         %---New Row---%
         Molniya-AllPlanets-output.m & 13.61546338 & 6.166077679 & NaN \\
         %---New Row---%
         Molniya-Drag-noOutput.m & 7.904138115 & 1.305483405 & NaN \\
         %---New Row---%
         Molniya-Drag-output.m & 9.090493606 & 2.762879112 & NaN \\
         %---New Row---%
         Molniya-NSG-noOutput.m & 6.50081572 & 3.759824535 & NaN \\
         %---New Row---%
         Molniya-NSG-output.m & 7.089750734 & 5.213419385 & NaN \\
         %---New Row---%
         Molniya-SRP-noOutput.m & 5.574052761 & 1.767514792 & NaN \\
         %---New Row---%
         Molniya-SRP-output.m & 6.668421333 & 3.22023354 & NaN \\
         %---New Row---%
         MoonPM-1PM-noOutput.m & NaN & NaN & NaN \\
         %---New Row---%
         MoonPM-1PM-output.m & NaN & NaN & NaN \\
         %---New Row---%
         Moon-1PM-noOutput.m & NaN & NaN & NaN \\
         %---New Row---%
         Moon-1PM-output.m & NaN & NaN & NaN \\
         %---New Row---%
         SunSyncPM-1PM-noOutput.m & 0.8684386917 & 0.8905438109 & NaN \\
         %---New Row---%
         SunSyncPM-1PM-output.m & 2.442609542 & 2.25524112 & NaN \\
         %---New Row---%
         SunSyncPM-AllPlanets-noOutput.m & 8.630934134 & 4.581268277 & NaN \\
         %---New Row---%
         SunSyncPM-AllPlanets-output.m & 10.24872065 & 5.949465982 & NaN \\
         %---New Row---%
         SunSyncPM-SRP-noOutput.m & 2.172100138 & 1.665132522 & NaN \\
         %---New Row---%
         SunSyncPM-SRP-output.m & 3.781204529 & 3.037495251 & NaN \\
      \hline\hline
      \label{Table: Performance2-2} 
\end{tabular}
\end{table}

\begin{table}[htbp!]
\centering
\caption{ Performance Test Case Comparisons}
      \begin{tabular}{lccc}
      \hline\hline
          Test Case & GMAT 10000xTime/Step & STK 10000xTime/Step & FF 10000xTime/Step \\
         \hline
         %---New Row---%
         SunSync-1PM-noOutput.m & 2.348181488 & 0.8905438109 & NaN \\
         %---New Row---%
         SunSync-1PM-output.m & 3.687035997 & 2.25524112 & NaN \\
         %---New Row---%
         SunSync-AllPlanets-noOutput.m & 10.24758043 & 4.581268277 & NaN \\
         %---New Row---%
         SunSync-AllPlanets-output.m & 11.84481254 & 5.949465982 & NaN \\
         %---New Row---%
         SunSync-Drag-noOutput.m & 6.382393873 & 1.712145917 & NaN \\
         %---New Row---%
         SunSync-Drag-output.m & 8.032878955 & 3.092757366 & NaN \\
         %---New Row---%
         SunSync-NSG-noOutput.m & 3.417686247 & 3.701970079 & NaN \\
         %---New Row---%
         SunSync-NSG-output.m & 5.027253917 & 5.070907558 & NaN \\
         %---New Row---%
         SunSync-SRP-noOutput.m & 4.003942303 & 1.665132522 & NaN \\
         %---New Row---%
         SunSync-SRP-output.m & 5.550127024 & 3.037495251 & NaN \\
      \hline\hline
      \label{Table: Performance2-3} 
\end{tabular}
\end{table}

\begin{table}[htbp!]
\centering
\caption{ GMAT/STK Performance Test Case Comparisons}
      \begin{tabular}{lccc}
      \hline\hline
          Test Case & \% Time to Run & \% Int. Steps & \% Time Per Step \\
         \hline
         %---New Row---%
         DeepSpacePM-AllPlanets-noOutput.m & NaN & NaN & NaN \\
         %---New Row---%
         DeepSpacePM-AllPlanets-output.m & NaN & NaN & NaN \\
         %---New Row---%
         DeepSpace-AllPlanets-noOutput.m & NaN & NaN & NaN \\
         %---New Row---%
         DeepSpace-AllPlanets-output.m & NaN & NaN & NaN \\
         %---New Row---%
         EML2PM-AllPlanets-noOutput.m & NaN & NaN & NaN \\
         %---New Row---%
         EML2PM-AllPlanets-output.m & NaN & NaN & NaN \\
         %---New Row---%
         EML2-AllPlanets-noOutput.m & NaN & NaN & NaN \\
         %---New Row---%
         EML2-AllPlanets-output.m & NaN & NaN & NaN \\
         %---New Row---%
         GEOPM-1PM-noOutput.m & 259.9267559 & 100 & 259.9267559 \\
         %---New Row---%
         GEOPM-1PM-output.m & 178.9126742 & 100 & 178.9126742 \\
         %---New Row---%
         GEOPM-AllPlanets-noOutput.m & 228.3760917 & 100 & 228.3760917 \\
         %---New Row---%
         GEOPM-AllPlanets-output.m & 204.4711672 & 100 & 204.4711672 \\
         %---New Row---%
         GEOPM-SRP-noOutput.m & 210.8549391 & 100 & 210.8549391 \\
         %---New Row---%
         GEOPM-SRP-output.m & 170.1252735 & 100 & 170.1252735 \\
         %---New Row---%
         GEO-1PM-noOutput.m & 558.3407415 & 100 & 558.3407415 \\
         %---New Row---%
         GEO-1PM-output.m & 447.5013412 & 100 & 447.5013412 \\
         %---New Row---%
         GEO-AllPlanets-noOutput.m & 316.2348076 & 100 & 316.2348076 \\
         %---New Row---%
         GEO-AllPlanets-output.m & 281.0330399 & 100 & 281.0330399 \\
         %---New Row---%
         GEO-Drag-noOutput.m & 843.4941054 & 100 & 843.4941054 \\
         %---New Row---%
         GEO-Drag-output.m & 456.2146691 & 100 & 456.2146691 \\
         %---New Row---%
         GEO-NSG-noOutput.m & 277.0018401 & 100 & 277.0018401 \\
         %---New Row---%
         GEO-NSG-output.m & 203.3652282 & 100 & 203.3652282 \\
         %---New Row---%
         GEO-SRP-noOutput.m & 594.7920731 & 100 & 594.7920731 \\
         %---New Row---%
         GEO-SRP-output.m & 333.271102 & 100 & 333.271102 \\
         %---New Row---%
         GPSPM-1PM-noOutput.m & 244.8833561 & 100 & 244.8833561 \\
         %---New Row---%
         GPSPM-1PM-output.m & 139.0700496 & 100 & 139.0700496 \\
         %---New Row---%
         GPSPM-AllPlanets-noOutput.m & 225.799743 & 100 & 225.799743 \\
         %---New Row---%
         GPSPM-AllPlanets-output.m & 198.2575216 & 100 & 198.2575216 \\
         %---New Row---%
         GPSPM-SRP-noOutput.m & 261.1777744 & 136.36679 & 191.5259386 \\
         %---New Row---%
         GPSPM-SRP-output.m & 181.8799373 & 136.36679 & 133.3755362 \\
         %---New Row---%
         GPS-1PM-noOutput.m & 535.6137164 & 100 & 535.6137164 \\
         %---New Row---%
         GPS-1PM-output.m & 341.0846731 & 100 & 341.0846731 \\
         %---New Row---%
         GPS-AllPlanets-noOutput.m & 312.2484345 & 100 & 312.2484345 \\
         %---New Row---%
         GPS-AllPlanets-output.m & 264.1379741 & 100 & 264.1379741 \\
         %---New Row---%
         GPS-Drag-noOutput.m & 816.0572608 & 100 & 816.0572608 \\
         %---New Row---%
         GPS-Drag-output.m & 415.2954637 & 100 & 415.2954637 \\
         %---New Row---%
         GPS-NSG-noOutput.m & 213.2824939 & 100.0115701 & 213.2578198 \\
         %---New Row---%
         GPS-NSG-output.m & 198.5924123 & 100.0115701 & 198.5694377 \\
         %---New Row---%
         GPS-SRP-noOutput.m & 581.777838 & 135.4301573 & 429.5777615 \\
         %---New Row---%
         GPS-SRP-output.m & 363.2531878 & 135.4301573 & 268.2217869 \\
         %---New Row---%
         ISSPM-1PM-noOutput.m & 95.45237251 & 101.6240633 & 93.92693956 \\
         %---New Row---%
         ISSPM-1PM-output.m & 109.8390356 & 101.6240633 & 108.0836881 \\
         %---New Row---%
         ISSPM-AllPlanets-noOutput.m & 192.1415705 & 101.6187567 & 189.0808122 \\
         %---New Row---%
         ISSPM-AllPlanets-output.m & 174.0931023 & 101.6187567 & 171.3198507 \\
         %---New Row---%
         ISSPM-SRP-noOutput.m & 224.6036251 & 175.6719342 & 127.8540173 \\
      \hline\hline
      \label{Table: Performance3-1} 
\end{tabular}
\end{table}

\begin{table}[htbp!]
\centering
\caption{ GMAT/STK Performance Test Case Comparisons}
      \begin{tabular}{lccc}
      \hline\hline
          Test Case & \% Time to Run & \% Int. Steps & \% Time Per Step \\
         \hline
         %---New Row---%
         ISSPM-SRP-output.m & 214.4379204 & 175.6719342 & 122.0672621 \\
         %---New Row---%
         ISS-1PM-noOutput.m & 256.7606306 & 101.6240633 & 252.6573158 \\
         %---New Row---%
         ISS-1PM-output.m & 165.2296067 & 101.6240633 & 162.5890575 \\
         %---New Row---%
         ISS-AllPlanets-noOutput.m & 224.274477 & 101.6187567 & 220.701851 \\
         %---New Row---%
         ISS-AllPlanets-output.m & 199.1408946 & 101.6187567 & 195.9686391 \\
         %---New Row---%
         ISS-Drag-noOutput.m & 376.0407585 & 102.0366281 & 368.5350696 \\
         %---New Row---%
         ISS-Drag-output.m & 264.8036438 & 102.0366281 & 259.5182227 \\
         %---New Row---%
         ISS-NSG-noOutput.m & 125.8025455 & 137.5370997 & 91.4680808 \\
         %---New Row---%
         ISS-NSG-output.m & 132.6395624 & 137.5370997 & 96.43911546 \\
         %---New Row---%
         ISS-SRP-noOutput.m & 421.4056159 & 175.6405903 & 239.9249599 \\
         %---New Row---%
         ISS-SRP-output.m & 323.4544431 & 175.6405903 & 184.1570007 \\
         %---New Row---%
         Mars1PM-1PM-noOutput.m & NaN & NaN & NaN \\
         %---New Row---%
         Mars1PM-1PM-output.m & NaN & NaN & NaN \\
         %---New Row---%
         Mars1PM-AllPlanets-noOutput.m & NaN & NaN & NaN \\
         %---New Row---%
         Mars1PM-AllPlanets-output.m & NaN & NaN & NaN \\
         %---New Row---%
         Mars1-1PM-noOutput.m & NaN & NaN & NaN \\
         %---New Row---%
         Mars1-1PM-output.m & NaN & NaN & NaN \\
         %---New Row---%
         Mars1-AllPlanets-noOutput.m & NaN & NaN & NaN \\
         %---New Row---%
         Mars1-AllPlanets-output.m & NaN & NaN & NaN \\
         %---New Row---%
         MolniyaPM-1PM-noOutput.m & 221.3788122 & 122.2183222 & 181.1338989 \\
         %---New Row---%
         MolniyaPM-1PM-output.m & 168.1680641 & 122.2183222 & 137.596443 \\
         %---New Row---%
         MolniyaPM-AllPlanets-noOutput.m & 251.8646623 & 122.2778803 & 205.9772885 \\
         %---New Row---%
         MolniyaPM-AllPlanets-output.m & 220.7126513 & 122.2778803 & 180.5008811 \\
         %---New Row---%
         MolniyaPM-SRP-noOutput.m & 244.8721049 & 152.6412776 & 160.4232542 \\
         %---New Row---%
         MolniyaPM-SRP-output.m & 247.0083897 & 152.6412776 & 161.8228002 \\
         %---New Row---%
         Molniya-1PM-noOutput.m & 521.0082077 & 122.2183222 & 426.2930453 \\
         %---New Row---%
         Molniya-1PM-output.m & 322.593476 & 122.2183222 & 263.9485391 \\
         %---New Row---%
         Molniya-AllPlanets-noOutput.m & 317.8156492 & 122.2778803 & 259.9126256 \\
         %---New Row---%
         Molniya-AllPlanets-output.m & 270.0047077 & 122.2778803 & 220.8123882 \\
         %---New Row---%
         Molniya-Drag-noOutput.m & 749.2361921 & 123.7472576 & 605.4568051 \\
         %---New Row---%
         Molniya-Drag-output.m & 407.1563062 & 123.7472576 & 329.0224884 \\
         %---New Row---%
         Molniya-NSG-noOutput.m & 213.2097251 & 123.3123949 & 172.9021038 \\
         %---New Row---%
         Molniya-NSG-output.m & 167.6930394 & 123.3123949 & 135.9904165 \\
         %---New Row---%
         Molniya-SRP-noOutput.m & 482.7270723 & 153.0712531 & 315.3610247 \\
         %---New Row---%
         Molniya-SRP-output.m & 316.9781312 & 153.0712531 & 207.0788112 \\
         %---New Row---%
         MoonPM-1PM-noOutput.m & NaN & NaN & NaN \\
         %---New Row---%
         MoonPM-1PM-output.m & NaN & NaN & NaN \\
         %---New Row---%
         Moon-1PM-noOutput.m & NaN & NaN & NaN \\
         %---New Row---%
         Moon-1PM-output.m & NaN & NaN & NaN \\
         %---New Row---%
         SunSyncPM-1PM-noOutput.m & 98.49366133 & 101.0007055 & 97.51779543 \\
         %---New Row---%
         SunSyncPM-1PM-output.m & 109.391978 & 101.0007055 & 108.3081326 \\
         %---New Row---%
         SunSyncPM-AllPlanets-noOutput.m & 190.2765419 & 100.9980927 & 188.3961735 \\
         %---New Row---%
         SunSyncPM-AllPlanets-output.m & 173.9822097 & 100.9980927 & 172.2628667 \\
         %---New Row---%
         SunSyncPM-SRP-noOutput.m & 227.269229 & 174.2246492 & 130.4460821 \\
         %---New Row---%
         SunSyncPM-SRP-output.m & 216.8823252 & 174.2246492 & 124.4842943 \\
      \hline\hline
      \label{Table: Performance3-2} 
\end{tabular}
\end{table}

\begin{table}[htbp!]
\centering
\caption{ GMAT/STK Performance Test Case Comparisons}
      \begin{tabular}{lccc}
      \hline\hline
          Test Case & \% Time to Run & \% Int. Steps & \% Time Per Step \\
         \hline
         %---New Row---%
         SunSync-1PM-noOutput.m & 266.3181575 & 101.0007055 & 263.6795023 \\
         %---New Row---%
         SunSync-1PM-output.m & 165.1234689 & 101.0007055 & 163.4874411 \\
         %---New Row---%
         SunSync-AllPlanets-noOutput.m & 225.9169328 & 100.9980927 & 223.6843557 \\
         %---New Row---%
         SunSync-AllPlanets-output.m & 201.0774543 & 100.9980927 & 199.0903482 \\
         %---New Row---%
         SunSync-Drag-noOutput.m & 378.3958205 & 101.5087555 & 372.7716084 \\
         %---New Row---%
         SunSync-Drag-output.m & 263.6506681 & 101.5087555 & 259.7319481 \\
         %---New Row---%
         SunSync-NSG-noOutput.m & 131.6939968 & 142.6483301 & 92.32074203 \\
         %---New Row---%
         SunSync-NSG-output.m & 141.4203213 & 142.6483301 & 99.13913554 \\
         %---New Row---%
         SunSync-SRP-noOutput.m & 419.2887231 & 174.3709665 & 240.4578764 \\
         %---New Row---%
         SunSync-SRP-output.m & 318.6115313 & 174.3709665 & 182.7205169 \\
      \hline\hline
      \label{Table: Performance3-3} 
\end{tabular}
\end{table}

\begin{table}[htbp!]
\centering
\caption{ GMAT/FF Performance Test Case Comparisons}
      \begin{tabular}{lccc}
      \hline\hline
          Test Case & \% Time to Run & \% Int. Steps & \% Time Per Step \\
         \hline
         %---New Row---%
         DeepSpacePM-AllPlanets-noOutput.m & NaN & NaN & NaN \\
         %---New Row---%
         DeepSpacePM-AllPlanets-output.m & NaN & NaN & NaN \\
         %---New Row---%
         DeepSpace-AllPlanets-noOutput.m & NaN & NaN & NaN \\
         %---New Row---%
         DeepSpace-AllPlanets-output.m & NaN & NaN & NaN \\
         %---New Row---%
         EML2PM-AllPlanets-noOutput.m & NaN & NaN & NaN \\
         %---New Row---%
         EML2PM-AllPlanets-output.m & NaN & NaN & NaN \\
         %---New Row---%
         EML2-AllPlanets-noOutput.m & NaN & NaN & NaN \\
         %---New Row---%
         EML2-AllPlanets-output.m & NaN & NaN & NaN \\
         %---New Row---%
         GEOPM-1PM-noOutput.m & NaN & NaN & NaN \\
         %---New Row---%
         GEOPM-1PM-output.m & NaN & NaN & NaN \\
         %---New Row---%
         GEOPM-AllPlanets-noOutput.m & NaN & NaN & NaN \\
         %---New Row---%
         GEOPM-AllPlanets-output.m & NaN & NaN & NaN \\
         %---New Row---%
         GEOPM-SRP-noOutput.m & NaN & NaN & NaN \\
         %---New Row---%
         GEOPM-SRP-output.m & NaN & NaN & NaN \\
         %---New Row---%
         GEO-1PM-noOutput.m & NaN & NaN & NaN \\
         %---New Row---%
         GEO-1PM-output.m & NaN & NaN & NaN \\
         %---New Row---%
         GEO-AllPlanets-noOutput.m & NaN & NaN & NaN \\
         %---New Row---%
         GEO-AllPlanets-output.m & NaN & NaN & NaN \\
         %---New Row---%
         GEO-Drag-noOutput.m & NaN & NaN & NaN \\
         %---New Row---%
         GEO-Drag-output.m & NaN & NaN & NaN \\
         %---New Row---%
         GEO-NSG-noOutput.m & NaN & NaN & NaN \\
         %---New Row---%
         GEO-NSG-output.m & NaN & NaN & NaN \\
         %---New Row---%
         GEO-SRP-noOutput.m & NaN & NaN & NaN \\
         %---New Row---%
         GEO-SRP-output.m & NaN & NaN & NaN \\
         %---New Row---%
         GPSPM-1PM-noOutput.m & NaN & NaN & NaN \\
         %---New Row---%
         GPSPM-1PM-output.m & NaN & NaN & NaN \\
         %---New Row---%
         GPSPM-AllPlanets-noOutput.m & NaN & NaN & NaN \\
         %---New Row---%
         GPSPM-AllPlanets-output.m & NaN & NaN & NaN \\
         %---New Row---%
         GPSPM-SRP-noOutput.m & NaN & NaN & NaN \\
         %---New Row---%
         GPSPM-SRP-output.m & NaN & NaN & NaN \\
         %---New Row---%
         GPS-1PM-noOutput.m & NaN & NaN & NaN \\
         %---New Row---%
         GPS-1PM-output.m & NaN & NaN & NaN \\
         %---New Row---%
         GPS-AllPlanets-noOutput.m & NaN & NaN & NaN \\
         %---New Row---%
         GPS-AllPlanets-output.m & NaN & NaN & NaN \\
         %---New Row---%
         GPS-Drag-noOutput.m & NaN & NaN & NaN \\
         %---New Row---%
         GPS-Drag-output.m & NaN & NaN & NaN \\
         %---New Row---%
         GPS-NSG-noOutput.m & NaN & NaN & NaN \\
         %---New Row---%
         GPS-NSG-output.m & NaN & NaN & NaN \\
         %---New Row---%
         GPS-SRP-noOutput.m & NaN & NaN & NaN \\
         %---New Row---%
         GPS-SRP-output.m & NaN & NaN & NaN \\
         %---New Row---%
         ISSPM-1PM-noOutput.m & NaN & NaN & NaN \\
         %---New Row---%
         ISSPM-1PM-output.m & NaN & NaN & NaN \\
         %---New Row---%
         ISSPM-AllPlanets-noOutput.m & NaN & NaN & NaN \\
         %---New Row---%
         ISSPM-AllPlanets-output.m & NaN & NaN & NaN \\
         %---New Row---%
         ISSPM-SRP-noOutput.m & NaN & NaN & NaN \\
      \hline\hline
      \label{Table: Performance4-1} 
\end{tabular}
\end{table}

\begin{table}[htbp!]
\centering
\caption{ GMAT/FF Performance Test Case Comparisons}
      \begin{tabular}{lccc}
      \hline\hline
          Test Case & \% Time to Run & \% Int. Steps & \% Time Per Step \\
         \hline
         %---New Row---%
         ISSPM-SRP-output.m & NaN & NaN & NaN \\
         %---New Row---%
         ISS-1PM-noOutput.m & NaN & NaN & NaN \\
         %---New Row---%
         ISS-1PM-output.m & NaN & NaN & NaN \\
         %---New Row---%
         ISS-AllPlanets-noOutput.m & NaN & NaN & NaN \\
         %---New Row---%
         ISS-AllPlanets-output.m & NaN & NaN & NaN \\
         %---New Row---%
         ISS-Drag-noOutput.m & NaN & NaN & NaN \\
         %---New Row---%
         ISS-Drag-output.m & NaN & NaN & NaN \\
         %---New Row---%
         ISS-NSG-noOutput.m & NaN & NaN & NaN \\
         %---New Row---%
         ISS-NSG-output.m & NaN & NaN & NaN \\
         %---New Row---%
         ISS-SRP-noOutput.m & NaN & NaN & NaN \\
         %---New Row---%
         ISS-SRP-output.m & NaN & NaN & NaN \\
         %---New Row---%
         Mars1PM-1PM-noOutput.m & NaN & NaN & NaN \\
         %---New Row---%
         Mars1PM-1PM-output.m & NaN & NaN & NaN \\
         %---New Row---%
         Mars1PM-AllPlanets-noOutput.m & NaN & NaN & NaN \\
         %---New Row---%
         Mars1PM-AllPlanets-output.m & NaN & NaN & NaN \\
         %---New Row---%
         Mars1-1PM-noOutput.m & NaN & NaN & NaN \\
         %---New Row---%
         Mars1-1PM-output.m & NaN & NaN & NaN \\
         %---New Row---%
         Mars1-AllPlanets-noOutput.m & NaN & NaN & NaN \\
         %---New Row---%
         Mars1-AllPlanets-output.m & NaN & NaN & NaN \\
         %---New Row---%
         MolniyaPM-1PM-noOutput.m & NaN & NaN & NaN \\
         %---New Row---%
         MolniyaPM-1PM-output.m & NaN & NaN & NaN \\
         %---New Row---%
         MolniyaPM-AllPlanets-noOutput.m & NaN & NaN & NaN \\
         %---New Row---%
         MolniyaPM-AllPlanets-output.m & NaN & NaN & NaN \\
         %---New Row---%
         MolniyaPM-SRP-noOutput.m & NaN & NaN & NaN \\
         %---New Row---%
         MolniyaPM-SRP-output.m & NaN & NaN & NaN \\
         %---New Row---%
         Molniya-1PM-noOutput.m & NaN & NaN & NaN \\
         %---New Row---%
         Molniya-1PM-output.m & NaN & NaN & NaN \\
         %---New Row---%
         Molniya-AllPlanets-noOutput.m & NaN & NaN & NaN \\
         %---New Row---%
         Molniya-AllPlanets-output.m & NaN & NaN & NaN \\
         %---New Row---%
         Molniya-Drag-noOutput.m & NaN & NaN & NaN \\
         %---New Row---%
         Molniya-Drag-output.m & NaN & NaN & NaN \\
         %---New Row---%
         Molniya-NSG-noOutput.m & NaN & NaN & NaN \\
         %---New Row---%
         Molniya-NSG-output.m & NaN & NaN & NaN \\
         %---New Row---%
         Molniya-SRP-noOutput.m & NaN & NaN & NaN \\
         %---New Row---%
         Molniya-SRP-output.m & NaN & NaN & NaN \\
         %---New Row---%
         MoonPM-1PM-noOutput.m & NaN & NaN & NaN \\
         %---New Row---%
         MoonPM-1PM-output.m & NaN & NaN & NaN \\
         %---New Row---%
         Moon-1PM-noOutput.m & NaN & NaN & NaN \\
         %---New Row---%
         Moon-1PM-output.m & NaN & NaN & NaN \\
         %---New Row---%
         SunSyncPM-1PM-noOutput.m & NaN & NaN & NaN \\
         %---New Row---%
         SunSyncPM-1PM-output.m & NaN & NaN & NaN \\
         %---New Row---%
         SunSyncPM-AllPlanets-noOutput.m & NaN & NaN & NaN \\
         %---New Row---%
         SunSyncPM-AllPlanets-output.m & NaN & NaN & NaN \\
         %---New Row---%
         SunSyncPM-SRP-noOutput.m & NaN & NaN & NaN \\
         %---New Row---%
         SunSyncPM-SRP-output.m & NaN & NaN & NaN \\
      \hline\hline
      \label{Table: Performance4-2} 
\end{tabular}
\end{table}

\begin{table}[htbp!]
\centering
\caption{ GMAT/FF Performance Test Case Comparisons}
      \begin{tabular}{lccc}
      \hline\hline
          Test Case & \% Time to Run & \% Int. Steps & \% Time Per Step \\
         \hline
         %---New Row---%
         SunSync-1PM-noOutput.m & NaN & NaN & NaN \\
         %---New Row---%
         SunSync-1PM-output.m & NaN & NaN & NaN \\
         %---New Row---%
         SunSync-AllPlanets-noOutput.m & NaN & NaN & NaN \\
         %---New Row---%
         SunSync-AllPlanets-output.m & NaN & NaN & NaN \\
         %---New Row---%
         SunSync-Drag-noOutput.m & NaN & NaN & NaN \\
         %---New Row---%
         SunSync-Drag-output.m & NaN & NaN & NaN \\
         %---New Row---%
         SunSync-NSG-noOutput.m & NaN & NaN & NaN \\
         %---New Row---%
         SunSync-NSG-output.m & NaN & NaN & NaN \\
         %---New Row---%
         SunSync-SRP-noOutput.m & NaN & NaN & NaN \\
         %---New Row---%
         SunSync-SRP-output.m & NaN & NaN & NaN \\
      \hline\hline
      \label{Table: Performance4-3} 
\end{tabular}
\end{table}

