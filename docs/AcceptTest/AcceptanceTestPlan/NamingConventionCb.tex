This section describes the naming convention for central body
dependent parameter scripts and output reports. The naming
convention consists of a case sensitive ordered series of option
strings, separated by underscores (\_ ). Currently, options are
allowed for the following fields, and will be present in the file
name:
\begin{enumerate}
  \item \emph{tool} - The tool used to generate the test case.
  \item \emph{traj} - The trajectory to use.  This includes initial conditions, physical parameters, and time
  step.
\end{enumerate}

CbParams precedes the \emph{tool} field and 2Body follows the
\emph{traj} field. The central body used can be determined based on
the \emph{traj} field. The final Cb file format is as followed:\\
CbParams\_\emph{tool}\_\emph{traj}\_2Body.report\\

The \emph{tool} field should always be the first option field. Each
field has a finite list of options, as follows (future options
should be added to this list):
\begin{enumerate}
  \item \emph{tool}\\
  \begin{tabular}{ll}
    STK  & - Satellite Toolkit HPOP or Astrogator\\
    FF   & - FreeFlyer\\
    GMAT & - General Mission Analysis Tool\\
  \end{tabular}

  \item \emph{traj}\\
  \begin{tabular}{ll}
    ISS & - leo orbit\\
    Mars1 & - eccentric low orbit\\
    Mercury1 & - eccentric low orbit\\
    Moon & - eccentric low orbit\\
    Pluto1 & - eccentric low orbit\\
    Venus1 & - eccentric low orbit\\
  \end{tabular}

NOTE:  Some test cases contain \emph{traj} variations. In this case
\emph{traj} precedes the modification. For example, if an ISS
trajectory is needed with a different Cd, \emph{traj} could be
ISSdiffCd1.

\end{enumerate}
