
To solve OD problems in GMAT, a user will have to create and
configure many objects that model physical components involved in
the measurement process such as spacecraft, receivers, clocks,
transmitters and celestial objects to name a few.  A user will also
have to configure measurement models, provide truth data or define
simulation parameters, and configure numerical solvers to determine
the best state estimate. Below we discuss how the user will provide
these and other types of information to define and solve OD problems
in GMAT.  In a sense, this portion of the document is a  short
user's guide  written before developing the software. This approach
is useful in several ways: it defines how a user will interact with
GMAT and allows analysts to provide feedback on the interfaces
before the system is written, and it gives some insight into various
designs for the the underlying software architecture.

We begin this chapter with a high level view of OD from the
perspective of a user who needs to apply GMAT to orbit
determination. We provide a categorization of models and algorithms
that a user will have control over and explain the overall
philosophy of the user interface in GMAT.  The high level
categorization explains where in the GMAT script and GUI a user
would need to go to set various types of information such as
dynamics models, process noise parameters, and measurement models to
name a few.  Next, we provide a detailed view of the primary objects
and commands required for orbit determination in GMAT.  These
sections go into detail on what fields are on each objects, and what
commands would be employed for different types of analysis.

\section{Overview and Philosophy}

To uniquely define an orbit determination problem, a user must often
provide hundreds of pieces of information ranging from clock drift
parameters, process noise characteristics, spacecraft physical
properties, ground station properties, and atmospheric modelling
parameters to name just a few. This section explains how these
pieces of information will be organized in GMAT and the philosophy
behind the organization of the data.  The goal is to provide an
intuitive interface for orbit determination analysis and explain
where a user would go to set different types of data, and why the
choices have been made.

To explain the organization of models and data, we present several
levels of detail into the organization of data and models for orbit
determination applications.  The first level view all models and
data into six categories and explains how these groups interact in
the estimation process.  These high level categories are:
measurement participants, measurements, estimators, dynamics models,
commands, and output.  Here we present which types of data are set
on each object and model. For example, you learn where clock bias
information is located and where process noise information is
located.   The second level view we present a simple estimation
example using the proposed script language and walk you through each
step to set up a batch least squares problem for a spacecraft and
one ground station.  Finally, we present a reference section that
describes all field names and settings contained in the five
categories. For example, in the reference section, you learn the
script syntax to set process noise parameters, and the different
process noise models that are available.

Let's begin with a introductory user perspective on to the primary
system components used in orbit determination.

\subsection{Introduction to System Components and Philosophy}

In GMAT, the objects and models used for orbit determination
applications can be categorized into 6 groups:  measurement
participants, measurements, dynamics models, estimators, commands,
and output.  Below we describe each of these groups and the
philosophy and functionality for each category in turn.  Fig.~
\ref{Fig:ODOrganization} shows a listing of each category with a
list of objects in models in each category.  Items in bold are
objects in GMAT that can be created using the \st{Create} command.
For example, you can create a clock named ``GPSClock" by using the
lines \st{Create Clock GPSClock}. Items that appear in italics in
Fig.~\ref{Fig:ODOrganization} are setttings and models available on
the object below which they appear.  For example, on a receiver, you
can set attenuation and bore site parameters.   Let's now look at
measurement participants in more detail.\\

\begin{figure}[htbp!]
    \begin{center}
    \begin{picture}(270,350)
    \special{psfile=ODOrganization.eps
    hscale= 85 vscale= 85 hoffset = -130 voffset = -340}
    \end{picture}
    \end{center}
    \vspace{0.2 in}
    \label{Fig:ODOrganization}
    \caption{Intermediate Level View: Objects and Commands Data and Settings }
\end{figure}


%\begin{figure}[htbp!]
%    \begin{center}
%    \begin{picture}(270,150)
%    \special{psfile=ODObjects.eps
%    hscale= 85 vscale= 85 hoffset = -130 voffset = -440}
%    \end{picture}
%    \end{center}
%    \vspace{0.2 in}
%    \label{Fig:ODObjects}
%    \caption{High Level View: Objects and Commands for OD Applications }
%\end{figure}


\textit{Measurement Participants}\\

Measurement participants are modles of physical objects that
``participate" in the process of creating a measured quantity that
is related to the spacecraft state.  Examples of measurement
participants include spacecraft, ground stations, celestial objects.
Measurement particpants can be actively or passively involved in the
measurement process. In the case of celestial navigation, a star may
serve as a passive participant.  In the a two-way range measurement,
the participants may be a ground station and a spacecraft.

In GMAT, measurement participants are created and configured
separately from measurement models (which are discussed below).  For
example, if a user requires a Doppler measurement between say Hubble
and Canberra, they first configure a spacecraft to model Hubble, and
then they configure a ground station to model Canberra. Once the
Hubble and Canberra objects are configured, the user creates a
measurement object and configures it to create a Doppler measurement
between Hubble and Canberra.  The philosophy is that measurement
participants exist independently from the measurement process, e.g.,
a spacecraft exists and advances in time even when measurements are
not being taken.  In this way, the separation of measurement
participants and measurements is similar to what happens in the real
applications.\\

\textit{Measurements}\\

In orbit determination, observed quantities used in the estimation
process are measurements of some property of electromagnetic wave
propagation among participants in the measurement process.
\cite{GTDS:89}  GMAT contains many types of measurements as shown in
Fig. \ref{Fig:ODOrganization} including Ground Station, TDRSS, GPS,
Celestial Object and Crosslink measurements.  An orbit determination
analyst may want to process simulated measurements to determine the
achievable accuracy for a given set of measurements and tracking
data schedule.  Operationally, analysts use observed measurement
values to generate best estimates for the spacecraft state. In GMAT,
a measurement object supports all of these functions by the
providing the following data:  observed measurement values (whether
they are read from a file or simulated), the computed (or expected)
value of the measurement, and the measurement partial derivatives.

Measurement objects are quite complex as a large amount of
user-provided data is required to perform the functions described
above.  For observations, you must provide GMAT with the file
format, name, and location.  When there is more data on a file than
you wish to process, you must provide data editing criteria and
information on which measurement types to include.  For computation
of the expected value you must provide the model with participants
for the measurements and a propagator to use to advance the state of
the participants.  Finally, for simulated data you must provide an
participants and a propagator along with measurement error models
such as bias and noise.\\

\textit{Estimators}\\

GMAT contains several estimators including batch, sequential, and
initial orbit determination solvers.  These algorithms solve for a
state estimate by processing measurements generated by measurement
objects configured by the user.  Hence, you must provide a list of
all measurements to include in a particular estimation run.  In
addition, the solver object is where the user specifies which
parameters are to be treated as solve-for and consider quantities.
Further data that is provided on the solver object is shown in Fig.
\ref{Fig:ODOrganization} and includes process noise parameters and
convergence tolerances to name a few.

\subsection{Intermediate Level View of System Components}


%To configure a measurement model, the measurement participants must
%be created and appropriately configured.   Fields on the Measurement
%Model object allow the user to define many different types of
%measurements given the specified list of participants.   If observed
%measurements are available from a standard file format, the user
%configures the measurement Model to read data from the desired file.
%In this case, the user must set object Ids on the measuement
%participants to match the Ids on the measurement file.  GMAT uses
%the file format and measurement Ids to determine how computed values
%are to be calculated.
%


\verbatiminput{GSMeasurement.script}
\verbatiminput{BatchLeastSquares.script}
\verbatiminput{ExtendedKalmanFilter.script}

%Estimators
%
%The third component required for an estimation problem is a solver.
%The user will have many solvers to choose from including but not
%limited to
%
%?   Initial orbit determination ?   Batch (least squares, other) ?
%Filters (SRIF, EKF, UKF)
%
%In GMAT the solver is a relatively simple object compared to the
%measurement model and the measurement  participants.  The job of the
%solver is to query configured measurement models for the observed
%and computed values  and the partial derivatives  and use this
%information  to generate state estimates.  Hence,  the estimator in
%GMAT knows little about the details of the measurement model
%computations or the configurations of the participants.
%
%A sample script segment for a BatchLeastSquare he user will specify
%the SolveFor and Consider parameters on the estimator, along with
%what measurements they would like the solver to process.
%
%%Create BatchLeastSquares BLSE BLSI.Measurements = BLSI.Propagator =
%%BLSE.SolveFor = BLSE.Consider = BLSE.SolutionEpoch =
%%BLSE.AbsoluteConvergenceTol = BLSE.Propagator =
%%
%%Create ExtendedKalmanFilter EKF EKF.Measurements = EKF.Propagator =
%%EKF.SolveFor = EKF.Consider = EKF.SolutionEpoch = EKF.Propagator =
%%EKF.ProcessNoiseModel = EKF.Smoothing = Dynamics Models
%
%Dynamics Models ?   Participant dynamics ?   Variational equations ?
%Process Noise
%
%Commands and Application Control
%
%Application Control Modes ?   RunEstimator ?   SimulateData ?
%RunEstimatorSequence
