\chapter{Creating Spacecraft}

Epoch, State, Ballistic and Mass Properties, Input Coordinate
System,


\noindent\begin{ScriptType} \noindent
Create Spacecraft Sat1\\
Sat1.Epoch.TAIGregorian  = 01 Jan 2000\\
12:00:00.000; \\
Sat1.SMA        = 8000; \\
Sat1.ECC        = .01;\\
Sat1.INC        = 28.5;\\
Sat1.AOP        = 0;\\
Sat1.RAAN = 90;\\
Sat1.TA         = 180;\\     %  Use M for mean anomaly, and E for eccentric anomaly
Sat1.Cd         = 2.0;\\
Sat1.Cr         = 1.4;\\
Sat1.DragArea   = 1;\\
Sat1.SRPArea    = 1;\\
Sat1.DryMass    = 100; 
\end{ScriptType}


\section{Spacecraft Epoch}

Alternative ways of Defining a Spacecraft's Epoch

Assuming a reference reference epoch for modified JD as MJD = JD -
2430000, the following epochs are all equivalent.

\noindent \% TAI in Modified Julian Date Format\\
 \begin{ScriptType}
Sat1.Epoch.TAIModJulian = 21545.0003703704 \end{ScriptType}

\noindent \% TAI in Gregorian Date Format\\
\begin{ScriptType} Sat1.Epoch.TAIGregorian  = 01 Jan
2000 12:00:32.0000;\end{ScriptType}

\noindent \% UTC in Modified Julian Date Format\\
  \begin{ScriptType}
 Sat1.Epoch.UTCModJulian  =
21545.00000; \end{ScriptType}

\noindent \% UTC in Gregorian Date Format\\
\begin{ScriptType} Sat1.Epoch.UTCGregorian  = 01 Jan
2000 12:00:00.000;  \end{ScriptType}

\section{Spacecraft Orbit State}

\%  Example: Creating a spacecraft using  the
Cartesian State\\
\begin{ScriptType}
Create Spacecraft Sat1;\\\
Sat1.X = 7082.960306079;\\
Sat1.Y = 7.052885901083;\\
Sat1.Z = -48.94363747128;\\
Sat1.VX = 0.05237087208446;\\
Sat1.VY = -1.069925309259;\\
Sat1.VZ = 7.424767278548;
\end{ScriptType}

\%  Example: Cartesian State: Method 2 \\
\begin{ScriptType}
Create Spacecraft Sat1;\\
Sat1.StateType = Cartesian;\\
Sat1.Element1 = 7000; \\   % X
Sat1.Element2 = 0.0;  \\  % Y
Sat1.Element3 = 0.0;  \\   % Z
Sat1.Element4 = 0.0;  \\   % VX
Sat1.Element5 = 8.0;  \\   % VY
Sat1.Element6 = 0.0;  \\   % VZ
\end{ScriptType}

\noindent \%  Example: Creating a spacecraft using  the
Keplerian Elements\\
\begin{ScriptType}
Create Spacecraft Sat1;\\
Sat1.SMA        = 8000;\\
Sat1.ECC        = .01;\\
Sat1.INC        = 28.5;\\
Sat1.AOP = 0; \\
Sat1.RAAN = 90;\\
Sat1.TA         = 180; \\    %  Use M for mean anomaly, and E for eccentric anomaly
\end{ScriptType}

\noindent \%  Keplerian Elements: Method 2\\
\begin{ScriptType}
Create Spacecraft Sat1;\\
Sat1.StateType   = Keplerian;\\
Sat1.AnomalyType = TA; \\   %  Alternatives:  MA, EA
Sat1.Element1 = 8000;   \\  % SMA
Sat1.Element2 = .01;   \\  % ECC
Sat1.Element3 = 28.5;  \\   % INC
Sat1.Element4 = 0;     \\   % RAAN
Sat1.Element5 = 90;    \\   % AOP
Sat1.Element6 = 180;   \\   % Anomaly
\end{ScriptType}
