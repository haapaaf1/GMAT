% $Id: CommandWrapperExample.tex,v 1.2 2008/02/16 00:54:58 dconway Exp $
\chapter{\label{chapter:CommandWrapperExample}Command Implementation: Sample
Code\index{Wrappers!Example}}
\chapauthor{Darrel J. Conway}{Thinking Systems, Inc.}

The wrapper classes described in Chapter~\ref{chapter:Commands} encapsulate the data used by
commands that need information at the single data element level, giving several disparate types a
common interface used during operation in the GMAT Sandbox.  This appendix provides sample code for
the usage of these wrappers, starting with sample setup code, and proceeding through
initialization, execution, and finalization.  The Vary command, used by the Solvers, is used to
demonstrate these steps.

\section{Sample Usage: The Maneuver Command}

Maneuver commands are used to apply impulsive velocity changes to a spacecraft.  They take the form

\begin{quote}
\begin{verbatim}
Maneuver burn1(sat1)
\end{verbatim}
\end{quote}

\noindent where burn1 is an ImpulsiveBurn object specifying the components of the velocity change
and sat1 is the spacecraft that receives the velocity change.  The Maneuver command overrides
InterpretAction using the following code:

\begin{quote}
\begin{verbatim}
//------------------------------------------------------------------------------
//  bool InterpretAction()
//------------------------------------------------------------------------------
/**
 * Parses the command string and builds the corresponding command structures.
 *
 * The Maneuver command has the following syntax:
 *
 *     Maneuver burn1(sat1);
 *
 * where burn1 is an ImpulsiveBurn used to perform the maneuver, and sat1 is the
 * name of the spacecraft that is maneuvered.  This method breaks the script
 * line into the corresponding pieces, and stores the name of the ImpulsiveBurn
 * and the Spacecraft so they can be set to point to the correct objects during
 * initialization.
 */
//------------------------------------------------------------------------------
bool Maneuver::InterpretAction()
{
   StringArray chunks = InterpretPreface();

   // Find and set the burn object name ...
   StringArray currentChunks = parser.Decompose(chunks[1], "()", false);
   SetStringParameter(burnNameID, currentChunks[0]);

   // ... and the spacecraft that is maneuvered
   currentChunks = parser.SeparateBrackets(currentChunks[1], "()", ", ");
   SetStringParameter(satNameID, currentChunks[0]);

   return true;
}
\end{verbatim}
\end{quote}

\noindent The maneuver command works with GMAT objects -- specifically ImpulsiveBurn objects and
Spacecraft -- but does not require the usage of the data wrapper classes.  The next example, the
Vary command, demonstrates usage of the data wrapper classes to set numeric values.

\section{Sample Usage: The Vary Command}

The Vary command has a much more complicated syntax than does the Maneuver command.  Vary commands
take the form
\begin{quote}
\begin{verbatim}
Vary myDC(Burn1.V = 0.5, {Pert = 0.0001, MaxStep = 0.05, Lower = 0.0, ...
    Upper = 3.14159, AdditiveScaleFactor = 1.5, MultiplicativeScaleFactor = 0.5);
\end{verbatim}
\end{quote}

\noindent The resulting InterpretAction method is a bit more complicated:

\begin{quote}
\begin{verbatim}
//------------------------------------------------------------------------------
//  void Vary::InterpretAction()
//------------------------------------------------------------------------------
/**
 * Parses the command string and builds the corresponding command structures.
 *
 * The Vary command has the following syntax:
 *
 *     Vary myDC(Burn1.V = 0.5, {Pert = 0.0001, MaxStep = 0.05, ...
 *               Lower = 0.0, Upper = 3.14159);
 *
 * where
 *
 * 1. myDC is a Solver used to Vary a set of variables to achieve the
 * corresponding goals,
 * 2. Burn1.V is the parameter that is varied, and
 * 3. The settings in the braces specify features about how the variable can
 * be changed.
 *
 * This method breaks the script line into the corresponding pieces, and stores
 * the name of the Solver so it can be set to point to the correct object
 * during initialization.
 */
//------------------------------------------------------------------------------
bool Vary::InterpretAction()
{
   // Clean out any old data
   wrapperObjectNames.clear();
   ClearWrappers();

   StringArray chunks = InterpretPreface();

   // Find and set solver object name --the only setting in Vary not in a wrapper
   StringArray currentChunks = parser.Decompose(chunks[1], "()", false);
   SetStringParameter(SOLVER_NAME, currentChunks[0]);

   // The remaining text in the instruction is the variable definition and
   // parameters, all contained in currentChunks[1].  Deal with those next.
   currentChunks = parser.SeparateBrackets(currentChunks[1], "()", ", ");

   // First chunk is the variable and initial value
   std::string lhs, rhs;
   if (!SeparateEquals(currentChunks[0], lhs, rhs))
      // Variable takes default initial value
      rhs = "0.0";

   variableName = lhs;
   variableId = -1;

   variableValueString = rhs;
   initialValueName = rhs;

   // Now deal with the settable parameters
   currentChunks = parser.SeparateBrackets(currentChunks[1], "{}", ", ");

   for (StringArray::iterator i = currentChunks.begin();
        i != currentChunks.end(); ++i)
   {
      SeparateEquals(*i, lhs, rhs);
      if (IsSettable(lhs))
         SetStringParameter(lhs, rhs);
      else
         throw CommandException("Setting \"" + lhs +
            "\" is missing a value required for a " + typeName +
            " command.\nSee the line \"" + generatingString +"\"\n");
   }

   MessageInterface::ShowMessage("InterpretAction succeeded!\n");
   return true;
}
\end{verbatim}
\end{quote}
