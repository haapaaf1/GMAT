\chapter{Time } \label{Ch:TimeSystems} \index{Time systems}

Time is the primary independent variable in GMAT.  Time is used in
integrating the equations of motion, and calculating planetary
ephemerides, the orientations of planets and moons, and atmospheric
density among others.  GMAT uses three types of time systems
depending on the type of calculations being performed: universal
time systems based on the Earth's rotation with respect to the Sun;
dynamic time systems that are based on the dynamic motion of the
solar system and take into account relativistic effects; and atomic
time systems based on the oscillation of the cesium atom. Each of
these time systems has specific uses and is discussed below.  In
addition,  universal, dynamic, and atomic time systems can be
expressed in different time formats.  The two time formats used in
GMAT are the Modified Julian Date (MJD) format, and the Gregorian
Date (GD) format.  In the next section, we'll take a look at the
time systems used in GMAT, and when GMAT uses each time system. Then
we'll look at the different time formats.

\section{Time Systems}

GMAT uses several different time systems in physical models and
spacecraft dynamics modelling.  The choice of time system for  a
particular calculation is determined by which time system is most
natural and convenient, as well as the  accuracy required.  In
general, for determining Earth's orientation at a given epoch, we
use one of several forms of Universal Time (UT), because universal
time is based on the Earth's rotation with respect to the Sun.
Planetary ephemerides are usually provided with time in a dynamic
time system, because dynamic time is the independent variable in the
dynamic theories and ephemerides.  The independent variable in
spacecraft equations of motion in GMAT is time expressed in an
atomic time system.  Let's look at each of these three systems,
starting with atomic time.

\subsection{Atomic Time:  TAI and A.1} \index{Time systems!TAI}
\index{Time systems!A.1} \label{Sec:AtomicTime}

Atomic time (AT) is a highly accurate time system which is
independent of the rotation of the Earth\cite{vallado2}.
Therefore, AT is a natural system for integrating a spacecraft's
equations of motion.  AT is defined in terms of the oscillations
of the cesium atom at mean sea level. The duration of the SI
second is defined to be 9,192,631,770 oscillations of the cesium
nuclide 133Ce. Two atomic times systems are are used in GMAT: A.1,
and international atomic time (TAI). A.1 is in advance of TAI by
0.0343817 seconds.
%
\begin{equation}
     A.1 = TAI + 0.0343817 sec
\end{equation}
%
where
%
\begin{equation}
   TAI = UTC + \Delta AT
\end{equation}
%
and $\Delta AT$ is the number of leap seconds, added since 1972,
needed to keep $\mid UTC - UT1 \mid \leq 0.9 $sec.  GMAT reads
$\Delta AT$ from the file named \emph{tai-utc.dat}. For times that
appear before the first epoch on the file, GMAT uses the first value
found in the file.  For times that appear after the last epoch, GMAT
uses the last value contained in the file.  Currently, GMAT uses A.1
time as the independent variable in the equations of motion. TAI is
used as a time system for defining spacecraft state information.

Now let's look at the universal time system.


\subsection{Universal Time: UTC and UT1}\index{Time systems!UTC}
\index{Time systems!UT1} \label{Sec:UniversalTime}

All of the universal time (UT) scales are based on the Earth's
rotation with respect to a fixed point (sidereal time) or with
respect to the Sun (solar time). The observed universal time (UTO)
is determined from observations of stellar transits to determine
mean local sidereal time. UT1 is UTO corrected for the Earth's polar
motion and is used when the instantaneous orientation of the Earth
is needed.  UTC is the basis for all civil time standards. It is
also known as Greenwich mean time (GMT) and Zulu time (Z). The UTC
time unit is defined to be an SI second, but UTC is kept within 0.9
seconds of UT1 by occasional leap second adjustments.  The equation
relating UTC and UT1 is
%
\begin{equation}
    UTC = UT1 - \Delta UT1
\end{equation}
%
In GMAT,  $\Delta UT1$ is read from the file \emph{eopc04.62-now}
provided by the International Earth Rotation and Reference Systems
Service (IERS). The file containing the latest measurements and
predictions can be found at
\\\emph{http://www.iers.org/}. For times past the last epoch
contained in the file, GMAT uses the last value of $ \Delta UT1$
contained in the file. GMAT uses UTC as a time system to define
spacecraft state information. UT1 is used to determine the Greenwich
hour angle and for the sidereal time portion of FK5 reduction.

\subsection{Dynamic Time: TT, TDB and TCB}\index{Time systems!TDB}
\index{Time systems!TT}\index{Time systems!TCB}
\label{Sec:DynamicTime}

Dynamical time is the independent variable in the dynamical
theories and ephemerides. This class of time scales contains
terrestrial time (TT),  Barycentric Dynamical time (TDB), and
Barycentric Coordinate Time (TCB).    TDB is the independent
variable in the equations of motion referred to the solar system
barycenter. It is also the coordinate time in the theory of
general relativity. Despite the fact that the Jet Propulsion
Laboratory (JPL) J2000.0 ephemerides are referred to in TDB, TT is
frequently used. This is because TT and TDB always differ by less
than 0.002 sec. As higher accuracy or more sensitive missions are
planned, the difference may need to be distinguished.  In this
section we'll discuss how to calculate TT, TDB and TCB, and
discuss where each is used in GMAT.

TT is the independent variable in the equations of motion referred
to the Earth's center. It is also the proper time in the theory of
general relativity. The unit of TT is a day of 86400 SI seconds at
mean sea level.  In GMAT, TT is used in FK5 reduction, and as an
intermediate time system in the calculation of TDB and TCB.  TT
can be calculated from the following equation:
%
\begin{equation}
     TT = TAI + 32.184 \hspace{.05 in}\mbox{sec}
\end{equation}
%

Calculating TDB exactly is a complicated process that involves
iteratively solving a transcendental equation.  For this reason,
it is convenient to use the following approximation
%
\begin{equation}
     TDB \approx TT + \underbrace{0.001658\sin{M_E} +
     0.00001385\sin{2M_E}}_{\mbox{units of seconds}}
\end{equation}
%
Note that the term in the underbrace has units of seconds, and
depending upon the units of $TT$, which is usually in days, a
conversion of the term may be necessary before performing the
addition with $TT$. $M_E$ is the Earth's mean anomaly with respect
to the sun and is given approximately as
%
\begin{equation}
     M_E \approx 357.5277233 + 35,999.05034 T_{TT}
\end{equation}
%
where $T_{TT}$ is the time in TT expressed in the Julian Century
format.  $T_{TT}$ can be calculated from
%
\begin{equation}
     T_{TT} = \frac{JD_{TT} - 2,451,545.0}{36,525}
\end{equation}
%
where $JD_{TT}$  is the time in TT expressed in the Julian Date
format.  For a more complete discussion of the TDB time system, see
Vallado\cite{vallado2} (pp. 195-198) and Seidelmann\cite{seidelmann}
(pp. 41-48).  GMAT uses TDB as the default time system in the JPL
ephemerides files. There is an option to use TT in the ephemerides
using the \st{UseTTForEphemeris} flag.

The last dynamic time system GMAT uses is Barycentic Coordinate
Time.  In 1992, the IAU adopted this system and clarified the
relationships between space-time coordinates\cite{seidelmann}.  In
general, calculating TCB requires a four-dimensional space-time
transformation that is well beyond the scope of this discussion.
However, TCB can be approximated using the following equation:
%
\begin{equation}
 TCB - TDB = L_B  (JD - 2443144.5)  86400
 \end{equation}
%
The present estimate of the value of $L_B$ is $1.550505 x 10^{-8}$
$(+/- 1 x 10^{-14})$ (Fukushima et al., Celestial Mechanics, 38,
215, 1986).  It is important to note that the main difference
between TDB and TCB is a secular drift, and that as of the J2000
Epoch, the difference was approximately 11.25 seconds and growing
\cite{Seidelmann:etal:92}. GMAT uses time in the TCB system to
evaluate the IAU data  for the spin axes and prime meridian
locations of all planets and moons except for
Earth.\cite{Seidelmann:etal:02}  Note that
Seidelmann\cite{Seidelmann:etal:02} mistakenly says that time in TCB
should be used in the equations given for the  pole and meridian
locations of the planets.  The correct time to use is TDB, and GMAT
uses this time system.

\section{Time Formats} \index{Time Formats} \label{Sec:TimeFormats}

There are two time formats that GMAT uses to represent time in the
systems discussed above.  These formats are called the Gregorian
Date (GD), and the Julian Date (JD).  The difference between the GD
and JD formats is how they represent the Year, Month, Day, Hours,
Minutes, and Seconds of a given date.  The GD format is well known,
and the J2000 epoch is expressed as, 01 Jan 2000 12:00:00.000 TT.
The reference epoch for the GD calendar is the beginning of the
Christian Epoch.  The JD format represents an epoch as a continuous
number containing the day and the fraction of day.

The J2000 epoch \index{J2000 Epoch} is commonly used in
astrodynamics as a reference epoch for planetary and other data. The
J2000 epoch occurred at 01 Jan 2000 12:00:00.000 TT.  The time
system, TT, is important for precise applications!  While the J2000
epoch is a specific instant in time, the numerical value changes
depending upon which time system you express it in. We can make an
analogy with vector algebra where we have an abstract quantity that
is a vector, and can't write down a set of numbers representing the
vector until we choose a coordinate system. Similarly, the J2000
epoch can be written in any of the different time systems and
formats. All of the following are equivalent definitions of the
J2000 Epoch:
%
\begin{eqnarray}
    2451545.0 & TT \nonumber \\
    2451544.9996274998411 & TAI \nonumber \\
    2451544.9992571294708 & UTC \nonumber\\
    2451544.9999999990686 & TDB \nonumber
\end{eqnarray}
%
\begin{eqnarray}
 & \mbox{01 Jan }   2000 \hspace{.1 in} 12:00:00.000  &\mbox{TT} \nonumber \\
& \mbox{01 Jan }  2000 \hspace{.1 in} 11:59:27.815986276  &\mbox{TAI} \nonumber \\
& \mbox{01 Jan }  2000 \hspace{.1 in} 11:59:55.815986276 &\mbox{UTC} \nonumber\\
 & \mbox{01 Jan }  2000 \hspace{.1 in} 11:59:59.999919534  &\mbox{TDB} \nonumber
\end{eqnarray}

In the next two sections we'll look at how to convert an epoch in a
given time system from the GD format to the JD format, and vice
versa.

\subsection{Julian Date and Modified Julian Date}\index{Time
formats!Julian date} \index{Time formats!modified Julian
date}\label{Sec:JDFormat}

The Julian date is a time format in which we can express a time
known in any of the Atomic, Universal or Dynamic time systems. The
Julian Date is composed of the Julian day number and the decimal
fraction of the current day.  Seidelmann\cite{seidelmann} (pp.
55-56) says ``The Julian day number represents the number of days
that has elapsed, at Greenwich noon on the day designated, since
...the epoch noon Jan 1 4713 B.C. in the Julian proleptic
calendar.  The Julian date (JD) corresponding to any instant is,
by simple extension to this concept, the Julian day number
followed by  the fraction of the day elapsed since the preceding
noon".

The fundamental epoch for most astrodynamic calculations is the
J2000.0 epoch\cite{seidelmann}.  This epoch is GD 01 Jan 2000
12:00:00.000 in the TT time system and is expressed as JD 2451545.0
TT. To convert between Julian Date format and Gregorian Date format,
GMAT uses Algorithm 14 from Vallado\cite{vallado2}
%
\begin{equation}
\begin{split}
    JD = 367Y - Int\left( \displaystyle\frac{ 7\left( Y + Int\left( \displaystyle\frac{M + 9}{12} \right) \right)}{4}
    \right)+ \\Int\left(\displaystyle\frac{275M}{9} \right) + D +
    1,721,013.5 + \frac{ \displaystyle\frac{ \displaystyle\frac{s}{60}+m}{60} + H}{24}
    \end{split}
\end{equation}
%
where $Y$ is the four digit year, $Int$ signifies real truncation,
and $M$, $D$, $H$, $m$ and $s$ are month, day, hour, minutes, and
seconds respectively. This equation is valid for the time period 01
Mar. 1900 to 28 Feb. 2100.

For numerical reasons it is often convenient to work in a Modified
Julian Date (MJD) format to ensure we can capture enough significant
figures using double precision computers.  In GMAT the MJD system is
defined as
%
\begin{equation}
     MJD = JD - 2,430,000.0
\end{equation}
%
where the reference epoch expressed in the GD format is 05 Jan 1941
12:00:00.000.  However, we must be careful in calculating the
Modified Julian Date, or we will lose the precision we are trying to
gain.  GMAT calculates the MJD as follows:
%
\begin{equation}
     \begin{split}
     \mbox{JDay} = & 367Y - Int\left( \displaystyle\frac{ 7\left( Y + Int\left( \displaystyle\frac{M + 9}{12} \right) \right)}{4}
    \right)+ \\ & Int\left(\displaystyle\frac{275M}{9} \right) + D +
    1,721,013.5 - 2,430,000.0
    \end{split}
\end{equation}
%
\begin{equation}
     \mbox{PartofDay} = \frac{ \displaystyle\frac{ \displaystyle\frac{s}{60}+m}{60} + H}{24}
\end{equation}
%
\begin{equation}
     \mbox{MJD} = (\mbox{JDay}- 2,430,000.0) + \mbox{PartofDay}
\end{equation}
%
The important subtlety is that we must subtract the MJD reference
from the JD, before we add the fraction of day, to avoid losing
precision in the MJD.

\subsection{Gregorian Date}\index{Time formats!Gregorian
date}\label{Sec:GregorianDateFormat}

The Gregorian Date format is primarily used as a time system in
which to enter state information in GMAT.  GD is not a convenient
time format for most mathematical calculations.  Hence, GMAT often
takes input in the GD format and converts it to a MJD format for
use internally.  The algorithm for converting from GD to MJD is
taken from Vallado\cite{vallado2} and reproduced here verbatim.

\begin{equation}
\begin{array}{llll}
      & T_{1900} = \displaystyle\frac{JD - 2,415,019.5}{365.25}\\
      & \\
      & Year = 1900 + \mbox{TRUNC}(T_{1900}) \\
      & \\
      & LeapYrs = \mbox{TRUNC}((Year - 1900 - 1)(0.25))\\
      & \\
      & Days = (JD - 2,415,019.5) - \\
      & \hspace{.5 in}((Year - 1900)(365.0) +
               LeapYrs)\\
      & \\
      & \mbox{IF} Days < 1.0 \mbox{ THEN}
      \end{array}\nonumber
\end{equation}
%
\begin{equation}
     \begin{array}{llll}
      & \\
      & \hspace{.5 in} Year = Year - 1 \\
      &  \\
      & \hspace{.5 in} LeapYrs = \mbox{TRUNC}((Year - 1900 - 1)(0.25))\\
      & \\
      & \hspace{.5 in}Days = (JD - 2,415,019.5) - \\
      & \hspace{1 in}((Year - 1900)(365.0) +
               LeapYrs)\\
      & \\
      & \mbox{If (\emph{Year} Mod 4)} = 0 \mbox{ Then} \\
      & \\
      &  \hspace{.5 in} LMonth[2] = 29 \\
       \end{array}\nonumber
\end{equation}
%
\begin{equation}
     \begin{array}{llll}
      & DayofYr = \mbox{TRUNC}(Days) \\
      & \\
      & \mbox{Sum days in each month until}\\
      & \hspace{.5 in} \mbox{\emph{LMonth} + 1 summation} > \mbox{\emph{DayofYr}} \\
      & \\
      &  Mon = \# \mbox{ of months in summation} \\
      & \\
      &  Day = DayofYr - LMonth \mbox{ summation}\\
      & \\
      &  \tau = (Days - DayofYr)24\\
      &  \\
      &  h = \mbox{TRUNC}(Temp)\\
      &  \\
      &  min = \mbox{TRUNC}((Temp - h)60) \\
      &  \\
      &  s = \left( Temp - h - \displaystyle\frac{min}{60} \right)3600
      %
       \end{array}\nonumber
\end{equation}

\section{Conclusions}

In this chapter we looked at three time systems that GMAT uses to
perform internal calculations:  atomic time, universal time, and
dynamic time.  Atomic time is used to integrate spacecraft
equations of motion, while universal time is used to determine the
sidereal time and greenwhich hour angle for use in FK5 reduction.
Dynamic time systems are used in the JPL ephemerides and in the
IAU planetary orientation data.   Time, in any of these time
systems, are represented in two formats:  the Gregorian Date, and
Julian Date. We looked at how to convert between different time
systems, and between different time formats.
