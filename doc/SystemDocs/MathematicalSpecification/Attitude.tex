\section{Attitude} \label{Ch:Attitude}

The attitude of a spacecraft can be defined qualitatively as how the
spacecraft is oriented in inertial space, and how that orientation
changes in time.  GMAT has the ability to model the orientation and
rate of rotation of a spacecraft using several different
mathematical models. Currently, GMAT assumes that a spacecraft is a
rigid body.

There are many ways to quantitatively describe the orientation and
rate of rotation of a spacecraft, just like there are many ways we
can quantitatively describe an orbit state.  Let's define any set of
numbers that can uniquely define the spacecraft attitude as an
attitude parameterization. GMAT allows the users to use several
common attitude parameterizations including quaternions, Euler
angles, the Direction Cosine Matrix (DCM), Euler angle rates, and
the angular velocity vector. Given an initial attitude state, GMAT
can propagate the attitude using one of several kinematic attitude
propagation models.

In this chapter, we discuss the attitude parameterizations supported
in GMAT, and how to convert between the different types.  We discuss
the internal state parameterization that GMAT uses.   Next we
investigate the types of attitude modes in GMAT and discuss in
detail how GMAT propagates the spacecraft attitude in all of the
Kinematic attitude modes.  We conclude the chapter with a discussion
of how GMAT converts between different attitude parameterizations.

\subsection{Attitude Propagation}

Given a set of initial conditions that define the attitude, GMAT can
propagate the attitude using several methods.  Currently, GMAT only
supports kinematic attitude propagation.  In Kinematic mode, the
attitude is defined by describing the desired orientation with
respect to other objects such as spacecraft or celestial bodies.
With this information, GMAT can calculate the required attitude to
satisfy the desired geometrical configuration.  This section
presents the different Kinematic attitude modes, and how GMAT
calculates the attitude state in each mode.  Let's begin by looking
at the internal attitude state representation and how the user can
define initial conditions.

\vspace{- .1 in} \subsubsection{Internal State Representation and
Attitude Initial Conditions}

Certain attitude parameterizations are more useful for attitude
propagation, while other attitude parameterizations are more
intuitive for providing attitude initial conditions or output. GMAT
uses different internal parameterizations of the attitude
orientation depending upon the attitude mode.  The type of
parameterization is chosen to make the attitude propagation
algorithms natural and convenient.  For the kinematic modes, GMAT
uses the DCM that represents the rotation from the inertial system
to the body axes as the attitude orientation parameterization. In
the future, when 6 degree of freedom attitude modelling is
implemented, GMAT will use the quaternion that represents the
rotation from the inertial system to the body axes. GMAT uses the
angular velocity of the body with respect to the inertial frame,
expressed in the body frame, $\{\boldsymbol\omega_{IB}\}_B$, as the
rate portion of the state vector.

For convenience, the user can choose a coordinate system in which to
define the initial attitude state.  Let's call this system
$\mathcal{F}_i$.  The user can define the initial attitude
orientation with respect to $\mathcal{F}_i$ using Euler angles, the
DCM, or quaternions.  The user can define the body rate with respect
to $\mathcal{F}_i$ by defining the angular velocity in
$\mathcal{F}_i$, $\{\boldsymbol\omega_{IB}\}_i$, or by defining the
Euler angle rates.  Note that not all attitude modes require these
three pieces of information.  The specific inputs for each attitude
mode are discussed below, along with details about how attitude
propagation is performed in each mode.

\subsubsection{Kinematic Attitude Propagation}

The Kinematic attitude mode allows a user to define a geometrical
configuration based on the relative position of a spacecraft with
respect to other spacecraft or celestial bodies, and with respect to
different coordinate systems.  In Kinematic mode, GMAT does not
integrate the attitude equations of motion, but rather calculates
the attitude based on the geometrical definition provided by the
user.  There are several Kinematic modes to choose from.  The
different modes allow the user to conveniently define the spacecraft
attitude depending on the type of attitude profile needed for a
specific mission.  To begin, let's look at how GMAT calculates the
attitude state in the Coordinate System Fixed attitude mode
(CSFixed).

\subsubsection{Coordinate System Fixed Mode}

In the CSFixed attitude mode, the user supplies two pieces of
information. They first specify a coordinate  system in which to fix
the attitude, $\mathcal{F}_i$.  $\mathcal{F}_i$ can be any of the
default coordinate systems or any user defined coordinate system.
Secondly, the user specifies how the body axis system,
$\mathcal{F}_B$ is oriented with respect to $\mathcal{F}_i$ by
defining $\mathbf{R}_{Bi}$ or an equivalent parameterization. With
this information, GMAT calculates the rotation from the inertial to
the body axes and the angular velocity of the body with respect to
the inertial frame, expressed in the body frame,
$\{\mathbf{\boldsymbol\omega}_{IB}\}_B$.

GMAT calculates the rotation matrix from $\mathcal{F}_i$  to
$\mathcal{F}_B$, $\mathbf{R}_{Bi}$, from the initial conditions
provided by the user.  For CSFixed mode, $\mathbf{R}_{Bi}$ is
constant and is stored for use in the equations below.  Knowing
$\mathbf{R}_{Bi}$, we can calculate the rotation matrix from the
inertial frame to the body frame, $\mathbf{R}_{BI}$, using the
following equation
%
\begin{equation}
     \mathbf{R}_{BI} = \mathbf{R}_{Bi}\mathbf{R}_{iI}
     \label{Eq:CSFixedRotationMatrix}
\end{equation}
%
$\mathbf{R}_{iI}$ is the rotation matrix from $\mathcal{F}_I$ to
$\mathcal{F}_i$ and GMAT knows how to calculate this matrix for all
allowable $\mathcal{F}_i$.  For details on the calculation of this
matrix for all coordinate systems in GMAT  see
Ch.~\ref{Ch:CoordinateSystems}.

To calculate $\{\mathbf{\boldsymbol\omega}_{IB}\}_B$, we start from
Eulers' equation:
%
\begin{equation}
   \dot{\mathbf{R}}_{BI} =
   -\{\mathbf{\boldsymbol\omega^\times}_{IB}\}_B\mathbf{R}_{BI}
   \label{Eq:CSFixedKinematics}
\end{equation}
%
where
%
\begin{equation}
       \{ \mathbf{\boldsymbol\omega^\times}_{IB}\}_B = \begin{pmatrix}
     0 & -\omega_3 & \omega_2\\
     \omega_3 & 0 & -\omega_1\\
     -\omega_2 & \omega_1 & 0\\
     \end{pmatrix}
\end{equation}
%
and $\{\mathbf{\boldsymbol\omega}_{IB}\}_B$ is the rotation of
$\mathcal{F_B}$ with respect to $\mathcal{F}_I$, expressed in
$\mathcal{F}_B$.
%
Solving Eq.~\ref{Eq:CSFixedKinematics} for
$\{\mathbf{\boldsymbol\omega}_{IB}^\times\}_B$ we obtain
%
\begin{equation}
   \{\mathbf{\boldsymbol\omega^\times}_{IB}\}_B =  -\dot{\mathbf{R}}_{BI}
   \mathbf{R}_{BI}^{T}
   \label{Eq:CSFixedKinematics2}
\end{equation}
%
Taking the derivative of Eq.~(\ref{Eq:CSFixedRotationMatrix}) with
respect to time yields
%
\begin{equation}
     {\mathbf{\dot R}_{BI}} = \mathbf{R}_{Bi}{\mathbf{\dot
     R}_{iI}}\label{Eq:CSFixedTimeDerivative}
\end{equation}
%
because by definition, for the CSFixed mode,  $ \mathbf{\dot R}_{Bi}
= \mathbf{0}$. Substituting Eq.~(\ref{Eq:CSFixedTimeDerivative})
into Eq.~(\ref{Eq:CSFixedKinematics2}) we obtain
%
%
\begin{equation}
   \{\mathbf{\boldsymbol\omega^\times}_{IB}\}_B =  -\mathbf{R}_{Bi}{\mathbf{\dot
     R}_{iI}} \mathbf{R}_{BI}^{T}
   \label{Eq:CSFixedKinematics3}
\end{equation}
%
where $\mathbf{R}_{Bi}$ is known from user input, and
$\mathbf{R}_{BI}$ is known from Eq.~(\ref{Eq:CSFixedRotationMatrix})
. GMAT knows how to calculate $\mathbf{\dot{R}}_{iI}$ for all
allowable $\mathcal{F}_i$ and details are contained in
Ch.~\ref{Ch:CoordinateSystems}.

In summary, in CSFixed mode, Eq.(\ref{Eq:CSFixedRotationMatrix}) is
used to calculate $\mathbf{R}_{BI}$, and
Eq.~(\ref{Eq:CSFixedKinematics3}) is used to calculate
$\{\mathbf{\boldsymbol\omega}_{IB}^x\}_B$.  If another attitude
parameterization is required, GMAT uses the algorithms in
Sec.~\ref{Sec:AttitudeParameterizations} to transform from
$\mathbf{R}_{BI}$ and $\{\mathbf{\boldsymbol\omega}_{IB}\}_B$ to the
required parameterization.  Now let's look at the spinning
spacecraft mode.

\subsubsection{Spinning Spacecraft Mode}

In spinning spacecraft mode, GMAT propagates the attitude by
assuming the spin axis direction is fixed in inertial space.  The
spacecraft attitude at some time, $t$, is determined from the
attitude initial conditions, the angular velocity vector, and the
elapsed time from the initial spacecraft epoch.  Let's take a closer
look at the calculations.

In the spinning spacecraft mode, the user provides three pieces of
information.  They first choose a coordinate system,
$\mathcal{F}_i$, in which to define the initial conditions.
Secondly, they define the initial orientation with respect to
$\mathcal{F}_i$ by providing $\mathbf{R}_{Bi}$ or an equivalent
parameterzation that is then converted to the DCM.   The user also
provides the angular velocity of the body axes with respect to the
inertial axes expressed in $\mathcal{F}_i$, $\{
\boldsymbol\omega_{IB}\}_i$.

To calculate $\mathbf{R}_{BI}(t)$ where $t$ is an arbitrary epoch,
we begin by calculating $\mathbf{R}_{B_{o}I}$ where
$\mathbf{R}_{B_{o}I} = \mathbf{R}_{BI}(t_o)$.  We calculate
$\mathbf{R}_{B_{o}I}$ using
%
\begin{equation}
     \mathbf{R}_{B_{o}I} =  \mathbf{R}_{Bi}\mathbf{R}_{iI}(t_o)
\end{equation}
%
where $\mathbf{R}_{Bi}$ comes from user provided data, and
$\mathbf{R}_{iI}(t_o)$ is calculated by GMAT and is dependent upon
$\mathcal{F}_i$.  See Ch.~\ref{Ch:CoordinateSystems} for details on
how GMAT calculates $\mathbf{R}_{iI}$ for all allowable coordinate
systems in GMAT.

Before calculating $\mathbf{R}_{BI}(t)$ we must determine the spin
axis in the body frame, $\{\boldsymbol\omega_{IB}\}_B$.  The user
provides $\{\boldsymbol\omega_{IB}\}_i$.  In spinning mode we assume
the spin axis direction is constant in inertial space and in the
body frame so $\{ \boldsymbol\omega_{IB} \}_B (t)$ $ = \{
\boldsymbol\omega_{IB} \}_B (t_o) = \{ \boldsymbol\omega_{IB} \}_B
$.  We can find the spin axis in the body frame using
$\mathbf{R}_{Bi}$ as follows
%
\begin{equation}
      \{ \boldsymbol\omega_{IB}\}_B = \mathbf{R}_{Bi} \{ \boldsymbol\omega_{IB}\}_i
\end{equation}
%
Once calculated, GMAT saves $\mathbf{R}_{B_{o}I}$ and $\{
\boldsymbol\omega_{IB}\}_B$ for use in calculating the attitude
orientation and rate at other epochs.

GMAT calculates $\mathbf{R}_{BI}(t)$ using the Euler axis/angle
rotation algorithm in Sec. \ref{Sec:AttitudeParameterizations}. The
Euler axis is simply the unitized angular velocity vector or,
%
\begin{equation}
     \mathbf{a} =   \frac{  \{ \boldsymbol\omega_{IB} \}_B  }{\omega_{IB} }
\end{equation}
%
where
%
\begin{equation}
     \omega_{IB} = \| \{\boldsymbol{\omega}_{IB} \}_B \|
\end{equation}
%
The Euler angle $\phi$ is calculated using
%
\begin{equation}
    \phi(t) = \omega_{IB}(t -t_o)
\end{equation}
%
where $t$ is the current epoch, and $t_o$ is the spacecraft's
initial epoch.  Let's define the rotation matrix that results from
the Euler axis/angle rotation using $\mathbf{a}$ and $\phi(t)$, as
$\mathbf{R}_{BB_{o}}(t)$.  We can calculate $\mathbf{R}_{BI}(t)$
using
%
\begin{equation}
     \mathbf{R}_{BI}(t) =
     \mathbf{R}_{BB_{o}}(t)\mathbf{R}_{B_{o}I}
\end{equation}
%

To summarize, in spinning mode the user provides $\mathbf{R}_{Bi}$
and  $\{ \boldsymbol\omega_{IB}\}_i$.  GMAT assumes that that the
spin axis direction is constant, and uses the Euler axis/angle
method to propagate the attitude to find $\mathbf{R}_{BI}$.

Now let's look at how GMAT performs conversions between the
different attitude parameterizations.

\subsection{6 DOF Modelling}

\begin{equation}
   \dot{\mathbf{X}} = \left[ \hspace{.05 in} \dot{\mathbf{r}}^T \hspace{.1 in} \dot{\mathbf{v}}^T \hspace{.1 in}  \dot{\mathbf{h}}^T \hspace{.1 in}  \dot{\mathbf{q}}^T  \hspace{.05 in} \right]^T
\end{equation}
%
\begin{equation}
   \dot{\mathbf{h}} = \mathbf{T} - \mathbf{\boldsymbol\omega}\times\mathbf{h}
\end{equation}
%
\begin{equation}
   \dot{\mathbf{q}} = \frac{1}{2}\mathbf{\Omega}\mathbf{q}
\end{equation}
%
where
%
\begin{equation}
    \mathbf{\Omega} = \left(\begin{array}{ccccc}
     0 & \omega_z  & -\omega_y & \omega_x\\
     -\omega_z & 0 & \omega_x & \omega_y\\
     \omega_y & -\omega_x & 0 & \omega_z\\
     -\omega_x & -\omega_y & -\omega_z & 0\\
  \end{array}\right)
\end{equation}
%
\begin{equation}
    \mathbf{\boldsymbol\omega} = \mathbf{I}^{-1}\mathbf{h}
\end{equation}

\subsection{Attitude Parameterizations \\ and Conversions}
\label{Sec:AttitudeParameterizations}

This section details how GMAT converts between different attitude
parameterizations.  For each conversion type, any singularities that
may occur are addressed.  The orientation parameterizations in GMAT
include the DCM, Euler Angles, quaternions, and Euler axis/angle.
The body rate parameterizations include Euler angle rates and
angular velocity.  We begin with the algorithm to transform from the
quaternions to the DCM.

\subsubsection{Conversion:  Quaternions to DCM}\label{sec:AttQuattoR}
\index{Attitude Parameterization!Quaternions to DCM}

Given:  $\mathbf{q}$, $q_4$

\noindent Find:  $\mathbf{R}$

\noindent Name:  \emph{ToCosineMatrix}

\begin{equation}
    \mathbf{q} = \left( q_1 \hspace{.1 in} q_2 \hspace{.1 in} q_3 \right)^T
\end{equation}
%
\begin{equation}
     \mathbf{q}^{\times} = \begin{pmatrix}
     0 & -q3 & q2\\
     q3 & 0 & -q1\\
     -q2 & q1 & 0\\
     \end{pmatrix}
\end{equation}
%
\begin{equation}
    c = \frac{1}{q_1^2 + q_2^2 + q_3^2 + q_4^2}
\end{equation}
%
\begin{equation}
     \mathbf{R} = c\left[ (q_4^2 - \mathbf{q}^T\mathbf{q})\mathbf{I}_3 +
      2\mathbf{q}\mathbf{q}^T -2q_4\mathbf{q}^{\times}\right]
\end{equation}


\subsubsection{Conversion:  DCM to Quaternions} \label{sec:DCMtoQuat}
\index{Attitude Parameterization!DCM to Quaternions}

Given:  $\mathbf{R}$

\noindent Find: $\mathbf{q}$, $q_4$

Define following vector
%
\begin{equation}
   \mathbf{v} = [ \hspace{.02 in} R_{11} \hspace{.1 in} R_{22}\hspace{.1 in}
   R_{33} \hspace{.1 in}  \mbox{trace}(\mathbf{R}) \hspace{.02 in}]
\end{equation}
%
Define $i_m$ as the index of the maximum component of $\mathbf{v}$
%

\noindent if $i_m = 1$
%
\begin{equation}
     \mathbf{q}''  = \begin{pmatrix}
     2v_{i_m} + 1 - \mbox{trace}(\mathbf{R})\\
     R_{12} + R_{21}\\
     R_{13} + R_{31}\\
     R_{23} - R_{32}\\
     \end{pmatrix}
\end{equation}
%
if $i_m = 2$
%
\begin{equation}
     \mathbf{q}''  = \begin{pmatrix}
     R_{21} + R_{12}\\
     2v_{i_m} + 1 - \mbox{trace}(\mathbf{R})\\
     R_{23} + R_{32}\\
     R_{31} - R_{13}\\
     \end{pmatrix}
\end{equation}
%
if $i_m = 3$
%
\begin{equation}
     \mathbf{q}''  = \begin{pmatrix}
     R_{31} + R_{13}\\
     R_{32} + R_{23}\\
     2v_{i_m} + 1 - \mbox{trace}(\mathbf{R})\\
     R_{12} - R_{21}\\
     \end{pmatrix}
\end{equation}
%
if $i_m = 4$
%
\begin{equation}
     \mathbf{q}''  = \begin{pmatrix}
     R_{23} - R_{32}\\
     R_{31} - R_{13}\\
     R_{12} - R_{21}\\
     1 + \mbox{trace}(\mathbf{R})\\
     \end{pmatrix}
\end{equation}
%
We normalize $\mathbf{q}''$ using
%
\begin{equation}
    \mathbf{q}' = \frac{\mathbf{q}''}{\| \mathbf{q}'' \|}
\end{equation}
%
Finally,
%
\begin{equation}
   \mathbf{q} = [\hspace{.05 in} q_{1}' \hspace{.2 in} q_{2}' \hspace{.2 in} q_3'
   \hspace{.05 in}]^T
\end{equation}
%
and
%
\begin{equation}
     q_4 = q_4'
\end{equation}
%Note: There is not a unique quaternion for a given DCM.  GMAT
%assumes that the ``+" sign is used in Eq.~(\ref{Eq:q_4}).

\subsubsection{Conversion:  DCM to Euler\\ Axis/Angle}

\index{Attitude Parameterization!DCM to Axis/Angle}

Given:  $\mathbf{R}$

\noindent Find: $\mathbf{a}$, $\phi$

\begin{equation}
     \mathbf{R}  = \begin{pmatrix}
     R_{11} & R_{12} & R_{13}\\
     R_{21} & R_{22} & R_{23}\\
     R_{31} & R_{32} & R_{33}\\
     \end{pmatrix}
\end{equation}
%
\begin{equation}
   \phi = \cos^{-1}\left( \frac{1}{2}\left(\mbox{trace}(\mathbf{R}) -
   1 \right)\right)
\end{equation}
%
\begin{equation}
    \mathbf{a} = \frac{1}{2\sin{\phi}}\begin{pmatrix}
     R_{23} - R_{32}\\
     R_{31} - R_{13}\\
     R_{12} - R_{21}\\
     \end{pmatrix}
\end{equation}
%
If $\|\sin{\phi} \| < 10^{-14}$, then we assume
%
\begin{equation}
    \mathbf{a} = \left[\hspace{.05 in} 1 \hspace{.1 in} 0 \hspace{.1 in}
    0 \hspace{.05 in} \right]^T
\end{equation}
%
Note that if $\|\sin{\phi} \| < 10^{-14}$ then $\cos{\phi} \approx
1 $ and we arrive at a DCM of $\mathbf{I}_3$.


\subsubsection{Conversion:  Euler Axis/Angle to DCM} \index{Attitude
Parameterization!Axis/Angle to DCM} \label{Sec:EulerAxis/AngletoDCM}

Given:  $\mathbf{a}$, $\phi$

\noindent Find: $\mathbf{R}$

\begin{equation}
     \mathbf{a}^{\times} = \begin{pmatrix}
     0 & -a_3 & a_2\\
     a_3 & 0 & -a_1\\
     -a_2 & a_1 & 0\\
     \end{pmatrix}
\end{equation}
%
\begin{equation}
    \mathbf{R} = \cos{\phi}\mathbf{I}_3 + (1 -
    \cos{\phi})\mathbf{a}\mathbf{a}^T -
    \sin{\phi}\mathbf{a}^{\times}
\end{equation}

\subsubsection{Conversion:  Euler Angles to DCM}
\label{sec:AttEulerAnglestoDCM}

Given:  Sequence order  ( i.e. 123, 121, .... 313),  $\theta_1$,
$\theta_2$, $\theta_3$

\noindent Find: $\mathbf{R}$

We'll give an example for a 321 rotation, and then present results
for the remaining 11 Euler angle sequences.  First, let's define
$\mathbf{R}_3(\theta_1)$, $\mathbf{R}_2(\theta_2)$, and
$\mathbf{R}_1(\theta_3)$.
%
\begin{equation}
    \mathbf{R}_3(\theta_1) =  \begin{pmatrix}
      \cos{\theta_1}  & \sin{\theta_1} & 0    \\
      -\sin{\theta_1} & \cos{\theta_1} & 0    \\
      0               & 0              & 1
     \end{pmatrix}
\end{equation}
%
\begin{equation}
    \mathbf{R}_2(\theta_2) =  \begin{pmatrix}
      \cos{\theta_2}  & 0              & -\sin{\theta_2}    \\
      0               & 1              & 0    \\
      \sin{\theta_2}  & 0              & \cos{\theta_2}
     \end{pmatrix}
\end{equation}
%
\begin{equation}
    \mathbf{R}_1(\theta_3) =  \begin{pmatrix}
      1               & 0                & 0    \\
      0               & \cos{\theta_3}   & \sin{\theta_3}    \\
      0               & -\sin{\theta_3}  & \cos{\theta_3}
     \end{pmatrix}
\end{equation}
%
Now we can write
%
\begin{eqnarray}
       &\mathbf{R}_{321} =  \mathbf{R}_1(\theta_3)\mathbf{R}_2(\theta_2)\mathbf{R}_3(\theta_1) = \nonumber\\
      &=         \begin{pmatrix}
      1               & 0                & 0    \\
      0               & c_3   & s_3    \\
      0               & -s_3  & c_3
     \end{pmatrix}
     %
      \begin{pmatrix}
      c_2  & 0              & -s_2    \\
      0               & 1              & 0    \\
      s_2  & 0              & c_2
     \end{pmatrix}
     %
     \begin{pmatrix}
      c_1   & s_1 & 0    \\
      -s_1  & c_1 & 0    \\
      0               & 0              & 1
     \end{pmatrix}\nonumber\\
\end{eqnarray}
%
where $c_1 =\cos{\theta_1}$, $s_1 = \sin{\theta_1}$ etc.  We can
rewrite $\mathbf{R}_{321} $ as
%
\begin{equation}
       \mathbf{R}_{321}  =\begin{pmatrix}
      c_2c_1              &  c_2s_1             & -s_2 \\
     -c_3s_1 + s_3s_2c_1  &  c_3c_1 + s_3s_2s_1 & s_3c_2 \\
     s_3s_1 + c_3s_2c_1   &  -s_3c_1 +c_3s_2s_1 & c_3c_2
     \end{pmatrix} \label{Eq:R321}
\end{equation}
%

The approach is similar for the remaining 11 Euler angle sequences.
Rather than derive the DCM matrices for the remaining 11 sequences,
we present them in Table \ref{table:EulerAnglestoDCM}.

\subsubsection{Conversion:  DCM to Euler Angles}
\label{sec:AttDCMtoEulerAngles}

Given:  Sequence order  ( i.e. 123, 121, .... 313), $\mathbf{R}$

\noindent Find:  $\theta_1$, $\theta_2$, $\theta_3$

We'll give an example for a 321 rotation, and then present results
for the remaining 11 Euler angle sequences.  Examining,
Eq.~(\ref{Eq:R321}), we see that
%
\begin{equation}
     \frac{ R_{21} }  { R_{11}  } = \frac{  \cos{\theta_2}\sin{\theta_1}     }
                                 {  \cos{\theta_2}\cos{\theta_1}    }
\end{equation}
%
From this we can see that
%
\begin{equation}
    \theta_1 =  \tan^{-1}{\frac{ R_{21} }  { R_{11}  }}
\end{equation}
%
Further inspection of Eq.~(\ref{Eq:R321}) shows us that
%
\begin{equation}
    \theta_2 = \sin^{-1}{R_{13}}
\end{equation}
%
At first glance, we may choose to calculate $\theta_3$ using
$\theta_3 = \tan^{-1}{(R_{23}/R_{33})}$.  However, in the case that
$\theta_2 = 90^\circ$, this would result in the indeterminate case,
$\theta_3 =$ $\tan^{-1}(R_{23}/R_{33})$ $= \tan^{-1}(0/0)$.  An
improved method, found in the ADEAS mathematical specifications
document, is to determine $\theta_3$ using
%
\begin{equation}
    \theta_3 = \tan^{-1} \left(\frac{ R_{31} \sin{\theta_1} - R_{32} \cos{\theta_1} }
    { -R_{21} \sin{\theta_1} + R_{22} \cos{\theta_1}} \right)
    \label{Eq:Rtotheta3}
\end{equation}
%
Substituting values from Eq.~(\ref{Eq:R321}) into
Eq.~(\ref{Eq:Rtotheta3}), and using abbreviated notation, we see
that
%
\begin{equation}
     \theta_3 = \tan^{-1} \left(  \frac{ s_1( s_3s_1 + c_3s_2c_1) - c_1(-s_3c_1 + c_3s_2s_1 )}
    { s_1(c_3s_1 - s_3s_2c_1  ) + c_1( c_3c_1 + s_3s_2s_1 ) }  \right)
\end{equation}
%
Now, if $\theta_2 = 90^\circ$, and we substitute $c_2 = 0$ and $s_2
= 1$ into the above equation, we see we get a determinate form.
Results for all twelve Euler Sequences are shown in Table
\ref{table:DCMtoEulerAngles}.

\noindent Note:  All $\tan^{-1}$ use a quadrant check ( equaivalent
to atan2 ) to make sure the the correct quadrant is chosen.

\subsubsection{Conversion:  Angular Velocity to \\ Euler Angles
Rates}

Given:  Sequence ( i.e. 123, 121, .... 313), $\theta_2$,
$\theta_3$ $\boldsymbol\omega$

\noindent Find: $\dot\theta_1$, $\dot\theta_2$, $\dot\theta_3$
%
\begin{equation}
    \begin{pmatrix}
         \dot\theta_1\\
         \dot\theta_2\\
         \dot\theta_3
    \end{pmatrix}
    %
    = \mathbf{S}^{-1}(\theta_2,\theta_3)\boldsymbol\omega
\end{equation}
%
$\mathbf{S}^{-1}(\theta_2,\theta_3)$ is dependent upon the Euler
sequence.  Table \ref{table:EulerAngleKinematics} contains the
different expressions for $\mathbf{S}^{-1}(\theta_2,\theta_3)$ for
each of the 12 unique Euler sequences.

Note:  Each of the forms of $\mathbf{S}^{-1}$ have a possible
singularity due to the appearance of either $\sin{\theta_2}$ or
$\cos{\theta_2}$ in the denominator.  If GMAT encounters a
singularity, an error message is thrown, and the zero vector is
returned.

\subsubsection{Conversion:  Euler Angles Rates to Angular Velocity}

\noindent Given: Sequence ( i.e. 123, 121, .... 313), $\theta_2$,
$\theta_3$, $\dot\theta_1$, $\dot\theta_2$, $\dot\theta_3$

\noindent Find: $\boldsymbol\omega$
%
\begin{equation}
    \boldsymbol\omega = \mathbf{S}(\theta_2,\theta_3)
    %
        \begin{pmatrix}
         \dot\theta_1\\
         \dot\theta_2\\
         \dot\theta_3
    \end{pmatrix}
\end{equation}
%
$\mathbf{S}(\theta_2,\theta_3)$ is dependent upon the Euler
sequence.  Table \ref{table:EulerAngleKinematics} contains the
different expressions for $\mathbf{S}^{-1}(\theta_2,\theta_3)$ for
each of the 12 unique Euler sequences.

\subsubsection{Conversion:  Quaternions to Euler Angles}

\noindent Given: $\mathbf{q}$, $q_4$, Euler Sequence

\noindent Find: $\theta_1$, $\theta_2$, and $\theta_3$

There is not a direct transformation to convert from the quaternions
to the Euler Angles.  GMAT first converts from the quaternion to the
DCM using the algorithm in Sec. \ref{sec:AttQuattoR}.  The DCM is
then used to calculate the Euler Angles for the given Euler angle
sequence using the algorithm in Sec. \ref{sec:AttDCMtoEulerAngles}.
%\section{Attitude Initial Conditions}

\subsubsection{Conversion:  Euler Angles to Quaternions}

\noindent Given: $\theta_1$, $\theta_2$, and $\theta_3$, Euler
Sequence

\noindent Find: $\mathbf{q}$, $q_4$

There is not a direct transformation to convert from Euler Angles to
quaternions.  GMAT first converts from the Euler Angles to the DCM
using the algorithm in Sec. \ref{sec:AttEulerAnglestoDCM}.  The DCM
is then used to calculate the quaternions using the algorithm in
Sec. \ref{sec:DCMtoQuat}.
%\section{Attitude Initial Conditions}



\subsubsection{Euler Angles to $\mathbf{R}$ Matrix}



\onecolumn
 \begin{table}[h]
        \centering
        \vspace{0 pt}
        \caption{Rotation Matrices for 12 Unique Euler Angle Rotation Sequences}
        \begin{tabular}{clccccccc}  \hline \hline
        \\
     %------------------------------------
     %------------------------------------
     $\mathbf{R}_3(\theta_3)\mathbf{R}_2(\theta_2)\mathbf{R}_1(\theta_1)$
     &
     $=\begin{pmatrix}
     c_3c_2    & c_3 s_2 s_1 + s_3 c_1  & -c_3 s_2 c_1 + s_1 s_3\\
     -s_3 c_2  & -s_3 s_2 s_1 + c_3 c_1 & s_3 s_2 c_1 + c_3 s_1\\
     s_2       & -c_2 s_1               & c_2 c_1\\
     \end{pmatrix}  \vspace{.1 in}$\\
     %
     %------------------------------------
     %------------------------------------
     $\mathbf{R}_2(\theta_3)\mathbf{R}_3(\theta_2)\mathbf{R}_1(\theta_1)$
     &
     $=\begin{pmatrix}
     c_3c_2   &   c_3 s_2 c_1 + s_1 s_3  &  c_3s_2s_1-s_3c_1 \\
      -s_2    &   c_2c_1                 &  c_2s_1\\
      s_3c_2  &   s_3s_2c_1 - c_3s_1     &  s_3s_2s_1 + c_3c_1 \\
     \end{pmatrix}   \vspace{.1 in}$\\
     %
     $\mathbf{R}_1(\theta_3)\mathbf{R}_3(\theta_2)\mathbf{R}_2(\theta_1)$
     &
     $=\begin{pmatrix}
     c_2c_1               &    s_2     &   -c_2s_1\\
     -c_3s_2c_1 + s_3 s_1 & c_3c_2     & c_3s_2s_1 + s_3c_1     \\
     s_3s_2c_1 +c_3s_1    &  -s_3c_2   &  -s_3s_2s_1 + c_3 c_1  \\
    \end{pmatrix}   \vspace{.1 in}$\\
     %
     %------------------------------------
     %------------------------------------
     $\mathbf{R}_3(\theta_3)\mathbf{R}_1(\theta_2)\mathbf{R}_2(\theta_1)$
     &
     $=\begin{pmatrix}
      c_3c_1 + s_3s_2s_1     &  s_3c_2    & -c_3s_1 + s_3s_2c_1    \\
      -s_3c_1 + c_3 s_2s_1   &  c_3c_2    & s_3s_1 + c_3s_2c_1    \\
      c_2s_1                 &  -s_2      & c_2c_1
     \end{pmatrix}  \vspace{.1 in}$\\
     %
     $\mathbf{R}_2(\theta_3)\mathbf{R}_1(\theta_2)\mathbf{R}_3(\theta_1)$
     &
     $=\begin{pmatrix}
      c_3c_1 - s_3s_2s_1     & c_3s_1 + s_3s_2c_1 & -s_3c_2 \\
      -c_2s_1                & c_2c_1             &  s_2  \\
      s_3c_1 + c_3s_2s_1     & s_3s_1 - c_3s_2c_1 & c_3c_2
     \end{pmatrix}  \vspace{.1 in}$\\
     %
     %------------------------------------
     %------------------------------------
     $\mathbf{R}_1(\theta_3)\mathbf{R}_2(\theta_2)\mathbf{R}_3(\theta_1)$
     &
     $=\begin{pmatrix}
      c_2c_1              &  c_2s_1             & -s_2 \\
     -c_3s_1 + s_3s_2c_1  &  c_3c_1 + s_3s_2s_1 & s_3c_2 \\
     s_3s_1 + c_3s_2c_1   &  -s_3c_1 +c_3s_2s_1 & c_3c_2
     \end{pmatrix}  \vspace{.1 in}$\\
     %
     $\mathbf{R}_1(\theta_3)\mathbf{R}_2(\theta_2)\mathbf{R}_1(\theta_1)$
     &
     $=\begin{pmatrix}
     c_2                  & s_2s_1              & -s_2c_1 \\
     s_3s_2               & c_3c_1 - s_3c_2s_1  & c_3s_1 + s_3c_2c_1\\
     c_3s_2               & -s_3c_1 - c_3c_2s_1 & -s_3s_1 +c_3c_2c_1
     \end{pmatrix}  \vspace{.1 in}$\\
          %
     %------------------------------------
     %------------------------------------
     $\mathbf{R}_1(\theta_3)\mathbf{R}_3(\theta_2)\mathbf{R}_1(\theta_1)$
     &
     $=\begin{pmatrix}
      c_2                  & s_2c_1              & s_2s_1 \\
     -c_3s_2              & c_3c_2c_1 - s_3s_1  & c_3c_2s_1 + s_3c_1\\
     s_3s_2               & -s_3c_2c_1 - c_3s_1 & -s_3c_2s_1 +   c_3c_1
     \end{pmatrix}  \vspace{.1 in}$\\
               %
     $\mathbf{R}_2(\theta_3)\mathbf{R}_1(\theta_2)\mathbf{R}_2(\theta_1)$
     &
     $=\begin{pmatrix}
     c_3c_1 - s_3c_2s_1   &  s_3s_2    &  -c_3s_1 -s_3c_2c_1\\
     s_2s_1               &  c_2       &  s_2c_1            \\
     s_3c_1 + c_3c_2s_1   &  -c_3s_2   &  -s_3s_1 + c_3c_2c_1
     \end{pmatrix}  \vspace{.1 in}$\\
                    %
     %------------------------------------
     %------------------------------------
     $\mathbf{R}_2(\theta_3)\mathbf{R}_3(\theta_2)\mathbf{R}_2(\theta_1)$
     &
     $=\begin{pmatrix}
     c_3c_2c_1 - s_3s_1   &  c_3s_2  & -c_3c_2s_1 - s_3c_1\\
     -s_2c_1              &  c_2     & s_2s_1             \\
     s_3c_2c_1 + c_3s_1   &  s_3s_2  & -s_3c_2s_1 + c_3c_1
     \end{pmatrix}  \vspace{.1 in}$\\
                         %
     $\mathbf{R}_3(\theta_3)\mathbf{R}_1(\theta_2)\mathbf{R}_3(\theta_1)$
     &
     $=\begin{pmatrix}
      c_3c_1 - s_3c_2s_1 & c_3s_1 + s_3c_2c_1  &  s_3s_2\\
     -s_3c_1 - c_3c_2s_1 & -s_3s_1 + c_3c_2c_1 &  c_3s_2\\
      s_2s_1             &  -s_2c_1            &  c_2
     \end{pmatrix}  \vspace{.1 in}$\\
                              %
     %------------------------------------
     %------------------------------------
     $\mathbf{R}_3(\theta_3)\mathbf{R}_2(\theta_2)\mathbf{R}_3(\theta_1)$
     &
     $=\begin{pmatrix}
      c_3c_2c_1-s_3s_1     & c_3c_2s_1 + s_3c_1  &  -c_3s_2\\
      -s_3c_2c_1 - c_3s_1  & -s_3c_2s_1 + c_3c_1 &  s_3s_2\\
      s_2c_1               & s_2s_1              &  c_2
     \end{pmatrix}  \vspace{.1 in}$\\
         \hline \hline
        \end{tabular}
        \label{table:EulerAnglestoDCM}
\end{table}
\twocolumn

\subsubsection{Euler Rates and Angular Velocity} \onecolumn
 \begin{table}[h]
        \centering
        \vspace{0 pt}
        \caption{Kinematics of Euler Angle Rotation Sequences}
        \begin{tabular}{cllcccccc}  \hline \hline
        Euler Sequence & $\mathbf{S}(\theta_2,\theta_3)$ &
        $\mathbf{S}^{-1}(\theta_2,\theta_3)$\\
        \hline
        \\
        %------------------
        %------------------
     $\mathbf{R}_3(\theta_3)\mathbf{R}_2(\theta_2)\mathbf{R}_1(\theta_1)$
     &
     $\begin{pmatrix}
        c3c2&          s3&                0\\
       -s3c2&          c3&                0\\
          s2&                0&                1\\
     \end{pmatrix}  \vspace{.1 in}$
     &
     $\begin{pmatrix}
                  c3/c2&         -s3/c2&                        0\\
                  s3&                  c3&                        0\\
                  -s2c3/c2&  s3s2/c2& 1\\
     \end{pmatrix}  \vspace{.1 in}$\\
     %
        %------------------
        %------------------
     $\mathbf{R}_2(\theta_3)\mathbf{R}_3(\theta_2)\mathbf{R}_1(\theta_1)$
     &
     $\begin{pmatrix}
         c3c2&        -s3&               0\\
        -s2&               0&               1\\
         s3c2&         c3&               0\\
     \end{pmatrix}  \vspace{.1 in}$
     &
     $\begin{pmatrix}
         c3/c2&                       0&         s3/c2\\
           -s3&                       0&                 c3\\
          s2c3/c2&                       1& s3s2/c2\\
     \end{pmatrix}  \vspace{.1 in}$\\
     %
     %------------------
     %------------------
     $\mathbf{R}_1(\theta_3)\mathbf{R}_3(\theta_2)\mathbf{R}_2(\theta_1)$
     &
     $\begin{pmatrix}
        s2&                0&                1\\
       c3c2&          s3&                0\\
      -s3c2&          c3&                0\\
     \end{pmatrix}  \vspace{.1 in}$
     &
     $\begin{pmatrix}
                        0&          c3/c2&         -s3/c2\\
                        0&                  s3&                  c3\\
                        1& -s2c3/c2&  s3s2/c2\\
     \end{pmatrix}  \vspace{.1 in}$\\
     %
     %------------------
     %------------------
     $\mathbf{R}_3(\theta_3)\mathbf{R}_1(\theta_2)\mathbf{R}_2(\theta_1)$
     &
     $\begin{pmatrix}
         s3c2&         c3&               0\\
         c3c2&        -s3&               0\\
        -s2&               0&               1\\
     \end{pmatrix}  \vspace{.1 in}$
     &
     $\begin{pmatrix}
         s3/c2&         c3/c2&                       0\\
                 c3&                -s3&                       0\\
 s3s2/c2& s2c3/c2&                       1\\
     \end{pmatrix}  \vspace{.1 in}$\\
     %
     %------------------
     %------------------
     $\mathbf{R}_2(\theta_3)\mathbf{R}_1(\theta_2)\mathbf{R}_3(\theta_1)$
     &
     $\begin{pmatrix}
       -s3c2&          c3&                0\\
          s2&                0&                1\\
        c3c2&          s3&                0\\
     \end{pmatrix}  \vspace{.1 in}$
     &
     $\begin{pmatrix}
         -s3/c2&                        0&          c3/c2\\
                  c3&                        0&                  s3\\
         s3s2/c2&                        1& -s2c3/c2\\
     \end{pmatrix}  \vspace{.1 in}$\\
     %------------------
     %------------------
     $\mathbf{R}_1(\theta_3)\mathbf{R}_2(\theta_2)\mathbf{R}_3(\theta_1)$
     &
     $\begin{pmatrix}
                -s2&               0&               1\\
       s3c2&         c3&               0\\
       c3c2&        -s3&               0\\
     \end{pmatrix}  \vspace{.1 in}$
     &
     $\begin{pmatrix}
                       0&         s3/c2&         c3/c2\\
                       0&                 c3&                -s3\\
                       1& s3s2/c2& s2c3/c2\\
     \end{pmatrix}  \vspace{.1 in}$\\
          %
     %------------------
     %------------------
     $\mathbf{R}_1(\theta_3)\mathbf{R}_2(\theta_2)\mathbf{R}_1(\theta_1)$
     &
     $\begin{pmatrix}
                c2&               0&               1\\
       s3s2&         c3&               0\\
      c3s2&        -s3&               0\\
     \end{pmatrix}  \vspace{.1 in}$
     &
     $\begin{pmatrix}
                        0&          s3/s2&          c3/s2\\
                        0&                  c3&                 -s3\\
                        1& -s3c2/s2& -c3c2/s2\\
     \end{pmatrix}  \vspace{.1 in}$\\
                    %
     %------------------
     %------------------
     $\mathbf{R}_1(\theta_3)\mathbf{R}_3(\theta_2)\mathbf{R}_1(\theta_1)$
     &
     $\begin{pmatrix}
          c2&                0&                1\\
       -c3s2&          s3&                0\\
        s3s2&          c3&                0\\
     \end{pmatrix}  \vspace{.1 in}$
     &
     $\begin{pmatrix}
                        0&         -c3/s2&          s3/s2\\
                        0&                  s3&                  c3\\
                        1&  c3c2/s2& -s3c2/s2\\
     \end{pmatrix}  \vspace{.1 in}$\\
                    %
     %------------------
     %------------------
     $\mathbf{R}_2(\theta_3)\mathbf{R}_1(\theta_2)\mathbf{R}_2(\theta_1)$
     &
     $\begin{pmatrix}
        s3s2&          c3&                0\\
          c2&                0&                1\\
       -c3s2&          s3&                0\\
     \end{pmatrix}  \vspace{.1 in}$
     &
     $\begin{pmatrix}
          s3/s2&                        0&         -c3/s2\\
                  c3&                        0&                  s3\\
         -s3c2/s2&                        1&  c3c2/s2\\
     \end{pmatrix}  \vspace{.1 in}$\\
                              %
     %------------------
     %------------------
     $\mathbf{R}_2(\theta_3)\mathbf{R}_3(\theta_2)\mathbf{R}_2(\theta_1)$
     &
     $\begin{pmatrix}
       c3s2&        -s3&               0\\
         c2&               0&               1\\
       s3s2&         c3&               0\\
     \end{pmatrix}  \vspace{.1 in}$
     &
     $\begin{pmatrix}
          c3/s2&                        0&          s3/s2\\
                 -s3&                        0&                  c3\\
        -c3c2/s2&                        1& -s3c2/s2\\
     \end{pmatrix}  \vspace{.1 in}$\\
                              %
     %------------------
     %------------------
     $\mathbf{R}_3(\theta_3)\mathbf{R}_1(\theta_2)\mathbf{R}_3(\theta_1)$
     &
     $\begin{pmatrix}
       s3s2&         c3&               0\\
       c3s2&        -s3&               0\\
         c2&               0&               1\\
     \end{pmatrix}  \vspace{.1 in}$
     &
     $\begin{pmatrix}
          s3/s2&          c3/s2&                        0\\
                  c3&                 -s3&                        0\\
      -s3c2/s2& -c3c2/s2&                        1\\
     \end{pmatrix}  \vspace{.1 in}$\\
     %
          %------------------
     %------------------
     $\mathbf{R}_3(\theta_3)\mathbf{R}_2(\theta_2)\mathbf{R}_3(\theta_1)$
     &
     $\begin{pmatrix}
        -c3s2&          s3&                0\\
         s3s2&          c3&                0\\
           c2&                0&                1\\
     \end{pmatrix}  \vspace{.1 in}$
     &
     $\begin{pmatrix}
         -c3/s2&          s3/s2&                        0\\
                  s3&                  c3&                        0\\
        c3c2/s2& -s3c2/s2&                        1\\
     \end{pmatrix}  \vspace{.1 in}$\\
         \hline \hline
        \end{tabular}
        \label{table:EulerAngleKinematics}
\end{table}

\begin{table}[h]
        \centering
        \vspace{0 pt}
        \caption{ Computation of Euler Angles from DCM}
        \begin{tabular}{llllllll}  \hline \hline \\
        Euler Sequence & Euler Angle Computations \\
        \hline \\
     $\mathbf{R}_3(\theta_3)\mathbf{R}_2(\theta_2)\mathbf{R}_1(\theta_1)$
     \hspace{.1 in}
     &
      $\theta_1 =  \tan^{-1}(-R_{32}/R_{33})$ &
      $\theta_2 =  \sin^{-1}(R_{31})$ &
      $\theta_3  = \tan^{-1}\left(\displaystyle\frac{R_{13}\sin{\theta_1}+
      R_{12}\cos{\theta_1}}{R_{23}\sin{\theta_1}+R_{22}\cos{\theta_1}}\right )$\vspace{.15 in}
     \\
     %
     $\mathbf{R}_2(\theta_3)\mathbf{R}_3(\theta_2)\mathbf{R}_1(\theta_1)$ \hspace{.1 in}
     &
     $\theta_1 =  \tan^{-1}(R_{23}/R_{22})$ &
         $\theta_2 =  \sin^{-1}(-R_{21})$ &
         $\theta_3  = \tan^{-1}\left(\displaystyle\frac{R_{12}\sin{\theta_1}- R_{13}\cos{\theta_1}}{-R_{32}\sin{\theta_1}+
         R_{33}\cos{\theta_1}}\right)$ \vspace{.15 in}
     \\
     $\mathbf{R}_1(\theta_3)\mathbf{R}_3(\theta_2)\mathbf{R}_2(\theta_1)$
     &
    $\theta_1 =  \tan^{-1}(-R_{13}/R_{11})$ &
         $\theta_2 =  \sin^{-1}(R_{12})$ &
         $\theta_3  = \tan^{-1}\left(\displaystyle\frac{R_{21}\sin{\theta_1}+ R_{23}\cos{\theta_1}}{R_{31}\sin{\theta_1}+
         R_{33}\cos{\theta_1}}\right)$ \vspace{.15 in}
     \\
     %
     $\mathbf{R}_3(\theta_3)\mathbf{R}_1(\theta_2)\mathbf{R}_2(\theta_1)$
     &
         $\theta_1 =  \tan^{-1}(R_{31}/R_{33})$ &
         $\theta_2 =  \sin^{-1}(-R_{32})$ &
         $\theta_3  = \tan^{-1}\left(\displaystyle\frac{R_{23}\sin{\theta_1}- R_{21}\cos{\theta_1}}{-R_{13}\sin{\theta_1}+
         R_{11}\cos{\theta_1}}\right)$ \vspace{.15 in}
     \\
     $\mathbf{R}_2(\theta_3)\mathbf{R}_1(\theta_2)\mathbf{R}_3(\theta_1)$
     &
     $\theta_1 =  \tan^{-1}(-R_{21}/R_{22})$ &
     $\theta_2 =  \sin^{-1}(R_{23})$ &
     $\theta_3  = \tan^{-1}\left(\displaystyle\frac{R_{32}\sin{\theta_1}+ R_{31}\cos{\theta_1}}{R_{12}\sin{\theta_1}+
     R_{11}\cos{\theta_1}}\right)$ \vspace{.15 in}
     \\
     %
     $\mathbf{R}_1(\theta_3)\mathbf{R}_2(\theta_2)\mathbf{R}_3(\theta_1)$
     &
     $\theta_1 =  \tan^{-1}(R_{12}/R_{11})$ &
     $\theta_2 =  \sin^{-1}(-R_{13})$ &
     $\theta_3  = \tan^{-1}\left(\displaystyle\frac{R_{31}\sin{\theta_1}- R_{32}\cos{\theta_1}}{-R_{21}\sin{\theta_1}+
     R_{22}\cos{\theta_1}}\right)$ \vspace{.15 in}\\
     %
     $\mathbf{R}_1(\theta_3)\mathbf{R}_2(\theta_2)\mathbf{R}_1(\theta_1)$
     &
     $\theta_1 =  \tan^{-1}(R_{12}/(-R_{13}))$ &
     $\theta_2 =  \cos^{-1}(R_{11})$ &
     $\theta_3  = \tan^{-1}\left(\displaystyle\frac{-R_{33}\sin{\theta_1}- R_{32}\cos{\theta_1}}{R_{23}\sin{\theta_1}+
     R_{22}\cos{\theta_1}}\right)$ \vspace{.15 in}\\
     %
     $\mathbf{R}_1(\theta_3)\mathbf{R}_3(\theta_2)\mathbf{R}_1(\theta_1)$
     &
     $\theta_1 =  \tan^{-1}(R_{13}/(R_{12}))$ &
     $\theta_2 =  \cos^{-1}(R_{11})$ &
     $\theta_3  = \tan^{-1}\left(\displaystyle\frac{-R_{22}\sin{\theta_1} + R_{23}\cos{\theta_1}}{-R_{32}\sin{\theta_1}+
     R_{33}\cos{\theta_1}}\right)$ \vspace{.15 in}\\
     %
     $\mathbf{R}_2(\theta_3)\mathbf{R}_1(\theta_2)\mathbf{R}_2(\theta_1)$
     &
     $\theta_1 =  \tan^{-1}(R_{21}/(R_{23}))$ &
     $\theta_2 =  \cos^{-1}(R_{22})$ &
     $\theta_3  = \tan^{-1}\left(\displaystyle\frac{-R_{33}\sin{\theta_1} + R_{31}\cos{\theta_1}}{-R_{13}\sin{\theta_1}+
     R_{11}\cos{\theta_1}}\right)$ \vspace{.15 in}\\
     %
     $\mathbf{R}_2(\theta_3)\mathbf{R}_3(\theta_2)\mathbf{R}_2(\theta_1)$
     &
     $\theta_1 =  \tan^{-1}(R_{23}/(-R_{21}))$ &
     $\theta_2 =  \cos^{-1}(R_{22})$ &
     $\theta_3  = \tan^{-1}\left(\displaystyle\frac{-R_{11}\sin{\theta_1} - R_{13}\cos{\theta_1}}{R_{31}\sin{\theta_1}+
     R_{33}\cos{\theta_1}}\right)$ \vspace{.15 in}\\
     %
     $\mathbf{R}_3(\theta_3)\mathbf{R}_1(\theta_2)\mathbf{R}_3(\theta_1)$
     &
     $\theta_1 =  \tan^{-1}(R_{31}/(-R_{32}))$ &
     $\theta_2 =  \cos^{-1}(R_{33})$ &
     $\theta_3  = \tan^{-1}\left(\displaystyle\frac{-R_{22}\sin{\theta_1} - R_{21}\cos{\theta_1}}{R_{12}\sin{\theta_1}+
     R_{11}\cos{\theta_1}}\right)$ \vspace{.15 in}\\
     %
     $\mathbf{R}_3(\theta_3)\mathbf{R}_2(\theta_2)\mathbf{R}_3(\theta_1)$
     &
     $\theta_1 =  \tan^{-1}(R_{32}/(R_{31}))$ &
     $\theta_2 =  \cos^{-1}(R_{33})$ &
     $\theta_3  = \tan^{-1}\left(\displaystyle\frac{-R_{11}\sin{\theta_1} + R_{12}\cos{\theta_1}}{-R_{21}\sin{\theta_1}+
     R_{22}\cos{\theta_1}} \right)$ \vspace{.15 in}\\
         \hline \hline
        \end{tabular}
        \label{table:DCMtoEulerAngles}
\end{table}
