\chapter{Numerical Algorithms} \label{Ch:NumericalAlgorithms}

\section{Lagrange Interpolation}

Below we describe the algorithm used for lagrange interpolation.
This includes specifying how points  from the known data are chosen
for interpolation, and how interpolation is performed for multiple
function values (i.e. position and velocity) at the desired
interpolation point.

Assume we have $m$ functions to interpolate and that each function
is known at $\ell$ values of the independent variable $x$ (in
ephemeris interpolation this is time) as illustrated in
Table~\ref{Table:InterpolationData}. Given some point, say $x'$, we
desire the interpolated values of the functions, or $\mathbf{f}'
\approx \mathbf{P}(x')$ where $P$ is the Lagrange interpolating
polynomial.

%
\begin{table}[ht] \centering
\caption{Example of Data for interpolation}
 \begin{tabular}{ccccc}  \hline \hline
         x & $\mathbf{f}_1$ & $\mathbf{f}_2$ & \dots & $\mathbf{f}_m$ \\ \hline
         $x_1$ & $f_1(x_1)$ &  $f_2(x_1)$ &  \dots & $f_m(x_1)$\\
         $x_2$ & $f_1(x_2)$ &  $f_2(x_2)$ &  \dots &$f_m(x_2)$\\
         $x_3$ & $f_1(x_3)$ &  $f_2(x_3)$ &  \dots &$f_m(x_3)$\\
         $x_4$ & $f_1(x_4)$ &  $f_2(x_4)$ &  \dots &$f_m(x_4)$\\
         \vdots   & \vdots &   \vdots &  $\ddots$ & \vdots\\
         $x_{\ell-1}$ & $f_1(x_{\ell-1})$ &  $f_2(x_{\ell-1})$ & \dots & $f_m(x_{\ell-1})$\\
         $x_{\ell}$ & $f_1(x_\ell)$ &  $f_2(x_\ell)$ & \dots & $f_m(x_\ell)$\\
         \hline \hline
         \label{Table:InterpolationData}
 \end{tabular}
 \end{table}
 %

Before proceeding we define the following variables:
%
\begin{center}
    \begin{minipage}[t]{5.0 in}
        \begin{tabbing}[htbp!]
            123456 \= dummy line \kill
            $\mathbf{x}$ \> $m_x$ x 1 Known values of the independent variable (monotonically increasing or decreasing) \\
            $x'$ \>  Value of independent variable a the required interpolation point\\
            $\mathbf{f}_i$ \> $m_x$ x $m_f$ Array of dependent variable values for the $i^{th}$ function\\
            $\mathbf{f}'$ \> 1 x $m_f$ Array of interpolated function values at $x'$\\
            $n$ \> Order of interpolation \\
            $m_x$  \>  Number of points in $\mathbf{x}$ and $\mathbf{f}_i$ \\
            $m_f$  \>  Number of functions to be interpolated \\
        \end{tabbing}
    \end{minipage}
\end{center}

For interpolation to be feasible, several conditions must be
satisfied.  First, $x'$ must lie between the minimum and maximum
value of $\mathbf{x}$, or at the boundaries ( otherwise we would be
extrapolating ):
%
\begin{equation}
     x' \geq \mbox{min}(\mathbf{x})
\end{equation}
%
and
%
\begin{equation}
     x' \leq \mbox{max}(\mathbf{x})
\end{equation}
%
Second, there must be enough data points in points in $\mathbf{x}$
to support interpolation to the desired order.  If the requested
interpolation is of order $n$, then we require $n+1$ data points to
perform the interpolation.  Hence the following criteria must be
met:
%
\begin{equation}
    m_x \geq n + 1
\end{equation}
%

If the requested interpolation is feasible, we must choose the
subset of $\mathbf{x}$ to use for interpolating the data at $x'$.
Define $x(q)$ as the $q^{th}$ element of $\mathbf{x}$. Given $x'$,
we choose $q$ to minimize the difference between $x'$ and the mean
of $x(q)$ and $x(q + n)$, or:
%
\begin{equation}
    \underset{q}{\mbox{min }} \left| \frac{x(q+n) + x(q)}{2} - x'\right|  \hspace{.3 in} ( q \in 1,2,3,... m_x)
\end{equation}
%
Choosing $q$ in this way places $x'$ as near to the center of the
interpolation interval as possible.

The standard formula for the Lagrange interpolating polynomial for a
single function is
%
\begin{equation}
    P(x) =  \sum_{k=1}^n P_j(x) \label{Eq:LagrangeInterp}
\end{equation}
%
where $P_j(x)$ is given by
%
\begin{equation}
   P_j(x) = y_j\prod_{k=1,k \neq j}^n\frac{x - x_j}{x_j - x_k} \label{Eq:LagrangePj}
\end{equation}
%
Lagrange interpolation is an efficient method when interpolating
multiple data sets available at the same independent variable
points.  This is due to the fact that the product term in
Eq.~(\ref{Eq:LagrangePj}) is only a function of the values of the
independent variables and the desired interpolation point and not on
the function being interpolated.  As a result, we can evaluate the
product term one time and use it for many functions (which is what
is done when interpolating an ephemeris file).  The algorithm for
implementing Lagrangian interpolation for multiple functions is
shown below.

\begin{center}
\begin{minipage}{6 in}
\begin{small}
\begin{algorithm}[H]

    \KwIn{$\mathbf{x}, \mathbf{f},x',q,n$}
    %
    \KwOut{$\mathbf{f}'$}
    %
    $\mathbf{f}' = \mathbf{0}_{(1 \mbox{ x } m_f)}$\;
    %
    \For{ i = q to $q+n$}
    {
        $\mathbf{prod} = f(i,:)$ \% The $i^{th}$ row of $\mathbf{f}$ \;
        \For{ $j = q$ to $q+n$}
        {
            \If{ $ i \neq j$}
            {
            \% The product in Eq.~(\ref{Eq:LagrangePj})\;
                    $ \mathbf{prod} = \mathbf{prod} \cdot \displaystyle\frac{x' - x(j)}{x(i) - x(j)}$\;
            }
        }
       \% The summation in Eq.~(\ref{Eq:LagrangeInterp})\;
       $ \mathbf{f}'= \mathbf{f}' + \mathbf{prod}$\;
    }
    \hspace{.2 in}
    %
    \label{alg:LagrangeInterpolation}\caption{Algorithm for Lagrangian Interpolation}
    %
\end{algorithm}
\end{small}
\end{minipage}
\end{center}


\section{Quadratic Polynomial Interpolation }


We are given three data points defined by a vector of independent
variables
%
\begin{equation}
     \mathbf{x} = [\hspace{.05 in} x_1 \hspace{.05 in} x_2 \hspace{.05 in}
     x_3 \hspace{.05
     in}]^T
\end{equation}
%
and a vector of corresponding dependent variables
%
\begin{equation}
     \mathbf{y} = [\hspace{.05 in} y_1 \hspace{.05 in} y_2 \hspace{.05 in}
     y_3  \hspace{.05
     in}]^T
\end{equation}
%
we wish to find a quadratic polynomial that fits the data such that
%
\begin{equation}
   y = Ax^2 + Bx +C
\end{equation}
%
We begin by forming the system of linear equations
%
\begin{equation}
    \left(%
    \begin{array}{ccc}
       x_1^2 & x_1 & 1 \\
       x_2^2 & x_2 & 1 \\
       x_3^2 & x_3 & 1 \\
    \end{array}%
    \right)
    %
     \left(%
    \begin{array}{ccc}
       A \\
       B \\
       C \\
    \end{array}%
    \right)
    =
     \left(%
    \begin{array}{ccc}
       y_1 \\
       y_2 \\
       y_3 \\
    \end{array}%
    \right)
\end{equation}
%
We can solve for the coefficients using
%
\begin{equation}
    A = \frac{
    %  Numerator
    \left|%
    \begin{array}{ccc}
       y_1 & x_1 & 1 \\
       y_2 & x_2 & 1 \\
       y_3 & x_3 & 1 \\
    \end{array}%
    \right|
    }
    %  Denominator
    {    \left|%
    \begin{array}{ccc}
       x_1^2 & x_1 & 1 \\
       x_2^2 & x_2 & 1 \\
       x_3^2 & x_3 & 1 \\
    \end{array}%
    \right|}
\end{equation}
%
\begin{equation}
    B = \frac{
    %  Numerator
    \left|%
    \begin{array}{ccc}
     x_1^2 & y_1 & 1 \\
     x_2^2 & y_2 & 1 \\
     x_3^2 & y_3 & 1 \\
    \end{array}%
    \right|
    }
    %  Denominator
    {    \left|%
    \begin{array}{ccc}
       x_1^2 & x_1 & 1 \\
       x_2^2 & x_2 & 1 \\
       x_3^2 & x_3 & 1 \\
    \end{array}%
    \right|}
\end{equation}
%
\begin{equation}
    C = y_3 - A x_3^2 - B x_3
\end{equation}

\section{Cubic Spline (Not-a-Knot) Interpolation }

We are given five data points defined by a vector of independent
variables
%
\begin{equation}
     \mathbf{x} = [\hspace{.05 in} x_1 \hspace{.05 in} x_2 \hspace{.05 in}
     x_3 \hspace{.05 in} x_4 \hspace{.05 in} x_5 \hspace{.05
     in}]^T
\end{equation}
%
and a vector of corresponding dependent variables
%
\begin{equation}
     \mathbf{y} = [\hspace{.05 in} y_1 \hspace{.05 in} y_2 \hspace{.05 in}
     y_3 \hspace{.05 in} y_4 \hspace{.05 in} y_5 \hspace{.05
     in}]^T
\end{equation}
%
we wish to find the four cubic polynomials, $i = 1,2,3,4$, such that
%
\begin{equation}
    p_i = a_i(x - x_i)^3 + b_i(x - x_i)^2 + c_i(x - x_i) + d_i
\end{equation}
%
where the values for $x_i$ are known from the inputs.

To calculate the coefficients $a_i$, $b_i$, $c_i$, and $d_i$ we
start by calculating the eight quantities
%
\begin{eqnarray}
     h_i      &=& x_{i+1} - x_i\\
     \Delta_i &=& \displaystyle\frac{y_{i+1} - y_i}{h_i}
\end{eqnarray}
%
Next we solve the following system of linear equations
%
\begin{equation}
     \mathbf{A}\mathbf{S} = \mathbf{B}
\end{equation}
%
where the components of $\mathbf{A}$ are given by
%
\begin{eqnarray}
     A_{11} &=& 2h_2 + h_1 \\
     A_{12} &=& 2h_1 + h_2 \\
     A_{13} &=& 0 \\
     A_{21} &=& 0 \\
     A_{22} &=& h_3 + 2h_4 \\
     A_{23} &=& 2h_3 + h_4 \\
     A_{31} &=& \displaystyle\frac{h_2^2}{h_1 + h_2} \\
     A_{32} &=& \displaystyle\frac{h_1h_2}{( h_1 + h_2 )} + 2( h_2 + h_3 ) + \displaystyle\frac{h_3h_4}{( h_3 + h_4
     )}\\
     A_{33} &=& \displaystyle\frac{h_3^2}{h_3 + h_4}
\end{eqnarray}
%
the components of $\mathbf{B}$ are
%
\begin{eqnarray}
    B_{11} &=& 6( \Delta_2 - \Delta_1 )\\
    B_{21} &=& 6( \Delta_4 - \Delta_3 )\\
    B_{31} &=& 6( \Delta_3 - \Delta_2 )
\end{eqnarray}
%
and $\mathbf{S} = [\hspace{.05 in} S_1 \hspace{.05 in} S_3
\hspace{.05 in} S_5 \hspace{.05 in}]^T$. We can solve for the
components of $\mathbf{S}$ using Cramer's Rule as follows
%
\begin{equation}
    S_1 = \frac{
    %  Numerator
    \left|%
    \begin{array}{ccc}
       B_{11} & A_{12} & A_{13} \\
       B_{21} & A_{22} & A_{23} \\
       B_{31} & A_{32} & A_{33} \\
    \end{array}%
    \right|
    }
    %  Denominator
    {\mathbf{| \mathbf{A}|}}
\end{equation}
%
\begin{equation}
    S_3 = \frac{
    %  Numerator
    \left|%
    \begin{array}{ccc}
       A_{11} & B_{11} & A_{13} \\
       A_{21} & B_{21} & A_{23} \\
       A_{31} & B_{31} & A_{33} \\
    \end{array}%
    \right|
    }
    %  Denominator
    {\mathbf{| \mathbf{A}|}}
\end{equation}
%
\begin{equation}
     S_5 = \frac{B_{31} - A_{31} S_1 - A_{32} S_3}{A_{33}}
\end{equation}
%
Now we can calculate $S_2$ and $S_4$ using
%
\begin{equation}
     S_2 = \frac{ h_2 S_1 + h_1 S_3  }{  h_1 + h_2 };
\end{equation}
%
\begin{equation}
     S_4 = \frac{ h_4 S_3 + h_3 S_5  }{  h_3 + h_4 };
\end{equation}
%

Finally, the coefficients for the $i^{th}$ cubic polynomial are
given by
\begin{eqnarray}
   a_i &=&  \frac{ S_{i+1} - S_i   }{ 6  h_i}\\
   b_i &=&  \frac{S_i}{2}\\
   c_i &=&  \frac{ y_{i+1} - y_i  }{ h_i } - \frac{ 2 h_i S_i + h_i S_{i+1} }{  6
   }\\
   d_i &=&  y_i
\end{eqnarray}


\section{Root Location using Brent's Method}

NOTE: The implementation used in GMAT's even locators follows the Algorithm
in Numerical Recipes 3rd edition which slightlyl different than the algorithm presented below.

Brent's\cite{Brent:73} method is a root finding algorithm that takes
advantage of superlinear convergence properties of the secant method
and inverse quadratic interpolation, and the robustness of the
bisection method.  The algorithm is a modification of Dekker's
method which combines the bisection method and the secant method
\cite{Dekker:69}.

The algorithm requires as inputs to real numbers that bound the
desired root.

\begin{tabbing}
    12345678912345 \= Reynolds number based on length $s$ \kill
    $f$         \>  Function whose root is desired \\
    $a$         \>  Initial guess for independent variable ($a$ and $b$ must bound the desired root)\\
    $b$         \>  Initial guess for independent variable ($a$ and $b$ must bound the desired root) \\
    $tol$     \>  Convergence tolerance.\\
    $N$       \> Maximum number of iterations \\
    $x_o$     \>  Root value\\
    $i$         \> Number of iterations\\
    $\epsilon$ \> Machine precision\\
\end{tabbing}

\begin{center}
\begin{minipage}{6 in}
\begin{small}
\begin{algorithm}[H]

    \KwIn{$f,a,b,tol,N,$}
    %
    \KwOut{$x_o,i$}
    %
    $f_a = f(a)$; $f_b = f(b)$\;
    \textbf{if} $f_a \cdot f_b > 0$; error and return; \textbf{end} \;
    $c = a;$ $f_c = f_a;$ $d = b - a;$ $e = d$\;
    \If{ $|f_c| \leq |f_b|$}
    {
     $a = b;$ $b = c;$ $c = a;$
    $f_a = f_b;$ $f_b = f_c;$ $f_c = f_a;$
    }
    $t = 2\epsilon | b | + tol$\;
    $m = 0.5(c - b)$ \;
    $i = 1$\;
    \While{$ | m | >= t$ and $| f_b | >= t$ and $ i \leq N$}
        {
        \eIf{ $|e | \leq t $ or $ |f_a| <  |fb|$}
          {
           $d = m$; $ e = m$\;
          }{
           $s = f_b/f_a$\;  % line 40
           \eIf {$ a = c $}
              {
              $p = 2.0 \cdot m \cdot s$\;
              $q = 1.0 - s$\;
               }
              {
                $q = f_a/f_c$; $r = f_b/f_c$\;
                $p = s(2.0 \cdot m \cdot q (q-r) - (b - a)(r - 1.0))$\;
                $q = (q - 1.0)(r - 1.0)(s - 1.0)$\;
                }
          \eIf{ $p >= 0.0$ }{ %line 60
            $q = -q$\;
          }{
            $p = -p$\;  % line 70
          }
        $s = e$\;  % line 80
        $e = d$\;
        \eIf {($2 \cdot p < 3.0 \cdot m \cdot q- |t\cdot q|) $ or $ ( p \leq | 0.5 \cdot s\cdot q | )$}
           {
            $d = p/q$\;
         }{
            $d = m$; $e = m$\;
         }
       }
       $a  = b$; $f_a = f_b$\;
       \eIf { $|d| \geq  t$ }{
          $b = b + d$;
        }
        {
        \eIf {$ m \geq 0.0$ }{
           $ b = b + t$;
        }{
           $ b = b - t$;
         }
        }
       $ f_b = f(b)$\;
        \eIf { ( $f_b > 0.0$ and $f_c > 0$ ) or ( $f_b < 0.0$ and $f_c < 0$ )} {
            $c = a;$ $f_c = fa;$ $d = b - a;$ $e = d$\;
            }
         {
               \If{ $|f_c| \leq |f_b|$}
            {
             $a = b;$ $b = c;$ $c = a;$
            $f_a = f_b;$ $f_b = f_c;$ $f_c = f_a;$
            }
                 }
        $t = 2\epsilon |b| + tol$; % line 3
        $m    = 0.5(c - b)$;
        $ i = i + 1$;

    }
    $x_o = b$\;
    \hspace{.2 in}
    %
    \label{alg:BrentsMethod}\caption{Brent's Method for Root Finding}
    %
\end{algorithm}
\end{small}
\end{minipage}
\end{center}
