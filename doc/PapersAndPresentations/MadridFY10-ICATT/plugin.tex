\subsection{The Plug-in Development Process}

GMAT's Plug-in capabilities let developers extend the classes derived from the
GmatBase class. Instances of these classes are constructed using GMAT's Factory
subsystem.  Plug-in authors capitalize on this design by creating custom
factories designed to support the new components that they are adding to the
system.

A GMAT plug-in is a shared library, linked against a shared library build of
GMAT's base code, that contains the class code for the new capability, one or
more supporting Factories for the new components, and a set of three C-style
interface functions that are accessed by GMAT to load the plug-in.  In the
following paragraphs,  we describe how to use each of these plug-in interfaces,
starting from the build requirements for GMAT, proceeding through the
interface functions and factory requirements, and finishing with the actual new
component that is being added.  The next section of this document describes a
sample plug-in which illustrates the process.

\subsection{Preparing GMAT for Plug-in Use}

GMAT plug-ins create classes that are derived from classes in GMAT's base code.
The plug-in needs to be linked against that code in order to use the
capabilities of the base classes, and to build the complete derived objects. In
order to do this, the plug-in library needs to be linked against the base code
that will be run when GMAT runs.

One option for a plug-in developer when compiling is to build the plug-in
using all of the required classes as part of the plug-in library. That approach
makes the plug-in much larger than necessary, and makes the prospect of
incompatibility between the plug-in and the evolving GMAT codebase likely.  The
preferred approach to plug-in development is to build GMAT's base code as one or
more shared libraries.  GMAT's build control file, BuildEnv.mk, has a setting
for this option.  A developer that is building a plug-in need only add this line
to the file:

\begin{quote}
\begin{verbatim}
SHARED_BASE = 1
\end{verbatim}
\end{quote}

\noindent and then clean and build GMAT.  The build process will build the
base code as a shared library -- named libGmatBase or libGmatBaseNoMatlab,
depending on the MATLAB build flags -- that can be used for plug-in
development. Once GMAT has been built this way, the plug-in developer is ready
to start coding the plug-in.

\subsection{The Plug-in Interface Functions}

GMAT accesses new user classes contained in plug-in libraries by calling three
methods in the plug-in library: GetFactoryCount(), GetFactoryPointer(), and
SetMessageReceiver(). GMAT uses these functions as the entry point into the
plug-in components.  They are defined as follows:

\begin{itemize}
\item \textbf{Integer GetFactoryCount()}:  This function reports the
number of Factory classes that are contained in the plug-in.  The current
implementation of GMAT requires that factories only support a single core type
because of an implementation limitation in the FactoryManager, so larger
plug-in libraries may need more than one supporting factory.
\item \textbf{Factory* GetFactoryPointer(Integer index)}:  This function
retrieves Factory pointers
from the plug-in.  Once GMAT knows the number of factories in the library, it
calls this function to retrieve the contained factories one at a time.
\item \textbf{void SetMessageReceiver(MessageReceiver* m)}: Messages posted
in GMAT are all sent to a MessageReceiver.  This optional function is used to
set the MessageReceiver for a plug-in if the developer incorporated GMAT's base
code in the plug-in library, rather than linking against a shared library.
\end{itemize}

%\noindent The code in Section~\ref{sec:InterfaceCode} shows one implementation
%of these methods.

\subsection{The Custom Factory}

GMAT's Factory subsystem is described in some detail in the Architectural
Specification\cite{ArchSpec}. GMAT uses this subsystem to create user objects
that are needed to run a mission.  The class diagram for the subsystem is shown
in Figure~\ref{fig:FactorySubsystem}.

\begin{figure*}[htb]
\centering
%\begin{picture}(180,280)(55,0)\special{psfile= Images/TheFactorySubsystem.ps
%hscale= 45 vscale= 45 hoffset = -40 voffset = 0}
%\end{picture}
\includegraphics*[scale=0.55]{Images/TheFactorySubsystem.ps}
\caption{\label{fig:FactorySubsystem} The Factory Subsystem}
\end{figure*}
%\begin{figure}[tb]
%\begin{center}
%%\includegraphics*[bb=0 0 200 147]{Images/TheFactorySubsystem.ps}
%\includegraphics*[scale=0.28]{Images/TheFactorySubsystem.ps}
%\caption{\label{fig:FactorySubsystem} The Factory Subsystem}
%\end{center}
%\end{figure}

The Factory base class defines the interface used to create user objects.  It includes subclass
specific interfaces for the core user class types, as can be seen from this portion of the class
definition:
\begin{scriptsize}
\begin{verbatim}
class GMAT_API Factory
{
public:
   // Return objects as generic type
   virtual GmatBase* CreateObject(
         const std::string &ofType,
         const std::string &withName = "");

   // return objects as specified types
   virtual SpaceObject*CreateSpacecraft(
         const std::string &ofType,
         const std::string &withName = "");
   virtual Propagator* CreatePropagator(
         const std::string &ofType,
         const std::string &withName = "");
   virtual ForceModel* CreateForceModel(
         const std::string &ofType,
         const std::string &withName = "");
   ...
\end{verbatim}
\end{scriptsize}

\noindent When the programmer has decided what type of new component needs to
be implemented, she creates a new factory that implements the corresponding
factory method from this group and calls the new component's constructor. Each
of the factory classes shown in Figure~\ref{fig:FactorySubsystem} is available
for browsing in the src/base/factory folder of GMAT's source tree, so the
developer should be able to select an appropriate Factory as a starting point
for the custom Factory. The sample code includes code for a Factory supporting a
new PhysicalModel class.

\subsection{The New Feature}

The purpose of all of the support code described above is, of course, the
implementation of a new user component.  GMAT provides a rich set of classes
that can be used as a starting point for the new component.  Programmers use
one of the existing classes as the base class for the new feature.  That
approach guarantees that GMAT has support for most or all of the plug-in
component in the core GMAT engine, significantly reducing integration
efforts.  The next section describes the design and implementation of one such
component: a custom force used in GMAT's force model.

\section{An Example}

This section presents the design for a complete GMAT plug-in library.  The
example shown here is a new force for the force model.  The new force used for
this example is a directed solar radiation pressure force, appropriate for solar
sailing, as described in Montenbruck and Gill\cite{mg}.  The complete source
code for this plug-in library, along with make files, is available from the Plug-in
project on SourceForge\cite{plugins}.

For the purposes of this example, the spacecraft attitude will be used to
calculate the direction of the normal to the reflecting surface, and thus the
direction of of the force vector.  More specifically, for this example the
spacecraft's x-axis, as specified by its attitude, will be treated as the
normal, $\hat \textbf{n}$, to the surface that the light hits. The spacecraft's
coefficient of reflectivity, $1 <= C_r <= 2$, determines the amount of light
that reflects off of the spacecraft; $C_r = 1$ means that all of the incident
light is absorbed, while $C_r = 2$ means that the light is all reflected.

The following sections define the new force, describe the class used to model
the force, and then present the code needed to add the new force to GMAT using
the plug-in architecture.

\subsection{The Physics of the SolarSail Model}

We'll begin by describing the model implemented in the code.  The vector from
the spacecraft to the Sun, $\mathbf{e}_s$, makes an angle $\theta$ with the surface
normal $\hat \textbf{n}$.  The absorbed radiation applies a force $\mathbf{f}_a$ directed
opposite to the sun vector.  The reflected radiation applies a force directed
anti-parallel to the normal vector, $\hat \textbf{n}$.

The magnitude of each of these forces is equal to the incident radiation
pressure, $P_r$, multiplied by the incident surface area, $A$, and then adjusted
to take into account the amount of light reflected or absorbed.  The force for
absorbed light is given by

\begin{equation}\label{eq:absorbedLight}
\textbf{F}_{abs} = -(1 - \varepsilon) P_r A \cos(\theta) \hat{\textbf{e}}_s
\end{equation}

\noindent while that of the reflected light is given by

\begin{equation}\label{eq:reflectedLight}
\textbf{F}_{ref} = -2 \varepsilon P_r A \cos^2(\theta) \hat{\textbf{n}}
\end{equation}

\noindent The constant $\varepsilon$ in these equations is the percentage of the
incident light that is reflected from the surface, and is related to the
coefficient of reflectivity through the equation

\begin{equation}\label{eq:CrEpsilon}
C_r = 1 + \varepsilon
\end{equation}

\noindent Finally, the factor of 2 in equation~\ref{eq:reflectedLight} accounts
for the reflectance effect of Newton's third law.  The cosine term in this
equation is squared because the reflected light applies its force exclusively in
the anti-normal direction; the force components parallel to the reflecting
surface from the incoming and outgoing light cancel out.

The incident radiation pressure, $P_r$, is a function of the distance from the
Sun to the spacecraft.  Spacecraft closer to the Sun experience a larger
incident radiation pressure than those further away.  This effect follows an
inverse square relationship; if the solar radiation pressure at one astronomical
unit from the Sun, $R_{AU}$, is written as $P_{AU}$, the radiation pressure at
an arbitrary distance $r_s$ is given by

\begin{equation}
P_r = P_{AU} \left(\frac{R_{AU}}{r_s}\right)^2
\end{equation}

Putting all of these pieces together, the force implemented in this plug-in is
given by

\begin{equation}
\textbf{F}_{sail} = -P_r A \cos\theta \{(1 -
\varepsilon) \hat{\textbf{e}}_s + 2 \varepsilon \cos\theta \hat{\textbf{n}}\}
\end{equation}

GMAT's equations of motion are expressed in terms of derivatives of the position
vectors.  That means that the function that models a force in GMAT,
\texttt{GetDerivatives()}, needs to express the effect of the force in terms of
an acceleration.  The Spacecraft model contains a reflectivity coefficient,
$C_r$, which matches the coefficient in equation~\ref{eq:CrEpsilon}. Using
equation~\ref{eq:CrEpsilon} and the relationship $\textbf{F} = m\textbf{a}$, the
resulting acceleration is

\begin{equation}
\textbf{a}_{sail} = -P_r \frac{A}{m} \cos\theta \{(2 -
C_r) \hat{\textbf{e}}_s + 2 (C_r - 1) \cos\theta \hat{\textbf{n}}\}
\end{equation}

\noindent This equation is encapsulated in the class, SolarSail, described below.

\subsection{The SolarSail Class}

GMAT's force model classes are all implementations of a base PhysicalModel
class. The SolarSail plug-in uses many of the features and structures already
implemented in the SolarRadiationPressure class, one of the members of the force
model subsystem.  The SolarSail class uses its own factory, implemented as the
Factory component of the plug-in library.  These additions are shown in the
ForceModel class hierarchy, shown in Figure~\ref{figure:ForceModelWithSail}.

\begin{figure}[tb]
\begin{center}
\includegraphics*[scale=0.35]{Images/ForceModelClassesWithSailPlugin.ps}
\caption[The Solar Sail Model Components]{\label{figure:ForceModelWithSail} The
Solar Sail Model Components.  The plug-in components are shown in blue.}
\end{center}
\end{figure}

The solar sail force uses many of the same calculations as are performed for
GMAT's solar radiation pressure model.  For that reason, the SolarSailForce
class is derived from the SolarRadiationPressure class.  The new class does need
to implement a different acceleration model, so it overrides the
GetDerivatives() method to provide accelerations as described above. It also
provides implementations for the four C++ default methods: the constructor, copy
constructor, destructor, and assignment operator.  The new force has data
structures that need to be initialized, so the Initialize() method is overridden
(and calls the SolarRadiationPressure::Initialize() method internally).  GMAT's
ForceModel class contains a method, IsUserForce(), which is called to determine
how to handle scripting for forces added by users.  This method is overridden to
report the new force as a user force. Finally, the Clone() method is overridden
so that GMAT can make copies of the new force from a GmatBase pointer.

The full source code for the SolarSailForce class is available from
SourceForge\cite{plugins} in the trunk/SolarSail folder of the project's
Subversion repository.

\subsection{The SailFactory and Interface code}

The SailFactory is used to create new instances of the SolarSailForce.  The code
is identical to many of the core factories found in GMAT's src/base/factory file
folder.  There are three sections specific to the SolarSailForce: the
CreatePhysicalModel() method:

\begin{quote}
\begin{verbatim}
PhysicalModel* SailFactory::
   CreatePhysicalModel(
      const std::string &ofType,
      const std::string &withName)
{
   if (ofType == "SailForce")
      return new SolarSailForce(
            withName);

   return NULL;
}
\end{verbatim}
\end{quote}

\noindent and the code in the constructor and copy constructor that populates
the list of creatable object names.  That code has this form:

\begin{quote}
\begin{verbatim}
   if (creatables.empty())
   {
      creatables.push_back(
            "SailForce");
   }
\end{verbatim}
\end{quote}

\noindent The rest of the factory code fills out the required elements: the
constructors, assignment operator, and destructor, as required in GMAT's coding
standards\cite{codeStandards}.

The code in the interface functions is nearly as transparent.  There are two
C-style functions that are used in the plug-in implementation: GetFactoryCount()
and GetFactoryPointer().  The GetFactoryCount() method returns the number of
factories in the plug-in -- one (1) for this example.  GetFactoryPointer()
creates an instance of the SailFactory and returns it to GMAT when it is called
with an input index of 0 (indicating the first factory in the plug-in), and
returns NULL for calls with other factory indices.  More complicated plug-in
libraries use this feature to support multiple factories that are loaded by
calling GetFactoryPointer for each factory in the library.

Once the code described above is in place, it can be compiled into a shared
library that meets GMAT's plug-in requirements.  GCC make files for the solar
sail library plug-in are included in the SourceForge repository, along with a
configuration file, SolarSailEnv.mk, for each of our supported platforms.  The
build process compiles the source files into object files, links those object
files with references to the GMAT base code shared library (libGmatBase or
libGmatBaseNoMatlab, described above), and produces a shared library compatible
with your GMAT build.

\subsection{Adding the Plug-in to GMAT}

Once you have built the plug-in library described above, place the resulting
code in the folder that contains your GMAT executable.  The plug-in will become
available in GMAT if you add a line to your GMAT startup file identifying the
library as a plug-in.  The required line looks like this for a plug-in named
libSolarSail:

\begin{quote}
\begin{verbatim}
PLUGIN = libSolarSail
\end{verbatim}
\end{quote}

\noindent The actual plug-in file name depends on your operating system -- on
Windows, the file name would be ``libSolarSail.dll''; on Linux, it would be
``libSolarSail.so'', and on Mac, ``libSolarSail.dylib''.  GMAT manages the file
extension internally based on the operating system, so the line in the startup file
does not explicitly specify the shared library extension.

Once this line is in place in your startup file, GMAT will attempt to load the
plug-in when it is started.  On success, the capabilities of the plug-in code
-- in this case, the new solar sail model -- will be available for use from a
GMAT script.

\subsection{Upcoming Capabilities}

The current plug-in capability provides new components to GMAT's model, along
with the factory code that GMAT uses in its engine to make that capability
available through GMAT's scripting language.  At this writing, GMAT does not
yet have the ability to generate complete user interface elements that extend
the wxWidgets based graphical user interface.  The plug-in interfaces for that
capability are currently being designed, and should be ready for use in the
second quarter of 2010. 