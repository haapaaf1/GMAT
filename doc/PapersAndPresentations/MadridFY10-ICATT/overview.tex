\section{System Architecture}

GMAT's System Architecture, described in the GMAT Architectural
Specification\cite{ArchSpec}, can be broken into three types of components: the
Model, consisting of all of the elements required for simulating a spacecraft
mission; the Engine, responsible for managing and connecting together the model
elements; and interfaces, consisting of the scripting and graphical user
interfaces, the MATLAB interface, and a simple console based interface.  Users
interact with GMAT through any of these interfaces to control the engine, and by
controlling the engine, create a model that simulates specific spacecraft
missions.

GMAT's architecture was designed so that the elements of the engine simulate
spacecraft missions by operating on objects through fairly abstract classes.
Specialization of these classes results in the ability to create objects that
specialize these abstract interfaces into detailed elements of the spacecraft
simulation.  In other words, GMAT models a spacecraft mission by specializing
high level classes into subclasses responsible for numerically simulating
spacecraft and their environment, orbital computations, and the timeline
defining the evolution of these elements.  These specialized subclasses are the
architectural components that a user sees as resources and commands, as
described earlier in this paper.

While the system contains subclasses that simulate most of the components
anticipated by the project team, we also recognized from the start that GMAT
could not contain everything that every user would need.  The philosophy of
starting from abstract classes defining interfaces that are specialized to meet
mission needs provides a powerful mechanism for extending these classes through
subclasses that meet the needs of the user community.  The GMAT development
team has used this approach repeatedly to add new capabilities to the system.

The user configurable components of GMAT are all derived from a general purpose
base class, GmatBase, which defines interfaces used throughout the system to
access common properties of the user classes.  This base class defines a
framework used by GMAT's engine to manage the objects used when modeling a
mission.  Classes derived from GmatBase are specialized to model different
aspects of the mission.  Classes located deeper in the GmatBase class hierarchy
are, in general, more specialized than those at the higher levels.  Details of
GMAT's class structure can be found in the GMAT Architectural
Specification\cite{ArchSpec}, or by running GMAT's source code through the
Doxygen source code documentation generator\cite{Doxygen}.

These subclasses of GmatBase provide the framework for extension of GMAT's
capabilities, either through work directly in GMAT's source code, or through
plug-in modules loaded by GMAT at runtime.  The remaining pages of this article
explain the design of GMAT's plug-in architecture, and include an overview of a
simple plug-in available in source form that illustrates the key elements of a
GMAT plug-in extension.

\section{GMAT Plug-ins}

Builds of GMAT made using development source code after June 25,
2008 have the ability to load shared libraries at run time and retrieve new user
objects from these libraries. The approach taken for this capability was built
on a prototype extension implemented at Thinking Systems in April, 2008 to meet
specific needs of the LCROSS mission, and documented as an extension to
GMAT\cite{pluginProp}. The following paragraphs explain how to use the plug-in
extensions to add new capabilities to GMAT.  A specific example -- the addition
of a new force for GMAT's force model -- is described in some detail, with
emphasis on the features necessary for incorporation into GMAT at run time.

GMAT has been extended through the incorporation of new capabilities, loaded at
run time using shared libraries, that incorporate a variety of features.  These
libraries, called GMAT plug-ins, have been used to add new optimizers, to
develop estimation capabilities, to drive proprietary visualization
elements, and to support ephemeris generation.

We'll begin this section by looking at the steps needed to construct a GMAT
plug-in. Once these steps have been described, the design of the example plug-in
code -- a basic solar sail model for GMAT's force model -- is described, along
with descriptions of the pieces needed to incorporate the new model through the
plug-in interface.  Finally, we present the steps needed to tell GMAT about the
new plug-in. 